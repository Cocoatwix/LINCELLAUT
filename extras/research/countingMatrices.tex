
\documentclass[a4paper, 12pt, reqno]{amsart}
%\usepackage[T1]{fontenc}
\usepackage{fontspec}

\usepackage[parfill]{parskip}

\usepackage[margin=3cm]{geometry}

\usepackage{setspace}
\setstretch{1.25}

\usepackage{natbib}

\usepackage{dsfont}
\usepackage{amssymb}

\usepackage{verbatim}

\newtheorem{prop}{Proposition}
\newtheorem{coro}{Corollary}

\DeclareMathOperator{\Span}{span}
\newcommand{\Z}{\mathbb{Z}}
\newcommand\divides{\ | \ }
\newcommand\Mat[2][]{\mathbf{#1}^{\!#2}}
\newcommand\smallmat[1]{\left[\begin{smallmatrix}#1\end{smallmatrix}\right]}
\newcommand{\cycsp}[1]{\mathcal{S}_{#1}}

\newcommand{\dq}[1]{``#1''}

\renewcommand{\qedsymbol}{\textbf{QED}}

%Allowing propositions to be clicked on
%This needs to be the last package imported
\usepackage{hyperref}
\usepackage{xcolor}
\hypersetup
{
    colorlinks=false,
	pdfborder={0 0 0},
    pdftitle={Counting 2 by 2 Matrices Modulo a Prime with Factorable Characteristic Polynomials},
    pdfpagemode=FullScreen,
	pdfauthor={Zach Strong},
}

\begin{document}
	Consider some vector space $\Z_p^2$ and a vector $\vec{v} \in \Z_p^2$ whose components satisfy some linear congruence. For example, if our vector is represented as 
	$\smallmat{x\\y}$, then one possible linear relation it could satisfy is $x + y \equiv 0 \bmod{p}$. Another could be $2x + 3y \equiv 0 \bmod{p}$. In general, we'll 
	consider vectors whose components satisfy equations of the form $ax + by = 0 \bmod{p}$ for constants $a, b \in \Z_p$, where one of $a$ or $b$ is nonzero.
	
	Let's say our vector space is $\Z_7^2$ and our vector $\vec{v} \in \Z_7^2$ satisfies $x + y \equiv 0 \bmod{7}$. Then, the following must be true:
	\begin{center}
		\begin{tabular}{l}
			$2x + 3(x + y) \equiv 2x \mod{7}$,\quad and \\
			$2y + 4(x + y) \equiv 2y \mod{7}$.
		\end{tabular}
	\end{center}
	Combining the left-hand side of the congruences, we get that
	\begin{center}
		\begin{tabular}{l}
			$5x + 3y \equiv 2x \mod{7}$,\quad and \\
			$4x + 6y \equiv 2y \mod{7}$.
		\end{tabular}
	\end{center}
	Rewriting these congruences using matrices and vectors, we see that
	\[
		\begin{bmatrix}5 & 3 \\ 4 & 6\end{bmatrix}
		\begin{bmatrix}x \\ y\end{bmatrix}
		\equiv
		2\!\begin{bmatrix}x \\ y\end{bmatrix}
		\mod{7}.
	\]
	Thus, any vector in $\Z_7^2$ that satisfies $x + y \equiv 0 \bmod{7}$ will be an eigenvector for the matrix $\smallmat{5&3\\4&6}$ modulo 7 with its eigenvalue being 2. As
	an example, the vector $\smallmat{4\\3}$ satisfies $x + y \equiv 0 \bmod{7}$, and
	\[
		\begin{bmatrix}5 & 3 \\ 4 & 6\end{bmatrix}
		\begin{bmatrix}4 \\ 3\end{bmatrix}
		\equiv
		\begin{bmatrix}1 \\ 6\end{bmatrix}
		\equiv
		2\!\begin{bmatrix}4 \\ 3\end{bmatrix}
		\mod{7}.
	\]
	
	This example suggests a general way to construct 2 by 2 matrices with a specific set of vectors being eigenvectors. If we want all vectors that satisfy 
	$ax + by \equiv 0 \bmod{p}$ to be eigenvectors with eigenvalue $\lambda$, then we create the matrix
	\begin{equation}
		\label{mat:desiredForm}
		\begin{bmatrix}
			(ra + \lambda) & rb \\
			sa & (sb + \lambda)
		\end{bmatrix} \mod{p},
	\end{equation}
	where $r$ and $s$ are arbitrary constants in $\Z_p$.
	
	In fact, we can show that \emph{any} 2 by 2 matrix with eigenvectors of eigenvalue $\lambda$ can be created this way. Let $\smallmat{h&i\\j&k}$ be a matrix modulo $p$ that
	has eigenvectors with eigenvalue $\lambda$. There are two cases we need to consider.
	
	\textbf{Case 1: For our eigenvectors $\smallmat{x\\y}$, one of $x$ or $y$ is zero.} Without loss of generality, we'll assume that $y \equiv 0 \bmod{p}$. The 
	$x \equiv 0 \bmod{p}$ case is handled in a similar way. In this case, our eigenvectors will satisfy \emph{any} relation of the form $0x + by \equiv 0 \bmod{p}$. As well, 
	for vectors of the form $\smallmat{x\\0}$ to be eigenvectors, it must be the case that the component $j$ in our given matrix is zero, and the component $h$ must be 
	$\lambda$. Thus, our matrix will look like
	\[
		\begin{bmatrix}
			\lambda & i \\
			0       & k
		\end{bmatrix} \mod{p}.
	\]
	With the correct choice of constants $r$, $s$, and $b$, we see that this matrix indeed matches the desired form of matrix \ref{mat:desiredForm}.
	
	\textbf{Case 2: None of the components of our eigenvectors are zero.} Note that, in this case, the congruence $ax + by \equiv 0 \bmod{p}$ that the eigenvectors satisfy 
	must have both $a$ and $b$ being nonzero. Otherwise, if only one of $a$ or $b$ was nonzero, the congruence would be impossible to satisfy with our given eigenvectors.

	Now, for any of these eigenvectors $\vec{v} \equiv \smallmat{x\\y}$, we have that
	\[
		\begin{bmatrix}h-\lambda & i \\ j & k-\lambda\end{bmatrix}
		\begin{bmatrix}x \\ y\end{bmatrix}
		\equiv
		\begin{bmatrix}0 \\ 0\end{bmatrix}
		\mod{p}
	\]
	just from the defining equation of the eigenvectors. Rewriting this matrix equation gives us two congruences the eigenvector must satisfy:
	\begin{center}
		\begin{tabular}{rl}
			$(h - \lambda)x + iy  \equiv 0 \mod{p}$, & \quad and \\
			$jx + (k - \lambda)y \equiv 0 \mod{p}$. &
		\end{tabular}
	\end{center}
	These two congruences, though appearing different, are actually just scalar multiples of each other. To see why, note that if a vector satisfies the congruence 
	$ax + by \equiv 0 \bmod{p}$, the only way it can also satisfy $rax + cy \equiv 0 \bmod{p}$ for some constants $r, c \in \Z_p$, $r \not\equiv 0$ is if 
	$c \equiv rb \bmod{p}$.\footnote{Since $\Z_p$ is a field, we can take $rax + cy \equiv 0 \bmod{p}$ and multiply both sides by $r^{-1}$ to give 
	$ax + r^{-1}cy \equiv 0 \bmod{p}$. If we subtract our known relation from this and rearrange, we get that $r^{-1}cy \equiv by \implies c \equiv rb \bmod{p}$.} Thus, 
	because $ra$ can be any element of $\Z_p^*$ (by virtue of $\Z_p$ being a field), once a vector satisfies some relation $ax + by \equiv 0 \bmod{p}$, the only other 
	relations of the same form it can satisfy are scalar multiples of it.
	
	Because the two congruences are scalar multiples of each other, we have that
	\begin{equation}
		\label{eqn:eigenEqnScalarMult}
		jx + (k - \lambda)y \equiv r(h - \lambda)x + riy \mod{p}
	\end{equation}
	for some constant $r \in \Z_p$. Then, using the congruences given by our defining eigenvector equation and congruence \ref{eqn:eigenEqnScalarMult}, we can rewrite our
	original matrix as
	\[
		\begin{bmatrix}
			(1(h - \lambda) + \lambda) & 1i \\
			1j & (1(k - \lambda) + \lambda)
		\end{bmatrix} 
		\equiv
		\begin{bmatrix}
			(1(h - \lambda) + \lambda) & 1i \\
			r(h - \lambda) & (ri + \lambda)
		\end{bmatrix}
		\mod{p}.
	\]
	This is the desired form of matrix \ref{mat:desiredForm}. 
	
	From these two cases, we can conclude that our method for constructing 2 by 2 matrices with a particular set of eigenvectors and a given eigenvalue can construct 
	\emph{all} possible 2 by 2 matrices with eigenvectors.
	
	One question we may ask about this method: can we use it to count the number of 2 by 2 matrices with eigenvectors? Put another way, can we use our particular construction 
	to count the number of 2 by 2 matrices modulo $p$ with factorable characteristic polynomials?
	
	To do this, we have to count the number of unique matrices we can obtain using our method. One way we can do this is by counting all the possible pairs of polynomials
	we can construct for the top and bottom row of our matrices. What do we mean by this? Recall the example system of congruences we used earlier:
	\begin{center}
		\begin{tabular}{l}
			$5x + 3y \equiv 2x \mod{7}$,\quad and \\
			$4x + 6y \equiv 2y \mod{7}$.
		\end{tabular}
	\end{center}
	This system corresponds to the matrix $\smallmat{5&3\\4&6} \bmod{7}$, which is one of the matrices modulo 7 with a factorable characteristic polynomial (due to it having 
	at least one eigenvalue in $\Z_7$, meaning its characteristic polynomial can necessarily be factored into two linear factors). The pair of polynomials associated with the
	matrix $\smallmat{5&3\\4&6} \bmod{7}$ can be written as $(5x+3y, 4x+6y)$. We see that every matrix will have a unique ordered pair of polynomials associated with it.
	Therefore, if we can count the number of unique ordered pairs of polynomials associated with the matrices our method creates, we will have counted the number of matrices 
	modulo $p$ with factorable characteristic polynomials.
	
	In general, for a given eigenvalue $\lambda$, a polynomial pair of the form 
	\[
		(r(ax + by) + \lambda x, s(ax + by) + \lambda y)
	\]
	for arbitrary constants $a,b,r,s \in \Z_p$ (where at least one of $a$ or $b$ is nonzero) will give us a valid pair associated to a matrix with eigenvectors of eigenvalue 
	$\lambda$. Trying to enumerate all of these pairs is a fairly arduous task. Using a particularly-arranged table, though, the task becomes somewhat doable:
	
	{\tiny
		\begin{center}
			\begin{tabular}{*{5}{c|}c}
				$\lambda x$        & $\lambda x$            & $\lambda x$             & $\cdots$ & $\lambda x$                 & $\lambda x$        \\
				$x+\lambda x$      & $x+y+\lambda x$        & $x+2y+\lambda x$        & $\cdots$ & $x+(p-1)y+\lambda x$        & $y+\lambda x$      \\
				$2x+\lambda x$     & $2(x+y)+\lambda x$     & $2(x+2y)+\lambda x$     & $\cdots$ & $2(x+(p-1))y+\lambda x$     & $2y+\lambda x$     \\
				$3x+\lambda x$     & $3(x+y)+\lambda x$     & $3(x+2y)+\lambda x$     & $\cdots$ & $3(x+(p-1))y+\lambda x$     & $3y+\lambda x$     \\
				$\vdots$           & $\vdots$               & $\vdots$                & $\vdots$ & $\vdots$                    & $\vdots$           \\
				$(p-1)x+\lambda x$ & $(p-1)(x+y)+\lambda x$ & $(p-1)(x+2y)+\lambda x$ & $\cdots$ & $(p-1)(x+(p-1))y+\lambda x$ & $(p-1)y+\lambda x$ \\
				\hline 
				$\lambda y$        & $\lambda y$            & $\lambda y$             & $\cdots$ & $\lambda y$                 & $\lambda y$        \\
				$x+\lambda y$      & $x+y+\lambda y$        & $x+2y+\lambda y$        & $\cdots$ & $x+(p-1)y+\lambda y$        & $y+\lambda y$      \\
				$2x+\lambda y$     & $2(x+y)+\lambda y$     & $2(x+2y)+\lambda y$     & $\cdots$ & $2(x+(p-1)y)+\lambda y$     & $2y+\lambda y$     \\
				$3x+\lambda y$     & $3(x+y)+\lambda y$     & $3(x+2y)+\lambda y$     & $\cdots$ & $3(x+(p-1)y)+\lambda y$     & $3y+\lambda y$     \\
				$\vdots$           & $\vdots$               & $\vdots$                & $\vdots$ & $\vdots$                    & $\vdots$           \\
				$(p-1)x+\lambda y$ & $(p-1)(x+y)+\lambda y$ & $(p-1)(x+2y)+\lambda y$ & $\cdots$ & $(p-1)(x+(p-1)y)+\lambda y$ & $(p-1)y+\lambda y$ 
				
			\end{tabular}
		\end{center}
	}
	
	The table is divided into two sections. The top section (above the horizontal dividing line) represents the first polynomial in the pair, while the bottom section (below 
	the horizontal line) represents the second polynomial in the pair. Each column of the table represents a different polynomial relation that a span of eigenvectors could 
	satisfy (e.g. $x+2y\equiv0\bmod{p}$ for the third row from the left). Each row represents a scalar multiple of this relation.
	
	To create a valid polynomial pair, we choose a column of the table, then select a polynomial from the top and bottom sections within that column. For example, using the
	table, the pairs $(2x+\lambda x, 3x+\lambda y)$, $(3(x+2y)+\lambda x, \lambda y)$, and $(\lambda x, \lambda y)$ are all valid pairs corresponding to matrices with 
	factorable characteristic polynomials.
	
	Enumerating the pairs in this way is still difficult, however, so we'll further simplify the problem by slightly changing the way we write our points. Label the first
	column of the table from the left as column $c=0$. The next column to the right will be column $c=1$, the next $c=2$, and so on. The last (rightmost) column, however, 
	will be handled separately, and so it won't be given a number.
	
	Using this notation, all our polynomial pairs will take the form
	\[
		((m_1+\lambda)x + m_1cy, (m_2+\lambda)x+(m_2c+c\lambda+\lambda)y) \mod{p}
	\]
	for columns with a number, and
	\[
		(\lambda x + m_1y, m_2y) \mod{p}
	\]
	for the rightmost column, where $m_1$ and $m_2$ are constants we choose from $\Z_p$.
	
	To further ease with explanation, we'll let $E \equiv m_1 + \lambda$, $F \equiv m_1c$, and $H \equiv m_2 + \lambda$. Some algebraic manipulations give us that 
	$\lambda \equiv E - m_1$, and so
	\begin{align*}
		m_2c + c\lambda + \lambda &\equiv Hc + \lambda \\
		                          &\equiv Hc + E - m_1 \mod{p}.
	\end{align*}
	If $c^{-1}$ exists, then we can also say $m_1 \equiv Fc^{-1}$, meaning
	\[
		Hc + E - m_1 \equiv Hc + E - Fc^{-1} \mod{p}.
	\]
	So, if $c^{-1}$ exists, we can write our polynomial pairs (for numbered columns) as
	\[
		(Ex + Fy, Hx + (Hc + E - Fc^{-1})y) \mod{p}.
	\]
	With all the necessary notation established, we can now start counting how many unique polynomial pairs we can construct using our method.
	
	\textbf{Case 1: Numbered column, $F \equiv 0$.} In this case, either $c \equiv 0$ or $m_1 \equiv 0$. We have control over what values we set our variables to (since 
	setting the values of variables here is equivalent to moving around on the table above), so let's choose $c \equiv 0$. In this case, our polynomial pairs simplify to
	\[
		(Ex, Hx + \lambda y) \mod{p}.
	\]
	By choosing appropriate values for $m_1$, $m_2$, and $\lambda$, we can create any possible polynomial pairing in our given form where $F \equiv 0$. Thus, this case accounts
	for $p^3$ unique polynomial pairs (since there are $p$ possible values for $E$, $p$ for $H$, and $p$ for $\lambda$).
	
	\textbf{Case 2: Numbered column, $F \not\equiv 0$, $H \equiv 0$.} In this case, our polynomial pairs simplify to
	\[
		(Ex + Fy, \lambda y) \mod{p}.
	\]
	Notice that, because $F \not\equiv 0$, neither $m_1$ nor $c$ can be zero. This limits the number of polynomial pairs we can construct. The value of $\lambda$ is ours to 
	choose, so there are $p$ possible values it can take on. Likewise, we can set $m_1$ to be whatever nonzero value we want, so this gives $p-1$ possible values for $E$. 
	Finally, we can set $c$ to be whatever nonzero value we'd like, so there are again $p-1$ possible values for $F$. In total, then, this case gives us $p(p-1)^2$ 
	unique polynomial pairs.
	
	\textbf{Case 3: Numbered column, $F \not\equiv 0$, $H \not\equiv 0$.} In this case, there are no simplifications we can make to our polynomial pairs. The number of unique
	pairs we can make comes down to how many different values $Hc + E - Fc^{-1}$ can take on for different values of $E$, $F$, and $H$.
	
	By choosing appropriate values for $m_1$ and $\lambda$, $E$ can take on any value, so there are $p$ possibilities for $E$. Now, let's analyse the modified expression 
	$Hc - Fc^{-1}$ from our polynomial pairing. We can safely ignore the $+E$ from the original expression since it doesn't affect how many unique values $Hc + E - Fc^{-1}$ 
	can take on. Let $Hc - Fc^{-1} \equiv X$. Multiplying by $c$, this becomes a quadratic in c:
	\begin{center}
		\begin{tabular}{rrl}
					   & $(Hc - Fc^{-1})c\ \equiv$ & $Xc$ \\
			$\implies$ & $Hc^2 - F       \ \equiv$ & $Xc$ \\
			$\implies$ & $Hc^2 - Xc - F  \ \equiv$ & $0 \mod{p}.$
		\end{tabular}
	\end{center}
	Both $H$ and $F$ are nonzero, so we're guaranteed to be working with a quadratic (not a linear polynomial) that doesn't have zero as a root.
	
	Now, by choosing an appropriate value for $m_2$, we can also assign $H$ to be whatever nonzero value we want, so there are $p-1$ possibilities for $H$. Also, because $H$ is
	nonzero, we can multiply by its inverse:
	\begin{center}
		\begin{tabular}{rrl}
			           & $(Hc^2 - Xc - F)H^{-1}\ \equiv$    & $0H^{-1}$ \\
			$\implies$ & $c^2 - XH^{-1}c - FH^{-1}\ \equiv$ & $0$       \\
			$\implies$ & $c^2 - Yc - Z\ \equiv$             & $0 \mod{p}$,
		\end{tabular}
	\end{center}
	where $Y$ and $Z$ are arbitrary constants, $Z \not\equiv 0$. If we can count how many of these quadratics have solutions solely in $\Z_p^*$, we can complete our count of 
	all the polynomial pairs in this case.
	
	Thankfully, it's straightforward to count how many have solutions in $\Z_p^*$---it's the same number of quadratics that can be written as $(c-r)(c-s)$ for nonzero 
	constants $r$ and $s$ in $\Z_p^*$. It turns out that there are $\frac{1}{2}p(p-1)$ such quadratics (just by counting how many factorisations we can write). Thus, there 
	are $\frac{1}{2}p(p-1)$ possibilities for our quadratic, meaning there are $(p)(p-1)(\frac{1}{2}p(p-1)) = \frac{1}{2}p^2(p-1)^2$ unique polynomial pairs given by this case.
	
	\textbf{Case 4: Non-numbered column.} In this case, the polynomial pairs we're considering take the form
	\[
		(\lambda x + m_1y, m_2y) \mod{p}.
	\]
	This is the same form as the pairs in case 2, except now there are no restrictions for the coefficients on $x$ and $y$. The restrictions in case 2 prevented us from having 
	pairs where $E \equiv \lambda$, so this case adds these $p(p-1)$ skipped polynomial pairs (where $F \not\equiv 0$, hence the factor of $p-1$).
	
	These are the only possible cases. Thus, adding the pair counts from each case will give us the total number of polynomial pairs our method can construct, which is 
	equivalent to the total number of 2 by 2 matrices modulo $p$ that have factorable characteristic polynomials. Our count comes out to be
	\begin{align*}
		       &\ p^3 + p(p-1)^2 + \frac{1}{2}p^2(p-1)^2 + p(p-1)                \\
		\equiv &\ p^3 + p(p^2 - 2p + 1) + \frac{1}{2}p^2(p^2 - 2p + 1) + p^2 - p \\
		\equiv &\ p^3 + p^3 - 2p^2 + p + \frac{1}{2}(p^4 - 2p^3 + p^2) + p^2 - p \\
		\equiv &\ \frac{1}{2}p^4 + p^3 - \frac{1}{2}p^2                          \\
		\equiv &\ \frac{1}{2}p^2(p^2 + 2p - 1).
	\end{align*}
	
	This formula agrees with the result of \citet{Olsavsky2003}.
	
	\bibliographystyle{plainnat}
	\bibliography{refs.bib}
\end{document}
