
\documentclass[a4paper, 12pt, reqno]{amsart}
\usepackage[T1]{fontenc}

\usepackage[parfill]{parskip}

\usepackage[margin=3cm]{geometry}

\usepackage{setspace}
\setstretch{1.25}

\usepackage{natbib}

\usepackage{dsfont}
\usepackage{amssymb}

\usepackage{verbatim}

\newcommand\vecspan[1]{\textup{span}\left( #1 \right)}

%\newcounter{propcounter}[section]
\newcounter{propcounter}
\newenvironment{proposition}[1]
{
	\refstepcounter{propcounter}
	\textbf{Proposition \thepropcounter} - \emph{#1} \\
	\emph{Justification.}
}
{
	\textbf{QED.} \\
}

\begin{document}
	Assume $p$ is a prime number, and $k$ and $L$ are positive integers.
	
	Let $\mathds{Z}_n$ represent the set of integers modulo $n$.
	
	Let $\mathds{Z}_n[X]$ represent the set of all polynomials with coefficients in $\mathds{Z}_n$.
	
	Let $\mathds{Z}_n^M$ represent the set of all $M \times 1$ vectors with entries in $\mathds{Z}_n$.
	
	Let $\mathds{Z}_n^{M \times N}$ represent the set of all $M \times N$ matrices with entries in $\mathds{Z}_n$.
	
	\section{Some Important Concepts}
		Here, some specific nonstandard terms and concepts are defined. A few are also introduced in section \ref{sec:maxBasis}.
		
		The \emph{multiplicative order of an invertible matrix $A \in \mathds{Z}_{p^k}^{L \times L}$} is the smallest positive integer $\omega$ where
		\[
			A^\omega \equiv I \mod{p^k}
		\]
		
		The \emph{multiplicative order of a vector $\vec{v} \in \mathds{Z}_{p^k}^L$ under an invertible matrix $A \in \mathds{Z}_{p^k}^{L \times L}$} is the smallest 
		positive integer $\omega$ where
		\[
			A^\omega\vec{v} \equiv \vec{v} \mod{p^k}
		\]
		The multiplicative orders of vectors and matrices are the chief interest of this document.
		
		The \emph{orbit space} or \emph{cycle space} of a vector $\vec{v} \in \mathds{Z}_{p^k}^L$ under a matrix $A \in \mathds{Z}_{p^k}^{L \times L}$ is the submodule
		\[
			\mathcal{S}_{\vec{v}} = \sum_{i\,=\,0}^\infty \vecspan{\{A^i\vec{v}\}} \mod{p^k}
		\]
		In essence, the cycle space of $\vec{v}$ is the set of all vectors that can be made as linear combinations of its iterates 
		($\vec{v}$, $A\vec{v}$, $A^2\vec{v}$, etc.). Though the cycle space is defined using an infinite sum, only the first few terms are needed. More specifically, the 
		cycle space of $\vec{v}$ can be equivalently defined as
		\[
			\mathcal{S}_{\vec{v}} = \sum_{i\,=\,0}^{d-1} \vecspan{\{A^i\vec{v}\}} \mod{p^k}
		\]
		where $d$ is the smallest positive integer such that
		\[
			A^d\vec{v} \in \sum_{i\,=\,0}^{d-1} \vecspan{\{A^i\vec{v}\}} \mod{p^k}
		\]
		
		The \emph{minimal annihilating polynomial} of a vector $\vec{v} \in \mathds{Z}_p^L$ under a matrix $A \in \mathds{Z}_p^{L \times L}$ is the monic polynomial 
		$\mu(x)$ of least degree such that
		\[
			\mu(A)\vec{v} \equiv \vec{0} \mod{p}
		\]
		
		Since there are a finite number of matrices in $\mathds{Z}_p^{L \times L}$, it must be the case that, for some $a, b \in \mathds{Z}^+$, $a \neq b$, we have that
		\[
			A^a \equiv A^b \mod{p}
		\]
		and so for any vector $\vec{v} \in \mathds{Z}_p^L$:
		\begin{align*}
			         & A^a\vec{v} \equiv A^b\vec{v} \mod{p} \\
			\implies & (A^a - A^b)\vec{v} \equiv \vec{0}    \\
			\implies & f(A)\vec{v} \equiv \vec{0} \text{ for } f(x) = x^a - x^b
		\end{align*}
		so at least one annihilating polynomial for $\vec{v}$ always exists. Since $\mathds{Z}_p$ is a field, $\mathds{Z}_p[X]$ is a principal ideal domain (see 
		\citet{enwiki:PID}), meaning the ideal of annihilating polynomials for $\vec{v}$ must have a unique generator (one can verify that an annihilating polynomial 
		multiplied by any other polynomial in $\mathds{Z}_p[X]$ is also an annihilating polynomial, so the set of annihilating polynomials indeed forms an ideal). This 
		generator is the minimal annihilating polynomial.
		
		Minimal annihilating polynomials and cycle spaces are important constructs for our purposes. The following propositions will allow us to work with them more easily.
		
		\begin{proposition}{Under a given matrix $A \in \mathds{Z}_p^{L \times L}$, the degree of a vector's minimal annihilating polynomial is equal to the dimension of 
		its cycle space.}
			\label{prop:degOfS}
			Let $\vec{v} \in \mathds{Z}_p^L$ be an arbitrary vector. Using the definition of cycle spaces, the dimension of $\mathcal{S}_{\vec{v}}$ will be the smallest
			value of $d$ such that the set
			\[
				\bigcup_{i\,=\,0}^d \{A^i\vec{v}\} \mod{p}
			\]
			is linearly dependent.
			
			Let $D$ be the degree of the minimal annihilating polynomial of $\vec{v}$. Assume $d < D$. The set $\bigcup_{i\,=\,0}^d \{A^i\vec{v}\}$ is linearly dependent, 
			so there exist nontrivial coefficients $a_0$ to $a_d$ in $\mathds{Z}_p$ such that
			\[
				\sum_{i\,=\,0}^d a_iA^i\vec{v} \equiv \vec{0} \mod{p}
			\]
			And so
			\[
				f(A)\vec{v} \equiv \vec{0} \text{ for } f(x) = \sum_{i\,=\,0}^d a_ix^i \mod{p}
			\]
			This is a contradiction since the degree of $f$, an annihilating polynomial, is less than the degree of the minimal annihilating polynomial. Therefore, the 
			dimension of $\mathcal{S}_{\vec{v}}$ must be at least $D$. 
			
			Let $m(x)$ be the minimal annihilating polynomial of $\vec{v}$. Our set $\bigcup_{i\,=\,0}^d \{A^i\vec{v}\}$ will always be linearly dependent when $d = D$ since 
			\begin{align*}
				m(A)\vec{v} \equiv& \ \sum_{i\,=\,0}^D a_iA^i\vec{v} \\
				            \equiv& \ \vec{0} \mod{p}
			\end{align*}
			for nontrivial constants $a_i \in \mathds{Z}_p$, so the dimension of $\mathcal{S}_{\vec{v}}$ is exactly $D$, the degree of the minimal annihilating polynomial for
			$\vec{v}$.
		\end{proposition}
		
		\begin{proposition}{For some $A \in \mathds{Z}_{p^k}^{L \times L}$, if $\vec{v} \in \mathds{Z}_{p^k}^L$ is a vector such that $\vec{v} \in \ker{(f(A))}$ for some 
		polynomial $f$, then $\mathcal{S}_{\vec{v}} \subseteq \ker{(f(A))}$.}
			\label{prop:SinK}
			First, we can show that any $A^n\vec{v} \in \ker{(f(A))}$:
			\begin{align*}
				\vec{v} \in \ker{(f(A))} \implies& f(A)\vec{v} \equiv \vec{0} \mod{p^k} \\
				                         \implies& A^nf(A)\vec{v} \equiv \vec{0}        \\
										 \implies& f(A)(A^n\vec{v}) \equiv \vec{0}      \\
										 \implies& A^n\vec{v} \in \ker{(f(A))} \text{ for some } n \in \mathds{Z}^+
			\end{align*}
			Since this is true, any linear combination of $A^n\vec{v}$ vectors is also in $\ker{(f(A))}$ (since kernels are submodules), and the set of all linear 
			combinations of $A^n\vec{v}$ vectors is exactly $\mathcal{S}_{\vec{v}}$, so $\mathcal{S}_{\vec{v}} \subseteq \ker{(f(A))}$.
		\end{proposition}
		
		It's also helpful to make explicit a particular property of modular arithmetic modulo $p^k$.
		
		\begin{proposition}{If $a \in \mathds{Z}^+$ is invertible modulo $p^k$, then $a$ is also invertible modulo $p^{k+\ell}$.}
			\label{prop:liftInverses}
			Let $\alpha$ be the inverse of $a$ modulo $p^k$. We know
			\[
				a\alpha \equiv 1 + p^kn \mod{p^{k+\ell}}
			\]
			for some $n \in \mathds{Z}_{p^\ell}$. Then
			\[
				a(\alpha - p^kn\alpha) \equiv 1 + p^kn - p^kn(1 + p^kn) \equiv 1 - p^{2k}n^2 \mod{p^{k+\ell}}
			\]
			since multiplication is associative and commutative in $\mathds{Z}_{p^{k+\ell}}$. Then
			\[
				a(1 - p^kn + p^{2k}n^2)\alpha \equiv 1 - p^{2k}n^2 + p^{2k}n^2(1 + p^kn) \equiv 1 + p^{3k}n^3 \mod{p^{k+\ell}}
			\]
			A general pattern emerges:
			\[
				a\alpha\sum_{i\,=\,0}^N (-1)^i p^{ik} n^i \equiv 1 + (-1)^N p^{(N+1)k}n^{N+1} \mod{p^{k+\ell}}
			\]
			for some $N \in \mathds{N}$. Then, if $(\mathcal{L}+1)k \geq k+\ell$,
			\[
				a\alpha\sum_{i\,=\,0}^\mathcal{L} (-1)^i p^{ik} n^i \equiv 1 + (-1)^\mathcal{L} p^{(\mathcal{L}+1)k}n^{\mathcal{L}+1} \equiv 1 \mod{p^{k+\ell}}
			\]
			So $\alpha\sum_{i\,=\,0}^\mathcal{L} (-1)^i p^{ik} n^i$ is the inverse of $a$ modulo $p^{k+\ell}$, meaning $a$ is invertible modulo $p^{k+\ell}$.
		\end{proposition}
	
	\section{Maximal Basis}
		\label{sec:maxBasis}
		A \emph{maximal vector} $\vec{v} \in \mathds{Z}_{p^k}^L$ under an invertible matrix $A \in \mathds{Z}_{p^k}^{L \times L}$ is a vector such that, if $\omega$ is the 
		multiplicative order of $A$, then $\omega$ is also the multiplicative order of $\vec{v}$.

		The name "maximal vector" comes from the fact that the multiplicative order of a vector is necessarily bounded by the multiplicative order of a matrix. Thus, if a 
		vector's multiplicative order matches a matrix's multiplicative order, it has the highest, or maximal, possible multiplicative order.
		
		Our interest is in proving that, for any modulus $p^k$ and any invertible matrix $A \in \mathds{Z}_{p^k}^{L \times L}$, there will always be at least one maximal 
		vector $\vec{v} \in \mathds{Z}_{p^k}^L$. This is known to be true for $k=1$ via proposition 3 of \citet{Mendivil2012}; proposition \ref{prop:maxVectGuarantee} aims
		to extend this for all $k$.
		
		First, though, an important property of maximal vectors we'll need to make use of:
		
		\begin{proposition}{If $\vec{v} \in \mathds{Z}_{p^k}^L$ is a maximal vector under the invertible matrix $A \in \mathds{Z}_{p^k}^{L \times L}$, then $A^n\vec{v}$ is
		also a maximal vector.}
			\label{prop:Amax}
			Let $\omega$ be the multiplicative order of $\vec{v}$ under $A$. Assume $A^n\vec{v}$ is not a maximal vector. Then there exists some number $\alpha < \omega$ 
			such that
			\[
				A^\alpha(A^n\vec{v}) \equiv A^n(A^\alpha\vec{v}) \equiv A^n\vec{v} \mod{p^k}
			\]
			Multiplying by $A^{-n}$, we see
			\[
				A^\alpha\vec{v} \equiv \vec{v} \mod{p^k}
			\]
			This contradicts the fact that $\vec{v}$ is a maximal vector. So, if $\vec{v}$ is a maximal vector, $A^n\vec{v}$ must also be a maximal vector.
		\end{proposition}
		
		Now, we can show that maximal vectors are rather common modulo $p$. In fact, under any invertible matrix modulo $p$, there will always be enough maximal vectors to 
		span the entirety of the vector space!
		
		\begin{proposition}{Given an invertible matrix $A \in \mathds{Z}_p^{L \times L}$, we can form a basis for $\mathds{Z}_p^L$ using maximal vectors under the matrix 
		$A$.}
			\label{prop:maxBasis}
			Let the minimal polynomial of $A$ be written as
			\[
				m(x) \equiv \prod_{i\,=\,1}^g f_i^{n_i}(x) \mod{p}
			\]
			for monic, coprime factors $\{f_i\}$ and positive integers $\{n_i\}$. By the Primary Decomposition Theorem, we know
			\[
				\mathds{Z}_p^L = \bigoplus_{i\,=\,1}^g \ker{(f_i^{n_i}(A))}
			\]
			where $\ker{(f_i^{n_i}(A))} \neq \{\vec{0}\}$. Let $K_i = \ker{(f_i^{n_i}(A))}$.
			
			By definition of the minimal polynomial, for each $K_i$, there exists a nonzero vector $\vec{x}_i \in K_i$ whose minimal annihilating polynomial is $f_i^{n_i}$. 
			Otherwise, if no such $\vec{x}_i$ existed, then $f_i^{n_i}$ wouldn't be a factor of the minimal polynomial---it wouldn't be needed to annihilate every vector in
			$\mathds{Z}_p^L$. By arguments used in proposition 3 of \citet{Mendivil2012}, then, we know the vector 
			\[
				\vec{\chi} \equiv \sum_{i\,=\,1}^g \vec{x}_i \mod{p}
			\]
			is a maximal vector. We also know the minimal annihilating polynomial of $\vec{\chi}$ is $m(x)$, since it's the smallest degree monic polynomial that annihilates 
			all $\vec{x}_i$ in the sum for $\vec{\chi}$:
			\begin{align*}
				m(A)\vec{\chi} \equiv& \ m(A)(\vec{x}_1 + \vec{x}_2 + \cdots + \vec{x}_g)       \\
				               \equiv& \ m(A)\vec{x}_1 + m(A)\vec{x}_2 + \cdots + m(A)\vec{x}_g \\
							   \equiv& \ \vec{0} + \vec{0} + \cdots + \vec{0}                   \\
							   \equiv& \ \vec{0} \mod{p}
			\end{align*}
			Therefore, by proposition \ref{prop:degOfS}, the dimension of $\mathcal{S}_{\vec{\chi}}$ is equal to the degree of $m(x)$. If we denote the degree of $m(x)$ by 
			$D$, then this means the set
			\[
				B_{\vec{\chi}} = \bigcup_{i\,=\,0}^{D-1} \{A^i\vec{\chi}\}
			\]
			is linearly independent. In fact, $B_{\vec{\chi}}$ is the defining set of basis vectors for $\mathcal{S}_{\vec{\chi}}$. Note that $|B_{\vec{\chi}}| = D$.
			
			If $D = L$, then by proposition \ref{prop:Amax}, the set $B_{\vec{\chi}}$ is a basis of maximal vectors for $\mathds{Z}_p^L$. Otherwise, we need to find $L-D$ 
			more vectors to form a basis. 
			
			If $d_i$ represents the degree of $f_i^{n_i}$, then 
			\[
				D = \sum_{i\,=\,1}^g d_i
			\]
			As well, $L$ can be written in terms of the dimensions of each $K_i$ by the Primary Decomposition Theorem:
			\[
				L = \sum_{i\,=\,1}^g \dim{(K_i)}
			\]
			So,
			\[
				L-D = \sum_{i\,=\,1}^g (\dim{(K_i)}-d_i)
			\]
			Each $d_i \leq \dim{(K_i)}$ since $d_i = \dim{(\mathcal{S}_{\vec{x}_i})}$ by proposition \ref{prop:degOfS}, and $\mathcal{S}_{\vec{x}_i} \subseteq K_i$ via 
			proposition \ref{prop:SinK}, therefore
			\[
				\dim{(K_i)} \geq \dim{(\mathcal{S}_{\vec{x}_i})} \implies \dim{(K_i)} \geq d_i
			\]
			
			This means we can rewrite $K_i$ as
			\[
				K_i = \mathcal{S}_{\vec{x}_i} \oplus \vecspan{\bigcup_{j\,=\,1}^{\dim{(K_i)}-d_i} \{\vec{b}_i^j\}}
			\]
			for some linearly-independent vectors $\vec{b}_i^j \in K_i \cap \mathcal{S}_{\vec{x}_i}'$ where $\mathcal{S}_{\vec{x}_i}'$ denotes all vectors not in 
			$\mathcal{S}_{\vec{x}_i}$. Let 
			\[
				B_i = \bigcup_{j\,=\,1}^{\dim{(K_i)}-d_i} \{\vec{b}_i^j\}
			\]
			Note that $|B_i| = \dim{(K_i)}-d_i$.
			
			For each $\vec{b}_i^j \in B_i$, consider the vector
			\[
				\vec{x}_i + \vec{b}_i^j \mod{p}
			\]
			Using the same reasoning as was used to show that the minimal annihilating polynomial of $\vec{\chi}$ was $m(x)$, the minimal annihilating polynomial of this 
			vector must be $f_i^{n_i}(x)$. Thus, the vector
			\[
				\vec{\epsilon_i^j} \equiv \vec{\chi} + \vec{b}_i^j \mod{p}
			\]
			must be a maximal vector for the same reason that $\vec{\chi}$ is a maximal vector.
			
			Is $\vec{\epsilon_i^j} \in \mathcal{S}_{\vec{\chi}}$? Assume it is. Then
			\begin{align*}
				\vec{\epsilon_i^j} \in \mathcal{S}_{\vec{\chi}} \implies& \vec{\chi} + \vec{b}_i^j \in \mathcal{S}_{\vec{\chi}} \\
				                                                \implies& \vec{b}_i^j \in \mathcal{S}_{\vec{\chi}}
			\end{align*}
			since we know $\vec{\chi} \in \mathcal{S}_{\vec{\chi}}$. We also know $\mathcal{S}_{\vec{\chi}} \subseteq \sum_{h\,=\,1}^g \mathcal{S}_{\vec{x}_h}$ since 
			$\vec{\chi} \equiv \sum_{h\,=\,1}^g \vec{x}_h$, so
			\[
				\vec{b}_i^j \in \sum_{h\,=\,1}^g \mathcal{S}_{\vec{x}_h}
			\]

			This means we can write $\vec{b}_i^j$ as a linear combination of vectors from each $\mathcal{S}_{\vec{x}_h}$. Let 
			$\vec{s}_h \in \mathcal{S}_{\vec{x}_h} \setminus \{\vec{0}\}$ and let $a_h \in \mathds{Z}_p$. Then for nontrivial constants $a_1$ to $a_g$, we have that
			\[
				\vec{b}_i^j \equiv a_1\vec{s}_1 + a_2\vec{s}_2 + \cdots + a_g\vec{s}_g \mod{p}
			\]
			We know $\vec{b}_i^j \in K_i$, so $f_i^{n_i}(A)\vec{b}_i^j \equiv \vec{0}$. Therefore
			\begin{align*}
				      & \ f_i^{n_i}(A)\vec{b}_i^j \\
				\equiv& \ f_i^{n_i}(A)(a_1\vec{s}_1 + a_2\vec{s}_2 + \cdots + a_g\vec{s}_g) \\
				\equiv& \ f_i^{n_i}(A)a_1\vec{s}_1 + \cdots + f_i^{n_i}(A)a_{i-1}\vec{s}_{i-1} + 
						  f_i^{n_i}(A)a_i\vec{s}_i + f_i^{n_i}(A)a_{i+1}\vec{s}_{i+1} + \cdots \\
				\equiv& \ f_i^{n_i}(A)a_1\vec{s}_1 + \cdots + f_i^{n_i}(A)a_{i-1}\vec{s}_{i-1} + 
				          f_i^{n_i}(A)a_{i+1}\vec{s}_{i+1} + \cdots + f_i^{n_i}(A)a_g\vec{s}_g \\
				\equiv& \ \vec{0} \mod{p}
			\end{align*}
			since $\vec{s}_i \in K_i$ by proposition \ref{prop:SinK}. Therefore
			\[
				a_1\vec{s}_1 + a_2\vec{s}_2 + \cdots + a_{i-1}\vec{s}_{i-1} + a_{i+1}\vec{s}_{i+1} + \cdots + a_g\vec{s}_g \in K_i
			\]
			However, we know
			\begin{align*}
				        & (K_1 + K_2 + \cdots + K_{i-1} + K_{i+1} + \cdots + K_g) \cap K_i = \{\vec{0}\} \\
				\implies& (\mathcal{S}_{\vec{x}_1} + \mathcal{S}_{\vec{x}_2} + \cdots + \mathcal{S}_{\vec{x}_{i-1}} + \mathcal{S}_{\vec{x}_{i+1}} + 
				           \cdots + \mathcal{S}_{\vec{x}_g}) \cap K_i = \{\vec{0}\}
			\end{align*}
			by the Primary Decomposition Theorem (each $K_h$ is disjoint except for the zero vector). This means
			\[
				a_1\vec{s}_1 + a_2\vec{s}_2 + \cdots + a_{i-1}\vec{s}_{i-1} + a_{i+1}\vec{s}_{i+1} + \cdots + a_g\vec{s}_g \equiv \vec{0}
			\]
			The set $\{\vec{s}_1,\, \vec{s}_2,\, \cdots,\, \vec{s}_{i-1},\, \vec{s}_{i+1},\, \cdots,\, \vec{s}_g\}$ is linearly independent because each $\vec{s}_h \in K_h$.
			Therefore, the only way this sum can be zero is if each $a_h \equiv 0$. Returning to our expression for $\vec{b}_i^j$, this means
			\[
				\vec{b}_i^j \equiv a_i\vec{s}_i
			\]
			We know $\vec{b}_i^j \not\in \mathcal{S}_{\vec{x}_i}$, so this is a contradiction. Therefore, $\vec{\epsilon_i^j} \not\in \mathcal{S}_{\vec{\chi}}$, and so each 
			$\vec{\epsilon_i^j}$ must be linearly independent to the vectors in $B_{\vec{\chi}}$ (which is a basis for $\mathcal{S}_{\vec{\chi}}$).
			
			Now, let's form the set of all such $\vec{\epsilon_i^j}$:
			\[
				E = \{\vec{\chi} + \vec{b}_i^j : 1 \leq i \leq g,\, \vec{b}_i^j \in B_i\}
			\]
			Note that $|E| = \sum_{i\,=\,1}^g |B_i| = \sum_{i\,=\,1}^g (\dim{(K_i)}-d_i) = L-D$.
			
			Is the set $E$ linearly independent? Assume it isn't. Then for nontrivial constants $a_{i,j} \in \mathds{Z}_p$ we have that
			\begin{align*}
				        & \sum_{i\,=\,1}^g \sum_{j\,=\,1}^{\dim{(K_i)}-d_i} a_{i,j}\vec{\epsilon_i^j} \equiv \vec{0} \\
				\implies& \sum_{i\,=\,1}^g \sum_{j\,=\,1}^{\dim{(K_i)}-d_i} a_{i, j}(\vec{\chi} + \vec{b}_i^j) \equiv \vec{0}
			\end{align*}
			Let $\alpha \equiv \sum_{i\,=\,1}^g \sum_{j\,=\,1}^{\dim{(K_i)}-d_i} a_{i,j}$. Then
			\[
				\implies \sum_{i\,=\,1}^g \left( \alpha\vec{x}_i + \sum_{j\,=\,1}^{\dim{(K_i)}-d_i} a_{i,j}\vec{b}_i^j \right) \equiv \vec{0}
			\]
			If $\alpha \equiv 0 \bmod{p}$, then
			\[
				\implies \sum_{i\,=\,1}^g \sum_{j\,=\,1}^{\dim{(K_i)}-d_i} a_{i,j}\vec{b}_i^j \equiv \vec{0}
			\]
			This is impossible for nontrivial constants $a_{i,j}$ since the set of vectors $\vec{b}_i^j$ forms a linearly-independent set by definition. So 
			$\alpha \not\equiv 0$. We have that
			\[
				\sum_{i\,=\,1}^g \left( \alpha\vec{x}_i + \sum_{j\,=\,1}^{\dim{(K_i)}-d_i} a_{i,j}\vec{b}_i^j \right) \equiv \vec{0}
			\]
			We know the vector $\alpha\vec{x}_i$ is linearly independent to $\sum_{j\,=\,1}^{\dim{(K_i)}-d_i} a_{i,j}\vec{b}_i^j$ since 
			\[
				\vecspan{B_i} \cap \mathcal{S}_{\vec{x}_i} = \{\vec{0}\}
			\]
			and since $\alpha\vec{x}_i \not\equiv \vec{0}$, there's no way the term
			\[
				\alpha\vec{x}_i + \sum_{j\,=\,1}^{\dim{(K_i)}-d_i} a_{i,j}\vec{b}_i^j
			\]
			can equal zero. As well, since this term is an element of $K_i$ (it is annihilated by $f_i^{n_i}(A)$), it is linearly independent to all other terms in the
			sum since they will all be elements of a $K_h \neq K_i$. Therefore, this sum can never give the zero vector since nonzero linearly-independent vectors cannot sum
			to zero. This means the set $E$ must be linearly independent.
			
			Finally, since $B_{\vec{\chi}}$ and $E$ are linearly-independent sets, and since all vectors in $E$ are linearly independent to the vectors in $B_{\vec{\chi}}$,
			this means the set $B_{\vec{\chi}} \cup E$ is a linearly-independent set of $D + (L - D) = L$ maximal vectors. Therefore, the set $B_{\vec{\chi}} \cup E$ forms
			a basis of maximal vectors for $\mathds{Z}_p^L$.
		\end{proposition}
		
		At a high level, proposition \ref{prop:maxBasis} says that if a matrix expression is to annihilate all maximal vectors in $\mathds{Z}_p^L$ under some invertible 
		matrix, it must annihilate \emph{all} vectors in $\mathds{Z}_p^L$. Since we can form a basis of maximal vectors, the span of maximal vectors is the entire space
		$\mathds{Z}_p^L$, so to annihilate all maximal vectors is to annihilate $\mathds{Z}_p^L$ entirely.
	
	\section{Maximal Vectors Modulo $p^k$}
		Before we can show that a maximal vector always exists modulo $p^k$ under some invertible matrix, we need to first understand a little bit about how matrices and
		vectors behave under different powers of a prime-power modulus.
		
		\begin{proposition}{Assume $p^k > p$ and that $p$ is an odd prime. If $A^\omega \equiv I + p^{k-1}B \bmod{p^k}$ for an invertible matrix 
		$A \in \mathds{Z}_{p^k}^{L \times L}$, a generic matrix $B \in \mathds{Z}_p^{L \times L}$, and a positive integer $\omega$, then 
		$A^{p\omega} \equiv I + p^kB \bmod{p^{k+1}}$.}
			\label{prop:liftIterate}
			If $A^\omega \equiv I + p^{k-1}B \bmod{p^k}$, then
			\[
				A^\omega \equiv I + p^{k-1}B + p^kC \mod{p^{k+1}}
			\]
			for some matrix $C \in \mathds{Z}_p^{L \times L}$. Calculating $A^{2\omega}$, we see
			\begin{align*}
				A^{2\omega} \equiv& \ (I + p^{k-1}B + p^kC)^2                                                   \\
							\equiv& \ I + 2p^{k-1}B + 2p^kC + p^{2k-2}B^2 + p^{2k-1}CB + p^{2k-1}BC + p^{2k}C^2 \\
							\equiv& \ I + 2p^{k-1}B + 2p^kC + \binom{2}{2}p^{2k-2}B^2 \mod{p^{k+1}}
			\end{align*}
			We notice that a possible general form for $A^{n\omega}$ arises:
			\[
				A^{n\omega} \overset{?}{\equiv} I + np^{k-1}B + np^kC + \binom{n}{2}p^{2k-2}B^2 \mod{p^{k+1}}
			\]
			Both $A^\omega$ and $A^{2\omega}$ follow this form. If we're given the expression
			\[
				A^{r\omega} \equiv I + rp^{k-1}B + rp^kC + \binom{r}{2}p^{2k-2}B^2 \mod{p^{k+1}}
			\]
			then
			\begin{align*}
				A^{(r+1)\omega} \equiv& \ (I + p^{k-1}B + p^kC)(I + rp^{k-1}B + rp^kC + \binom{r}{2}p^{2k-2}B^2) \\
				                \equiv& \ I + p^{k-1}B + p^kC + rp^{k-1}B + rp^{2k-2}B^2 + rp^{2k-1}CB + rp^kC \\
								      & \ + rp^{2k-1}BC + rp^{2k}C^2 + \binom{r}{2}p^{2k-2}B^2 + \binom{r}{2}p^{3k-3}B^3 + \binom{r}{2}p^{3k-2}CB^2 \\
								\equiv& \ I + (r+1)p^{k-1}B + (r+1)p^kC + \left(r + \binom{r}{2}\right)p^{2k-2}B^2 \mod{p^{k+1}} \\
								\equiv& \ I + (r+1)p^{k-1}B + (r+1)p^kC + \binom{r+1}{2}p^{2k-2}B^2
			\end{align*}
			So
			\begin{align*}
				        & A^{r\omega}     \equiv I + rp^{k-1}B     + rp^kC     + \binom{r}{2}p^{2k-2}B^2   \mod{p^{k+1}} \\
				\implies& A^{(r+1)\omega} \equiv I + (r+1)p^{k-1}B + (r+1)p^kC + \binom{r+1}{2}p^{2k-2}B^2 \mod{p^{k+1}}
			\end{align*}
			Therefore, by induction, we can say
			\[
				A^{n\omega} \equiv I + np^{k-1}B + np^kC + \binom{n}{2}p^{2k-2}B^2 \mod{p^{k+1}}
			\]
			Setting $n=p$ yields
			\begin{align*}
				A^{p\omega} \equiv& \ I + pp^{k-1} + pp^kC + \binom{p}{2}p^{2k-2}B^2 \\
				            \equiv& \ I + p^kB + p^{k+1}C + \frac{p-1}{2}p^{2k-1}B^2 \\
							\equiv& \ I + p^kB \mod{p^{k+1}}
			\end{align*}
			since $p$ is an odd prime.
		\end{proposition}
		
		\begin{proposition}{Suppose that $\omega$ is the multiplicative order of $A$ modulo $p^k$. Then, the multiplicative order of $A$ modulo $p^{k+1}$ is either $\omega$ 
		or $p\omega$. As well, for odd primes $p$, if the multiplicative order of $A$ increased by a factor of $p$ for a lower power of the modulus, then the
		multiplicative order must increase by a factor of $p$ for all higher powers of the modulus.}
			\label{prop:cycLenInc}
			Let the multiplicative order of $A$ modulo $p^{k+1}$ be $\Omega$. Assume that $\omega \nmid \Omega$. Then $\Omega = m\omega + n$ for some nonnegative integers 
			$m$ and $n$ where $1 \leq n < \omega$. This means
			\begin{align*}
				        & A^\Omega \equiv I \mod{p^{k+1}}   \\
				\implies& A^\Omega \equiv I \mod{p^k}       \\
				\implies& A^{m\omega}A^n \equiv I \mod{p^k} \\
				\implies& A^n \equiv I \mod{p^k}
			\end{align*}
			This is a contradiction since $n < \omega$, yet $\omega$ is the multiplicative order of $A$ modulo $p^k$. Therefore $\omega \mid \Omega$.
			
			We know that $A^\omega \equiv I \bmod{p^k}$, so
			\[
				A^\omega \equiv I + p^kB \mod{p^{k+1}}
			\]
			for some $B \in \mathds{Z}_p^{L \times L}$. If $B \equiv 0$, then we have
			\[
				A^\omega \equiv I \mod{p^{k+1}}
			\]
			In this case, since $\omega \mid \Omega$, we know $\Omega = \omega$.
			
			Otherwise, if $B \not\equiv 0$, we can check successive powers of $A^\omega$ to find the first that is equivalent to the identity matrix. Since 
			$\omega \mid \Omega$, we can be sure that the first value of $n\omega$ such that $A^{n\omega} \equiv I \bmod{p^{k+1}}$ is indeed the multiplicative order of $A$
			modulo $p^{k+1}$. Computing $A^{2\omega}$, we see
			\begin{align*}
				A^{2\omega} \equiv& \ (I + p^kB)^2 \\
				            \equiv& \ I + 2p^kB + p^{2k}B^2 \\
							\equiv& \ I + 2p^kB \mod{p^{k+1}}
			\end{align*}
			It seems that
			\[
				A^{n\omega} \overset{?}{\equiv} I + np^kB \mod{p^{k+1}}
			\]
			This is certainly the case for $n=1$ and $n=2$. Assuming we're given the expression
			\[
				A^{r\omega} \equiv I + rp^kB \mod{p^{k+1}}
			\]
			then
			\begin{align*}
				A^{(r+1)\omega} \equiv& \ (I + p^kB)(I + rp^kB) \\
				                \equiv& \ I + p^kB + rp^kB + rp^{2k}B \\
								\equiv& \ I + (r+1)p^kB \mod{p^{k+1}}
			\end{align*}
			So
			\[
				A^{r\omega} \equiv I + rp^kB \implies A^{(r+1)\omega} \equiv I + (r+1)p^kB \mod{p^{k+1}}
			\]
			By induction, we can say that
			\[
				A^{n\omega} \equiv I + np^kB \mod{p^{k+1}}
			\]
			Setting $n=p$, we see
			\begin{align*}
				A^{p\omega} \equiv& \ I + pp^kB \\
							\equiv& \ I \mod{p^{k+1}}
			\end{align*}
			No positive value of $n$ smaller than $p$ will cause this cancellation. Therefore, in the case where $B \not\equiv 0$, $\Omega = p\omega$.
			
			These are all the possibilities we need to consider, so $\Omega = \omega$ or $\Omega = p\omega$.
			
			Now, we'll prove that the multiplicative order of $A$ increasing by a factor of $p$ for a lower power of the modulus means that the multiplicative order of $A$ 
			must always increase by a factor of $p$ for higher powers of the modulus. We'll now assume that $p$ is an odd prime. If, let's say, $\psi$ is the multiplicative 
			order of $A$ modulo $p^\ell$ for some $\ell \in \mathds{Z}^+$, and $p\psi$ is the multiplicative order of $A$ modulo $p^{\ell + 1}$, then we know
			\[
				A^\psi \equiv I + p^\ell C \mod{p^{\ell+1}}
			\]
			for some nonzero $C \in \mathds{Z}_p^{L \times L}$. Proposition \ref{prop:liftIterate} guarantees that
			\[
				A^{p^w\psi} \equiv I + p^{\ell+w}C \mod{p^{\ell+1+w}}
			\]
			for positive integers $w$. There is no possible way $A^{p^w\psi} \equiv I \bmod{p^{\ell+1+w}}$, meaning $p^w\psi$ can never be the multiplicative order of $A$
			modulo $p^{\ell+1+w}$. This means the multiplicative order of $A$ must increase by a factor of $p$ for each increment of $w$.
		\end{proposition}
		
		\begin{proposition}{A set of basis vectors for $\mathds{Z}_{p^k}^L$ is also a set of basis vectors for $\mathds{Z}_{p^{k+\ell}}^L$.}
			\label{prop:basisLift}
			Assume the set $\mathcal{B} = \{\vec{b}_1,\, \vec{b}_2,\, \cdots,\, \vec{b}_L\}$ forms a basis for $\mathds{Z}_{p^k}^L$. Because the set $\mathcal{B}$ is
			linearly independent, the matrix
			\[
				B = \begin{bmatrix}
					\vec{b}_1 & \vec{b}_2 & \cdots & \vec{b}_L
				\end{bmatrix} \mod{p^k}
			\]
			must be invertible by the Invertible Matrix Theorem, meaning the determinant must also be invertible. 
			
			Considering the matrix $B$ modulo $p^{k+\ell}$, we see the determinant of this matrix must still be invertible by proposition \ref{prop:liftInverses}. 
			Therefore, by the Invertible Matrix Theorem, $B$ modulo $p^{k+\ell}$ must also be invertible, meaning its column vectors form a basis for 
			$\mathds{Z}_{p^{k+\ell}}^L$ (since there are $L$ linearly-independent column vectors).
		\end{proposition}
		
		\begin{proposition}{Given an invertible matrix $A \in \mathds{Z}_{p^k}^{L \times L}$ with a multiplicative order modulo $p^k$ of $\omega$, and a maximal vector 
		$\vec{v} \in \mathds{Z}_{p^k}^L$ modulo $p^k$, then the multiplicative order of $\vec{v}$ modulo $p^{k+1}$ is either $\omega$ or $p\omega$.}
			\label{prop:maxCycLenInc}
			Let $\Omega$ be the multiplicative order of $\vec{v}$ modulo $p^{k+1}$. Assume that $\omega \nmid \Omega$. Then $\Omega = m\omega + n$ for nonnegative integers 
			$m$ and $n$ where $1 \leq n < \omega$.
			But then
			\begin{align*}
				        & A^\Omega\vec{v} \equiv \vec{v} \mod{p^{k+1}}   \\
				\implies& A^\Omega\vec{v} \equiv \vec{v} \mod{p^k}       \\
				\implies& A^{m\omega}A^n\vec{v} \equiv \vec{v} \mod{p^k} \\
				\implies& A^n\vec{v} \equiv \vec{v} \mod{p^k}
			\end{align*}
			which is a contradiction since $n < \omega$, yet $\vec{v}$ is a maximal vector modulo $p^k$. Therefore, $\omega \mid \Omega$.
			
			We know $A^\omega \equiv I \bmod{p^k}$, so
			\[
				A^\omega\vec{v} \equiv \vec{v} + p^kB\vec{v} \mod{p^{k+1}}
			\]
			for some matrix $B \in \mathds{Z}_p^{L \times L}$.
			
			If $B \equiv 0$, then we have
			\[
				A^\omega\vec{v} \equiv \vec{v} \mod{p^{k+1}}
			\]
			and since $\omega \mid \Omega$, this means $\Omega = \omega$.
			
			Otherwise, if $B \not\equiv 0$, we can look at successive powers of $A^\omega$ and find the first where a product with itself and $\vec{v}$ returns $\vec{v}$.
			If $\vec{v} \in \ker{(p^kB)}$, then
			\begin{align*}
				A^\omega\vec{v} \equiv& \ \vec{v} + p^kB\vec{v} \\
				                \equiv& \ \vec{v} \mod{p^{k+1}}
			\end{align*}
			and in this case $\Omega = \omega$.
			
			Calculating $A^{2\omega}$, we see
			\begin{align*} 
				A^{2\omega}\vec{v} \equiv& \ (I + p^kB)^2\vec{v} \\
				                   \equiv& \ (I + 2p^kB + p^{2k}B^2)\vec{v} \\
								   \equiv& \ \vec{v} + 2p^kB\vec{v} \mod{p^{k+1}}
			\end{align*}
			Note that if $\vec{v} \in \ker{(2p^kB)}$, then
			\[
				2p^kB\vec{v} \equiv \vec{0} \implies p^kB\vec{v} \equiv \vec{0} \mod{p^{k+1}}
			\]
			In fact, $\vec{v} \in \ker{(qp^kB)} \Leftrightarrow \vec{v} \in \ker{(p^kB)}$ for all $q \in \mathds{Z}_p^+$, so whenever $\vec{v}$ is annihilated by $qp^kB$, 
			$\Omega = \omega$. From now on, assume $\vec{v}$ is not annihilated by $qp^kB$.
			
			The above expression for $A^{2\omega}\vec{v}$ suggests a general form for $A^{n\omega}\vec{v}$:
			\[
				A^{n\omega}\vec{v} \overset{?}{\equiv} \vec{v} + np^kB\vec{v} \mod{p^{k+1}}
			\]
			We've confirmed this pattern is true for $n=1$ and $n=2$. If we're given the expression
			\[
				A^{r\omega}\vec{v} \equiv \vec{v} + rp^kB\vec{v} \mod{p^{k+1}}
			\]
			then
			\begin{align*}
				A^{(r+1)\omega}\vec{v} \equiv& \ (I + p^kB)(I + rp^kB)\vec{v} \\
				                       \equiv& \ (I + (r+1)p^kB + rp^{2k}B^2)\vec{v} \\
									   \equiv& \ \vec{v} + (r+1)p^kB\vec{v} \mod{p^{k+1}}
			\end{align*}
			So
			\[
				A^{r\omega}\vec{v} \equiv \vec{v} + rp^kB\vec{v} \implies A^{(r+1)\omega}\vec{v} \equiv \vec{v} + (r+1)p^kB\vec{v} \mod{p^{k+1}}
			\]
			By induction, this means
			\[
				A^{n\omega}\vec{v} \equiv \vec{v} + np^kB\vec{v} \mod{p^{k+1}}
			\]
			Plugging $n=p$ into the above relation gives us
			\begin{align*}
				A^{p\omega}\vec{v} \equiv& \ \vec{v} + pp^kB\vec{v} \\
				                   \equiv& \ \vec{v} \mod{p^{k+1}}
			\end{align*}
			$n=p$ is the first such substitution that causes this cancellation. Therefore, when $\vec{v} \not\in \ker{(p^kB)}$ and when $B \not\equiv 0$, $\Omega = p\omega$.
			
			These are all the cases we need to consider, so $\Omega = \omega$ or $\Omega = p\omega$.
		\end{proposition}
		
		Now, we have everything we need to show that a maximal vector modulo $p^k$ always exists under an invertible matrix.
		
		\begin{proposition}{Given an invertible matrix $A \in \mathds{Z}_{p^k}^{L \times L}$, a maximal vector exists modulo $p^k$ for odd primes $p$.}
			\label{prop:maxVectGuarantee}
			An invertible matrix modulo $p^k$ remains invertible modulo $p$ since, by the Invertible Matrix Theorem, a matrix is invertible when its determinant is 
			invertible, and if the determinant of a matrix is invertible modulo $p^k$, it must also be invertible modulo $p$ (via modular reduction).
			
			So, $A \bmod{p}$ is invertible, meaning proposition \ref{prop:maxBasis} guarantees that a basis of maximal vectors under $A \bmod{p}$ exists for $\mathds{Z}_p^L$.
			We'll call this basis $\mathcal{B}$.
			
			Now, assume that, modulo $p$, $\omega$ is the multiplicative order of $A$. Because each $\vec{b} \in \mathcal{B}$ is a maximal vector, we know $\omega$ is also 
			the multiplicative order of $\vec{b}$ modulo $p$.
			
			If we now look at $A \bmod{p^2}$ (which is also invertible by the same argument as above), proposition \ref{prop:cycLenInc} tells us that the multiplicative order
			of $A$ modulo $p^2$, which we'll label as $\Omega$, is either $\omega$ or $p\omega$. If $\Omega = \omega$, then all the vectors in $\mathcal{B}$ must still be 
			maximal vectors. Otherwise, if some $\vec{b} \in \mathcal{B}$ was no longer a maximal vector modulo $p^2$, then we'd have that
			\[
				A^n\vec{b} \equiv \vec{b} \mod{p^2} \implies A^n\vec{b} \equiv \vec{b} \mod{p}
			\]
			where $n < \omega$, which is a contradiction to the fact that $\vec{b}$ is a maximal vector modulo $p$. In the case where $\Omega = \omega$, we don't need to do
			any more work to find maximal vectors; our set $\mathcal{B}$ contains $L$ of them.
			
			It's possible that, for every increase of our modulus, from $p$ to $p^2$ to $p^3$, all the way to $p^k$, the multiplicative order of $A$ remains as $\omega$. If 
			this is the case, then $\mathcal{B}$ will still contain $L$ maximal vectors; we don't need to do any more work to find them.
			
			Otherwise, let's assume that $p^\ell$, where $p < p^\ell < p^k$, is the first modulus where the multiplicative order of $A$ increases from $\omega$ to 
			$p\omega$. In this case, we have that
			\[
				A^\omega \equiv I + p^{\ell-1}B \mod{p^\ell}
			\]
			for some nonzero matrix $B \in \mathds{Z}_p^{L \times L}$. By proposition \ref{prop:maxCycLenInc}, the multiplicative order of $\vec{b} \in \mathcal{B}$ modulo 
			$p^\ell$ is either $\omega$ or $p\omega$. Calculating $A^\omega\vec{b}$, we see
			\[
				A^\omega\vec{b} \equiv \vec{b} + p^{\ell-1}B\vec{b} \mod{p^\ell}
			\]
			It's possible that $\vec{b} \in \ker{(p^{\ell-1}B)}$, in which case the multiplicative order of $\vec{b}$ is $\omega$. However, by proposition 
			\ref{prop:basisLift}, $\mathcal{B}$ forms a basis for $\mathds{Z}_{p^\ell}^L$, so at least one vector $\vec{\beta} \in \mathcal{B}$ must not be in the kernel of $p^{\ell-1}B$. Otherwise, if no such vector 
			existed, then $p^{\ell-1}B$ would annihilate all the vectors of a basis, meaning it would necessarily be the zero matrix. We know $p^{\ell-1}B \not\equiv 0$, 
			since $A^\omega \not\equiv I$ by assumption, so such a vector $\vec{\beta}$ must exist.
			
			Therefore,
			\begin{align*}
				A^\omega\vec{\beta} \equiv& \ \vec{\beta} + p^{\ell-1}B\vec{\beta} \mod{p^\ell} \\
				                \not\equiv& \ \vec{\beta}
			\end{align*}
			The multiplicative order of $\vec{\beta}$ cannot possibly be $\omega$ from the above equation, so it must be $p\omega$, meaning it is a maximal vector modulo 
			$p^\ell$. As well, by proposition \ref{prop:liftIterate}, the matrix $B$ in the expression for $A^\omega \bmod{p^\ell}$ will be the same matrix $B$ in the 
			expressions for $A^{p^x\omega} \bmod{p^{\ell+x}}$ for positive integers $x$, so all higher powers will have that
			\begin{align*}
				A^{p^x\omega}\vec{\beta} \equiv& \ \vec{\beta} + p^{\ell+x-1}B\vec{\beta} \mod{p^{\ell+x}} \\
				                     \not\equiv& \ \vec{\beta}
			\end{align*}
			So long as $\vec{\beta}$ is a maximal vector modulo $p^{\ell+x-1}$, the above equation shows that it must also be a maximal vector modulo $p^{\ell+x}$ by 
			applying proposition \ref{prop:maxCycLenInc} (since proposition \ref{prop:cycLenInc} guarantees that $p^{x+1}\omega$ is the multiplicative order of $A$ modulo 
			$p^{\ell+x}$). Therefore, by applying the above reasoning up to modulo $p^k$, $\vec{\beta}$ must also be a maximal vector modulo $p^k$.
		\end{proposition}
	
	\bibliographystyle{plainnat}
	\bibliography{refs.bib}
\end{document}
