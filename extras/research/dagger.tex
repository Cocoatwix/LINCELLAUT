
\documentclass[a4paper, 12pt, reqno]{amsart}
\usepackage[T1]{fontenc}

\usepackage[parfill]{parskip}
\usepackage[margin=3cm]{geometry}

\usepackage{setspace}
\setstretch{1.25}

\usepackage{dsfont}
\usepackage{amssymb}
\usepackage{wasysym} %\newmoon
\usepackage{tipa} %fancy f

%For drawing tree structures 
%tex.stackexchange.com/questions/5447
\usepackage{qtree}
\qtreecenterfalse                  %Don't automatically center trees
\setlength\tabcolsep{0.5cm}        %Padding for trees
\renewcommand\qtreeunaryht{0.58cm} %Length of lines in tree

\usepackage{verbatim}

%ceiling function
\newcommand\ceil[1]{\left\lceil #1 \right\rceil}

%floor function
\newcommand\floor[1]{\left\lfloor #1 \right\rfloor}

\newcounter{propcounter}[section]
\newenvironment{proposition}[1]
{
	\refstepcounter{propcounter}
	\textbf{Proposition \thesection.\thepropcounter \ - }\emph{#1} \\ \\
	\emph{Justification.}
}
{
	\textbf{QED.} \\
}

%p factor counting function
\newcommand\fancyf[1]{\textup{\texthtbardotlessj}(#1)}

\begin{document}
	\section{The $\dag$ Function}
		\label{sec:dagger}
		Say we're given two $L \times L$ matrices $A$ and $C$ modulo a prime power $p^k$, so $A, C \in \mathds{Z}_{p^k}^{L \times L}$. If we took the sum of $A$ and $C$ and
		raised it to some power, we'd get
		\[
			(A + C)^x \mod{p^k}
		\]
		In general, matrix multiplication between $A$ and $C$ doesn't commute, so we can't make use of the Binomial Theorem to write a general formula for this expression. 
		The first few powers of $(A + C)^x$ are:
		\begin{alignat*}{2}
			(A + C)^0              & \equiv I                                                     & \mod{p^k} \\
			(A + C)^1              & \equiv A + C                                                 & \mod{p^k} \\
			(A + C)^2              & \equiv A^2 + AC + CA + C^2                                   & \mod{p^k} \\
			(A + C)^3              & \equiv A^3 + {A^2}C + ACA + CA^2 + {C^2}A + CAC + AC^2 + C^3 & \mod{p^k} \\
			\vdots \hspace{0.85cm} & \hspace{5.33cm} \vdots                                       & \vdots \hspace{0.5cm}
		\end{alignat*}
		Even with fairly low exponents, these expressions quickly become both monstrous to work with and monstrous to write out. A simpler notation, at the very least, would
		make expressing these sums far easier. There are plenty of ways to do this, though the one most useful for our purposes involves grouping elements together 
		depending on how many $C$ matrices are present in the product. From above, we see that
		\[
			(A + C)^3 \equiv A^3 + {A^2}C + ACA + CA^2 + {C^2}A + CAC + AC^2 + C^3 \mod{p^k}
		\]
		Grouping the terms together by the number of $C$ matrices they have, we get
		\[
			(A + C)^3 \equiv (A^3) + ({A^2}C + ACA + CA^2) + ({C^2}A + CAC + AC^2) + (C^3) \mod{p^k}
		\]
		Interestingly, it seems each sum within a set of brackets is the sum of all unique orderings of a particular set of matrices. For example, the first set of
		brackets contains the sum of all possible orderings of 3 $A$ matrices and 0 $C$ matrices into a product. The second set of brackets contains the sum of all possible
		orderings of 2 $A$ matrices and 1 $C$ matrix into a product. This pattern continues.
		
		In turns out that, in general, powers of $(A + C)$ can always be grouped into sums of all possible orderings of a certain number of $A$ matrices and $C$ matrices.
		
		\begin{proposition}{For matrices $A, C \in \mathds{Z}_{p^k}^{L \times L}$, the expression $(A + C)^x$ for $x \in \mathds{Z}^+$ can be written as
		\[
			(A + C)^x \equiv \sum_{i\,=\,0}^x \dag_i^x \mod{p^k}
		\]
		where $\dag_a^b$ is the sum of all unique products of $(a)$ $C$ matrices and $(b-a)$ $A$ matrices.}
			\label{prop:daggerDef}
			The base case of $x = 1$ can be verified by directly looking at the expression for $(A + C)^1$:
			\[
				(A + C)^1 \equiv A + C \mod{p^k}
			\]
			The only unique product that can be made with 1 $A$ matrix and $0$ C matrices is $A$, so $A = \dag_0^1$. Similarly, the only unique product that can be made 
			with 0 $A$ matrices and 1 $C$ matrix is $C$, so $C = \dag_1^1$. Therefore,
			\[
				(A + C)^1 \equiv \sum_{i\,=\,0}^1 \dag_i^1 \mod{p^k}
			\]
			
			We can now use induction to prove that all higher powers of $(A + C)$ will also follow this pattern. Assume we have
			\[
				(A + C)^y \equiv \sum_{i\,=\,0}^y \dag_i^y \mod{p^k}
			\]
			Finding an expression for $(A + C)^{y+1}$, we see
			\begin{align*}
				(A + C)^{y+1} & \equiv (A + C)(A + C)^y                 \\
				              & \equiv (A + C)\sum_{i\,=\,0}^y \dag_i^y \\
							  & \equiv \left( \sum_{i\,=\,0}^y A\dag_i^y \right) + \left( \sum_{j\,=\,0}^y C\dag_j^y \right)
			\end{align*}
			With a little thinking, we see that for $1 \leq r \leq y$, $A\dag_r^y + C\dag_{r-1}^y = \dag_r^{y+1}$ since $A\dag_r^y$ gives all the unique products of $(r)$ 
			$C$ matrices and $(y+1-r)$ $A$ matrices where an $A$ matrix is guaranteed to be the leftmost matrix in the product, and $C\dag_{r-1}^y$ gives all the unique 
			products of $(r)$ $C$ matrices and $(y+1-r)$ $A$ matrices where a $C$ matrix is guaranteed to be the leftmost matrix in the product. Therefore, we can rewrite 
			our above summations as
			\[
				\equiv A\dag_0^y + \left( \sum_{i\,=\,1}^y \dag_i^{y+1} \right) + C\dag_y^y
			\]
			By definition of $\dag$, $\dag_0^y = A^y$ and $\dag_y^y = C^y$, so we can rewrite the above sum as
			\begin{align*}
				\equiv & \, A(A^y) + \left( \sum_{i\,=\,1}^y \dag_i^{y+1} \right) + C(C^y)   \\
				\equiv & \, A^{y+1} + \left( \sum_{i\,=\,1}^y \dag_i^{y+1} \right) + C^{y+1}
			\end{align*}
			$\dag_0^{y+1} = A^{y+1}$ and $\dag_{y+1}^{y+1} = C^{y+1}$, so we can further simplify our above sum to
			\begin{align*}
				\equiv & \, \dag_0^{y+1} + \left( \sum_{i\,=\,1}^y \dag_i^{y+1} \right) + \dag_{y+1}^{y+1} \\
				\equiv & \, \sum_{i\,=\,0}^{y+1} \dag_i^{y+1}
			\end{align*}
			Therefore, we've shown that
			\[
				(A + C)^y \equiv \sum_{i\,=\,0}^y \dag_i^y \ \implies\  (A + C)^{y+1} \equiv \sum_{i\,=\,0}^{y+1} \dag_i^{y+1} \mod{p^k}
			\]
			Since we also know $(A + C)^1 \equiv \sum_{i\,=\,0}^1 \dag_i^1 \bmod{p^k}$, the principle of induction says that, for $x \in \mathds{Z}^+$,
			\[
				(A + C)^x \equiv \sum_{i\,=\,0}^x \dag_i^x \mod{p^k}
			\]
		\end{proposition}
		
		Proposition \ref{sec:dagger}.\ref{prop:daggerDef} gives us a much more compact way to write the expansions of expressions of the form $(A + C)^x$. It also gives us a
		slightly combinatorial way to view the expansions of these expressions; each $\dag$ term is simply the sum of all possible multiplication orderings of some number of
		$A$ and $C$ matrices. As we'll see later in this document, thinking about these expressions in a combinatorial way will prove useful in finding 
		specific properties these expressions must have.
		
	\section{Non-stable Matrix Lifts}
		For a matrix $A \in \mathds{Z}_{p^k}^{L \times L}$ with $\ker(A) = \{\vec{0}\}$, we say the \emph{cycle length} or \emph{orbit length} of the matrix is the smallest
		$\omega \in \mathds{Z}^+$ such that
		\[
			A^\omega \equiv I \mod{p^k}
		\]
		
		A common technique used to analyse the dynamics of matrices (particularly with regard to their cycle length) is to look at \emph{lifts} of matrices. A
		\emph{lift} of a matrix $A \in \mathds{Z}_{p^k}^{L \times L}$ is any matrix $\hat{A} \in \mathds{Z}_{p^{k+l}}^{L \times L}$ (with $l \in \mathds{Z}^+$) where
		\[
			\hat{A} \equiv A + p^kB \mod{p^{k+l}},\quad B \in \mathds{Z}_{p^l}^{L \times L}
		\]
		Due to how lifts are constructed, the cycle length of a lift must be a multiple of the cycle length of the matrix it was lifted from. In fact, if the cycle length of
		$A \bmod{p^k}$ is $\omega$, then for a lift of $A \bmod{p^{k+l}}$, the most common cycle length for it to have is $p^l\omega$.
		
		To see why, imagine we take a lift of $A$ from modulo $p^k$ to modulo $p^{k+1}$, with $\ker{(A)} = \{\vec{0}\}$:
		\[
			\hat{A} \equiv A + p^kB \mod{p^{k+1}},\quad B \in \mathds{Z}_p^{L \times L}
		\]
		If the cycle length of $A \bmod{p^k}$ is $\omega$, then
		\[
			\hat{A}^\omega \equiv (A + p^kB)^\omega \equiv I + p^kR \mod{p^{k+1}},\quad R \in \mathds{Z}_p^{L \times L}
		\]
		since any power of $\hat{A}$ must reduce to the same power of $A$ modulo $p^k$ (by construction of $\hat{A}$), and $A^\omega \equiv I \bmod{p^k}$. This is why 
		the cycle length of $\hat{A} \bmod{p^{k+1}}$ must be a multiple of $\omega$: otherwise, if $\hat{A}^r \equiv I \bmod{p^{k+1}}$ and $\omega \nmid r$, then
		$A^r \equiv I \bmod{p^k}$ via modular reduction, which is a contradiction since the cycle length of $A$ is $\omega$---$A$ can only cycle back to $I \bmod{p^k}$ with 
		an exponent that's a multiple of $\omega$ (by definition of the cycle length). 
		
		It's possible that $R \equiv 0$, in which case the cycle length of $\hat{A} \bmod{p^{k+1}}$ is equal to $\omega$, but in the vast majority of cases, 
		$R \not\equiv 0$.
		
		Now, if we imagine raising $\hat{A}$ to $2\omega$, we get:
		\begin{align*}
			\hat{A}^{2\omega} & \equiv \hat{A}^\omega\hat{A}^\omega \\
			                  & \equiv (I + p^kR)(I + p^kR)         \\
							  & \equiv I + 2p^kR + p^{2k}R^2        \\
						      & \equiv I + 2p^kR \mod{p^{k+1}}
		\end{align*}
		We see that $\hat{A}^{2\omega} \equiv I + 2p^kR \bmod{p^{k+1}}$. In general, if we have
		\[
			\hat{A}^{r\omega} \equiv I + rp^kR \mod{p^{k+1}}
		\]
		then
		\begin{align*}
			\hat{A}^{(r+1)\omega} & \equiv \hat{A}^{\omega}\hat{A}^{r\omega} \\
			                      & \equiv (I + p^kR)(I + rp^kR)             \\
							      & \equiv I + rp^kR + p^kR + rp^{2k}R^2     \\
								  & \equiv I + (r+1)p^kR \mod{p^{k+1}}
		\end{align*}
		This means
		\[
			\hat{A}^{r\omega} \equiv I + rp^kR \ \implies\  \hat{A}^{(r+1)\omega} \equiv I + (r+1)p^kR \mod{p^{k+1}}
		\]
		By the principle of induction, we can say
		\[
			\hat{A}^{n\omega} \equiv I + np^kR \mod{p^{k+1}}
		\]
		for all $n \in \mathds{Z}^+$.
		
		Now, since the cycle length of $\hat{A}$ must be a multiple of $\omega$, we need to find the smallest value of $n$ where the expression for $\hat{A}^{n\omega}$ is
		equivalent to the identity. By inspection, we see that $n = p$ is the first such value:
		\begin{align*}
			\hat{A}^{p\omega} & \equiv I + pp^kR    \\
			                  & \equiv I + p^{k+1}R \\
							  & \equiv I \mod{p^{k+1}}
		\end{align*}
		So, in the most common case, the cycle length of $\hat{A} \bmod{p^{k+1}}$ is $p\omega$. Using this exact logic, we could then find the most common cycle length for
		a lift of $\hat{A} \bmod{p^{k+2}}$, then mod $p^{k+3}$, etc., etc., until we find the most common cycle length of a lift mod $p^{k+l}$, which would happen to
		be $p^l\omega$ (since an extra factor of $p$ would accumulate for each increase in the modulus' power).
		
		If we start with a matrix $A \in \mathds{Z}_p^{L \times L}$ (so $A$ is a matrix modulo $p$) with $\ker(A) = \{\vec{0}\}$ and cycle length $\omega$, and we take a 
		lift of the matrix $\hat{A} \in \mathds{Z}_{p^k}^{L \times L}$, the matrix $\hat{A}$ is called a \emph{non-stable lift} of $A$ if its cycle length is $p^{k-1}\omega$,
		which happens to be the most common possible value as described above. The name is derived from the fact that, for the sequence of prime-power moduli up to $p^k$
		(so mod $p$, $p^2$, $p^3$, $\cdots$, $p^{k-1}$, $p^k$), $\hat{A}$ will have a different cycle length under all of them; at each increase of the modulus, the cycle 
		length of $\hat{A}$ will gain an extra factor of $p$.
		
		Non-stable lifts are easier to work with than the alternative since, if its cycle length under \emph{any} of the prime-power moduli in the sequence from $p$ to $p^k$
		is known, we can immediately figure out its cycle length under any other modulus in the sequence. As well, non-stable lifts are, as described above, the most common
		type of lift, so any results shown for them will apply to the vast majority of matrices.
		
	\section{The Lifting Problem}
		\label{sec:liftingProblem}
		Something interesting happens when we begin to play around with different lifts of a matrix.
		
		As an example, let
		\[
			A = \begin{bmatrix}
				1 & 1 \\
				2 & 4
			\end{bmatrix} \mod{5}
		\]
		One can verify that $\ker(A) = \{\vec{0}\}$ and that its cycle length under a modulus of 5 is $\omega = 8$. There are many different lifts we can take of this matrix.
		Let's let
		\[
			C_0 = \begin{bmatrix}
				0 & 2 \\
				2 & 1
			\end{bmatrix},\quad
			C_1 = \begin{bmatrix}
				1 & 2 \\
				0 & 2
			\end{bmatrix},\quad
			C_2 = \begin{bmatrix}
				4 & 2 \\
				4 & 3
			\end{bmatrix}
		\]
		and consider the three lifts $(A + 5C_0)$, $(A + 5C_1)$, and $(A + 5C_2)$ mod 125. All three of these lifts have cycle lengths of $25\omega = 200$, so they are all
		non-stable lifts. This means that, mod 25, these matrices should have a cycle length of $5\omega = 40$. If this is true, raising these lifts to the 40th power mod
		125 should give us matrices that, when reduced mod 25, give the identity. A few quick calculations shows this to be true:
		\begin{alignat*}{2}
			(A + 5C_0)^{40} & \equiv \begin{bmatrix}
				101 & 25 \\
				 50 & 51
			\end{bmatrix} & \mod{125} \\
			(A + 5C_1)^{40} & \equiv \begin{bmatrix}
				101 & 50 \\
				100 &  1
			\end{bmatrix} & \mod{125} \\
			(A + 5C_2)^{40} & \equiv \begin{bmatrix}
				101 & 100 \\
				 75 &  26
			\end{bmatrix} & \mod{125}
		\end{alignat*}
		
		Another set of lifts we can consider are $(A + 25C_0)$, $(A + 25C_1)$, and $(A + 25C_2)$ mod 125. Again, the cycle lengths of these three lifts happen to be 
		$25\omega = 200$, so these are all non-stable lifts. Using the same reasoning as above, raising these matrices to the 40th power should give us matrices that, when 
		reduced mod 25, give us the identity. Calculating the relevant powers, we see that this is indeed true:
		\begin{alignat*}{2}
			(A + 25C_0)^{40} & \equiv \begin{bmatrix}
				26 & 75 \\
				25 &  1
			\end{bmatrix} & \mod{125} \\
			(A + 25C_1)^{40} & \equiv \begin{bmatrix}
				26 & 75 \\
				25 &  1
			\end{bmatrix} & \mod{125} \\
			(A + 25C_2)^{40} & \equiv \begin{bmatrix}
				26 & 75 \\
				25 &  1
			\end{bmatrix} & \mod{125}
		\end{alignat*}
		This time, however, something strange happens. Despite using three different lifts, each one iterates to the same resulting matrix. What's more, it appears the 
		resulting matrices are exactly the same as the resulting matrix of raising $A$ itself to the 40th power:
		\[
			A^{40} \equiv \begin{bmatrix}
				26 & 75 \\
				25 &  1
			\end{bmatrix} \mod{125}
		\]
		The extra matrices we added to get our lifts above appeared to not affect the result at all! Recalling our $\dag$ notation, we can write our matrix lift expressions
		as
		\begin{alignat*}{2}
			(A + 25C_0)^{40} = \dag_0^{40} + 25\dag_1^{40} \equiv \begin{bmatrix}
				26 & 75 \\
				25 &  1
			\end{bmatrix} + & 25\begin{bmatrix}
				115 & 15 \\
				  5 & 35
			\end{bmatrix} & \mod{125} \\
			(A + 25C_1)^{40} = \dag_0^{40} + 25\dag_1^{40} \equiv \begin{bmatrix}
				26 & 75 \\
				25 &  1
			\end{bmatrix} + & 25\begin{bmatrix}
				15 & 45 \\
				40 & 25
			\end{bmatrix} & \mod{125} \\
			(A + 25C_2)^{40} = \dag_0^{40} + 25\dag_1^{40} \equiv \begin{bmatrix}
				26 & 75 \\
				25 &  1
			\end{bmatrix} + & 25\begin{bmatrix}
				90 & 105 \\
				85 &   5
			\end{bmatrix} & \mod{125}
		\end{alignat*}
		Note that the $\dag$ function used $C_0$, $C_1$, and $C_2$ as its "$C$" matrices, not $25C_0$, $25C_1$, and $25C_2$. The extra factors of 25 were brought outside the
		$\dag$ to better organise the resulting matrices. This convention will be used for the rest of the document.
		
		These expressions show exactly why the different lifts seemed to behave the same: each of their $\dag_1^{40}$ matrices had at least one extra factor of 5 in it, 
		which, when multiplied by 25, sent the entire term to zero mod 125. Comparing these expressions to the ones we get for the earlier lifts:
		\begin{align*}
			(A + 5C_0)^{40} = \dag_0^{40} + 5\dag_1^{40} + 25\dag_2^{40} \equiv \begin{bmatrix}
				26 & 75 \\
				25 &  1
			\end{bmatrix} +& 5\begin{bmatrix}
				115 & 15 \\
				  5 & 35
			\end{bmatrix} + 25\begin{bmatrix}
				40 & 30 \\
				80 & 55
			\end{bmatrix} \bmod{125} \\
			(A + 5C_1)^{40} = \dag_0^{40} + 5\dag_1^{40} + 25\dag_2^{40} \equiv \begin{bmatrix}
				26 & 75 \\
				25 &  1
			\end{bmatrix} +& 5\begin{bmatrix}
				15 & 45 \\
				40 & 25
			\end{bmatrix} + 25\begin{bmatrix}
				105 & 45 \\
				 10 & 15
			\end{bmatrix} \bmod{125} \\
			(A + 5C_2)^{40} = \dag_0^{40} + 5\dag_1^{40} + 25\dag_2^{40} \equiv \begin{bmatrix}
				26 & 75 \\
				25 &  1
			\end{bmatrix} +& 5\begin{bmatrix}
				90 & 105 \\
				85 &   5
			\end{bmatrix} + 25\begin{bmatrix}
				 0 & 70 \\
				15 & 25
			\end{bmatrix} \bmod{125}
		\end{align*}
		
		It seems gaining a few extra factors of 5 in the $\dag$ matrices is common. Why is this happening? Is it generic behaviour?
		
	\section{Understanding the Lifting Problem}
		\label{sec:understanding1}
		In fact, the behaviour observed in section \ref{sec:liftingProblem} is fairly generic, so long as the lifts of $A$ being considered are non-stable lifts. For the 
		rest of this section, assume $A \in \mathds{Z}_p^{L \times L}$, $\ker(A) = \{\vec{0}\}$, the cycle length of $A \bmod{p}$ is $\omega$ and assume $p$ is an odd prime.
		
		The first thing to do is generalise the expressions we were getting for the lifts above. Say we're given a non-stable lift of $A$ in the form $(A + p^nC) \bmod{p^k}$
		for some matrix $C \in \mathds{Z}^{L \times L}$ (no modulus assumed) and some $n \in \mathds{Z}^+$. Since our lift is non-stable, its cycle length will be 
		$\omega p^{k-1}$ modulo $p^k$ and $\omega p^{k-2}$ modulo $p^{k-1}$. This means that $(A + p^nC)^{\omega p^{k-2}} \bmod{p^k}$ should be a matrix that, when reduced 
		mod $p^{k-1}$, gives the identity. Using the $\dag$ function, we can expand $(A + p^nC)^{\omega p^{k-2}}$ as
		\[
			(A + p^nC)^{\omega p^{k-2}} \equiv \sum_{i\,=\,0}^{\omega p^{k-2}} p^{ni}\dag_i^{\omega p^{k-2}} \mod{p^k}
		\]
		
		The above calculations we did for the lifts seem to suggest that each $\dag_i^{\omega p^{k-2}}$ should have a factor of $p^{k - ni}$ in it so that it's
		annihilated when multiplied by $p^{ni}$, except for $\dag_0^{\omega p^{k-2}}$ and maybe $\dag_1^{\omega p^{k-2}}$ when $n = 1$. If this guess is true, then we've 
		shown that any non-stable lift in the form $(A + p^nC)$ with $n > 1$ will behave identically to $A$ when raised to the $\omega p^{k-2}$-th power.
		
		To gain some bearings, let's assume for a moment that $p = 3$, $\omega = 2$, and $k = 3$. This allows us to have some concrete numbers to work with. Using these
		numbers, our above expression would become
		\begin{align*}
			(A + 3^nC)^6 & \equiv \dag_0^6 + 3^n\dag_1^6 + 3^{2n}\dag_2^6 + 3^{3n}\dag_3^6 + 3^{4n}\dag_4^6 + 3^{5n}\dag_5^6 + 3^{6n}\dag_6^6 \mod{27} \\
			             & \equiv \dag_0^6 + 3^n\dag_1^6 + 3^{2n}\dag_2^6 \mod{27}
		\end{align*}
		
		For this particular example, if our guess from above is correct, then $\dag_2^6$ should have a factor of 3 in it. Unfortunately, looking at the actual sum for 
		$\dag_2^6$ doesn't give us very much info about whether any factors of 3 are present:
		\begin{align*}
			\dag_2^6 =& \ C^2A^4    + CACA^3  + CA^2CA^2 + CA^3CA + CA^4C \\
			         +& \ AC^2A^3   + ACACA^2 + ACA^2CA  + ACA^3C         \\
					 +& \ A^2C^2A^2 + A^2CACA + A^2CA^2C                  \\
					 +& \ A^3C^2A   + A^3CAC                              \\
					 +& \ A^4C^2 \mod{27}
		\end{align*}
		
		However, we can gleam some info from this expression by first multiplying it by 9. Because $A \bmod{27}$ is technically a lift of $A \bmod{3}$, we know that 
		$A^2 \bmod{27}$ must reduce to $I$ when reduced mod 3 since the cycle length of $A \bmod{3}$ is 2 (in this particular example). This means 
		\[
			A^2 \equiv I + 3R \mod{27},\quad R \in \mathds{Z}_9^{L \times L}
		\]
		So, if we multiply by 9, we'd get
		\[
			9A^2 \equiv 9I + 27R \equiv 9I \mod{27}
		\]
		When multiplied by 9, $A^2$ behaves exactly the same as $I$. Using this, multiplying $\dag_2^6$ by 9 allows us to do some simplification:
		\begin{align*}
			9\dag_2^6 =& \ 9C^2A^4    + 9CACA^3  + 9CA^2CA^2 + 9CA^3CA + 9CA^4C \\
			          +& \ 9AC^2A^3   + 9ACACA^2 + 9ACA^2CA  + 9ACA^3C          \\
				      +& \ 9A^2C^2A^2 + 9A^2CACA + 9A^2CA^2C                    \\
				 	  +& \ 9A^3C^2A   + 9A^3CAC                                 \\
					  +& \ 9A^4C^2 \mod{27}                                     \\
			    \equiv & \ 9C^2IA^2 + 9CACAI + 9CICA^2 + 9CIACA + 9CIA^2C \\
				      +& \ 9AC^2IA  + 9ACACI + 9ACICA  + 9ACIAC           \\
					  +& \ 9IC^2A^2 + 9ICACA + 9ICA^2C                    \\
					  +& \ 9IAC^2A  + 9IACAC                              \\
					  +& \ 9IA^2C^2 \mod{27}                              \\
				\equiv & \ 9C^2I  + 9CACA + 9C^2I  + 9CACA + 9CIC \\
				      +& \ 9AC^2A + 9ACAC + 9AC^2A + 9ACAC        \\
					  +& \ 9C^2I  + 9CACA + 9CIC                  \\
					  +& \ 9AC^2A + 9ACAC                         \\
					  +& \ 9IC^2 \mod{27}                         \\
				\equiv & \ 9C^2   + 9CACA + 9C^2   + 9CACA + 9C^2 \\
				      +& \ 9AC^2A + 9ACAC + 9AC^2A + 9ACAC        \\
					  +& \ 9C^2   + 9CACA + 9C^2                  \\
					  +& \ 9AC^2A + 9ACAC                         \\
					  +& \ 9C^2 \mod{27}                          \\
				\equiv & \ 9(6C^2 + 3CACA + 3AC^2A + 3ACAC) \mod{27} \\
				\equiv & \ 9(3(2C^2 + CACA + AC^2A + ACAC)) \mod{27}
		\end{align*}
		On top of the 9 we multiplied $\dag_2^6$ by, an extra factor of 3 appeared! This is quite interesting, since multiplying by 9 can't make an additional factor of 3
		appear in $\dag_2^6$. This factor must have already been in $\dag_2^6$; multiplying by 9 just made it easier to find. Therefore, we've shown that $\dag_2^6$ has at
		least one factor of 3 in this example, so
		\begin{align*}
			(A + 3^nC)^6 \equiv & \ \dag_0^6 + 3^n\dag_1^6 + 3^{2n}\dag_2^6                                   \\
			             \equiv & \ \dag_0^6 + 3^n\dag_1^6 + 3(3^{2n})D,\quad D \in \mathds{Z}_9^{L \times L} \\
						 \equiv & \ \dag_0^6 + 3^n\dag_1^6 \mod{27} \quad \text{(since } n \geq 1)
		\end{align*}
		In fact, we can play this same simplification game with $\dag_1^6$ as well:
		\begin{align*}
			9\dag_1^6 &      = 9CA^5  + 9ACA^4  + 9A^2CA^3 + 9A^3CA^2 + 9A^4CA  + 9A^5C  \\
			          & \equiv 9CIA^3 + 9ACIA^2 + 9ICA^3   + 9IACA^2  + 9IA^2CA + 9IA^3C \\
					  & \equiv 9CIA   + 9ACI    + 9CIA     + 9ACI     + 9ICA    + 9IAC   \\
					  & \equiv 9CA    + 9AC     + 9CA      + 9AC      + 9CA     + 9AC    \\
					  & \equiv 9(3CA + 3AC)                                              \\
					  & \equiv 9(3(CA + AC))
		\end{align*}
		Once again, by multiplying by 9, we see $\dag_1^6$ must have a factor of 3. So
		\begin{align*}
			(A + 3^nC)^6 \equiv & \ \dag_0^6 + 3^n\dag_1^6                                    \\
			             \equiv & \ \dag_0^6 + 3^{n+1}D,\quad D \in \mathds{Z}_9^{L \times L} \\
						 \equiv & \ A^6 + 3^{n+1}D \mod{27}
		\end{align*}
		Thus, when $n \geq 2$, the above expression will become $A^6$, proving that all lifts modulo 27 using our given form with $n \geq 2$ will give the same matrix when 
		raised to the $\omega p^{k-2}$-th power.

	\section{Understanding the Lifting Problem, Part 2}
		\label{sec:understanding2}
		Can we make sense of this lifting behaviour in the general case, without specific values for $p$, $k$, or $\omega$? Let's assume we're given a matrix 
		$A \in \mathds{Z}_p^{L \times L}$ where $\ker(A) = \{\vec{0}\}$ and the cycle length of $A \bmod{p}$ is $\omega$. Also assume we're given a non-stable lift of $A$, 
		say $(A + p^nC) \bmod{p^k}$ for $n \in \mathds{Z}^+$ and $C \in \mathds{Z}^{L \times L}$. As before, we know that
		\[
			(A + p^nC)^{\omega p^{k-2}} \equiv \sum_{i\,=\,0}^{\omega p^{k-2}} p^{ni} \dag_i^{\omega p^{k-2}} \mod{p^k}
		\]
		
		Let's look at a particular $\dag$ matrix, say $\dag_a^{\omega p^{k-2}}$ where $a$ is a nonnegative integer such that $0 \leq a \leq \omega p^{k-2}$. If we multiply
		$\dag_a^{\omega p^{k-2}}$ by $p^{k-1}$, we'll be able to perform the same simplification trick as in section \ref{sec:understanding1}. However, since we're dealing
		with a generic $\dag$ matrix, it's rather difficult to write out the exact matrices we'll get from this simplification. We need a better way to "extract" the factor
		of $p$ we're looking for.
		
		Instead, let's consider a generic term in the resulting expression and figure out how many times it'll appear in our sum. Let's say we're given the term
		\[
			p^{k-1}A^{\chi_0}CA^{\chi_1}C\cdots A^{\chi_{a-2}}CA^{\chi_{a-1}}CA^{\chi_a}
		\]
		where $0 \leq \chi_i < \omega$ and $\omega \mid \sum_{i\,=\,0}^a (\chi_i + 1)$. These restraints are in place so that the term we're considering is actually a term 
		which could be obtained by multiplying $\dag_a^{\omega p^{k-2}}$ by $p^{k-1}$.
		
		In order to count the number of times this term appears in the sum for $p^{k-1}\dag_a^{\omega p^{k-2}}$, we need to consider all the various ways in which 
		$A^\omega$ matrices could've been placed in between the matrices---$A^\omega$ disappears when we do our simplification, so to count how many times our term appears, 
		we need to count how many terms before the simplification would reduce to our term, and all these pre-simplification terms would look like our post-simplification 
		term with additional $A^\omega$ matrices placed between the matrices.
		
		How many $A^\omega$ matrices would be present in the pre-simplification terms? Since we're looking at terms in the sum for $\dag_a^{\omega p^{k-2}}$, there should be
		$\omega p^{k-2}$ total matrices in each term. Subtracting the number of matrices present in our post-simplification term, we see there must be
		\begin{align*}
			 & \ \omega p^{k-2} - \sum_{i\,=\,0}^a (\chi_i + 1) \\
			=& \ \omega p^{k-2} - \omega x                      \\
			=& \ \omega(p^{k-2} - x)
		\end{align*}
		extra matrices in the pre-simplification term compared to the post-simplification term. Each $A^\omega$ matrix contributes $\omega$ matrices, so there must be
		$(p^{k-2} - x)$ extra $A^\omega$ matrices in the pre-simplification term.
		
		What possible values can $x$ take? The value of $x$ totally depends on how many matrices are present in our post-simplification term. A post-simplification term with
		the smallest number of matrices possible would look like
		\[
			p^{k-1}\underbrace{CCC\cdots CC}_{a}A^{(\omega - a) \bmod{\omega}}
		\]
		meaning the number of extra matrices in the pre-simplification terms would be given by
		\begin{align*}
			 & \ \omega p^{k-2} - \sum_{i\,=\,0}^a (\chi_i + 1)    \\
			=& \ \omega p^{k-2} - (a + (\omega - a \bmod{\omega})) \\
			=& \ \omega p^{k-2} - \omega\ceil{\frac{a}{\omega}}    \\
			=& \ \omega\left(p^{k-2} - \ceil{\frac{a}{\omega}}\right)
		\end{align*}
		So $x = \ceil{\frac{a}{\omega}}$ when our post-simplification term has the smallest number of matrices possible. A post-simplification term with the largest number of
		matrices possible would look like
		\[
			p^{k-1}\underbrace{(CA^{\omega-1})(CA^{\omega-1})(\cdots)(CA^{\omega-1})}_a
		\]
		meaning the number of extra matrices in the pre-simplification terms would be given by
		\begin{align*}
			 & \ \omega p^{k-2} - \sum_{i\,=\,0}^a (\chi_i + 1) \\
			=& \ \omega p^{k-2} - \omega a                      \\
			=& \ \omega(p^{k-2} - a)
		\end{align*}
		So $x = a$ when our post-simplification term has the largest number of matrices possible. Therefore, the value of $x$ can be any integer between 
		$\ceil{\frac{a}{\omega}}$ and $a$, inclusive.
		
		Now, we have all the details needed to find an explicit formula for counting how many times our post-simplification term appears in the sum for 
		$p^{k-1}\dag_a^{\omega p^{k-2}}$. Let's write our generic post-simplification term as
		\[
			p^{k-1}(A^{\chi_0}CA^{\chi_1})(CA^{\chi_2})(\cdots)(CA^{\chi_{a-1}})(CA^{\chi_a})
		\]
		We want to see how many ways we can arrange our $A^\omega$ matrices inside this expression to get a pre-simplification term that simplifies to this, so we only need
		to consider the expression
		\[
			(A^{\chi_0}CA^{\chi_1})(CA^{\chi_2})(\cdots)(CA^{\chi_{a-1}})(CA^{\chi_a})
		\]
		Notice that multiplying an $A^\omega$ anywhere inside a pair of brackets would be the same as multiplying it somewhere outside the brackets, so we only need to 
		consider multiplying $A^\omega$ outside the brackets. We can see that there are $a$ pairs of brackets in our expression, and from above we know there are
		$(p^{k-2} - x)$ $A^\omega$ matrices for us to use. We can imagine, from a string of $(p^{k-2} - x + a)$ $A^\omega$ matrices, we choose $a$ places for our bracket 
		pairs to go, replacing the $A^\omega$ matrices in the process (ensuring our brackets remain in the correct order, of course).
		
		Using the $\dag_2^6$ example from section \ref{sec:understanding1}, imagine we wanted to find all the possible pre-simplification terms that reduced to $C^2$. In our
		bracketed form, $C^2$ would be written as
		\[
			(C)(C)
		\]
		The number of $A^\omega$ matrices we'd have to use in this example would be 2, since each pre-simplification term must have 6 matrices (the superscript of the 
		$\dag$ function), our 2 $C$ matrices take up 2 of those, and the remaining 4 matrices would be used up by 2 $A^2$ matrices (recall that $\omega = 2$ in this example).
		So, our string of $A^\omega$ matrices would look like
		\[
			(A^2)(A^2)(A^2)(A^2)
		\]
		We have two $C$ matrices to place down here. One possible placement could be
		\[
			(C)(C)(A^2)(A^2) = C^2A^4
		\]
		Another possible placement could be
		\[
			(A^2)(C)(A^2)(C) = A^2CA^2C
		\]
		A third could be
		\[
			(C)(A^2)(A^2)(C) = CA^4C
		\]
		etc. The total number of possible placements is the choose function $\binom{4}{2} = 6$, which means there are 6 $C^2$ terms in the sum for $9\dag_2^6$ (not 
		including the initial multiple of 9 we multiplied by). Sure enough, looking back at section \ref{sec:understanding1} confirms this.
		
		Relating this example to our general case, for a particular post-simplification term, there should be $\binom{p^{k-2} - x + a}{a}$ terms present in the 
		$p^{k-1}\dag_a^{\omega p^{k-2}}$ sum.
		
	\section{Understanding the Lifting Problem, Part 3}
		\label{sec:understanding3}
		Given $A \in \mathds{Z}_p^{L \times L}$ with $\ker(A) = \{\vec{0}\}$ and a cycle length modulo $p$ of $\omega$, section \ref{sec:understanding2} has shown that, given
		a non-stable lift of $A$, say $(A + p^nC)$ with $n \in \mathds{Z}^+$ and $C \in \mathds{Z}^{L \times L}$, individual terms in the sum for 
		$p^{k-1}\dag_a^{\omega p^{k-2}}$, $0 \leq a \leq \omega p^{k-2}$ should appear $\binom{p^{k-2}-x+a}{a}$ times (not including the factor of $p^{k-1}$ we multiply by), 
		where $x$ is an integer between $\ceil{\frac{a}{\omega}}$ and $a$.
		
		In order to find how many factors of $p$ are in $\dag_a^{\omega p^{k-2}}$, we need to find the minimum number of factors of $p$ present in $\binom{p^{k-2}-x+a}{a}$.
		This is because the number of factors of $p$ in $\dag_a^{\omega p^{k-2}}$ is given by how many factors of $p$ we can factor out from each of the coefficients on each 
		term in the sum for $p^{k-1}\dag_a^{\omega p^{k-2}}$. The coefficients are given by $\binom{p^{k-2}-x+a}{a}$, so the minimum number of factors of $p$ in 
		$\binom{p^{k-2}-x+a}{a}$ for all possible $x$ between $\ceil{\frac{a}{\omega}}$ and $a$ will give us how many factors of $p$ are in $\dag_a^{\omega p^{k-2}}$.
		
		How do we find this minimum?
		
		First, we can note that the number $\binom{p^{k-2}-x+a}{a}$ is equivalent to
		\begin{align*}
			\binom{p^{k-2}-x+a}{a} &= \frac{(p^{k-2}-x+a)!}{a!(p^{k-2}-x)!}\\
			                       &= \frac{\prod_{i\,=\,1}^a (p^{k-2}-x+i)}{a!}
		\end{align*}
		No matter the value of $x$, the number of factors of $p$ in $\binom{p^{k-2}-x+a}{a}$ will always be proportional to the factors of $p$ in $\frac{1}{a!}$. How many 
		factors of $p$ are in $\frac{1}{a!}$? Thankfully, there's a relatively nice expression for this:
		\[
			\frac{1}{\prod_{i\,=\,1}^\infty p^{\floor{\frac{a}{p^i}}}}
		\]
		
		Now, we need to find the minimum possible number of factors of $p$ in
		\[
			\prod_{i\,=\,1}^a (p^{k-2}-x+i)
		\]
		If $x = a$, then this expression becomes
		\[
			= (p^{k-2}-a+1)(p^{k-2}-a+2)(\cdots)(p^{k-2}-1)(p^{k-2})
		\]
		If $x = \ceil{\frac{a}{\omega}}$, then this expression becomes
		\[
			= (p^{k-2}-\ceil{\frac{a}{\omega}}+1)(p^{k-2}-\ceil{\frac{a}{\omega}}+2)(\cdots)(p^{k-2}-\ceil{\frac{a}{\omega}}+a)
		\]
		Looking at when $x = a$, if we start at the $(p^{k-2})$ term, then for every $p$ terms to the left of $(p^{k-2})$ in the product, another factor of $p$ will 
		be added to the product. For example, if we look $p$ terms to the left of $(p^{k-2})$, the product will look like
		\begin{align*}
			 & \ \cdots(p^{k-2}-p)(p^{k-2}-p+1)(\cdots)(p^{k-2}-1)(p^{k-2}) \\
			=& \ \cdots p(p^{k-3}-1)(p^{k-2}-p+1)(\cdots)(p^{k-2}-1)(p^{k-2})
		\end{align*}
		An extra factor of $p$ appears in the product. This will happen every additional $p$ terms we look to the left of $(p^{k-2})$. As well, if $x \neq a$, then we'll be
		able to look to the right of $(p^{k-2})$, and the same logic will apply. 
		
		However, the behaviour is a little more complex than that. If we look $p^2$ terms in either direction, an additional factor of $p$ will appear in the product, 
		\emph{on top of} the factors of $p$ gained every $p$ terms. The same happens every $p^3$ terms, every $p^4$ terms, etc. In order to minimise the number factors of $p$
		in $\prod_{i\,=\,1}^a (p^{k-2}-x+i)$, we need to find the value of $x$ such that the number of factors of $p$ accumulated on each side of the $(p^{k-2})$ term is
		minimised.
		
		First, are we sure that the $(p^{k-2})$ term always exists in this product? Well, for $x = a$, we saw that it was on the product. For $x = a - y$ for some nonnegative
		integer $y$, the product becomes
		\[
			(p^{k-2}-a+y+1)(p^{k-2}-a+y+2)(\cdots)(p^{k-2}-1)(p^{k-2})(\cdots)(p^{k-2}+y-1)(p^{k-2}+y)
		\]
		So long as $a - y > 0$, the term $(p^{k-2})$ will always be in the product. We know $x$ is bounded between $\ceil{\frac{a}{\omega}}$ and $a$, so as long as 
		$a \neq 0$, $(p^{k-2})$ will always be in our product.\footnote{In fact, since we're not concerned with making conclusions about $\dag_0^{\omega p^{k-2}}$, we'll 
		assume that $a \neq 0$ for the rest of our calculations.}
		
		This allows us to represent our product in a more graphical way. Imagine $(p^{k-2})$ is the root node to a tree. Each factor in the product is another node in the
		tree. The root node has two branches, one to its left and one to its right. Nodes on the left branch represent terms that are to the left of $(p^{k-2})$ in the 
		product, while nodes on the right branch represent terms that are to the right in the product. For example, if $\omega = 2$, $a = 5$, and 
		$x = \ceil{\frac{a}{\omega}} = 3$, then our product becomes
		\[
			(p^{k-2} - 2)(p^{k-2} - 1)(p^{k-2})(p^{k-2} + 1)(p^{k-2} + 2)
		\]
		and our tree representation becomes
		%I'm so sorry
		\vspace*{-2.3cm}
		\begin{center}
			\begin{tabular}{c}
				\vspace{3.3cm} \\
				1 \\
				  \\
				2
			\end{tabular}
			\hspace*{0.5cm}
			\Tree
			[
				.$(p^{k-2})$
				[.$(p^{k-2}-1)$ [.$(p^{k-2}-2)$ ]] 
				[.$(p^{k-2}+1)$ [.$(p^{k-2}+2)$ ]]
			]
		\end{center}
			
		In fact, the specific terms on the branches under $(p^{k-2})$ don't really matter. All we care about is whether we can factor $p$ out of them, and this can be 
		determined solely by the constant we're adding/subtracting to/from $p^{k-2}$. This constant is given by the number of the layer the node is on, so all we really
		need to keep track of for each node is what layer it's on:
		\vspace*{-2cm}
		\begin{center}
			\begin{tabular}{c}
				\vspace{2.8cm} \\
				1 \\
				  \\
				2
			\end{tabular}
			\hspace*{0.5cm}
			\Tree
			[
				.$(p^{k-2})$
				[.$\newmoon$ [.$\newmoon$ ]] 
				[.$\newmoon$ [.$\newmoon$ ]]
			]
		\end{center}
		Whenever a node reaches a layer that's a multiple of $p$, we know that we'll be able to factor out the same number of factors of $p$ as we can from the layer number,
		up to a maximum of $k-2$ factors of $p$ (since each term takes the form $(p^{k-2} \pm i)$). Now, our question of finding the minimum number of factors of $p$ within
		a product becomes a question of minimising the number of factors of $p$ our nodes accumulate. Given a certain number of nodes, how can we arrange them on our tree so
		that the least number of factors of $p$ are crossed on the layer numbers?
		
		One little detail we need to address: can we treat the two sides of the tree as the same? The number of nodes we can have on either branch of the tree depends on how
		many terms we can have on either side of $(p^{k-2})$ in the product. For our purposes, the nodes on both branches act the same; all that matters is what layer they
		reach, so we can swap the branches no problem. However, the possible positions of $(p^{k-2})$ in the product can cause restrictions on what tree shapes are possible.
		
		For example, it we assume the only two possible products are
		\[
			(p^{k-2}-4)(p^{k-2}-3)(p^{k-2}-2)(p^{k-2}-1)(p^{k-2})
		\]
		and
		\[
			(p^{k-2}-3)(p^{k-2}-2)(p^{k-2}-1)(p^{k-2})(p^{k-2}+1)
		\]
		then we can only make four of the five possible trees containing five nodes:
		\vspace*{0cm}
		\begin{center}
			\begin{tabular}{*{4}{c}}
				\Tree
				[
					.$(p^{k-2})$
					[.$\newmoon$ [.$\newmoon$ [.$\newmoon$ [.$\newmoon$ ]]]]
					[ ]
				] &
				\Tree
				[
					.$(p^{k-2})$
					[.$\newmoon$ [.$\newmoon$ [.$\newmoon$ ]]]
					[.$\newmoon$ ]
				] &
				\Tree
				[
					.$(p^{k-2})$
					[.$\newmoon$ ]
					[.$\newmoon$ [.$\newmoon$ [.$\newmoon$ ]]]
				] &
				\Tree
				[
					.$(p^{k-2})$
					[ ]
					[.$\newmoon$ [.$\newmoon$ [.$\newmoon$ [.$\newmoon$ ]]]]
				]
			\end{tabular}
		\end{center}
		The first and second trees are obtained directly from the products, while the third and fourth are mirror images (which are allowed since both branches behave 
		identically). However, there's no way to get the fifth tree:
		\begin{center}
			\Tree
			[
				.$(p^{k-2})$
				[.$\newmoon$ [.$\newmoon$ ]]
				[.$\newmoon$ [.$\newmoon$ ]]
			]
		\end{center}
		Do we ever need to worry about restrictions like this? In cases where $\omega \geq 2$, it turns out that we don't. So long as the $(p^{k-2})$ term can get to the
		"center" of the product, the place where the difference in the number of terms on either side of $(p^{k-2})$ is no more than 1, then every tree structure is possible.
		
		Going back and looking at previous examples, we see that the value of $x$ dictates where the $(p^{k-2})$ term goes in the product. If the leftmost term in the product
		is position 1, then the $(p^{k-2})$ term will always be placed at position $x$. So, if the bounds of $x$ allow $(p^{k-2})$ to reach the "center" point of the product,
		then all tree structures are possible. By inspection, the "center" point will always be located at position $\ceil{\frac{a}{2}}$. For all $\omega \geq 2$, the 
		lower bound of $x$ is $\ceil{\frac{a}{\omega}} \leq \ceil{\frac{a}{2}}$ for $a \geq 1$, so $(p^{k-2})$ will certainly reach the "center" point in these cases
		(since $(p^{k-2})$ can go from position $a$ to a position further left than the "center" point). Therefore, so long as $\omega \geq 2$, all possible tree structures 
		can be formed.\footnote{We'll deal with the $\omega = 1$ case later in this document.}
		
		Now, to find the tree arrangement so that the fewest number of factors of $p$ can be factored from the layer numbers. I can't guarantee that my approach gives the
		optimal solution, but I have a hard time seeing how another approach could possibly yield fewer factors of $p$. My process was as follows:
		\begin{enumerate}
			\item{Every time we add a new node to the tree, check to see if we can place the node on a layer that doesn't have a factor of $p$ in its layer number. If so, 
			place the node there.}
			\item{If we must place the node on a layer whose layer number contains a factor of $p$, check to see on which branch the layer number would have the fewest number
			of factors of $p$, and place it there.}
		\end{enumerate}
		
		By following this algorithm, an expression for the fewest number of factors of $p$ possible in the layer numbers emerges:
		\[
			\prod_{i\,=\,1}^\infty p^{\floor{\frac{|a-p^i|}{p^i}}}
		\]
		Since each term in our original product takes the form $(p^{k-2} \pm i)$, this sum doesn't actually need to be an infinite sum; we'll only ever be able to factor out a
		maximum of $p^{k-2}$ from each term, so the sum we're interested in is given by:
		\[
			\prod_{i\,=\,1}^{k-2} p^{\floor{\frac{|a-p^i|}{p^i}}}
		\]
		
		By construction of the tree, if we multiply this sum by $(p^{k-2})$, it should give the fewest number of factors of $p$ possible in the product
		$\prod_{i\,=\,1}^a (p^{k-2}-x+i)$ for all possible values of $x$, which is what we wanted.
		
		Putting this all together, this means that the minimum number of factors of $p$ in $\binom{p^{k-2}-x+a}{a}$ for all possible values of $x$ between 
		$\ceil{\frac{a}{\omega}}$ and $a$ is given by
		\begin{align*}
			 & (p^{k-2})\frac{\prod_{i\,=\,1}^{k-2} p^{\floor{\frac{|a-p^i|}{p^i}}}}{\prod_{i\,=\,1}^\infty p^{\floor{\frac{a}{p^i}}}} \\
			=&\, p^{k-2 + \left(\sum_{i\,=\,1}^{k-2} \floor{\frac{|a-p^i|}{p^i}} \right) - \left(\sum_{i\,=\,1}^\infty \floor{\frac{a}{p^i}} \right)}
		\end{align*}
		
	\section{Understanding the Lifting Problem, Part 4}
		\label{sec:understanding4}
		Given $A \in \mathds{Z}_p^{L \times L}$ with $\ker(A) = \{\vec{0}\}$ and a cycle length modulo $p$ of $\omega > 1$, section \ref{sec:understanding3} has shown that, 
		the minimum factors of $p$ present in $\dag_a^{\omega p^{k-2}}$ for $1 \leq a < \omega p^{k-2}$ is given by
		\[
			p^{k-2 + \left(\sum_{i\,=\,1}^{k-2} \floor{\frac{|a-p^i|}{p^i}} \right) - \left(\sum_{i\,=\,1}^\infty \floor{\frac{a}{p^i}} \right)}
		\]
		Recall that the original expression we're interested in understanding is
		\[
			(A + p^nC)^{\omega p^{k-2}} \equiv \sum_{i\,=\,0}^{\omega p^{k-2}} p^{ni} \dag_i^{\omega p^{k-2}} \mod{p^k}
		\]
		If we can show that the number of factors of $p$ in $\dag_a^{\omega p^{k-2}}$ is always greater than or equal to $k-na$ for $n \geq 2$, then we can be certain that 
		the merging lift behaviour we observed in section \ref{sec:understanding1} will always occur when $n \geq 2$. This means the expression
		\[
			k-2 + \left(\sum_{i\,=\,1}^{k-2} \floor{\frac{|a-p^i|}{p^i}} \right) - \left(\sum_{i\,=\,1}^\infty \floor{\frac{a}{p^i}} \right)
		\]
		must be $\geq k-na$. Let's say
		\[
			S = k-2 + \left(\sum_{i\,=\,1}^{k-2} \floor{\frac{|a-p^i|}{p^i}} \right) - \left(\sum_{i\,=\,1}^\infty \floor{\frac{a}{p^i}} \right)
		\]
		Then
		\[
			S-k+2 = \left(\sum_{i\,=\,1}^{k-2} \floor{\frac{|a-p^i|}{p^i}} \right) - \left(\sum_{i\,=\,1}^\infty \floor{\frac{a}{p^i}} \right)
		\]
		By inspection of our sum for $(A+p^nC)^{\omega p^{k-2}}$, we see that when $a > k$, $p^{an} > p^k$, so $\dag_a^{\omega p^{k-2}}$ will be annihilated. This means we 
		only need to look at values of $a$ less than $k$. Using this, the right hand side is bounded from below by $-\log_p(a)$ when $0 < a < k$, which are the values of $a$
		we're interested in. So,
		\begin{align*}
			               \ & S-k+2 \geq -\log_p(a) \\
			\Leftrightarrow\ & S \geq k-\log_p(a)-2 \\
		\end{align*}
		When is the right hand side $\geq k-na$?
		\begin{align*}
			                 & k-\log_p(a)-2 \geq k-na \\
			\Leftrightarrow\ & na \geq \log_p(a)+2 \\
		\end{align*}
		So long as $n \geq 2$, this inequality is always true for $0 < a < k$. By transitivity, we can say
		\[
			S \geq k - na \quad\text{ for } n \geq 2
		\]
		Since $S$ is the minimum number of factors of $p$ in $\dag_a^{\omega p^{k-2}}$, we can be certain that, for $n \geq 2$ and $\omega \neq 1$, all 
		$p^{na}\dag_a^{\omega p^{k-2}}$ terms with $a > 0$ will be annihilated in the sum for $(A+p^nC)^{\omega p^{k-2}}$, thus proving the lift merging behaviour we observed 
		in section \ref{sec:understanding1}.
		
	\section{Edge Cases}
		Given $A \in \mathds{Z}_p^{L \times L}$ with $\ker(A) = \{\vec{0}\}$ and a cycle length modulo $p$ of $\omega > 1$, section \ref{sec:understanding3} has shown that, 
		the minimum factors of $p$ present in $\dag_a^{\omega p^{k-2}}$ for $1 \leq a < \omega p^{k-2}$ is given by
		\[
			p^{k-2 + \left(\sum_{i\,=\,1}^{k-2} \floor{\frac{|a-p^i|}{p^i}} \right) - \left(\sum_{i\,=\,1}^\infty \floor{\frac{a}{p^i}} \right)}
		\]
		What changes when $\omega = 1$? Recall that
		\[
			\binom{p^{k-2}-x+a}{a} = \frac{\prod_{i\,=\,1}^a (p^{k-2}-x+i)}{a!}
		\]
		The factors of $p$ in $\frac{1}{a!}$ remains the same:
		\[
			\frac{1}{\prod_{i\,=\,1}^\infty p^{\floor{\frac{a}{p^i}}}}
		\]
		However, the factors of $p$ present in
		\[
			\prod_{i\,=\,1}^a (p^{k-2}-x+i)
		\]
		becomes much easier to determine. When $\omega = 1$, we have that $\ceil{\frac{a}{\omega}} = \ceil{\frac{a}{1}} = a$, so $x$ can only be one possible value: $a$.
		By inspection, this means the factors of $p$ in this product are given by
		\[
			(p^{k-2})\prod_{i\,=\,1}^{k-2} p^{\floor{\frac{a-1}{p^i}}}
		\]
		Therefore, the total factors of $p$ (not just the minimum as was the case when $\omega \neq 1$) in $\binom{p^{k-2}-x+a}{a}$ is given by
		\begin{align*}
			      &\, (p^{k-2})\frac{\prod_{i\,=\,1}^{k-2} p^{\floor{\frac{a-1}{p^i}}}}{\prod_{i\,=\,1}^\infty p^{\floor{\frac{a}{p^i}}}} \\
			\equiv&\, p^{k-2+\left(\sum_{i\,=\,1}^{k-2} \floor{\frac{a-1}{p^i}}\right) - \left(\sum_{i\,=\,1}^\infty \floor{\frac{a}{p^i}}\right)}
		\end{align*}
		Once again, if we can show that
		\[
			k-2+\left(\sum_{i\,=\,1}^{k-2} \floor{\frac{a-1}{p^i}}\right) - \left(\sum_{i\,=\,1}^\infty \floor{\frac{a}{p^i}}\right)
		\]
		is $\geq k - na$, then the lift merging behaviour from section \ref{sec:understanding1} will also to apply to when $\omega = 1$. Let
		\[
			S = k-2+\left(\sum_{i\,=\,1}^{k-2} \floor{\frac{a-1}{p^i}}\right) - \left(\sum_{i\,=\,1}^\infty \floor{\frac{a}{p^i}}\right)
		\]
		Then
		\[
			S-k+2 = \left(\sum_{i\,=\,1}^{k-2} \floor{\frac{a-1}{p^i}}\right) - \left(\sum_{i\,=\,1}^\infty \floor{\frac{a}{p^i}}\right)
		\]
		For $0 < a < k$ (still the values of $a$ of interest), we again see that the right hand side is bounded from below by $-\log_p(a)$, so
		\[
			S-k+2 \geq -\log_p(a)
		\]
		The same logic as in section \ref{sec:understanding4} follows. Therefore, when $\omega = 1$, the lift merging behaviour observed in section \ref{sec:understanding1}
		still occurs for $n \geq 2$.
		
		One other small edge case to consider is when $k=2$. In this case, the lift merging behaviour is trivial since, when $n\geq2$, we have that
		\[
			(A+p^nC)^{\omega p^{k-2}} \equiv A^\omega
		\]
		and so all lifts of this form must necessarily give the same resulting matrix when $n \geq 2$. This edge case is listed here since some of the calculations performed
		in this document implicitly assume that $k > 2$, which is alright since the $k=2$ case trivially produces the lift merging behaviour.
		
		\begin{comment}
			%I'm so sorry
			\vspace*{-5cm}
			\begin{center}
				\begin{tabular}{c}
					\vspace{6.4cm} \\
					1 \\
					  \\
					2 \\
					  \\
					3 \\
					  \\
					4 \\
					  \\
					5
				\end{tabular}
				\hspace*{0.5cm}
				\Tree
				[.$p^{k-2}$
					[.$\newmoon$ [.$\newmoon$ [.$\newmoon$ [.$\newmoon$ [.$\newmoon$ ]]]]] 
					[.$\newmoon$ [.$\newmoon$ [.$\newmoon$ [.$\newmoon$ [.$\newmoon$ ]]]]]
				]
			\end{center}
		\end{comment}
\end{document}
