
\documentclass[a4paper, 12pt, reqno]{amsart}

\usepackage[T1]{fontenc}

\usepackage[margin=3cm]{geometry}

\usepackage{setspace}
\setstretch{1.25}

\usepackage[parfill]{parskip}

\usepackage{dsfont}
\usepackage{marvosym} %Dumb astrological signs

%Defining an environment for propositions
\newcounter{propcounter}
\newenvironment{proposition}[1]
{
	\vspace{1em}
	\refstepcounter{propcounter} %Increment environment counter
	\textbf{Proposition \thepropcounter.} \emph{#1}
	
	\emph{Justification.}
}
{
	QED. \\
}

\begin{document}
	%All fields are principal ideal domains
	\section{Annihilating Polynomials Modulo Some Prime}
		Say we're given some matrix $A \in \mathds{Z}_{p}^{L \times L}$, where $\mathds{Z}_{p}^{L \times L}$ represents the set of all $L \times L$ matrices with entries
		from the integers modulo $p$. For the rest of this document, assume $p$ is some prime. A polynomial $r(x)$ is an \emph{annihilating polynomial of $A$} if
		\[
			r(A) \equiv 0
		\]
		At least one such annihilating polynomial must exist for $A$ modulo $p$ by the Cayley-Hamilton Theorem: the characteristic polynomial. The set of all annihilating
		polynomials for $A$ forms an ideal since any polynomial with an annihilating polynomial as a factor is also an annihilating polynomial. Since the ring of
		polynomials with coefficients from the integers modulo $p$ (denoted as $\mathds{Z}_{p}[X]$) is a principal ideal domain, the ideal of annihilating polynomials
		modulo $p$ is a principal ideal---i.e. generated using only a single element. The monic generating element of this ideal is called the \emph{minimal polynomial of
		$A$}.
		
		The minimal polynomial of $A$ is of interest when analysing the dynamics of vectors under repeated multiplication by $A$; it encodes the simplest algebraic
		property $A$ must follow. However, when moving up to modulo $p^2$ and beyond, the annihilating polynomials---now in the ring $\mathds{Z}_{p^k}[X]$---no longer 
		necessarily form a principal ideal, so there isn't necessarily a single minimal polynomial that characterises the algebraic behaviour of $A$. However, under 
		specific circumstances, the set of annihilating polynomials modulo $p^k$ may still be a principal ideal.
		
	\section{Annihilating Polynomials Modulo Some Prime Squared}
		Before we look at when annihilating polynomials form principal ideals modulo $p^k$, it's helpful to understand the concept of a \emph{lift}. Generally, a \emph{lift}
		$W \bmod{p^{k+m}}$ of some object $V \bmod{p^k}$ is an object that reduces down to $V$:
		\[
			W \equiv V \mod{p^k}
		\]
		where $p^{k+m} > p^{k}$.
		
		For example, the polynomial $6x^{2} + 15x + 4 \bmod{25}$ is a lift of the polynomial \\ $x^{2} + 4 \bmod{5}$ since
		\[
			6x^{2} + 15x + 4 \equiv x^{2} + 4 \mod{5}
		\]
		Numerous other objects can have lifts, such as matrices, vectors, and scalars.
		
		Another important concept is the concept of \emph{embeds}. Generally, an \emph{embed} $W \bmod{p^{k+m}}$ of an object $V \bmod{p^k}$ is an object that's related to
		$V$ in the following way:
		\[
			W \equiv p^mV \mod{p^{k+m}}
		\]
		$W$ is the embed, while $V$ is the object that's been embedded.
		Embeds are useful as their behaviour mimics the object they've embedded, even though the modulus governing them is different. For instance, with the ring of
		polynomials $\mathds{Z}_{p^k}[X]$, we can define a mapping $\phi_{n} : \mathds{Z}_{p^k}[X] \rightarrow p^n\mathds{Z}_{p^{k+n}}[X]$, $\phi_{n}(X) = p^nX$. This
		mapping creates a group homomorphism under addition between $\mathds{Z}_{p^k}[X]$ and $p^n\mathds{Z}_{p^{k+n}}[X]$ since, for $Y, Z \in \mathds{Z}_{p^k}[X]$,
		\[
			\phi_n(Y + Z) = p^n(Y + Z) = p^nY + p^nZ = \phi_n(Y) + \phi_n(Z)
		\]
		This mapping is also bijective (and therefore a group isomorphism) since 
		\begin{center}
			\begin{tabular}{*3{l}}
						   & $\phi_n(Y) \equiv \phi_n(Z)$ & $\mod{p^{k+n}}$ \\
				$\implies$ & $p^nY \equiv p^nZ$           & $\mod{p^{k+n}}$ \\
				$\implies$ & $p^nY - p^nZ \equiv 0$       & $\mod{p^{k+n}}$ \\
				$\implies$ & $p^n(Y - Z) \equiv 0$        & $\mod{p^{k+n}}$ \\
				$\implies$ & $Y - Z \equiv 0$             & $\mod{p^k}$     \\
				$\implies$ & $Y \equiv Z$                 & $\mod{p^k}$
			\end{tabular}
		\end{center}
		so $\phi_n$ is injective, and for every $p^nY \in \mathds{Z}_{p^{k+n}}[X]$ there's a $Y \in \mathds{Z}_{p^k}[X]$ such that $\phi_n(Y) = p^nY$, so $\phi_n$ is 
		surjective.
		
		This shows that the set of all embed polynomials in $\mathds{Z}_{p^{k+n}}[X]$ which are embeds of the polynomials in $\mathds{Z}_{p^k}[X]$ is isomorphic to all 
		polynomials in $\mathds{Z}_{p^k}[X]$.
		
		Using lifts and embeds, we can link concepts modulo $p$ with those same concepts modulo $p^k$. The following proposition makes use of both lifts 
		and embeds.
		
		\begin{proposition}{Given $A \in \mathds{Z}_{p}^{L \times L}$, no nonzero polynomial with lesser degree than the minimal polynomial of $A$ modulo $p$ can annihilate 
		$A$ modulo $p^2$.}
			\label{prop:noLowerZeros}
			Let m$(x)$ be the minimal polynomial of $A$ modulo $p$. Assume that a nonzero polynomial r$(x)$ exists with lower degree than m$(x)$ such that
			\[
				\text{r}(A) \equiv 0 \mod{p^2}
			\]
			Either r$(x)$ reduces to 0 modulo $p$, or it doesn't. If it doesn't reduce to 0, then r$(x)$ must be the lift of some nonzero polynomial modulo $p$. However,
			since r$(x)$ annihilates $A$ modulo $p^2$, it would also annihilate $A$ modulo $p$. Such an r$(x)$ can't possibly exist since r$(x)$ modulo $p$ would have a 
			lower degree than m$(x)$, and the degree of m$(x)$ is defined to be the lowest degree a nonzero annihilating polynomial can have modulo $p$.
			
			This means r$(x)$ reduces to 0 modulo $p$. Then r$(x)$ must be of the form $p$t$(x)$ for some polynomial t$(x)$ modulo $p$. Note that the degree of t$(x)$ 
			modulo $p$ is necessarily the same as the degree of r$(x)$. Since r$(x) = p\text{t}(x)$, r$(x)$ is an embed of some nonzero polynomial modulo $p$, so
			\[
				\text{r}(x) = \phi_1(\text{t}(x))
			\]
			Because the embed mapping $\phi_1$ is an isomorphism, and since r$(x)$ is an annihilating polynomial, t$(x)$ must also be an annihilating polynomial modulo $p$:
			\begin{center}
				\begin{tabular}{lcl}
					           & r$(A) \equiv 0$                   & $\mod{p^2}$ \\
					$\implies$ & $\phi_1($t$(A)) \equiv \phi_1(0)$ & $\mod{p^2}$ \\
					$\implies$ & t$(A) \equiv 0$                   & $\mod{p}$
				\end{tabular}
			\end{center}
			Again, this causes a contradiction since there can be no nonzero annihilating polynomial modulo $p$ with lesser degree than m$(x)$.
			
			Therefore, the only polynomial in $\mathds{Z}_{p^2}[X]$ with lesser degree than m$(x)$ which annihilates $A$ modulo $p^2$ is the zero polynomial.
		\end{proposition}
		
		Using some properties of polynomials, we can also show that, when a monic annihilating polynomial q$(x)$ of a matrix $A$ modulo $p^2$ is the same degree as the 
		minimal polynomial of $A$ modulo $p$, then no polynomial with degree greater than or equal to the minimal polynomial's degree can annihilate $A$ modulo $p^2$ 
		without being a multiple of q$(x)$. First, though, we need to define a particular equivalence relation.
		
		Assume q$(x)$ is a monic annihilating polynomial of some $A \in \mathds{Z}_{p}^{L \times L}$ modulo $p^2$ with the same degree as the minimal polynomial of $A$ 
		modulo $p$. Let 
		\begin{align*}
			\text{a}(x) &= \text{s}_\text{a}(x)\text{q}(x) + \text{r}_\text{a}(x) \\
			\text{b}(x) &= \text{s}_\text{b}(x)\text{q}(x) + \text{r}_\text{b}(x) \\
			\text{c}(x) &= \text{s}_\text{c}(x)\text{q}(x) + \text{r}_\text{c}(x)
		\end{align*}
		for some 
		s$_\text{a}(x), \text{s}_\text{b}(x), \text{s}_\text{c}(x), \text{r}_\text{a}(x), \text{r}_\text{b}(x), \text{r}_\text{c}(x) \in \mathds{Z}_{p^2}[X]$ where
		r$_\text{a}(x)$, r$_\text{b}(x)$, and r$_\text{c}(x)$ are all of lesser degree than q$(x)$. Note that, because of the restrictions on r$_\text{a}(x)$, 
		r$_\text{b}(x)$, and r$_\text{c}(x)$, and because q$(x)$ is monic, the division algorithm for polynomials ensures the representation for any polynomials 
		a$(x)$, b$(x)$, and c$(x)$ in this form is unique.
		
		We'll say a$(x)$ and b$(x)$ are related by the relation \Taurus\ if r$_\text{a}(x) \equiv \text{r}_\text{b}(x) \bmod{p^2}$. Denote a$(x)$ and b$(x)$ being related 
		by a$(x)$ \Taurus\ b$(x)$; a$(x)$ and b$(x)$ are \Taurus-equivalent.
		
		Note that if a polynomial is \Taurus-equivalent to 0, then that polynomial is a polynomial multiple of q$(x)$.
		
		The \Taurus\ relation is indeed an equivalence relation. It is reflexive since for any a$(x)$ as defined above, r$_\text{a}(x) \equiv \text{r}_\text{a}(x)$, so
		a$(x)$ \Taurus\ a$(x)$. It is transitive since if a$(x)$ \Taurus\ b$(x)$ and b$(x)$ \Taurus\ c$(x)$, then $\text{r}_\text{a}(x) \equiv \text{r}_\text{b}(x)$ 
		and $\text{r}_\text{b}(x) \equiv \text{r}_\text{c}(x)$, so $\text{r}_\text{a}(x) \equiv \text{r}_\text{c}(x)$ which means a$(x)$ \Taurus\ c$(x)$. It is also
		symmetric since if a$(x)$ \Taurus\ b$(x)$, then $\text{r}_\text{a}(x) \equiv \text{r}_\text{b}(x) \implies \text{r}_\text{b}(x) \equiv \text{r}_\text{a}(x)$ and
		so b$(x)$ \Taurus\, a$(x)$. This is enough to show that \Taurus\ is an equivalence relation.
		
		The following proposition will further establish \Taurus\ as a form of equivalence between polynomials.
		
		\begin{proposition}{If a$(x)$ \Taurus\ b$(x)$, then a$(x) + \text{c}(x)$ \Taurus\ b$(x) + \text{c}(x)$ and \\ a$(x)\text{c}(x)$ \Taurus\ b$(x)\text{c}(x)$.}
			\label{prop:taurusPreservation}
			Let
			\begin{align*}
				\text{a}(x) &= \text{s}_\text{a}(x)\text{q}(x) + \text{r}_1(x) \\
				\text{b}(x) &= \text{s}_\text{b}(x)\text{q}(x) + \text{r}_1(x) \\
				\text{c}(x) &= \text{s}_\text{c}(x)\text{q}(x) + \text{r}_2(x)
			\end{align*}
			Then
			\begin{align*}
				\text{a}(x) + \text{c}(x) &= \bigl(\text{s}_\text{a}(x) + \text{s}_\text{c}(x)\bigr)\text{q}(x) + \bigl(\text{r}_1(x) + \text{r}_2(x)\bigr) \\
				\text{b}(x) + \text{c}(x) &= \bigl(\text{s}_\text{b}(x) + \text{s}_\text{c}(x)\bigr)\text{q}(x) + \bigl(\text{r}_1(x) + \text{r}_2(x)\bigr) \\
			\end{align*}
			By inspection, a$(x) + \text{c}(x)$ \Taurus\ b$(x) + \text{c}(x)$. As well,
			\begin{align*}
				\text{a}(x)\text{c}(x) &= \text{s}_\text{a}(x)\text{s}_\text{c}(x)\text{q}^2(x) + 
				                          \text{s}_\text{a}(x)\text{r}_2(x)\text{q}(x) +
										  \text{s}_\text{c}(x)\text{r}_1(x)\text{q}(x) +
										  \text{r}_1(x)\text{r}_2(x) \\
									   &= \bigl(\text{s}_\text{a}(x)\text{s}_\text{c}(x)\text{q}(x) +
									       \text{s}_\text{a}(x)\text{r}_2(x) +
										   \text{s}_\text{c}(x)\text{r}_1(x)\bigr)\text{q}(x) +
										   \text{r}_1(x)\text{r}_2(x)
			\end{align*}
			and
			\begin{align*}
				\text{b}(x)\text{c}(x) &= \text{s}_\text{b}(x)\text{s}_\text{c}(x)\text{q}^2(x) + 
				                          \text{s}_\text{b}(x)\text{r}_2(x)\text{q}(x) +
										  \text{s}_\text{c}(x)\text{r}_1(x)\text{q}(x) +
										  \text{r}_1(x)\text{r}_2(x) \\
									   &= \bigl(\text{s}_\text{b}(x)\text{s}_\text{c}(x)\text{q}(x) +
									       \text{s}_\text{b}(x)\text{r}_2(x) +
										   \text{s}_\text{c}(x)\text{r}_1(x)\bigr)\text{q}(x) +
										   \text{r}_1(x)\text{r}_2(x)
			\end{align*}
			By definition of the \Taurus\ relation, a$(x)\text{c}(x)$ \Taurus\ r$_1(x)\text{r}_2(x)$ and b$(x)\text{c}(x)$ \Taurus\ r$_1(x)\text{r}_2(x)$. The \Taurus\ 
			relation is an equivalence relation, so a$(x)\text{c}(x)$ \Taurus\ b$(x)\text{c}(x)$.
		\end{proposition}
		
		Proposition \ref{prop:taurusPreservation} allows us to create some useful relations using the polynomial q$(x)$. Let q$(x)$ be written as
		\[
			\text{q}(x) = x^n + \alpha_{n-1}x^{n-1} + \alpha_{n-2}x^{n-2} + \cdots + \alpha_1x + \alpha_0
		\]
		
		By definition of \Taurus-equivalent polynomials, q$(x)$ \Taurus\ 0. Then using proposition \ref{prop:taurusPreservation},
		\begin{align}
			         & x^n + \alpha_{n-1}x^{n-1} + \alpha_{n-2}x^{n-2} + \cdots + \alpha_1x + \alpha_0 \ \text{\Taurus}\ 0 \nonumber \\
			\implies & x^n \ \text{\Taurus}\ -(\alpha_{n-1}x^{n-1} + \alpha_{n-2}x^{n-2} + \cdots + \alpha_1x + \alpha_0)
			\label{eq:taurusSub}
		\end{align}
		
		Using q$(x)$, we can obtain a \Taurus-equivalent polynomial for $x^n$. As well, we can find a \Taurus-equivalent polynomial for any higher power of $x$ simply by 
		multiplying both sides of equation \ref{eq:taurusSub} by some power of $x$.
		
		With equation \ref{eq:taurusSub}, we can reduce \emph{any} polynomial in $\mathds{Z}_{p^2}[X]$ to a \Taurus-equivalent polynomial with degree less than the 
		degree of q$(x)$ by substituting any high-order terms in the polynomial with their \Taurus-equivalent expression guaranteed by equation \ref{eq:taurusSub}. This 
		\Taurus-equivalent polynomial preserves the original polynomial's remainder when divided by q$(x)$ (by definition of \Taurus-equivalent polynomials). As well, 
		reducing a polynomial using equation \ref{eq:taurusSub} preserves the value of that polynomial when plugging in the relevant matrix $A$ (due to q$(x)$ annihilating 
		$A$).
		
		Now, with the \Taurus\ relation specified, we can now show that, when a monic annihilating polynomial q$(x)$ exists modulo $p^2$ for some matrix $A$ with the same 
		degree as the matrix's minimal polynomial modulo $p$, then all annihilating polynomials modulo $p^2$ are polynomial multiples of q$(x)$.

		\begin{proposition}{If a monic annihilating polynomial q$(x)$ of $A \in \mathds{Z}_{p}^{L \times L}$ modulo $p^2$ has the same degree as the minimal polynomial of $A$
		modulo $p$, then no polynomial with a degree greater than or equal to the minimal polynomial can annihilate $A$ modulo $p^2$ without being a multiple of q$(x)$.}
			\label{prop:noHigherZeros}
			Assume we have another annihilating polynomial r$(x)$ modulo $p^2$ with degree greater than or equal to the degree of q$(x)$. Using equation \ref{eq:taurusSub}, 
			we can reduce r$(x)$ down to a \Taurus-equivalent polynomial t$(x)$ which is guaranteed to have a degree less than q$(x)$. Because t$(x)$ is
			\Taurus-equivalent to r$(x)$, t$(x)$ is also an annihilating polynomial for $A$ modulo $p^2$. By proposition \ref{prop:noLowerZeros}, 
			t$(x) \equiv 0 \bmod{p^2}$. By definition of \Taurus-equivalent polynomials, r$(x)$ must be a multiple of q$(x)$.
		\end{proposition}
		
		Because of propositions \ref{prop:noLowerZeros} and \ref{prop:noHigherZeros}, given a matrix $A \in \mathds{Z}_{p}^{L \times L}$, whenever a monic annihilating 
		polynomial q$(x)$ modulo $p^2$ has the same degree as the minimal polynomial of $A$ modulo $p$, then the set of all annihilating polynomials modulo $p^2$ must form
		a principal ideal with q$(x)$ being the generating element. It's fitting to label q$(x)$ the \emph{minimal polynomial of $A$ modulo $p^2$}.
		
	\section{Annihilating Polynomials Modulo Some Prime Power}
		
		If it's established that the set of annihilating polynomials modulo $p^2$ is a principal ideal, then the set of annihilating polynomials modulo $p^k$, with
		$p^k > p^2$, can also be established as principal ideals using an induction-type argument.
		
		\begin{proposition}{Given $A \in \mathds{Z}_{p}^{L \times L}$, if it's known that the set of annihilating polynomials modulo $p^k$ forms a principal ideal, then no 
		nonzero polynomial with lesser degree than the minimal polynomial of $A$ modulo $p^k$ can annihilate $A$ modulo $p^{k+1}$.}
			\label{prop:pkNoLowerZeros}
			Let m$(x)$ be the minimal polynomial of $A$ modulo $p^k$. Assume that a nonzero polynomial r$(x)$ exists with lower degree than m$(x)$ such that
			\[
				\text{r}(A) \equiv 0 \mod{p^{k+1}}
			\]
			Either r$(x)$ reduces to 0 modulo $p^k$, or it doesn't. If it doesn't reduce to 0, then r$(x)$ must be the lift of some nonzero polynomial modulo $p^k$. However,
			since r$(x)$ annihilates $A$ modulo $p^{k+1}$, it would also annihilate $A$ modulo $p^k$. Such an r$(x)$ can't possibly exist since r$(x)$ modulo $p^k$ would 
			have a lower degree than m$(x)$, and the degree of m$(x)$ is defined to be the lowest degree a nonzero annihilating polynomial can have modulo $p^k$.
			
			This means r$(x)$ reduces to 0 modulo $p^k$. Then r$(x)$ must be of the form $p^k$t$(x)$ for some polynomial t$(x)$ modulo $p$. Note that the degree of t$(x)$ 
			modulo $p$ is necessarily the same as the degree of r$(x)$. Since r$(x) = p^k\text{t}(x)$, r$(x)$ is an embed of some nonzero polynomial modulo $p$, so
			\[
				\text{r}(x) = \phi_k(\text{t}(x))
			\]
			Because the embed mapping $\phi_k$ is an isomorphism, and since r$(x)$ is an annihilating polynomial, t$(x)$ must also be an annihilating polynomial modulo $p$:
			\begin{center}
				\begin{tabular}{lcl}
					           & r$(A) \equiv 0$                   & $\mod{p^{k+1}}$ \\
					$\implies$ & $\phi_k($t$(A)) \equiv \phi_k(0)$ & $\mod{p^{k+1}}$ \\
					$\implies$ & t$(A) \equiv 0$                   & $\mod{p}$
				\end{tabular}
			\end{center}
			This causes a contradiction since 
			\[
				\text{t}(A) \equiv 0 \bmod{p} \implies \phi_{k-1}(\text{t}(A)) \equiv 0 \bmod{p^k}
			\]
			and there can be no nonzero annihilating polynomial modulo $p^k$ with lesser degree than m$(x)$.
			
			Therefore, the only polynomial in $\mathds{Z}_{p^{k+1}}[X]$ with lesser degree than m$(x)$ which annihilates $A$ modulo $p^{k+1}$ is the zero polynomial.
		\end{proposition}
		
		The \Taurus\ relation stays largely unchanged when considering higher moduli. Assume q$(x)$ is a monic annihilating polynomial of some 
		$A \in \mathds{Z}_{p}^{L \times L}$ modulo $p^{k+1}$ with the same degree as the minimal polynomial of $A$ modulo $p^k$. Let 
		\begin{align*}
			\text{a}(x) &= \text{s}_\text{a}(x)\text{q}(x) + \text{r}_\text{a}(x) \\
			\text{b}(x) &= \text{s}_\text{b}(x)\text{q}(x) + \text{r}_\text{b}(x) \\
			\text{c}(x) &= \text{s}_\text{c}(x)\text{q}(x) + \text{r}_\text{c}(x)
		\end{align*}
		for some 
		s$_\text{a}(x), \text{s}_\text{b}(x), \text{s}_\text{c}(x), \text{r}_\text{a}(x), \text{r}_\text{b}(x), \text{r}_\text{c}(x) \in \mathds{Z}_{p^{k+1}}[X]$ where
		r$_\text{a}(x)$, r$_\text{b}(x)$, and r$_\text{c}(x)$ are all of lesser degree than q$(x)$. 
		
		We'll say a$(x)$ and b$(x)$ are related by the relation \Taurus\ if r$_\text{a}(x) \equiv \text{r}_\text{b}(x) \bmod{p^{k+1}}$. \Taurus\ still forms an equivalence
		relation in the same way, and proposition \ref{prop:taurusPreservation} still applies. As well, equation \ref{eq:taurusSub} can still be used to reduce any
		polynomial in $\mathds{Z}_{p^{k+1}}[X]$ to a \Taurus-equivalent polynomial with degree less than q$(x)$. With this, we can restate proposition 
		\ref{prop:noHigherZeros} in terms of this more general case.
		
		\begin{proposition}{If a monic annihilating polynomial q$(x)$ of $A \in \mathds{Z}_{p}^{L \times L}$ modulo $p^{k+1}$ has the same degree as the minimal polynomial 
		of $A$ modulo $p^k$, and the set of annihilating polynomials for $A$ modulo $p^k$ is a principal ideal, then no polynomial with a degree greater than or equal to 
		the minimal polynomial can annihilate $A$ modulo $p^{k+1}$ without being a multiple of q$(x)$.}
			\label{prop:pkNoHigherZeros}
			Assume we have another annihilating polynomial r$(x)$ modulo $p^{k+1}$ with degree greater than or equal to the degree of q$(x)$. Using equation 
			\ref{eq:taurusSub}, we can reduce r$(x)$ down to a \Taurus-equivalent polynomial t$(x)$ which is guaranteed to have a degree less than q$(x)$. Because t$(x)$ is
			\Taurus-equivalent to r$(x)$, t$(x)$ is also an annihilating polynomial for $A$ modulo $p^{k+1}$. By proposition \ref{prop:pkNoLowerZeros}, 
			t$(x) \equiv 0 \bmod{p^{k+1}}$. By definition of \Taurus-equivalent polynomials, r$(x)$ must be a multiple of q$(x)$.
		\end{proposition}
		
		Therefore, using propositions \ref{prop:pkNoLowerZeros} and \ref{prop:pkNoHigherZeros}, whenever a monic annihilating polynomial q$(x)$ modulo $p^{k+1}$ of some 
		matrix $A$ has the same degree as the minimal polynomial of $A$ modulo $p^k$, then the set of annihilating polynomials modulo $p^{k+1}$ forms a principal ideal
		with q$(x)$ as its generator.
		
		% An easy example to apply this idea to is the set of all matrices in $\mathds{Z}_{p}^{L \times L}$ whose minimal polynomials are equal to their characteristic 
		% polynomials. The characteristic polynomial of a matrix can always be expressed as a monic polynomial. As well, the degree of the characteristic polynomial for these 
		% matrices doesn't change modulo $p^k$ for any value of $k$. This means they'll always be some monic annihilating polynomial with the same degree as the lower
		% moduli: the characteristic polynomial. In these cases, the set of annihilating polynomials should always form a principal ideal, no matter the modulus.
\end{document}
