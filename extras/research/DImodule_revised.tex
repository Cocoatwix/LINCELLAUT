
\documentclass[a4paper, 12pt, reqno]{amsart}
\usepackage[T1]{fontenc}

\usepackage[parfill]{parskip}

\usepackage[margin=3cm]{geometry}

\usepackage{setspace}
\setstretch{1.25}

\usepackage{dsfont}

\usepackage{verbatim}

%Max
\newcommand\mathmax[1]{\textup{max}(#1)}

%Span
\newcommand\vecspan[1]{\textup{span}(#1)}

%Cores
\newcommand\core[3]{\textup{core}_{#1}^{#2}(#3)}

%Defining environment for propositions
\newcounter{propcounter}[section]
\newenvironment{proposition}[1]
{
	\refstepcounter{propcounter}
	\textbf{Proposition \thesection.\thepropcounter.} \emph{#1}
	
	\emph{Justification.}
}
{
	\textbf{QED.} \\
}

\begin{document}
	For this document, let $\mathds{Z}_n$ represent the integers modulo $n$. Let $\mathds{Z}_{n}^{M \times N}$ represent the set of all $M$ by $N$ matrices with elements 
	from $\mathds{Z}_n$. Let $\mathds{Z}_{n}^{M}$ represent the set of all $M$ by $1$ column vectors with elements from $\mathds{Z}_n$. Note that $\mathds{Z}_{n}^M$ forms a 
	$\mathds{Z}_{n}$-module.
	
	All relevant calculations are done modulo $p^k$ unless otherwise stated.
		
	\section{Dimensional Independence}
		\label{sec:DI}
		In $\mathds{Z}_{p^k}^L$, where $p$ is prime and $k \in \mathds{Z}^+$, a set of vectors $V$ is \emph{dimensionally independent} iff
		\[
			\forall \vec{v} \in V, \quad \vecspan{\vec{v}} \cap \vecspan{V \setminus \{\vec{v}\}} = \{\vec{0}\}
		\]
		and
		\[
			\forall \vec{v} \in V, \quad \vec{v} \not\equiv \vec{0}
		\]
		
		As an example for why this definition may be useful, consider the vectors 
		$
		\begin{bmatrix}
			\begin{smallmatrix}
				5 \\
				0 \\
				0
			\end{smallmatrix}
		\end{bmatrix}
		$, $
		\begin{bmatrix}
			\begin{smallmatrix}
				0 \\
				5 \\
				0
			\end{smallmatrix}
		\end{bmatrix}
		$, and $
		\begin{bmatrix}
			\begin{smallmatrix}
				0 \\
				0 \\
				5
			\end{smallmatrix}
		\end{bmatrix}
		$ modulo 25. These vectors aren't linearly independent since
		\[
			5\begin{bmatrix}
				5 \\
				0 \\
				0
			\end{bmatrix} + 
			5\begin{bmatrix}
				0 \\
				5 \\
				0
			\end{bmatrix} + 
			5\begin{bmatrix}
				0 \\
				0 \\
				5
			\end{bmatrix} \equiv \vec{0} \mod{25}
		\]
		However, in some sense, these vectors are still independent since one cannot be made from a linear combination of the others. Other than the zero vector, their
		spans are disjoint, similar to the spans of linearly independent vectors. Although these vectors aren't linearly independent, there may still be certain properties
		of linear independence that apply to these vectors. Dimensional independence aims to capture these properties.
		
		Before we start working with dimensionally-independent vectors, it's helpful to understand what it means for a specific vector to \emph{not} be dimensionally
		independent to a set of vectors. Given some vector $\vec{u}$ and a set $V = \{\vec{v}_1, \vec{v}_2, \cdots, \vec{v}_g\}$ that's dimensionally independent, if 
		$\vec{u}$ isn't dimensionally-independent to $V$, then there must be some nonzero vector in $\vecspan{\vec{u}}$ that's also in $\vecspan{V}$. Symbolically, this
		means
		\[
			\vecspan{\vec{u}} \cap \vecspan{V} \neq \{\vec{0}\}
		\]
		So, there exists a smallest $p^c$, $c \in \mathds{Z}^+$, such that
		\[
			p^c\vec{u} \in \vecspan{V}
		\]
		and
		\[
			p^c\vec{u} \not\equiv \vec{0}
		\]
		If, for some invertible element $a \in \mathds{Z}_{p^k}$, we had
		\[
			ap^c\vec{u} \in \vecspan{V}
		\]
		then $a^{-1}ap^c\vec{u} \equiv p^c\vec{u}$ would also be in $\vecspan{V}$, so we only need to consider prime-power multiples of $\vec{u}$. In fact, in any case 
		where we need to check whether a vector's span intersects with some other set, it's sufficient to check only prime-power multiples of that vector for this very 
		reason. 
		
		Since $p^c\vec{u} \in \vecspan{V}$, we can express $p^c\vec{u}$ as a linear combination of vectors in $V$:
		\[
			p^c\vec{u} \equiv a_1\vec{v}_1 + a_2\vec{v}_2 + \cdots + a_g\vec{v}_g
		\]
		So, for every such vector $\vec{u}$, there's some minimal prime-power multiple that can be written as a linear combination of the vectors in $V$. All higher 
		prime-power multiples will also be able to be written as a linear combination of vectors in $V$ (simply by multiplying both sides of the above equation by the
		appropriate power of $p$). This fact will be used repeatedly throughout the rest of the document.
		
		The following propositions will help solidify the connection between linear independence and dimensional independence.
		
		\begin{proposition}{If $V = \{\vec{v}_1, \vec{v}_2, \cdots, \vec{v}_g\}$ is a dimensionally-independent set, then 
		$\{a_1\vec{v}_1, a_2\vec{v}_2, \cdots, a_g\vec{v}_g\}$ is also a dimensionally-independent set for nonzero vectors $a_1\vec{v}_1$ to $a_g\vec{v}_g$.}
			\label{prop:DImultset}
			For any integer $a_i$, $\vecspan{a_i\vec{v}_i} \subseteq \vecspan{\vec{v}_i}$, so
			\[
				\vecspan{\vec{v}_i} \cap \vecspan{V \setminus \{\vec{v}_i\}} = \{\vec{0}\} \implies
				\vecspan{a_i\vec{v}_i} \cap \vecspan{V \setminus \{\vec{v}_i\}} = \{\vec{0}\}
			\]
			As well, for any $\vec{v}_j \in V \setminus \{\vec{v}_i\}$,
			\begin{align*}
				& \vecspan{v_j} \cap \vecspan{\{\vec{v}_1, \vec{v}_2, \cdots, \vec{v}_i, \cdots, \vec{v}_g\} \setminus \{\vec{v}_j\}} = \{\vec{0}\} \\
				\implies & 
				\vecspan{v_j} \cap \vecspan{\{\vec{v}_1, \vec{v}_2, \cdots, a_i\vec{v}_i, \cdots, \vec{v}_g\} \setminus \{\vec{v}_j\}} = \{\vec{0}\}
			\end{align*}
			This means swapping any $\vec{v}_i$ in $V$ with $a_i\vec{v}_i$ (so long as $a_i\vec{v}_i \not\equiv \vec{0}$) would keep $V$ dimensionally independent.
		\end{proposition}
		
		Proposition \ref{sec:DI}.\ref{prop:DImultset} essentially says that, if a set of vectors is dimensionally independent, then any multiples of those vectors will also 
		be dimensionally independent, so long as none of those multiples are the zero vector.
		\\ \\ \\ \\
		\begin{proposition}{If $V = \{\vec{v}_1, \vec{v}_2, \cdots, \vec{v}_g\}$ is dimensionally independent, then the equation
		\[
			a_1\vec{v}_1 + a_2\vec{v}_2 + \cdots + a_g\vec{v}_g \equiv \vec{0}
		\]
		has only the solution where each $a_i\vec{v}_i \equiv \vec{0}$.}
			\label{prop:DIzerosum}
			Assume that a second solution to
			\[
				a_1\vec{v}_1 + a_2\vec{v}_2 + \cdots + a_g\vec{v}_g \equiv \vec{0}
			\]
			exists where at least one term is nonzero. Rearranging, we see
			\[
				-a_i\vec{v}_i \equiv a_1\vec{v}_1 + a_2\vec{v}_2 + \cdots + a_{i-1}\vec{v}_{i-1} + a_{i+1}\vec{v}_{i+1} + \cdots + a_g\vec{v}_g
			\]
			for some nonzero term $-a_i\vec{v}_i$. But then
			\[
				-a_i\vec{v}_i \in \vecspan{V \setminus \{\vec{v}_i\}}
			\]
			which implies
			\[
				\vecspan{\vec{v}_i} \cap \vecspan{V \setminus \{\vec{v}_i\}} \neq \{\vec{0}\}
			\]
			This is a contradiction since $V$ is dimensionally independent. Therefore no such second solution can exist.
		\end{proposition}
		
		Much like linearly-independent vectors, dimensionally-independent vectors can only sum to zero if each vector itself is zero. However, since nonzero multiples of
		dimensionally-independent vectors can be zero, there may be multiple distinct linear combinations (that is, distinct sets of coefficients for each linear combination)
		that result in a sum of zero.
		
		It turns out that, aside from dimensionally-independent vectors behaving similarly to linearly-independent vectors, there's also a direct link between sets of 
		linearly-independent vectors and dimensionally-independent vectors.

		\begin{proposition}{Given a set of dimensionally-independent vectors \\ 
		$V = \{p^{\alpha_1}\vec{v}_1, p^{\alpha_2}\vec{v}_2, \cdots, p^{\alpha_g}\vec{v}_g\}$ for integers $0 \leq \alpha_i < k$ and where \\
		$\forall \vec{v}_i \in V,\, \vec{v}_i \not\equiv \vec{0} \bmod{p}$, then the set 
		\[
			W = \{\vec{v}_1, \vec{v}_2, \cdots, \vec{v}_g\}
		\]
		is linearly independent.}
			\label{prop:DItoLI}
			Assume $W$ isn't linearly independent. Then there exists a non-trivial solution to
			\[
				\sum_i p^{\beta_i}a_i\vec{v}_i \equiv \vec{0}
			\]
			where $a_i \not\equiv 0 \bmod{p}$ and $\beta_i \in \mathds{Z},\, \beta_i \geq 0$. Define $P$ as
			\[
				P = \mathmax{\{p^{\alpha_1-\beta_1},\, p^{\alpha_2-\beta_2},\, \cdots,\, p^{\alpha_g-\beta_g},\, 1\}}
			\]
			Then
			\begin{equation}
				\label{eq:DItoLIsum}
				\sum_i Pp^{\beta_i}a_i\vec{v}_i \equiv \vec{0}
			\end{equation}
			
			By construction of $P$, each $Pp^{\beta_i}a_i\vec{v}_i$ is now a multiple of some vector in $V$. At least one $Pp^{\beta_i}a_i\vec{v}_i$ is nonzero, since 
			if $\alpha_j-\beta_j$ is the maximum difference between any $\alpha_i-\beta_i$, then $Pp^{\beta_j}a_j\vec{v}_j \equiv p^{\alpha_j}a_j\vec{v}$ which cannot be 
			zero since $a_j$ has no factors of $p$. By proposition \ref{sec:DI}.\ref{prop:DIzerosum}, equation \ref{eq:DItoLIsum} implies the set $V$ isn't dimensionally 
			independent. This is a contradiction, so $W$ must be linearly independent.
		\end{proposition}
		
		Proposition \ref{sec:DI}.\ref{prop:DItoLI} shows that, given a set of dimensionally-independent vectors, there's a collection of corresponding sets of 
		linearly-independent vectors. This fact will prove useful in section \ref{sec:PPC}, where we use it to strengthen the result of proposition 
		\ref{sec:CDIS}.\ref{prop:DIsubmodule} for a particular type of submodule.
		
		We can also say something about dimensionally-independent vectors if they sum to a vector containing factors of $p$, the base of our prime-power modulus.
		
		\begin{proposition}{Let $\{\vec{v}_1, \vec{v}_2, \cdots, \vec{v}_g\}$ be a dimensionally-independent set. If
		\[
			\vec{v}_1 + \vec{v}_2 + \cdots + \vec{v}_g \equiv p^c\vec{w}
		\]
		where $\vec{w} \not\equiv \vec{0} \bmod{p}$ and $0 \leq c < k$, then each $\vec{v}_i \equiv \vec{0} \bmod{p^c}$. Furthermore, at least one 
		$\vec{v}_i \not\equiv \vec{0} \bmod{p^{c+1}}$.}
			\label{prop:DIsumwithp}
			Multiplying both sides of the given equation by $p^{k-c}$, we get
			\[
				p^{k-c}(\vec{v}_1 + \vec{v}_2 + \cdots + \vec{v}_g) \equiv p^k\vec{w} \equiv \vec{0}
			\]
			From Proposition \ref{sec:DI}.\ref{prop:DIzerosum}, the sum $\sum_{i=1}^{g} p^{k-c}\vec{v}_i$ can only equal zero if each vector in the sum is the zero vector. 
			This means each $p^{k-c}\vec{v}_i \equiv \vec{0}$, and so each $\vec{v}_i$ must have at least a factor of $p^c$ in it. Therefore, 
			$\vec{v}_i \equiv \vec{0} \bmod{p^c}$ for all $\vec{v}_i$.
			
			Now, let $\vec{v}_i \equiv p^c\vec{\nu}_i$. Then
			\[
				p^c(\vec{\nu}_1 + \vec{\nu}_2 + \cdots + \vec{\nu}_g) \equiv p^c\vec{w}
			\]
			$\vec{w} \not\equiv \vec{0} \bmod{p}$, so $p^{k-1}\vec{w} \not\equiv \vec{0}$. This implies
			\[
				p^{k-1}\vec{w} \equiv p^{k-1}(\vec{\nu}_1 + \vec{\nu}_2 + \cdots + \vec{\nu}_g) \not\equiv \vec{0}
			\]
			Therefore, at least one $p^{k-1}\vec{\nu}_i \not\equiv \vec{0}$, which means $\vec{\nu}_i \not\equiv \vec{0} \bmod{p}$, so then
			\[
				p^c\vec{\nu}_i \equiv \vec{v}_i \not\equiv \vec{0} \mod{p^{c+1}}
			\]
			for at least one $\vec{v}_i$.
		\end{proposition}
		
		Proposition \ref{sec:DI}.\ref{prop:DIsumwithp} essentially guarantees that, when a set of dimensionally-independent vectors sum to a vector with a certain number
		of factors of $p$, then the dimensionally-independent vectors themselves must have a certain number of factors of $p$. This fact ends up being crucial for showing
		the result of proposition \ref{sec:CDIS}.\ref{prop:DIsubmodule}.
		
	\section{Creating Dimensionally-Independent Sets}
		\label{sec:CDIS}
		If we have a dimensionally-independent set, it's helpful to have an easy way to check whether adding a new vector to it will keep it dimensionally independent.
		
		\begin{proposition}{If $V = \{\vec{v}_1, \vec{v}_2, \cdots, \vec{v}_g\}$ is dimensionally independent and $\vec{n} \not\equiv \vec{0}$ is a vector such that
		$\vecspan{\vec{n}} \cap \vecspan{V} = \{\vec{0}\}$, then 
		\[
			V \cup \{\vec{n}\}
		\]
		is a dimensionally-independent set.}
			\label{prop:easyDIcheck}
			To show $V \cup \{\vec{n}\}$ is dimensionally independent, we must show that, for all $\vec{v} \in V$,
			\[
				\vecspan{\vec{v}} \cap \vecspan{(V \setminus \{\vec{v}\}) \cup \{\vec{n}\}} = \{\vec{0}\}
			\]
			Assume there exists some vector $\vec{v}_i \in V$ that doesn't follow the above relation. Then there exists a positive integer $c$ where
			\[
				p^c\vec{v}_i \equiv a_1\vec{v}_1 + a_2\vec{v}_2 + \cdots + a_{i-1}\vec{v}_{i-1} + a_{i+1}\vec{v}_{i+1} + \cdots + a_g\vec{v}_g + a_n\vec{n}
			\]
			and
			\[
				p^c\vec{v}_i \not\equiv \vec{0}
			\]
			Note that $a_n\vec{n} \not\equiv \vec{0}$ since $V$ is dimensionally independent. Rearranging, we see
			\[
				a_n\vec{n} \equiv p^c\vec{v_i} - a_1\vec{v}_1 - a_2\vec{v}_2 - \cdots - a_{i-1}\vec{v}_{i-1} - a_{i+1}\vec{v}_{i+1} - \cdots - a_g\vec{v}_g
			\]
			This implies
			\[
				\vecspan{\vec{n}} \cap \vecspan{V} \neq \{\vec{0}\}
			\]
			This is a contradiction, so no such $\vec{v}_i$ can exist. Then $V \cup \{\vec{n}\}$ is dimensionally independent.
		\end{proposition}
		
		Proposition \ref{sec:CDIS}.\ref{prop:easyDIcheck} shows that, if all nonzero multiples of a vector are not in the span of a dimensionally-independent set, we can add
		the vector to that set while retaining dimensional independence within the set.
		
		It's also useful to be able to say something about when adding a vector to a dimensionally-independent set does \emph{not} retain dimensional independence.
		
		\begin{proposition}{Let $V = \{\vec{v}_1, \vec{v}_2, \cdots, \vec{v}_g\}$ be a dimensionally-independent set, and let $\vec{u}$ be a nonzero vector such that
		$V \cup \{\vec{u}\}$ is not dimensionally independent. If the maximum number of factors of $p$ of any vector in $V$ is less than or equal to the number of factors
		of $p$ in $\vec{u}$, then given the sum
		\[
			p^c\vec{u} \equiv \sum_i a_i\vec{v}_i
		\]
		where $a_i\vec{v}_i \not\equiv \vec{0}$, then $a_i \equiv 0 \bmod{p^c}$ for all $a_i$.}
			\label{prop:minpfactor}
			By proposition \ref{sec:DI}.\ref{prop:DImultset} and \ref{sec:DI}.\ref{prop:DIsumwithp}, each of $a_i\vec{v}_i$ must have at least the same number of factors of 
			$p$ as $p^c\vec{u}$. If each of $\vec{v}_i$ has at most the same number of factors of $p$ as $\vec{u}$, then each of $a_i$ must have at least a factor of $p^c$ 
			in order for proposition \ref{sec:DI}.\ref{prop:DIsumwithp} to hold. So $a_i \equiv 0 \bmod{p^c}$ for all $a_i$.
		\end{proposition}
		
		Intuitively, proposition \ref{sec:CDIS}.\ref{prop:minpfactor} says that, given a vector $\vec{u}$ that's not dimensionally independent to a 
		dimensionally-independent set, the number of factors of $p$ in $\vec{u}$ can give some idea for what an expression for $p^c\vec{u}$ would look like in terms of the 
		vectors in the dimensionally-independent set. Ultimately, it's this proposition which allows the next proposition to be shown.
		
		\begin{proposition}{Any nonempty submodule $M$ of $\mathds{Z}_{p^k}^L$ can be expressed as the span of a set of dimensionally-independent vectors.}
			\label{prop:DIsubmodule}
			If $M = \{\vec{0}\}$, then let $V = \emptyset$. $\vecspan{V} = \{\vec{0}\}$, and $V$ is dimensionally independent.
			
			Otherwise, every submodule is closed under taking linear combinations, so every submodule of $\mathds{Z}_{p^k}^L$ can be represented as the span of a set of 
			vectors. Let 
			\[
				\hat{M} = \{\vec{m}_1, \vec{m}_2, \cdots, \vec{m}_g\}
			\]
			where $M = \vecspan{\hat{M}}$. Without loss of generality, assume $\vec{0} \not\in \hat{M}$.
			
			The following steps will produce a dimensionally-independent set of vectors whose span is $M$.
			
			1. Order the vectors in $\hat{M}$ by the number of factors of $p$ they have (least to greatest) and store them in an ordered list $S$. For example, if $\hat{M}$ 
			were to look like 
			\[
				\hat{M} = \{p^2\vec{n}_1,\, p^3\vec{n}_2,\, \vec{n}_3,\, p^2\vec{n}_4,\, \vec{n}_5\}
			\]
			where $\vec{n}_1$ to $\vec{n}_5$ have no factors of $p$, then a valid list $S$ could look like
			\[
				S = [\vec{n}_3,\, \vec{n}_5,\, p^2\vec{n}_1,\, p^2\vec{n}_4,\, p^3\vec{n}_2]
			\]
			Another valid list $S$ could be
			\begin{equation}
				\label{eq:exampleS}
				S = [\vec{n}_5,\, \vec{n}_3,\, p^2\vec{n}_4,\, p^2\vec{n}_1,\, p^3\vec{n}_2]
			\end{equation}
			The $i$-th element in $S$ is indexed as $S_i$. The number of elements in $S$ is represented as $|S|$. As an example, for the list $S$ specified by equation 
			\ref{eq:exampleS}, $S_2 = \vec{n}_3$ and $|S| = 5$. We'll also initialise a set $V$ to be the empty set.
			
			2. Set $\vec{s} = S_1$.
			
			3. Check to see whether $V \cup \{\vec{s}\}$ is a dimensionally-independent set. If it is, replace $V$ with $V \cup \{\vec{s}\}$ and proceed to step 5.
			Otherwise, continue to step 4.
			
			4. If $V \cup \{\vec{s}\}$ is not a dimensionally-independent set, then for some smallest $p^c$, $0 \leq c < k$, $c \in \mathds{Z}$, we have that
			\[
				p^c\vec{s} \equiv \sum_i a_i\vec{v}_i
			\]
			for vectors $\vec{v}_i \in V$ and nonzero $a_i \in \mathds{Z}_{p^k}$ where $a_i\vec{v}_i \not\equiv \vec{0}$ (refer to pp. 1-2). Because of the ordering of $S$, 
			the vectors in $V$ are guaranteed to have at most the same number of factors of $p$ in them as $\vec{s}$. This means we can make use of proposition 
			\ref{sec:CDIS}.\ref{prop:minpfactor} and rewrite the sum for $p^c\vec{s}$ as
			\[
				p^c\vec{s} \equiv p^c\sum_i b_i\vec{v}_i
			\]
			for different nonzero constants $b_i \in \mathds{Z}_{p^k}$ where $b_i\vec{v}_i \not\equiv \vec{0}$. Define $\vec{x}$ to be
			\[
				\vec{x} \equiv \vec{s} - \sum_i b_i\vec{v}_i
			\]
			If $\vec{x}$ happens to be the zero vector, discard it and proceed to step 5. 
			
			By construction, $p^c\vec{x} \equiv \vec{0}$. As well, for all $0 \leq f < c$, $f \in \mathds{Z}$, we have that
			\begin{center}
				\begin{tabular}{*{3}{l}}
							   & $p^f\vec{x}$ & $\equiv\, p^f\vec{s} - p^f\sum_i b_i\vec{v}_i$ \\
					$\implies$ & $p^f\vec{x}$ & $\not\in\, \vecspan{V}$
				\end{tabular}
			\end{center}
			since $p^f\vec{x}$ is the sum of a vector in $\vecspan{V}$ and a vector not in $\vecspan{V}$. Note that $p^f\vec{s} \not\in \vecspan{V}$ since $p^c\vec{s}$ is 
			the first nonzero multiple of $\vec{s}$ in $\vecspan{V}$.
			
			By proposition \ref{sec:CDIS}.\ref{prop:easyDIcheck}, this means the set $V \cup \{\vec{x}\}$ is dimensionally independent. Also, 
			$\vecspan{V \cup \{\vec{s}\}} = \vecspan{V \cup \{\vec{x}\}}$ since
			\[
				\vec{x} \equiv \vec{s} - \sum_i b_i\vec{v}_i \implies \vec{s} \equiv \vec{x} + \sum_i b_i\vec{v}_i
			\]
			so any linear combination made with the vectors in $V \cup \{\vec{s}\}$ can be made with the vectors in $V \cup \{\vec{x}\}$, and vice versa.
			
			If the number of factors of $p$ in $\vec{x}$ is at most the same as the number of factors of $p$ in $\vec{s}$, replace $V$ with $V \cup \{\vec{x}\}$ and proceed 
			to step 5. 
			
			Otherwise, if $\vec{s} = S_i$, insert $\vec{x}$ into $S$ somewhere greater than the $i$-th position so that the ordering of $S$ is preserved (i.e. the 
			vectors in $S$ are still ordered from least number of factors of $p$ to greatest). Note that we're not removing the current $\vec{s}$ from $S$; we're 
			increasing the number of vectors in $S$ by inserting $\vec{x}$. The list $S$, then, can be imagined as a queue of sorts, with the current $\vec{s}$ representing 
			which element in the queue we're currently using.
			
			The instruction above ensures that the vector $\vec{x}$ can always be created in step 4; each vector $\vec{s}$ will always have at least the same number of 
			factors of $p$ as any vector in $V$.
			
			5. If $\vec{s} = S_i$ where $i < |S|$, set $\vec{s} = S_{i+1}$ and return to step 3. Otherwise, continue to step 6. 
			
			Note that each vector in $S$ is equal to $\vec{s}$ only once, and so each vector in $S$ can lead to the insertion of only one extra vector into $S$ (via step 
			4). However, the extra vector inserted into $S$ must necessarily have a greater number of factors of $p$ than the vector that led to its insertion (see step 4). 
			Eventually, if vectors continue to be added, these new inserted vectors will be unable to lead to more inserted vectors since they'll have so many factors of $p$
			that any vectors with a greater	number of factors of $p$ will be congruent to the zero vector (which step 4 will discard). Therefore, this process is guaranteed 
			to eventually proceed to step 6; it is impossible to indefinitely insert vectors into $S$.
			
			6. $V$ should now be a dimensionally-independent set of vectors whose span is $M$.
		\end{proposition}
		
		Proposition \ref{sec:CDIS}.\ref{prop:DIsubmodule} shows yet another link between linear independence and dimensional independence: every subspace can be expressed
		as the span of a set of linearly-independent vectors, and every submodule can be expressed as the span of a set of dimensionally-independent vectors. While the
		submodule case isn't as strong as the subspace case (e.g. every submodule doesn't necessarily have a basis), proposition \ref{sec:CDIS}.\ref{prop:DIsubmodule}
		gives a guaranteed way to represent any nonzero submodule, which allows for stronger conditions to be derived for more specific types of submodules, which we'll see
		in the next section.
		
	\section{Prime-Power Cores}
		\label{sec:PPC}
		Given a matrix $A \in \mathds{Z}_{p^k}^{L \times L}$ and the $\mathds{Z}_{p^k}$-module $\mathds{Z}_{p^k}^L$, the \emph{core of $A$}, represented as 
		$\core{p^k}{L}{A}$ is defined to be the largest submodule $S \subseteq \mathds{Z}_{p^k}^L$ such that
		\[
			AS = S
		\]
		In other words, $\core{p^k}{L}{A}$ is the largest submodule of $\mathds{Z}_{p^k}^L$ where the action of multiplying by $A$ creates a bijection between 
		$\core{p^k}{L}{A}$ and itself.
		
		It's known that $\core{p}{L}{A}$ will always be a free module---that is, a module that has a basis---since $\mathds{Z}_p^L$ is a vector space, so $\core{p}{L}{A}$ is 
		also a vector space, meaning it necessarily has a basis. This section's goal is to show that $\core{p^k}{L}{A}$ is also always a free module. First, though, there 
		are a few important concepts to introduce.
		
		For a vector $\vec{v} \in \mathds{Z}_{p^k}^L$ and a matrix $A \in \mathds{Z}_{p^k}^{L \times L}$, denote the \emph{orbit of $\vec{v}$ under $A$} as the set
		$\{\vec{v}, A\vec{v}, A^2\vec{v}, \cdots\}$. 
		
		Because $\mathds{Z}_{p^k}^L$ has a finite number of elements, the orbit of every vector in $\mathds{Z}_{p^k}^L$ is guaranteed to be a finite set; once 
		$A^i\vec{v} \equiv A^I\vec{v}$ (for some $i, I \in \mathds{Z}$, $i, I \geq 0$, $i > I$) each subsequent $A^j\vec{v}$ (for some $j \in \mathds{Z},\, j > i$) 
		will be able to be written as $A^J(A^I\vec{v})$ for $J \in \mathds{Z}$, $0 \leq J < i-I$, and therefore won't add a new vector to the orbit of $\vec{v}$.
		
		The orbits of vectors in the core of a matrix have a few especially nice properties, one of which is described in the following proposition.

		\begin{proposition}{If and only if $\vec{v} \in \core{p^k}{L}{A}$, then for some smallest $c \in \mathds{Z}^+$, 
		\[
			A^c\vec{v} \equiv \vec{v}
		\]}
			\label{prop:coreOrbits}
			First, we'll show $\vec{v} \in \core{p^k}{L}{A} \implies A^c\vec{v} \equiv \vec{v}$.
			
			Assume $A^c\vec{v} \not\equiv \vec{v}$ for all $c$. Then for some $c,d \in \mathds{Z}^+, \, d < c$, we have
			\begin{equation}
				\label{eq:coreOrbitRelation}
				A^c\vec{v} \equiv A^d\vec{v}
			\end{equation}
			where $c$ is the first such number with this property. This $c,d$ must exist as there are a finite number of vectors in $\mathds{Z}_{p^k}^L$. However, since 
			$d \neq 0$, we also know
			\[
				A(A^{d-1}\vec{v}) \equiv A^d\vec{v}
			\]
			and
			\begin{equation}
				\label{eq:coreOrbitImpossibleRelation}
				A^{d-1}\vec{v} \not\equiv A^{c-1}\vec{v}
			\end{equation}
			If equation \ref{eq:coreOrbitImpossibleRelation} wasn't true, it would contradict $c$ being the first number to satisfy equation \ref{eq:coreOrbitRelation}. So
			\[
				A(A^{c-1}\vec{v}) \equiv A(A^{d-1}\vec{v}) \equiv A^d\vec{v}
			\]
			meaning $A$ sends two different vectors to $A^d\vec{v}$. This is a contradiction since multiplication by $A$ must act as a bijection on vectors in 
			$\core{p^k}{L}{A}$.	Therefore, there must be some smallest $c \in \mathds{Z}^+$ where $A^c\vec{v} \equiv \vec{v}$.
			
			Now, we'll show $A^c\vec{v} \equiv \vec{v} \implies \vec{v} \in \core{p^k}{L}{A}$.
			
			Let the orbit of $\vec{v}$ be denoted as $\{\vec{v}, A\vec{v}, A^2\vec{v}, \cdots\}$. By assumption,
			\[
				A^d\vec{v} \not\equiv \vec{v}, \quad d < c 
			\]
			The only vector $A$ maps to $\vec{v}$ is $A^{c-1}\vec{v}$. As well, a relation such as
			\[
				A^i\vec{v} \equiv A^d\vec{v}
			\]
			for some $i, d \in \mathds{Z}^+,\, i, d < c,\, i \neq d$ cannot exist as, if it did, then it would contradict the fact that $c$ is the smallest nonnegative 
			integer where $A^c\vec{v} \equiv \vec{v}$ (using such a relation, $A^c\vec{v}$ could be reduced to some $A^x\vec{v}$ with $x < c$). Thus, every vector in the 
			orbit of $\vec{v}$ is mapped to by only one unique vector, so $A$ acts as a bijection on the orbit of $\vec{v}$. This implies $\vec{v} \in \core{p^k}{L}{A}$.
		\end{proposition}
		
		Via proposition \ref{sec:PPC}.\ref{prop:coreOrbits}, the core of a matrix can be thought of as the set of all vectors whose orbits form a loop---that is, every vector
		in the orbit is mapped to by only one other vector. Under repeated multiplication by the matrix, vectors in the core will visit every vector in their orbit exactly 
		once before returning to themselves.
		
		Given a single matrix $A \in \mathds{Z}_{p^{k+n}}^{L \times L}$, and given two modules $\mathds{Z}_{p^k}^{L}$ and $\mathds{Z}_{p^{k+n}}^L$, there are a few ways to 
		relate the orbits of their vectors under this matrix. One such way is through \emph{embedding} vectors from $\mathds{Z}_{p^k}^L$ to $\mathds{Z}_{p^{k+n}}^L$.
		
		Define $\phi : \mathds{Z}_{p^{k}}^{L} \rightarrow p\mathds{Z}_{p^{k+1}}^{L}$ as $\phi(\vec{x}) = p\vec{x}$. This mapping is a module homomorphism since,
		for $\vec{u}, \vec{v} \in \mathds{Z}_{p^k}^L$, $s, t \in \mathds{Z}_{p^k}$,
		\[
			\phi(s\vec{u} + t\vec{v}) = p(s\vec{u} + t\vec{v}) = ps\vec{u} + pt\vec{v} = sp\vec{u} + tp\vec{v} = s\phi(\vec{u}) + t\phi(\vec{v})
		\]
		This mapping also creates a bijection between the modules $\mathds{Z}_{p^k}^L$ and $p\mathds{Z}_{p^{k+1}}^L$ since
		\begin{center}
			\begin{tabular}{*{3}{l}}
						   & $\phi(\vec{u}) \equiv \phi(\vec{v})$           &                 \\
				$\implies$ & $\phi(\vec{u}) - \phi(\vec{v}) \equiv \vec{0}$ &                 \\
				$\implies$ & $\phi(\vec{u} - \vec{v}) \equiv \vec{0}$       &                 \\
				$\implies$ & $p(\vec{u} - \vec{v}) \equiv \vec{0}$          & $\mod{p^{k+1}}$ \\
				$\implies$ & $\vec{u} - \vec{v} \equiv \vec{0}$             & $\mod{p^k}$     \\
				$\implies$ & $\vec{u} \equiv \vec{v}$                       & $\mod{p^k}$     \\
			\end{tabular}
		\end{center}
		so $\phi$ is injective, and for any vector of the form $p\vec{v} \in \mathds{Z}_{p^{k+1}}^L$, there's a vector $\vec{v} \in \mathds{Z}_{p^k}^L$ where 
		$\phi(\vec{v}) = p\vec{v}$, so $\phi$ is surjective. Therefore, $\phi$ is a module isomorphism.
		
		One important property of $\phi$ to note is that it preserves matrix multiplication:
		\begin{align*}
			\phi(A\vec{v}) & \equiv p(A\vec{v} \bmod{p^k})     \\
			               & \equiv p(A\vec{v}) \bmod{p^{k+1}} \\
						   & \equiv A(p\vec{v}) \bmod{p^{k+1}} \\
						   & \equiv A\phi(\vec{v}) \bmod{p^{k+1}}
		\end{align*}
		
		Given a vector $\vec{v} \in \mathds{Z}_{p^k}^L$, the \emph{embed vector of $\vec{v}$ in $\mathds{Z}_{p^{k+1}}^L$} is given by $\phi(\vec{v})$. Embed vectors give us
		an easy way to relate vectors in the core of a matrix between modules of differing prime-power moduli.
		
		\begin{proposition}{$\vec{v} \in \core{p^k}{L}{A} \Longleftrightarrow \phi(\vec{v}) \in \core{p^{k+1}}{L}{A}$.}
			\label{prop:embedsInCore}
			First, we'll show $\vec{v} \in \core{p^k}{L}{A} \implies \phi(\vec{v}) \in \core{p^{k+1}}{L}{A}$.
			
			If $\vec{v} \in \core{p^k}{L}{A}$, then by proposition \ref{sec:PPC}.\ref{prop:coreOrbits}, for some smallest $c \in \mathds{Z}^+$,
			\begin{equation}
				\label{eq:nonEmbeddedCoreRelation1}
				A^c\vec{v} \equiv \vec{v}
			\end{equation}
			$\phi$ is a module isomorphism, so
			\begin{equation}
				\label{eq:embeddedCoreRelation1}
				\phi(A^c\vec{v}) \equiv \phi(\vec{v}) \mod{p^{k+1}}
			\end{equation}
			No smaller $c$ can satisfy equation \ref{eq:embeddedCoreRelation1}. If such a smaller number existed, then it could be swapped in to equation 
			\ref{eq:embeddedCoreRelation1} and $\phi^{-1}$ could be applied, which would contradict $c$ being the smallest such number that satisfies equation 
			\ref{eq:nonEmbeddedCoreRelation1}. Since $\phi$ preserves matrix multiplication,
			\[
				A^c\phi(\vec{v}) \equiv \phi(\vec{v}) \mod{p^{k+1}}
			\]
			Since $c$ is the smallest possible positive integer with this property, proposition \ref{sec:PPC}.\ref{prop:coreOrbits} guarantees that 
			$\phi(\vec{v}) \in \core{p^{k+1}}{L}{A}$.
			
			Next, we'll show $\phi(\vec{v}) \in \core{p^{k+1}}{L}{A} \implies \vec{v} \in \core{p^k}{L}{A}$.
			
			If $\phi(\vec{v}) \in \core{p^{k+1}}{L}{A}$, then by proposition \ref{sec:PPC}.\ref{prop:coreOrbits}, there exists some smallest $c \in \mathds{Z}^+$ where
			\begin{equation}
				\label{eq:embeddedCoreRelation2}
				A^c\phi(\vec{v}) \equiv \phi(\vec{v}) \mod{p^{k+1}}
			\end{equation}
			$\phi$ preserves matrix multiplication, so
			\[
				\phi(A^c\vec{v}) \equiv \phi(\vec{v}) \mod{p^{k+1}}
			\]
			$\phi$ is a module isomorphism, so $\phi^{-1}$ can be applied on both sides of the above equation:
			\[
				A^c\vec{v} \equiv \vec{v}
			\]
			If $c$ wasn't the smallest positive integer where the above relation holds, then swapping $c$ for the actual smallest positive integer and applying $\phi$ 
			on both sides would contradict $c$ being the smallest positive integer that satisfies equation \ref{eq:embeddedCoreRelation2}. Therefore $c$ is the smallest 
			such integer, and proposition \ref{sec:PPC}.\ref{prop:coreOrbits} guarantees that $\vec{v} \in \core{p^k}{L}{A}$.
		\end{proposition}
		
		Embed vectors are not the only possible way to relate similar cores. Another vector type we can use to relate cores is the \emph{lift vector}. For a vector 
		$\vec{v} \in \mathds{Z}_{p^k}^L$, a \emph{lift vector of $\vec{v}$} is any vector $\vec{\ell} \in \mathds{Z}_{p^{k+n}}^L$ such that
		\[
			\vec{\ell} \equiv \vec{v} \mod{p^k}
		\]
		where $n \in \mathds{Z}^+$. This means $\vec{\ell}$ must take the form
		\[
			\vec{\ell} \equiv \vec{v} + p^k\vec{u} \mod{p^{k+n}}, \quad \vec{u} \in \mathds{Z}_{p^n}^L
		\]
		
		Much like embed vectors, lift vectors provide a way to relate vectors within two related cores.
		
		\begin{proposition}{If $\vec{v} \in \mathds{Z}_{p^k}^L$ is in $\core{p^k}{L}{A}$, then a lift vector $\vec{\ell} \in \mathds{Z}_{p^{k+n}}^L$ of $\vec{v}$ exists
		such that
		\[
			\vec{\ell} \in \core{p^{k+n}}{L}{A}
		\]}
			\label{prop:dropsInCore2Lifts}
			$\vec{v} \in \core{p^k}{L}{A}$, so by proposition \ref{sec:PPC}.\ref{prop:coreOrbits} there exists a smallest $c \in \mathds{Z}^+$ such that
			\[
				A^c\vec{v} \equiv \vec{v}
			\]
			Using properties of modular arithmetic, we can lift this equation to a higher-power modulus:
			\[
				A^c\vec{v} \equiv \vec{v} + p^k\vec{u} \mod{p^{k+n}}, \quad \vec{u} \in \mathds{Z}_{p^n}^L
			\]
			Taking this new vector and multiplying it by $A^c$, we get
			\[
				A^c(\vec{v} + p^k\vec{u}) \equiv \vec{v} + p^k(\vec{u} + A^c\vec{u}) \mod{p^{k+n}}
			\]
			In general, if we continue taking these vectors and multiplying them by $A^c$, we'll get a sequence of vectors where each term looks like
			\[
				\vec{v} + p^k\vec{w}_i \mod{p^{k+n}}, \quad \vec{w}_i \in \mathds{Z}_{p^n}^L
			\]
			There are only a finite number of vectors in $\mathds{Z}_{p^n}^L$, so eventually $\vec{w}_x \equiv \vec{w}_y$ for some $x,y \in \mathds{Z},\, x > y$. Let 
			$\vec{W}$ be the first such $\vec{w}_x$ where this occurs. Then, if we let
			\[
				\vec{\ell} \equiv \vec{v} + p^k\vec{W} \mod{p^{k+n}}
			\]
			then $\vec{\ell}$ is guaranteed to be in $\core{p^{k+n}}{L}{A}$ by construction. $\vec{\ell}$ also happens to be a lift of $\vec{v}$ since
			\[
				\vec{\ell} \equiv \vec{v} \mod{p^k}
			\]
		\end{proposition}
		
		While proposition \ref{sec:PPC}.\ref{prop:dropsInCore2Lifts} doesn't explicitly give us a corresponding vector for a given vector in a related core, it does 
		guarantee such a vector exists. Using the concepts of embed and lift vectors, we can show that the cores of matrices in modules with prime-power moduli---which are 
		submodules---must necessarily be spanned by a set of linearly-independent vectors, strengthening the result of proposition \ref{sec:CDIS}.\ref{prop:DIsubmodule}.
		
		\begin{proposition}{For any matrix $A \in \mathds{Z}_{p^k}^{L \times L}$, $\core{p^k}{L}{A}$ can be represented as the span of a set of linearly-independent vectors.
		\footnote{Thank you to Dr. Franklin Mendivil for helping to condense this proposition!}}
			\label{prop:coreBasis}
			If $\core{p^k}{L}{A} = \{\vec{0}\}$, then let $M = \emptyset$. $\vecspan{M} = \{\vec{0}\}$, and $M$ is linearly independent.
			
			Otherwise, let $M = \{\vec{m}_1, \vec{m}_2, \cdots, \vec{m}_g\}$ be a set of dimensionally-independent vectors where $\vecspan{M} = \core{p^k}{L}{A}$ guaranteed 
			by proposition \ref{sec:CDIS}.\ref{prop:DIsubmodule}.
			
			For each vector $\vec{m}_i \in M$, the following argument shows that $\vec{m}_i \not\equiv \vec{0} \bmod{p}$: 
			
			Let $\vec{m}_i \equiv p^{t_i}\vec{n}_i$ where $t_i \in \mathds{Z},\, t_i < k$ and $\vec{n}_i \not\equiv \vec{0} \bmod{p}$. By repeated application of proposition
			\ref{sec:PPC}.\ref{prop:embedsInCore}, we know
			\[
				\vec{n}_i \in \core{p^{k-t_i}}{L}{A}
			\]
			
			By proposition \ref{sec:PPC}.\ref{prop:dropsInCore2Lifts}, there exists a lift vector $\vec{\ell}_i$ of $\vec{n}_i$ such that 
			\[
				\vec{\ell}_i \in \core{p^k}{L}{A}. 
			\]
			Note that $p^{t_i}\vec{\ell}_i \equiv p^{t_i}(\vec{n}_i + p^{k-t_i}\vec{z}) \equiv p^{t_i}\vec{n}_i + p^k\vec{z} \equiv p^{t_i}\vec{n}_i \equiv \vec{m}_i$ for
			some vector $\vec{z} \in \mathds{Z}_{p^k}^L$ since $\vec{\ell}_i$ is a lift of $\vec{n}_i$.
			
			Since $\vec{\ell}_i$ is in $\core{p^k}{L}{A}$, it must also be in $\vecspan{M}$. This means $\vec{\ell}_i$ can be represented as a linear combination of vectors
			in $M$:
			\[
				\vec{\ell}_i \equiv \sum_j a_j\vec{m}_j
			\]
			where each $a_j \in \mathds{Z}_{p^k}$ and each $a_j\vec{m}_j \not\equiv \vec{0}$. This means
			\[
				\vec{m}_i \equiv p^{t_i}\vec{\ell}_i \equiv p^{t_i}\sum_j a_j\vec{m}_j
			\]
			
			$M$ is a dimensionally-independent set, so 
			\[
				\vec{m}_i \equiv p^{t_i} \sum_j a_j\vec{m}_j \equiv p^{t_i}a_i\vec{m}_i
			\]
			The only way this equation can be true is if $t_i \equiv 0$. Therefore, $\vec{m}_i \equiv \vec{n}_i$, and $\vec{n} \not\equiv \vec{0} \bmod{p}$.
			
			With this argument, we can be sure that no vector in $M$ has a multiple of $p$ in it. By proposition \ref{sec:DI}.\ref{prop:DItoLI}, then, $M$ is guaranteed 
			to be a set of linearly-independent vectors.
		\end{proposition}
		
		Proposition \ref{sec:PPC}.\ref{prop:coreBasis} shows that the core of any matrix under a prime-power modulus is always a free module, meaning it has a basis. This
		means any prime-power LCA system restricted to its core acts as an invertible prime-power LCA system.
\end{document}
