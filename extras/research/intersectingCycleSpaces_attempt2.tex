
\documentclass[a4paper, 12pt, reqno]{amsart}
\usepackage[T1]{fontenc}

\usepackage[parfill]{parskip}

\usepackage[margin=3cm]{geometry}

\usepackage{setspace}
\setstretch{1.25}

\usepackage{natbib}

\usepackage{dsfont}
\usepackage{amssymb}

\usepackage{verbatim}

\newtheorem{prop}{Proposition}
\newtheorem{coro}{Corollary}

\DeclareMathOperator{\Span}{span}
\newcommand{\Z}{\mathbb{Z}}
\newcommand\divides{\ | \ }
\newcommand\Mat[2][]{\mathbf{#1}^{\!#2}}
\newcommand\smallmat[1]{\left[\begin{smallmatrix}#1\end{smallmatrix}\right]}
\newcommand{\cycsp}[1]{\mathcal{S}_{#1}}

\newcommand{\dq}[1]{``#1''}

\renewcommand{\qedsymbol}{\textbf{QED}}

%Allowing propositions to be clicked on
%This needs to be the last package imported
\usepackage{hyperref}
\usepackage{xcolor}
\hypersetup
{
    colorlinks=false,
	pdfborder={0 0 0},
    pdftitle={Intersecting Cycle Spaces},
    pdfpagemode=FullScreen,
	pdfauthor={Zach Strong},
}

\begin{document}
	An object of interest to us is the span of a vector's set of iterates under a given update matrix. We call this space the \emph{cycle space} of a vector.
	
	\begin{defn}
		The \emph{cycle space} of a vector $\vec{v} \in \Z_p^L$ under a matrix $\Mat[A]{} \in \Z_p^{L \times L}$ for some prime $p$, denoted as $\cycsp{\vec{v}}$, is 
		defined to be the subspace spanned by the iterates of $\vec{v}$ under iteration by $\Mat[A]{}$. Symbolically,
		\[
			\cycsp{\vec{v}} = \Span\!\left(\bigcup_{i\,=\,0}^{\infty} \{\Mat[A]{i}\vec{v}\}\right) \mod{p}.
		\]
		The vector $\vec{v}$ is said to be the \emph{generating vector} for $\cycsp{\vec{v}}$.
	\end{defn}
	
	For primes $p$, because $\Z_p^L$ is a finite vector space, cycle spaces, which are subspaces of $\Z_p^L$, are also finite, despite our definition including an infinite
	union. In fact, for the definition of the cycle space, it would be sufficient to take the union from $0 \leq i \leq \omega+\tau-1$, where $\omega$ is the cycle length of 
	the generating vector, and $\tau$ is the transient length. This is because all the iterates for a vector after the $\omega+\tau-1$-th iterate will have been iterated to 
	before by definition of the cycle length and transient length.
	
	Cycle spaces are of interest to us as they represent the finest decomposition of an LCA's configuration space into invariant\footnote{As a reminder, a subspace is 
	\emph{invariant} under a matrix $\Mat[A]{}$ if, for all vectors $\vec{v}$ in the subspace, $\Mat[A]{}\vec{v}$ is also in the subspace.} subspaces under multiplication by 
	the update matrix. It isn't too difficult to see why. Say we wanted the smallest possible invariant subspace under a matrix $\Mat[A]{}$ that contained the vector 
	$\vec{v}$. For the space to be invariant under $\Mat[A]{}$, it better be the case that $\Mat[A]()\vec{v}$, $\Mat[A]{2}\vec{v}$, etc., are in the subspace. In order for us 
	to have a subspace, it must also be the case that any linear combination of $\vec{v}$, $\Mat[A]{}\vec{v}$, $\Mat[A]{2}\vec{v}$, etc., is also in the subspace. The 
	definition of cycle spaces satisfies these criteria without introducing any additional vectors which don't need to be in the subspace, and so cycle spaces are indeed the
	smallest possible invariant subspaces that contain a given vector.
	
	There are many reasons to care about invariant subspaces. In a more general context, one may want to consider the eigenspaces of a linear transformation in order to
	characterise the transformation. Eigenspaces are, by definition, invariant subspaces. For our purposes, we wish to better understand the dynamics of the vectors in an
	LCA's configuration space, and one way to do that is by understanding the action of the LCA's update matrix on a vector's iterates. The cycle space of a vector provides
	the smallest subspace in which the iterates of a vector are contained, and so it is in some sense the simplest space to consider for this purpose. In both cases, having a
	better understanding of how cycle spaces behave could prove useful, and so we'll dedicate this section of the thesis to proving some results regarding these spaces.
	
	A property we can deduce about a vector's cycle space right away is its dimension---the number of linearly-independent vectors needed to span it.
	
	\begin{prop}
		\label{prop:cycSpaceDim}
		Given a vector $\vec{v} \in \Z_p^L$, $p$ prime, with minimal annihilating polynomial $m(x) \in \Z_p[X]$ under $\Mat[A]{} \in \Z_p^{L \times L}$, the dimension of 
		the cycle space of $\vec{v}$ is equal to the degree of $m(x)$. Symbolically, $\dim(\cycsp{\vec{v}}) = \deg(m(x))$.\footnote{Note that proofs of this fact have appeared
		previously. An example is Theorem 2 on page 69 of \citet{Jacobson1953}.}
	\end{prop}
	\begin{proof}
		The minimal annihilating polynomial is the smallest polynomial $m(x)$ such that $m(\Mat[A]{})\vec{v} \equiv \vec{0}$. By rearranging this relation, we find that 
		$\Mat[A]{\deg(m(x))}\vec{v}$ is the first iterate of $\vec{v}$ that can be written as a linear combination of its previous iterates (i.e. $\vec{v}$ to 
		$\Mat[A]{\deg(m(x))-1}\vec{v}$). So, $\cycsp{\vec{v}}$ requires $\deg(m(x))$ basis vectors ($\vec{v}$ to $\Mat[A]{\deg(m(x))-1}$), and so 
		$\dim(\cycsp{\vec{v}}) = \deg(m(x))$.
	\end{proof}
	
	We can also show that, given a subspace whose minimal annihilating polynomial is a particular degree, these must exist a cycle space within that subspace whose dimension
	equals the degree of the polynomial.
	
	\begin{prop}
		\label{prop:cycSpaceMandatoryCycSpace}
		Given an invariant subspace $V \subseteq \Z_p^L$ under $\Mat[A]{} \in \Z_p^{L \times L}$, $p$ prime, with minimal annihilating polynomial $m(x) \in \Z_p[X]$, there 
		exists a vector $\vec{v} \in V$ such that $\dim(\cycsp{\vec{v}}) = \deg(m(x))$. Thus, the dimension of $V$ is at least $\deg(m(x))$.
	\end{prop}
	\begin{proof}
		Let $m(x) = (f_1(x))^{n_1}\times(f_2(x))^{n_2}\times\cdots\times(f_k(x))^{n_k}$ for irreducible, monic, relatively-prime polynomials $f_i(x) \in \Z_p[X]$ and positive 
		integers $n_i$. The Primary Decomposition Theorem says that
		\[
			V = \bigoplus_{i\,=\,1}^k \ker((f_i(\Mat[A]{}))^{n_i})
		\]
		where each $\ker((f_i(\Mat[A]{}))^{n_i}) \neq \{\vec{0}\}$. Thus, for each $\ker((f_i(\Mat[A]{}))^{n_i})$, there must exist at least one vector $\vec{v}_i$ such that
		$(f_i(x))^{n_i}$ is the minimal annihilating polynomial of $\vec{v}_i$. If this wasn't the case, then there would exist some $\eta_i < n_i$ where 
		$\ker((f_i(\Mat[A]{}))^{\eta_i}) = \ker((f_i(\Mat[A]{}))^{n_i})$ and so 
		\[
			(f_1(x))^{n_1}\times\cdots\times(f_{i-1}(x))^{n_{i-1}}\times(f_i(x))^{\eta_i}\times(f_{i+1}(x))^{n_{i+1}}\times\cdots\times(f_k(x))^{n_k}
		\]
		would be an annihilating polynomial for $V$. However, this polynomial has a smaller degree than $m(x)$, which contradicts the fact that $m(x)$ is the minimal 
		annihilating polynomial for $V$, and so this cannot happen.
		
		Because each $\ker((f_i(\Mat[A]{}))^{n_i})$ shares only the zero vector with all the other kernels, the sum $\vec{v} \equiv \vec{v}_1 + \vec{v}_2 + \cdots + \vec{v}_k$
		has minimal annihilating polynomial 
		\[
			(f_1(x))^{n_1}\times(f_2(x))^{n_2}\times\cdots\times(f_k(x))^{n_k} \equiv m(x).
		\]
		By proposition \ref{prop:cycSpaceDim}, then, $\cycsp{\vec{v}}$ has dimension $\deg(m(x))$. As well, because $V$ is invariant under $\Mat[A]{}$, 
		$\cycsp{\vec{v}} \subseteq V$, so $\dim(V) \geq \dim(\cycsp{\vec{v}}) = \deg(m(x))$.
	\end{proof}
	
	Propositions \ref{prop:cycSpaceDim} and \ref{prop:cycSpaceMandatoryCycSpace} show that vectors' cycle spaces are directly related to minimal annihilating polynomials. 
	It's natural, then, to wonder if cycle spaces could prove to be as useful in analysing LCAs as the subspaces given by the Primary Decomposition Theorem. After all, both 
	types of subspaces are directly linked to minimal annihilating polynomials, and both share similar properties (such as being invariant under multiplication by the LCA's 
	update matrix). However, as cycle spaces provide a more granular decomposition of an LCA's configuration space, it's possible they could reveal somehow \dq{finer} details 
	about an LCA's behaviour.\footnote{Our cycle spaces are a special case of a more general space: a \emph{cyclic subspace}. Cyclic subspaces can be used to show lots of 
	interesting results in areas of linear algebra (such as proving the Cayley-Hamilton Theorem; see pages 280 to 285 of \citet{Friedberg1989}), so it makes sense to wonder 
	whether they can be used for our purposes, too.}
	
	Though cycle spaces share many traits with the subspaces given by the Primary Decomposition Theorem, we quickly notice that they don't behave the same. One immediate 
	difference we notice is that the intersections of cycle spaces need not contain only the zero vector, as is the case with the Primary Decomposition Theorem's subspaces. 
	As an example, take the LCA $\left(\Z_5, \Z_5^2, \smallmat{1 & 1 \\ 1 & 0}\right)$ and consider the two vectors $\vec{v} = \smallmat{1 \\ 2}$ and 
	$\vec{w} = \smallmat{1 \\ 0}$. We see that
	\[
		\mathcal{S}_{\vec{v}} = \Span\!\left(\left\{\begin{bmatrix}1 \\ 2\end{bmatrix}\right\}\right) \mod{5}
	\]
	and
	\[
		\mathcal{S}_{\vec{w}} = \Span\!\left(\left\{\begin{bmatrix}1 \\ 0\end{bmatrix}\!\!,\ \begin{bmatrix}0 \\ 1\end{bmatrix}\right\}\right) \mod{5},
	\]
	and so
	\[
		\mathcal{S}_{\vec{v}} \cap \mathcal{S}_{\vec{w}} = \mathcal{S}_{\vec{v}}.
	\]
	This example shows that the intersections of cycle spaces are not as predictable as the intersections of the subspaces provided by the Primary Decomposition Theorem. 
	However, our example also displays a potential pattern: is the intersection of two cycle spaces always another cycle space? In this case, the result is trivial since one
	of the cycle spaces is the entirety of our LCA's configuration space, and so the intersection comes out as the other cycle space. If we were to try other examples, though,
	we'd see the same behaviour: intersections of cycle spaces seem to yield other cycle spaces. Thus, it may be worthwhile to think about this question. What can we deduce 
	about the intersection of cycle spaces?
	
	Because cycle spaces are invariant subspaces under multiplication by an update matrix, anything we prove about general invariant subspaces will also apply to cycle spaces.
	The following proposition is an example.
	
	\begin{prop}
		\label{prop:cycSpaceIntersectSubspace}
		Let $V$ and $W$ be invariant subspaces of $\Z_p^L$ under $\Mat[A]{} \in \Z_p^{L \times L}$, $p$ prime. Then $V \cap W$ is a subspace of $\Z_p^L$ which is invariant 
		under $\Mat[A]{}$.
	\end{prop}
	\begin{proof}
		Let $\vec{x}, \vec{y} \in V \cap W$. This means $\vec{x}$ and $\vec{y}$ are in both $V$ and $W$. We can then say $\vec{x} + \vec{y}$ is in both $V$ and $W$ since 
		subspaces are closed under addition. Thus, $\vec{x} + \vec{y} \in V \cap W$.
		
		Furthermore, we can say $t\vec{x}$ is in both $V$ and $W$ for $t \in \Z_p$ since subspaces are closed under taking scalar multiples. Thus, $t\vec{x} \in V \cap W$ for 
		$t \in \Z_p$.
		
		Finally, because $V \cap W \subseteq \Z_p^L$ by virtue of both $V$ and $W$ being within $\Z_p^L$, we have enough to show that $V \cap W$ is a subspace of $\Z_p^L$.
		
		To show $V \cap W$ is invariant under $\Mat[A]{}$, note that $\Mat[A]{}\vec{x}$ is in both $V$ and $W$ since both subspaces are invariant under $\Mat[A]{}$ by 
		assumption. Thus, $\Mat[A]{}\vec{x} \in V \cap W$, and so any arbitrary vector in the intersection stays within the intersection under multiplication by $\Mat[A]{}$. 
		In other words, $V \cap W$ is invariant under $\Mat[A]{}$.
	\end{proof}

	Our current conjecture is that the intersection of cycle spaces is itself a cycle space. Proposition \ref{prop:cycSpaceIntersectSubspace} may have us wondering whether
	something more general is true. Perhaps \emph{any} combination of two cycle spaces that results in an invariant subspace will be a cycle space. One such combination we
	could try is the sum of two cycle spaces.
	
	Note that, when we when we refer to the sum of two subspaces, say $V + W$ for subspaces $V$ and $W$, we mean the set of vectors 
	$\{\vec{v} + \vec{w} : \vec{v} \in V, \vec{w} \in W\}$. This is slightly different from the \emph{direct} sum of subspaces $V$ and $W$ which requires that for every
	$\vec{x} \in V + W$ there exists a \emph{unique} pair of vectors $\vec{v} \in V$ and $\vec{w} \in W$ such that $\vec{x} = \vec{v} + \vec{w}$. For this thesis, $\oplus$ will
	be used to denote the direct sum, while $+$ will be used for this looser sum.
	
	\begin{prop}
		\label{prop:cycSpaceSumSubspace}
		Let $V$ and $W$ be invariant subspaces of $\Z_p^L$ under $\Mat[A]{} \in \Z_p^{L \times L}$, $p$ prime. Then $V + W$ is a subspace of $\Z_p^L$ which is invariant under 
		$\Mat[A]{}$.
	\end{prop}
	\begin{proof}
		Let $\vec{x}, \vec{y} \in V + W$. This means both $\vec{x}$ and $\vec{y}$ can be written as a sum of two vectors: one from $V$ and one from $W$. Therefore, 
		$\vec{x} + \vec{y}$ can also be written in this form simply by adding the respective vectors from each subspace (which can be done since subspaces are closed under 
		addition). So, $\vec{x} + \vec{y} \in V + W$.
		
		Furthermore, a scalar multiple of a sum of two vectors, one from each subspace, will result in another sum of two vectors, one from each subspace, since subspaces are
		closed under scalar multiplication. Thus, $t\vec{x} \in V + W$ for $t \in \Z_p$.
		
		By virtue of $V$ and $W$ being subsets of $\Z_p^L$, $V + W$ must also be a subset of $\Z_p^L$ due to $\Z_p^L$ being a subspace (which is closed under addition). This, 
		combined with what was shown above, is enough to show that $V + W$ is a subspace of $\Z_p^L$.
		
		As with scalar multiples, a sum of two vectors, one from each subspace, multiplied by $\Mat[A]{}$, must also give a sum of two vectors, one from each subspace, since
		both $W$ and $V$ are invariant under $\Mat[A]{}$. Therefore, $\Mat[A]{}\vec{x} \in V + W$, meaning $V + W$ is invariant under $\Mat[A]{}$.
	\end{proof}
	
	Proposition \ref{prop:cycSpaceSumSubspace} proves that the sum of two cycle spaces is indeed an invariant subspace. Unfortunately, examples of sums of cycle spaces exist
	where the resulting sum is not a cycle space. For instance, take the LCA $\left(\Z_3, \Z_3^3, \smallmat{2 & 0 & 0\\0 & 0 & 1\\0 & 1 & 0}\right)$ and add the two cycle 
	spaces generated by the vectors $\smallmat{1\\0\\0}$ and $\smallmat{0\\1\\0}$. The resulting space, $\Z_3^3$, cannot be spanned by a single vector under iteration by the 
	given update matrix as the minimal polynomial for $\smallmat{2 & 0 & 0\\0 & 0 & 1\\0 & 1 & 0}$ modulo 3 is a degree 2 polynomial, meaning every vector in $\Z_3^3$ under 
	this matrix will have a minimal annihilating polynomial of degree 2 or less, and by proposition \ref{prop:cycSpaceDim}, this means every vector's cycle space will be of 
	dimension 2 or less.
	
	So, it isn't the case that all \dq{combinations} of cycle spaces that result in invariant subspaces give cycle spaces. Thus, if there is some unique property that 
	intersections of cycle spaces possess so that they themselves are cycle spaces, it goes beyond the fact that they're invariant subspaces. The following proposition will 
	give us some insight into what that unique property may be.
	
	\begin{prop}
		\label{prop:cycSpaceInteriorRestrictions}
		Let $\vec{v} \in \Z_p^L$ be a vector, $p$ prime, with cycle space $\cycsp{\vec{v}}$ under $\Mat[A]{} \in \Z_p^{L \times L}$. If $S$ is an invariant subspace such that 
		$S \subseteq \cycsp{\vec{v}}$, then
		\[
			\dim(S) = \deg(m(x)),
		\]
		where $m(x) \in \Z_p[X]$ is the minimal annihilating polynomial of $S$.
	\end{prop}
	\begin{proof}
		Let $D = \dim(\cycsp{\vec{v}})$. To show what we want to show, there are two cases we need to consider.
		
		\textbf{Case 1: $S = \cycsp{\vec{v}}$.} Since $S = \cycsp{\vec{v}}$, the minimal annihilating polynomial for $S$ will be equal to the minimal 
		annihilating polynomial for $\cycsp{\vec{v}}$. By proposition \ref{prop:cycSpaceDim}, the degree of the minimal annihilating polynomial for
		$\cycsp{\vec{v}}$ is $D$, which is the dimension of $\cycsp{\vec{v}}$ and therefore the dimension of $S$.
		
		\textbf{Case 2: $S \subset \cycsp{\vec{v}}$.} Assume the opposite of our proposition. Then, we have that $S \subset \cycsp{\vec{v}}$, but $\dim(S) > \deg(m(x))$. Note 
		that it isn't possible for the dimension of an invariant subspace to be less than the degree of its minimal annihilating polynomial by proposition 
		\ref{prop:cycSpaceMandatoryCycSpace}, and $S$ is certainly an invariant subspace by assumption. So, if $\dim(S) \neq \deg(m(x))$, then it must be the case that the 
		dimension is greater, not less.
		
		To ease with notation, let $n = \deg(m(x))$ and $d = \dim(S)$.
		
		Now, let $B = \{\vec{b}_1, \vec{b}_2, \cdots, \vec{b}_d, \vec{b}_{d+1}, \cdots, \vec{b}_D\}$ be a basis for $\cycsp{\vec{v}}$ where 
		$\{\vec{b}_1, \vec{b}_2, \cdots, \vec{b}_d\}$ is a basis for $S$. As well, let $\tilde{B} = \{\vec{b}_{d+1}, \vec{b}_{d+2}, \cdots, \vec{b}_D\}$. 
		Then, we can write $\Mat[A]{i}\vec{v}$ as
		\[
			\Mat[A]{i}\vec{v} \equiv \vec{t}_i + \vec{g}_i \mod{p},
		\]
		where $\vec{t}_i \in S$ and $\vec{g}_i \in \Span(\tilde{B})$.
		
		Because $|\tilde{B}| = D-d$ (and because $\tilde{B}$ is a basis for $\Span(\tilde{B})$), the maximum number of vectors in a subset of $\Span(\tilde{B})$ that can be 
		linearly independent is $D-d$. Thus, it must be the case that the set $\{\vec{g}_0, \vec{g}_1, \cdots, \vec{g}_{D-d}\}$ is linearly dependent, meaning there are
		constants $a_0$ to $a_{D-d}$ in $\Z_p$ such that
		\[
			a_0\vec{g}_0 + a_1\vec{g}_1 + \cdots + a_{D-d}\vec{g}_{D-d} \equiv \vec{0} \mod{p}.
		\]
		Let $r(x) = a_0 + a_1x + a_2x^2 + \cdots + a_{D-d}x^{D-d}$. Note that $r(x)$ is a degree $D-d$ polynomial. We have that
		\begin{align*}
			       & \ r(\Mat[A]{})\vec{v} \\
			\equiv & \ a_0\vec{v} + a_1\Mat[A]{}\vec{v} + a_2\Mat[A]{2}\vec{v} + \cdots + a_{D-d}\Mat[A]{D-d}\vec{v} \\
			\equiv & \ a_0(\vec{t}_0 + \vec{g}_0) + a_1(\vec{t}_1 + \vec{g}_1) + \cdots + a_{D-d}(\vec{t}_{D-d} + \vec{g}_{D-d}) \\
			\equiv & \ a_0\vec{t}_0 + a_1\vec{t}_1 + \cdots + a_{D-d}\vec{t}_{D-d} + a_0\vec{g}_0 + a_1\vec{g}_1 + \cdots + a_{D-d}\vec{g}_{D-d} \\
			\equiv & \ a_0\vec{t}_0 + a_1\vec{t}_1 + \cdots + a_{D-d}\vec{t}_{D-d} + \vec{0} \\
			\in    & \ S.
		\end{align*}
		Because $m(x)$ is the minimal annihilating polynomial of $S$, this means
		\[
			m(\Mat[A]{})r(\Mat[A]{})\vec{v} \equiv \vec{0} \mod{p}.
		\]
		This implies that an annihilating polynomial of degree $n + D - d$ exists for $\vec{v}$, and so by proposition \ref{prop:cycSpaceDim}, 
		$\dim(\cycsp{\vec{v}}) \leq n + D - d < D$ since $d > n$. This is a contradiction since $\dim(\cycsp{\vec{v}}) = D$. Thus, our assumption that 
		$\dim(S) > \deg(m(x))$ must be false.
	\end{proof}
	
	Proposition \ref{prop:cycSpaceInteriorRestrictions} massively restricts what sorts of cycle space behaviour are possible within another cycle space. For example, say we had
	a cycle space $\cycsp{\vec{v}}$ with minimal annihilating polynomial $(f(x))^3$ for some irreducible $f(x)$, and within $\cycsp{\vec{v}}$ we had two cycle spaces 
	$\cycsp{1}$ and $\cycsp{2}$ with minimal annihilating polynomials $f(x)$ and $(f(x))^2$, respectively. By proposition \ref{prop:cycSpaceInteriorRestrictions}, it must be 
	the case that $\cycsp{1} \subset \cycsp{2}$. Otherwise, $\cycsp{1} + \cycsp{2}$ (an invariant subspace by proposition \ref{prop:cycSpaceSumSubspace}) would have a 
	dimension greater than $2\deg(f(x))$, while the minimal annihilating polynomial would have a degree of exactly $2\deg(f(x))$, and proposition 
	\ref{prop:cycSpaceInteriorRestrictions} guarantees that this isn't possible.
	
	Reading proposition \ref{prop:cycSpaceInteriorRestrictions}, we see what sets apart the intersection of two cycle spaces from other \dq{combinations} of cycle spaces (such 
	as a sum): the intersection is necessarily contained within another cycle space. Thus, the proposition applies, and we find that our initial conjecture is correct; the
	intersection of two cycle spaces must also be a cycle space.
	
	\begin{coro}
		Let $\vec{v}, \vec{w} \in \Z_p^L$, $p$ prime. Then for some vector $\vec{x} \in \Z_p^L$,
		\[
			\cycsp{\vec{v}} \cap \cycsp{\vec{w}} = \cycsp{\vec{x}}
		\]
		under some matrix $\Mat[A]{} \in \Z_p^{L \times L}$.
	\end{coro}
	\begin{proof}
		Let $S = \cycsp{\vec{v}} \cap \cycsp{\vec{w}}$. By proposition \ref{prop:cycSpaceIntersectSubspace}, $S$ is an invariant subspace under $\Mat[A]{}$. As well, by 
		definition of intersections, we have that $S \subseteq \cycsp{\vec{v}}$. Thus, proposition \ref{prop:cycSpaceInteriorRestrictions} tells us that the dimension of $S$
		is equal to the degree of its minimal annihilating polynomial, which we'll label as $d$. By proposition \ref{prop:cycSpaceMandatoryCycSpace}, there must exist some 
		vector $\vec{x} \in S$ whose cycle space has dimension $d$. Because $\cycsp{\vec{x}} \subseteq S$, we have that $\cycsp{\vec{x}} = S$ since their dimensions are the
		same.
	\end{proof}
	
	<<is there anything I can add here? something to close off the section with a bit of wisdom?>>
	
	\bibliographystyle{plainnat}
	\bibliography{refs.bib}
\end{document}
