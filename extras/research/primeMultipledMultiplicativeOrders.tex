
\documentclass[12pt, a4paper, reqno]{amsart}
\usepackage[T1]{fontenc}

\usepackage[parfill]{parskip}

\usepackage[margin=3cm]{geometry}

\usepackage{setspace}
\setstretch{1.25}

\usepackage{natbib}

\usepackage{dsfont}

%Comments?
\usepackage{verbatim}

%Defining all the stuff we need for proofs
\usepackage{amsthm}
%\newtheorem{prop}{Proposition}[section]
\newtheorem{prop}{Proposition}
\renewcommand\qedsymbol{\textbf{QED}}

\newcommand\notdivides{\ \nmid \ }
\newcommand\divides{\ | \ }
\newcommand\lcm[1]{\textup{lcm}(#1)}

\begin{document}
	\begin{prop}
		\label{prop:extensionOrder}
		Let $p$ be an odd prime and $K$ be an algebraic extension field of $\mathds{Z}_p$. Then, the multiplicative order of any element in $K^*$---the set of invertible 
		elements in $K$---divides $p^d - 1$, where $d$ is the degree of the extension.
	\end{prop}
	\begin{proof}
		The finite abelian group $K^*$ under the operation of multiplication modulo $p$ has $p^d - 1$ elements, and thus its order is $p^d - 1$. By Lagrange's Theorem, 
		any element in $K^*$ must have a multiplicative order dividing the group's order. %(https://mathworld.wolfram.com/LagrangesGroupTheorem.html)
	\end{proof}
	\begin{comment}
	\begin{proof}
		Let $A \in \mathds{Z}_{p}^{d \times d}$ be an invertible matrix whose minimal polynomial is $f(x)$. Such matrices can always be created by finding the companion 
		matrix to $f(x)$ and performing a change of basis on it using an arbitrary matrix whose determinant is one. Note that $A$ will satisfy the same algebraic properties
		as $\alpha$, mainly that $f(A) \equiv 0 \bmod{p}$. Therefore, the multiplicative order of $A$ will be equal to the multiplicative order of $\alpha$. (CAN I JUSTIFY 
		THIS MORE FORMALLY?)
		
		Since the minimal polynomial of $A$ is irreducible, the minimal annihilating polynomial under $A$ for all nonzero vectors $\vec{v} \in \mathds{Z}_p^d$ 
		is $f(x)$ (since the minimal annihilating polynomial under $A$ for any vector in a vector space must divide the minimal polynomial of $A$). By the Minimal 
		Polynomial Theorem (via \citet{Patterson2008}), this means every nonzero vector in $\mathds{Z}_p^d$ has the same multiplicative order. Let $r$ be this 
		multiplicative order. There are $p^d - 1$ nonzero vectors in $\mathds{Z}_p^d$, and so for all nonzero vectors to have the same multiplicative order under an 
		invertible matrix, $r \divides p^d - 1$.
		
		As well, since $A^{r}\vec{v} \equiv \vec{v} \bmod{p}$ for all vectors $\vec{v} \in \mathds{Z}_p^d$, this means $A^r \equiv I \bmod{p}$ since the identity is the
		only matrix with this property. Thus, the multiplicative order of $A$ is $r$, and so the multiplicative order of $\alpha$ is also $r$. Therefore, the multiplicative
		order of $\alpha$ divides $p^d - 1$.
	\end{proof}
	\end{comment}
	
	\begin{prop}
		\label{prop:staticMinPoly}
		Let $p$ be an odd prime, $L$ a positive integer, $A \in \mathds{Z}_p^{L \times L}$, and $K$ an algebraic extension field of $\mathds{Z}_p$. Then the minimal 
		polynomial of $A$ over $K$ is the same as over $\mathds{Z}_p$.
	\end{prop}
	\begin{proof}
		We see that $\mathds{Z}_p \subseteq K$, and so the minimal polynomial of $A$ over $K$ must contain the same factors as the minimal polynomial over $\mathds{Z}_p$ if 
		it is to annihilate all vectors in $K^{L \times L}$. As well, the minimal polynomial of $A$ over $\mathds{Z}_p$ is still an annihilating polynomial for $A$ over 
		$K$, so no additional factors are needed for the minimal polynomial of $A$ over $K$.
	\end{proof}
	
	\begin{prop}
		Let $p$ be an odd prime and $L$ be a positive integer. Let $A \in \mathds{Z}_{p}^{L \times L}$ be an invertible matrix modulo $p$. Then the multiplicative order of
		$A$ is a multiple of $p$ if and only if the minimal polynomial of $A$ has a repeated factor.
	\end{prop}
	\begin{proof}
		Let the minimal polynomial of $A$ be represented as $m(x)$, and let $K$ be the splitting field of $m(x)$ over $\mathds{Z}_p$. By proposition 
		\ref{prop:staticMinPoly}, the minimal polynomial of $A$ remains the same over any algebraic extension field of $\mathds{Z}_p$. Thus, we 
		can work over the field $K$ to deduce properties of $m(x)$.
		
		As well, let the multiplicative order of $A$ be denoted as $\omega$.
		
		First, we'll show that $m(x)$ having a repeated factor implies that $\omega$ is a multiple of $p$.
		
		Assume that $m(x)$ has a repeated factor. If $m(x)$ has a repeated factor over $\mathds{Z}_p$, then it'll surely have a repeated linear factor over $K$, so we can 
		write $m(x) = (x - c)^2\mu(x)$, where $c \in K$ and $\mu(x) \in K[X]$.
		
		Now, note that, since $\omega$ is the multiplicative order of $A$, and since $A$ is invertible, we have that
		\begin{alignat*}{2}
			         & A^\omega \equiv I & \mod{p} \\
			\implies & A^\omega - I \equiv 0 & \mod{p}.
		\end{alignat*}
		This shows $x^\omega - 1$ is an annihilating polynomial for $A$. Over $\mathds{Z}_p$, all annihilating polynomials for $A$ are polynomial multiples of the minimal
		polynomial of $A$ (since $\mathds{Z}_p[X]$ is a principle ideal domain), so we have that
		\begin{alignat*}{3}
			         & m(x) & \divides & x^\omega - 1      \\
			\implies & (x - c)^2 & \divides & x^\omega - 1 \\
			\implies & (x - c) & \divides & x^\omega - 1.
		\end{alignat*}
		Dividing $(x^\omega - 1)$ by $(x - c)$, we see that
		\begin{equation}
			\label{eq:firstDivision}
			\frac{x^\omega - 1}{x - c} = \sum_{i\,=\,0}^{\omega-1} c^ix^{\omega - 1 - i}.
		\end{equation}
		For this division to make sense, multiplying the constant term of the quotient by the constant term of the divisor should result in the constant term in the
		dividend. Therefore,
		\begin{align*}
			         & c^{\omega - 1}(-c) \equiv -1 \mod{p} \\
			\implies & c^\omega \equiv 1 \mod{p}.
		\end{align*}
		For this congruence to hold, $\omega$ must be a multiple of the multiplicative order of $c$. Since $c$ is an eigenvalue for $A$, this will always hold true.
		
		We also know that $(x - c)^2 \divides (x^\omega - 1)$, so $(x - c)$ should divide the quotient obtained in equation \ref{eq:firstDivision}:
		\[
			\frac{\sum_{i\,=\,0}^{\omega-1} c^ix^{\omega - 1 - i}}{x - c} = \sum_{i\,=\,0}^{\omega - 2} (i+1)c^ix^{\omega - 2 - i}.
		\]
		Again, multiplying the constant term of the quotient by the constant term of the divisor should result in the constant term in the dividend. Therefore,
		\begin{alignat*}{4}
			         & (\omega - 1)c^{\omega - 2}(-c) \ & \equiv & \ c^{\omega - 1} & \mod{p} \\
			\implies & (1 - \omega)c^{\omega - 1}     \ & \equiv & \ c^{\omega - 1} & \mod{p} \\
		\end{alignat*}
		The only way this congruence can be satisfied is if $(1 - \omega) \equiv 1 \bmod{p}$, which implies that $\omega \equiv 0 \bmod{p}$. So, $\omega$ is a multiple of 
		$p$.
		
		Now, we'll show that $\omega$ being a multiple of $p$ implies $m(x)$ has a repeated factor.
		
		To show this, assume otherwise. That is, assume that $p \divides \omega$, but $m(x)$ does \emph{not} have a repeated factor. Thus, over the splitting field $K$, 
		$m(x)$ can be written as
		\[
			m(x) = \prod_{i\,=\,0}^{\ell} (x - c_i), \quad c_i \in K,
		\]
		for some positive integer $\ell$, where for all $0 \leq i \leq \ell$ and $0 \leq j \leq \ell$, $i \neq j \implies c_i \not\equiv c_j \bmod{p}$. Then, by the Primary 
		Decomposition Theorem, this means the vector space $K^L$ can be decomposed as
		\[
			K^L = \bigoplus_{i\,=\,0}^{\ell} \ker(A - c_iI)
		\]
		where each kernel contains at least one nonzero vector. Due to the fact that each factor in $m(x)$ is linear, each kernel corresponds to an 
		eigenspace with an eigenvalue given by one of $c_i$. Let $E_i = \ker(A - c_iI)$.
		
		Now, by proposition 3 of \citet{Mendivil2012}, there exists at least one nonzero vector from each of $E_i$ we can sum together to obtain a maximal vector, a vector 
		whose multiplicative order equals $\omega$. From this same proposition, we know $\omega = \lcm{\omega_1, \omega_2, \cdots, \omega_\ell}$ where $\omega_i$ is the 
		maximum multiplicative order possible for a vector in $E_i$. 
		
		Since each $E_i$ is an eigenspace, each $\omega_i$ is simply the multiplicative order of the eigenvalue of $E_i$. If the eigenvalue of $E_i$ is in $\mathds{Z}_p$, 
		then by Fermat's Little Theorem we know the multiplicative order of the eigenvalue must divide $p-1$, so $\omega_i$ must also divide $p-1$. Otherwise, if the 
		eigenvalue of $E_i$ is in $K \setminus \mathds{Z}_p$, then proposition \ref{prop:extensionOrder} tells us that the multiplicative order of the eigenvalue must 
		divide $p^d - 1$, where $d$ is the degree of the particular extension where the eigenvalue is obtained from. So, $\omega_i$ must also divide $p^d - 1$.
		
		Importantly, we see that $p$ doesn't divide $p-1$ nor $p^d-1$. This means that $\omega = \lcm{\omega_1, \omega_2, \cdots, \omega_\ell}$ cannot be a multiple of $p$. 
		This is a contradiction since $p \divides \omega$. Thus, our assumption that $m(x)$ doesn't contain a repeated factor must be incorrect.
		
	\end{proof}
	
	\bibliographystyle{plainnat}
	\bibliography{../refs.bib}
\end{document}
