
%Maybe I should specify that the generators of annideals that I'm talking about are dimensionally-independent?

\documentclass[a4paper, reqno, 12pt]{amsart}
\usepackage[T1]{fontenc}
\usepackage[parfill]{parskip}
\usepackage[margin=3cm]{geometry}
\usepackage{setspace}
\setstretch{1.25}

\usepackage{dsfont}

\usepackage{natbib}

\usepackage{verbatim}

%vector span
\newcommand\vecspan[1]{\textup{span}\left(#1\right)}

%small vectors
\newcommand\smallvec[1]{\left[\begin{smallmatrix}#1\end{smallmatrix}\right]}

%does not exist
\newcommand\DNE{\textup{DNE}}


%proposition environment
\newcounter{propcounter}[section]
\newenvironment{proposition}[1]
{
	\refstepcounter{propcounter}
	\textbf{Proposition \thesection.\thepropcounter.} \emph{#1}
	
	\emph{Justification.}
}
{
	\textbf{QED.} \\
}

%addendum environment
\newcounter{addcounter}[section]
\newenvironment{addendum}[1]
{
	\refstepcounter{addcounter}
	\textbf{Addendum \thesection.\theaddcounter.} \emph{#1} \\
}
{
	$\circ$ \\
}

\begin{document}
	For the entirety of this document, let $p$ be a prime. Also let $L$ and $k$ be positive integers with $k > 1$.
	
	\section{Annihilating Polynomials Modulo $p^k$}
		\label{sec:AP}
		Given a vector $\vec{v} \in \mathds{Z}_{p^k}^{L}$ and a matrix $A \in \mathds{Z}^{L \times L}$, an \emph{annihilating polynomial of $\vec{v}$ under $A \bmod{p^k}$} 
		is any polynomial $f(x)$ such that
		\[
			f(A)\vec{v} \equiv \vec{0} \mod{p^k}
		\]
		
		Modulo $p$, a vector $\vec{v}$ has a \emph{minimal annihilating polynomial}, a monic annihilating polynomial that divides every nonzero annihilating polynomial of 
		$\vec{v}$ (see section 2 of \citet{Mendivil2012}).
		
		Our interest in this document is considering what the set of all annihilating polynomials for a given vector $\vec{v}$ will look like.
		
		Immediately, we know the set of all annihilating polynomials for a vector $\vec{v} \in \mathds{Z}_{p^k}^L$ will be an ideal of $\mathds{Z}_{p^k}[X]$, the set of all 
		polynomials with coefficients in $\mathds{Z}_{p^k}$. To see why, assume we have a set of arbitrary annihilating polynomials for $\vec{v}$, denoted as $a_1(x)$, 
		$a_2(x)$, etc. If all of these are annihilating polynomials under a matrix $A$, then for arbitrary polynomials $\alpha_1(x)$, $\alpha_2(x)$, etc.,
		\begin{align*}
				   & \, \alpha_1(A)a_1(A)\vec{v} + \alpha_2(A)a_2(A)\vec{v} + \cdots \\
			\equiv & \, \alpha_1(A)\vec{0} + \alpha_2(A)\vec{0} + \cdots             \\
			\equiv & \, \vec{0}
		\end{align*}
		A sum of arbitrary polynomial multiples of our annihilating polynomials is also an annihilating polynomial. Therefore, the set of annihilating polynomials is closed 
		under addition and taking polynomial multiples, meaning it is an ideal of the set $\mathds{Z}_{p^k}[X]$.
		
		If we concern ourselves only with the modulo $p$ case, then the ideal of annihilating polynomials for $\vec{v}$ becomes a principal ideal since the minimal 
		polynomial divides all nonzero annihilating polynomials, meaning every annihilating polynomial is some polynomial multiple of the minimal polynomial, and therefore 
		the minimal polynomial is the sole generator for the ideal.
		
		In the prime-power case (modulo $p^k$), there can be more than one generator for the set of annihilating polynomials. For example, take the matrix
		\[
			A = \begin{bmatrix}
				2 & 7 & 1 & 6 \\
				0 & 5 & 3 & 6 \\
				4 & 5 & 7 & 5 \\
				5 & 1 & 1 & 2
			\end{bmatrix} \mod{9}
		\]
		The vector 
		\[
			\vec{v} = \begin{bmatrix}
				1 \\
				0 \\
				0 \\
				0
			\end{bmatrix}
		\]
		is annihilated by the polynomials $3 + 6A + 3A^2 + 3A^3$, and $1 + 6A + 2A^3 + A^4$. Neither one of these polynomials is a polynomial multiple of the other, so 
		both of them are needed to generate the vector's ideal of annihilating polynomials.
		
		Is there anything we can say about the generators of the ideal of annihilating polynomials in the prime-power case?
		
		\begin{proposition}{Say we're given $\vec{v} \in \mathds{Z}_{p^k}^{L}$, $A \in \mathds{Z}^{L \times L}$, and $f(x)$, an annihilating polynomial of 
		$\vec{v} \bmod{p^k}$, with leading term $p^{\alpha}a_dx^d$, $\alpha < k$, and where $a_d^{-1}$ exists. Any annihilating polynomial with leading term 
		$p^{\alpha+\beta}b_{d+q}x^{d+q}$, $\beta, q \geq 0$, $\alpha+\beta < k$, $b_{d+q} \not\equiv 0$, can be written as the sum of a polynomial multiple of 
		$f(x)$ and another annihilating polynomial with degree less than $d+q$.}
			\label{prop:splitBiggerPoly}
			Let 
			\[
				f(x) = p^{\alpha}a_dx^d + a_{d-1}x^{d-1} + a_{d-2}x^{d-2} + \cdots + a_2x^2 + a_1x + a_0
			\]
			and let an arbitrary annihilating polynomial matching our assumptions be written as
			\[
				g(x) = p^{\alpha+\beta}b_{d+q}x^{d+q} + b_{d+q-1}x^{d+q-1} + b_{d+q-2}x^{d+q-2} + \cdots + b_2x^2 + b_1x + b_0
			\]
			Multiplying $f(x)$ by $p^{\beta}b_{d+q}a_d^{-1}x^q$ makes its leading term equal to the leading term in $g(x)$:
			\[
				p^{\beta}b_{d+q}a_d^{-1}x^qf(x) = p^{\alpha+\beta}b_{d+q}x^{d+q} + p^{\beta}b_{d+q}a_d^{-1}x^q(a_{d-1}x^{d-1} + a_{d-2}x^{d-2} + \cdots + a_1x + a_0)
			\]
			Therefore, there exists some polynomial $r(x)$ with degree less than $d+q$ such that
			\[
				p^{\beta}b_{d+q}a_d^{-1}x^qf(x) + r(x) = g(x)
			\]
			Also, since $f(x)$ and $g(x)$ are annihilating polynomials, $r(x)$ must also be an annihilating polynomial since
			\begin{center}
				\begin{tabular}{crclclc}
							   & $p^{\beta}b_{d+q}a_d^{-1}x^qf(A)\vec{v}$ & $+$ & $r(A)\vec{v}$ & $=$ & $g(A)\vec{v}$ & $\mod{p^k}$ \\
					$\implies$ &                                $\vec{0}$ & $+$ & $r(A)\vec{v}$ & $=$ & $\vec{0}$     & $\mod{p^k}$
				\end{tabular}
			\end{center}
		\end{proposition}
		
		\begin{proposition}{Modulo $p^k$, a maximum of one polynomial per degree is needed as a generator for a vector's ideal of annihilating polynomials---namely, a 
		polynomial with the fewest factors of $p$ on its leading term's coefficient.}
			\label{prop:oneGenPerDegree}
			Assume we're given an annihilating polynomial $q(x)$ of degree $n$ with the fewest number of factors of $p$ on its leading term's coefficient (when compared to 
			all other degree $n$ annihilating polynomials), say $p^{\alpha}$. Proposition \ref{sec:AP}.\ref{prop:splitBiggerPoly} then says that any other annihilating 
			polynomial $s(x)$ of degree $n$ with at least a factor of $p^\alpha$ on its leading term's coefficient can be written as
			\[
				s(x) = aq(x) + r(x)
			\]
			where $a \in \mathds{Z}_{p^k}$ and where $r(x)$ is at most an $n-1$ degree annihilating polynomial. Therefore, any polynomial of degree $n$ can be written as 
			the sum of a scaled $q(x)$ and a smaller degree annihilating polynomial, meaning $q(x)$ is the only degree $n$ polynomial needed as a generator for the ideal of
			annihilating polynomials.
		\end{proposition}
		
		\begin{proposition}{Let $q(x)$ be an annihilating polynomial for some vector of degree $n$ with a factor of $p^\alpha$ in its leading term's coefficient. Any 
		annihilating polynomial of degree $n+1$ with at least a factor of $p^\alpha$ in its leading term's coefficient can't be a generator for the ideal of annihilating 
		polynomials for the given vector.}
			\label{prop:sameFactorHigherPoly}
			Let
			\[
				q(x) = a_np^\alpha x^n + \sum_{i\,=\,0}^{n-1} a_ix^i
			\]
			with $a_n$ being invertible, and let
			\[
				s(x) = b_{n+1}p^\alpha x^{n+1} + \sum_{i\,=\,0}^{n} b_ix^i
			\]
			be an arbitrary annihilating polynomial for our vector with degree $n+1$ that has at least a factor of $p^\alpha$ in its leading term's coefficient. By 
			proposition \ref{sec:AP}.\ref{prop:splitBiggerPoly}, we can write $s(x)$ as
			\[
				s(x) = b_{n+1}a_n^{-1}xq(x) + r(x)
			\]
			with $r(x)$ an annihilating polynomial for the vector of degree at most $n$. Since $q(x)$ and $r(x)$ are both degree $n$ annihilating polynomials, they must be
			generated by polynomials in the ideal of annihilating polynomials with at most degree $n$. Therefore, $s(x)$ cannot possibly be a generator for the ideal since
			it can be represented as the sum of annihilating polynomials that are generated by generators with at most degree $n$.
		\end{proposition}
		
		Proposition \ref{sec:AP}.\ref{prop:sameFactorHigherPoly} shows that each new higher-degree generator we find for the ideal of annihilating polynomials of some 
		vector must have a lesser number of factors of $p$ on its leading term's coefficient. Otherwise, it can't possibly be a new generator.
		
		Before we continue, we must define a mapping which will prove useful in the next proposition. Define 
		$\phi_n : \mathds{Z}_{p^k}[X] \rightarrow p^n\mathds{Z}_{p^{k+n}}[X]$ as $\phi_n(f(x)) = p^nf(x)$. This mapping is injective since
		\begin{center}
			\begin{tabular}{ccccl}
						   & $\phi_n(f(x))$      & $\equiv$ & $\phi_n(g(x))$ & $\mod{p^{k+n}}$ \\
				$\implies$ & $p^nf(x)$           & $\equiv$ & $p^ng(x)$      & $\mod{p^{k+n}}$ \\
				$\implies$ & $p^nf(x) - p^ng(x)$ & $\equiv$ & $0$            & $\mod{p^{k+n}}$ \\
				$\implies$ & $p^n(f(x) - g(x))$  & $\equiv$ & $0$            & $\mod{p^{k+n}}$ \\
				$\implies$ & $f(x) - g(x)$       & $\equiv$ & $0$            & $\mod{p^k}$     \\
				$\implies$ & $f(x)$              & $\equiv$ & $g(x)$         & $\mod{p^k}$
			\end{tabular}
		\end{center}
		
		The mapping is also surjective since for each $p^ng(x) \in p^n\mathds{Z}_{p^{k+n}}[X]$, there exists a $g(x) \in \mathds{Z}_{p^k}[X]$ such that 
		$\phi_n(g(x)) = p^ng(x) \bmod{p^{k+n}}$. Therefore, $\phi_n$ is an isomorphism between $\mathds{Z}_{p^k}[X]$ and $p^n\mathds{Z}_{p^{k+n}}[X]$.
		
		\begin{proposition}{Given a vector $\vec{v} \in \mathds{Z}_{p^k}^{L}$ and a matrix $A \in \mathds{Z}^{L \times L}$, if the minimal annihilating polynomial of 
		$\vec{v}$ has degree $d$, then the smallest degree annihilating polynomials for $\vec{v} \bmod{p^k}$ will always be of degree $d$.}
			\label{prop:smallestAnnihDegree}
			First, we'll show that the degree of any annihilating polynomial for $\vec{v} \bmod{p^k}$ must have a degree of at least $d$. Assume that there exists some 
			nonzero annihilating polynomial $f(x) \bmod{p^k}$ with degree less than $d$. There are two cases to consider.
			
			\emph{Case 1 - At least one term in $f(x)$ does not have a factor of $p$ in its coefficient.} In this case, we can make use of the fact that congruences mod 
			$p^k$ must also hold mod $p$. Since one term has no factors of $p$, $f(x)$ will reduce modulo $p$ to a 
			nonzero polynomial. Therefore,
			\[
				f(A)\vec{v} \equiv \vec{0} \mod{p^k} \implies f(A)\vec{v} \equiv \vec{0} \mod{p}
			\]
			This isn't possible since all nonzero annihilating polynomials mod $p$ must be polynomial multiples of the minimal polynomial, and the degree of $f(x)$ is too 
			low to be such a multiple.
			
			\emph{Case 2 - All terms in $f(x)$ have at least one factor of $p$ in their coefficients.} In this case, we can use the $\phi_n$ isomorphism to "unembed" the 
			polynomial into some lower modulus. Symbolically, if $p^\alpha g(x) = f(x)$, where $g(x)$ has no factors of $p$ in at least one of its coefficients, and 
			$z(x) = 0$, then
			\begin{center}
				\begin{tabular}{crcll}
							   &                     $p^\alpha g(A)\vec{v}$ & $\equiv$ & $\vec{0}$                         & $\mod{p^k}$            \\
					$\implies$ &                     $p^\alpha g(A)\vec{v}$ & $\equiv$ & $z(A)\vec{v}$                     & $\mod{p^k}$            \\
					$\implies$ & $\phi_{\alpha}^{-1}(p^\alpha g(A))\vec{v}$ & $\equiv$ & $\phi_{\alpha}^{-1}(z(A))\vec{v}$ & $\mod{p^k}$            \\
					$\implies$ &                              $g(A)\vec{v}$ & $\equiv$ & $z(A)\vec{v}$                     & $\mod{p^{k - \alpha}}$ \\
					$\implies$ &                              $g(A)\vec{v}$ & $\equiv$ & $\vec{0}$                         & $\mod{p^{k - \alpha}}$
				\end{tabular}
			\end{center}
			The last implication puts us back in case 1, so
			\[
				g(A)\vec{v} \equiv \vec{0} \mod{p^{k - \alpha}} \implies g(A)\vec{v} \equiv \vec{0} \mod{p}
			\]
			As with case 1, this isn't possible since all nonzero annihilating polynomials mod $p$ must be polynomial multiples of the minimal polynomial, and the degree 
			of $g(x)$ is too low to be such a multiple.
			
			Both cases lead to a contradiction, meaning our assumption that such a $f(x)$ exists is false. Therefore, there is no nonzero annihilating polynomial mod $p^k$ 
			with degree less than $d$.
			
			Next, we'll show that there will always exist an annihilating polynomial mod $p^k$ with a degree of $d$. Let the minimal annihilating polynomial of $\vec{v}$ mod
			$p$ be denoted as $m(x)$. Then
			\[
				m(A)\vec{v} \equiv \vec{0} \mod{p}
			\]
			Using the $\phi_n$ isomorphism, we can embed this relation to higher power moduli. The above relation becomes
			\[
				p^{k-1}m(A)\vec{v} \equiv \vec{0} \mod{p^k}
			\]
			meaning the polynomial $p^{k-1}m(x)$ is an annihilating polynomial of degree $d$ modulo $p^k$. Since all annihilating polynomials must have a degree of at least $
			d$, this means the smallest degree annihilating polynomials modulo $p^k$ will always have a degree of $d$.
		\end{proposition}
		
		Proposition \ref{sec:AP}.\ref{prop:smallestAnnihDegree} shows that, when looking for generators for ideals of annihilating polynomials modulo $p^k$, we don't need to
		look through polynomials of degree less than the relevant vector's minimal polynomial modulo $p$.
		
	\section{Predicting Generators}
		\label{sec:PG}
		Consider an arbitrary vector $\vec{v} \in \mathds{Z}_{p^k}^L$ and an arbitrary matrix $A \in \mathds{Z}^{L \times L}$. It turns out that the relationship between the
		vectors $\vec{v},\, A\vec{v},\, A^2\vec{v},\,$ etc., determines how many generators are needed for a particular vector's ideal of annihilating polynomials. The 
		following propositions will shed light on this.
		
		\begin{proposition}{If and only if $p^jA^n\vec{v}$, $0 \leq j < k$, nonnegative $n \in \mathds{Z}$, can be represented as a linear combination of the vectors 
		$\vec{v},\, A\vec{v},\, A^2\vec{v},\, \cdots,\, A^{n-1}\vec{v}$, then there will exist an annihilating polynomial for $\vec{v}$ of degree $n$ with the coefficient 
		on its leading term having a factor of $p^j$.}
			\label{prop:annihPolyExistence}
			First, we'll show $p^jA^n\vec{v}$ having a linear combination representation leads to an annihilating polynomial existing.
			
			Let
			\[
				p^jA^n\vec{v} \equiv \sum_{i\,=\,0}^{n-1} a_iA^i\vec{v}
			\]
			for constants $a_i \in \mathds{Z}_{p^k}$. Then
			\[
				p^jA^n\vec{v} - \sum_{i\,=\,0}^{n-1} a_iA^i\vec{v} \equiv \vec{0}
			\]
			and so
			\[
				f(x) = p^jx^n - \sum_{i\,=\,0}^{n-1} a_ix^i
			\]
			is an annihilating polynomial of degree $n$ for $\vec{v}$ with a factor of $p^j$ on the coefficient of its leading term.
			
			Next, we'll show the annihilating polynomial existing leads to the linear combination representation for $p^jA^n\vec{v}$.
			
			Let 
			\[
				f(x) = p^jx^n - \sum_{i\,=\,0}^{n-1} a_ix^i
			\]
			be an annihilating polynomial for $\vec{v}$. Then
			\[
				f(A)\vec{v} = p^jA^n\vec{v} - \left(\sum_{i\,=\,0}^{n-1} a_iA^i \right)\vec{v} \equiv \vec{0}
			\]
			and so
			\[
				p^jA^n\vec{v} \equiv \sum_{i\,=\,0}^{n-1} a_iA^i\vec{v}
			\]
			so $p^jA^n\vec{v}$ can be written as a linear combination of the vectors $\vec{v}$, $A\vec{v}$, $A^2\vec{v}$, $\cdots$, $A^{n-1}\vec{v}$.
		\end{proposition}
		
		Note that if $\alpha p^jA^n\vec{v}$ can be represented as a linear combination, with $\alpha$ being invertible, then $p^jA^n\vec{v}$ can also be represented as a 
		linear combination (and vice versa) since $\alpha^{-1}$ can simply be multiplied to remove the constant. Therefore, we only need to concern ourselves with the 
		representations of prime-power multiples of iterations of $\vec{v}$ when applying proposition \ref{sec:PG}.\ref{prop:annihPolyExistence}.
		%Before we make use of proposition 2.1, it's important to establish...
		Proposition \ref{sec:PG}.\ref{prop:annihPolyExistence} proves especially useful in predicting the ideal generators for a vector's annihilating polynomials when 
		lifting said vector to a higher-power modulus. First, though, it's important to establish some notation that'll help us work with prime-power multiples of a 
		particular vector.
		
		For an arbitrary vector $\vec{v} \in \mathds{Z}_{p^k}^{L}$ and an arbitrary matrix $A \in \mathds{Z}^{L \times L}$, let $p^{j_n}$ represent the smallest power of $p$
		such that 
		\[
			p^{j_n}A^n\vec{v} \in \vecspan{\{\vec{v},\, A\vec{v},\, A^2\vec{v},\, \cdots,\, A^{n-1}\vec{v}\}}
		\]
		and let $P_{\vec{v}} = [j_0,\, j_1,\, j_2,\, \cdots]$ be an ordered list of the exponents of each $p^{j_n}$ (in order of increasing $n$). If no such $p^{j_n}$ exists 
		for a given $n$, then set $j_n$ as DNE in $P_{\vec{v}}$.
		
		\begin{proposition}{Let $j_m$ represent the $m$-th element of $P_{\vec{v}}$ for some $\vec{v} \in \mathds{Z}_{p^k}^L$ and some $A \in \mathds{Z}^{L \times L}$. 
		If $j_m$ exists, then $j_{m+1} \leq j_m$.}
			\label{prop:monotonicP}
			If $j_m$ exists, then we have
			\[
				p^{j_m}A^m\vec{v} \in \vecspan{\{\vec{v},\, A\vec{v},\, A^2\vec{v},\, \cdots,\, A^{m-1}\vec{v}\}}
			\]
			By proposition \ref{sec:PG}.\ref{prop:annihPolyExistence}, an annihilating polynomial of degree $m$ exists with a factor of $p^{j_m}$ in the leading term's 
			coefficient. Denote it as
			\[
				f(x) = p^{j_m}x^m + \sum_{i\,=\,0}^{m-1} a_ix^i
			\]
			for constants $a_i \in \mathds{Z}_{p^k}$. $f(x)$ is an annihilating polynomial for $\vec{v}$, so
			\[
				f(A)\vec{v} \equiv \vec{0}
			\]
			Then
			\[
				Af(A)\vec{v} \equiv \vec{0}
			\]
			which implies
			\[
				xf(x) = p^{j_m}x^{m+1} + \sum_{i\,=\,0}^{m-1} a_ix^{i+1}
			\]
			is also an annihilating polynomial of $\vec{v}$. By proposition 2.1, this implies
			\[
				p^{j_m}A^{m+1}\vec{v} \in \vecspan{\{\vec{v},\, A\vec{v},\, A^2\vec{v},\, \cdots,\, A^m\vec{v}\}}
			\]
			So then the value of $j_{m+1}$ must be at most the value of $j_m$.
		\end{proposition}
		
		Proposition \ref{sec:PG}.\ref{prop:monotonicP} shows that the sequence of all existing elements in $P_{\vec{v}}$ is nonincreasing. 
		
		Finally, we have enough to describe a general process for finding all the generators of a vector's ideal of annihilating polynomials.
		
		First, generate $P_{\vec{v}}$ for your given vector $\vec{v} \in \mathds{Z}_{p^k}^L$. Let $j_n$ index the $n$-th element of $P_{\vec{v}}$. Next, find the first 
		value in $P_{\vec{v}}$ that exists. By proposition \ref{sec:AP}.\ref{prop:smallestAnnihDegree}, this should occur at $j_d$, where $d$ is the degree of the minimal 
		annihilating polynomial of $\vec{v}$. Then, proposition \ref{sec:PG}.\ref{prop:annihPolyExistence} guarantees that, using the expression for $p^{j_d}A^d\vec{v}$ 
		written as a linear combination of the terms $A^0\vec{v}$ to $A^{d-1}\vec{v}$, an annihilating polynomial of degree $d$ can be formed. Since $p^{j_d}$ is the least 
		number of factors of $p$ needed to multiply $A^d\vec{v}$ by in order to find its linear combination in terms of the other iterations of $\vec{v}$, we can be sure 
		that no annihilating polynomials exist of degree $d$ with less factors of $p$ on its leading term's coefficient.
		
		Proposition \ref{sec:AP}.\ref{prop:oneGenPerDegree} tells us that we only need one generator per degree, so we can safely start looking for higher-degree 
		annihilating polynomials once we find one of a particular degree. Repeatedly applying proposition \ref{sec:AP}.\ref{prop:sameFactorHigherPoly}, we see that any 
		higher-degree generator must have less factors of $p$ on its leading term than all the generators that came before it. This can only occur for degrees where the 
		corresponding value in the sequence $P_{\vec{v}}$ decreases. Therefore, via proposition \ref{sec:PG}.\ref{prop:annihPolyExistence}, if $j_n$ represents a value in 
		$P_{\vec{v}}$ that's smaller than all the values that came before it in the sequence, then an annihilating polynomial of degree $n$ can be formed in the same manner 
		as above, and this polynomial will serve as a generator for the ideal. This same logic can be applied repeatedly to find all generators.
		
		Once we find a $j_i$ where $j_i = 0$, that'll mark the last generator needed since having no factors of $p$ on a leading term's coefficient is the lowest number of 
		factors of $p$ possible; the entries of $P_{\vec{v}}$ can't continue to decrease after this. $P_{\vec{v}}$ must eventually have a value of zero somewhere in its 
		sequence since by proposition \ref{sec:PG}.\ref{prop:monotonicP}, $P_{\vec{v}}$ is nonincreasing, and eventually, any vector $\vec{v}$ will fall into a cycle, 
		meaning for some powers of $A$, we'll have
		\[
			A^x\vec{v} \equiv A^y\vec{v}, \quad x,y \in \mathds{Z}^+ \cup \{0\}, \quad y < x
		\]
		and so $A^x\vec{v} \in \vecspan{A^y\vec{v}}$, meaning $j_x = 0$. This is enough to show a last generator will always exist.
		
		This process can be implemented fairly easily into computer software to find the generators of a vector's ideal of annihilating polynomials. This process is much
		faster than manually checking every possible polynomial to see if it's annihilating; all this process requires is iterating the vector $\vec{v}$ to obtain its
		iterations, then using matrix row reduction with these vectors to find how to write each relevant vector iteration (or a prime-power multiple of it) in terms of the
		other vectors.
		
		This method of finding generators for an ideal of annihilating polynomials has some interesting implications.
		
		\begin{proposition}{Modulo $p^k$, the maximum number of generators for the ideal of annihilating polynomials of a vector $\vec{v}$ is $k$.}	
			\label{prop:maxGens}
			Via proposition \ref{sec:AP}.\ref{prop:sameFactorHigherPoly}, we know that if a generator exists of degree $d$, then any generators with degree $> d$ must have 
			a lesser factor of $p$ on its leading term's coefficient. Therefore, the maximum number of new generators we can find is the maximum number of numbers we can 
			list where the number of factors of $p$ decreases with each term. Modulo $p^k$, the maximum number of terms in such a sequence is $k$.
		\end{proposition}
		
	\section{Miscellaneous}
		\label{sec:M}
		It is very much possible for a vector modulo $p^k$ to have more than one generator for its ideal of annihilating polynomials (as we've seen), though it's also
		possible for it to have \emph{only} one. If, for a vector $\vec{v} \in \mathds{Z}_{p^k}^L$, $P_{\vec{v}}$ were to look something like
		\[
			P_{\vec{v}} = [\DNE,\, \DNE,\, \cdots,\, \DNE,\, 0,\, 0,\, 0,\, \cdots]
		\]
		then the vector $\vec{v}$ would have only one generator. In such a case, we can make use of tools previously reserved only for when the modulus is prime. 
		
		\begin{proposition}{At least one generator for a vector's ideal of annihilating polynomials must be monic.}
			\label{prop:monicExistence}
			For any vector $\vec{v} \in \mathds{Z}_{p^k}^L$ and a given $A \in \mathds{Z}^{L \times L}$, $\vec{v}$ is guaranteed to eventually fall into a cycle under 
			iteration by $A$ (since there are a finite number of vectors in $\mathds{Z}_{p^k}^L$). Therefore, for some nonnegative $c,f \in \mathds{Z}$, $c > f$, we have that
			\[
				A^c\vec{v} \equiv A^f\vec{v}
			\]
			and so
			\[
				A^c\vec{v} - A^f\vec{v} \equiv \vec{0}
			\]
			which means the polynomial $g(x) = x^c - x^f$ is an annihilating polynomial for $\vec{v}$. No sum of polynomials with factors of $p$ in their leading terms' 
			coefficients could ever construct such a polynomial since polynomials with invertible leading term coefficients would be needed to create the $x^c$ term. 
			Therefore, a monic annihilating polynomial is always guaranteed as one of the generators in a vector's ideal of annihilating polynomials.
		\end{proposition}
		
		Proposition \ref{sec:M}.\ref{prop:monicExistence} is straightforward, maybe even obvious, yet it allows for a very strong tool to be used that was previously only
		applied to when the modulus is prime. If a vector has only one generator in its ideal of annihilating polynomials, then proposition 
		\ref{sec:M}.\ref{prop:monicExistence} guarantees that the generator is monic. This allows for the \emph{order} of the generator to be found (as defined in 
		\citet{Patterson2008}). As such, the Minimal Polynomial Theorem outlined in \citet{Patterson2008} can be applied to vectors whose ideal of annihilating polynomials 
		has only a single generator. Nowhere in the proof does it make use of the fact that the arithmetic is being done modulo a prime. It relies on two facts: that 
		the ideal of annihilating polynomials is principal, and that the order of the ideal's generator can be calculated. As long as the vector's ideal of annihilating 
		polynomials has a single generator, these two facts hold.
	
	\bibliographystyle{plainnat}
	\bibliography{refs.bib}
\end{document}
