
\documentclass[a4paper, reqno, 12pt]{amsart}

\usepackage[T1]{fontenc}
\usepackage[margin=3cm]{geometry}
\usepackage[parfill]{parskip}

\usepackage{setspace}
\setstretch{1.25}

\usepackage{dsfont}
%\usepackage{amsmath}

\usepackage{natbib}

\begin{document}
	\section{Preamble}
		This document outlines an idea for proving that all invertible LCAs with prime-power moduli have vectors with maximal cycle length. That is, there should 
		always exist a vector whose cycle length is equivalent to the matrix's cycle length. This idea should also apply to non-invertible systems if 
		\citet{Strong2022core} is correct in asserting that the cores of all prime-powered LCAs are isomorphic to invertible prime-powered LCAs. 
		
		This idea is based on a previous, weaker idea for proving maximal cycle length vectors exist. The earlier idea only applies to LCAs with matrices of 
		particular forms. This idea, \emph{in theory}, should apply to all LCAs with prime-power moduli. The setup is quite similar to the previous idea, but with a 
		few adjustments that \emph{hopefully} solve the problems encountered with the last approach.
		The next section outlines some terms and definitions that are used throughout the document. The third section goes over the idea itself. 
		
		As of now, this isn't a proof since no one else has looked it over and it probably isn't very formal.
		
	\section{Terms and Definitions}
		\begin{description}
			\item[$\mathds{Z}_{m}$] The set of all integers modulo $m$.
			
			\item[$\mathds{Z}_{m}^{L}$] The set of all $L \times 1$ vectors with components in $\mathds{Z}_{m}$.
			
			\item[$\mathds{Z}_{m}^{L \times L}$] The set of all $L \times L$ matrices with components in $\mathds{Z}_{m}$.
			
			\item[$R$-Module] An $R$-module is an object consisting of some abelian group $M$ (which is usually referred to as the module on its own) and some operation
			that maps an element from $R$ and an element from $M$ to another element in $M$. For this document, an $R$-module can be thought of as a vector space with
			more relaxed conditions.
			
			\item[Linear Cellular Automata (LCA)] An LCA is an object defined by a modulus $m \in \mathds{Z} : m > 1$, the $\mathds{Z}_{m}$-module 
			$\mathds{Z}_{m}^{L}$ with $L \in \mathds{Z}^{+}$, and a matrix $A \in \mathds{Z}_{m}^{L \times L}$ that acts on the vectors in the module.
			
			\item[Prime-Power(ed) LCA] An LCA with a power of a prime as its modulus. Usually, the case of $p^1$ being the modulus, $p$ being a prime, isn't
			included in this case.
			
			\item[Cycle Length] The cycle length of a matrix $A$ is the smallest positive integer $\omega$ such that $A^{\omega} \equiv I \bmod{m}$, where $m$ is
			the relevant modulus. The cycle length of a vector $\vec{v}$ under the matrix $A$ is the smallest positive integer $\omega$ such that 
			$A^{\omega}\vec{v} \equiv \vec{v} \bmod{m}$. The relevant modulus will be obvious from context. 
			
			\item[Maximal Cycle Length] The maximal cycle length of an LCA is the cycle length of its matrix.
			
			\item[Iteration] The process of multiplying a vector by some update matrix.
		\end{description}
		
	\section{The Idea}
		Say we're given a matrix $A \in \mathds{Z}_{p}^{L \times L}$ with cycle length $\omega$. We'll also assume $A$ to be invertible. From Proposition 3 in \citet{Mendivil2012}, we know at least one
		vector of maximal cycle length will exist in this case. Let's name one of these maximal cycle length vectors $\vec{v}$. If our modulus is $p$, with $p$ being 
		prime, we have that
		\[
			A^{\omega} \equiv I \mod{p}
		\]
		
		If we increase our modulus to $p^2$, the above statement becomes
		\[
			A^{\omega} \equiv I + pB \mod{p^2}, \quad B \in \mathds{Z}_{p}^{L \times L}
		\]
		
		There are two possibilities in this case: either $B \equiv 0$, or $B \not\equiv 0$. If $B \equiv 0$, then the matrix $A$ must have the same cycle length
		modulo $p^{2}$ as it did modulo $p$: $\omega$. If this is the case, then the vector $p\vec{v}$ will have maximal cycle length modulo $p^{2}$. To see why,
		we can define a mapping $\phi : \mathds{Z}_{p}^{L} \rightarrow \mathds{Z}_{p^{2}}^{L}, \, \phi(\vec{a}) = p\vec{a}$. This mapping preserves matrix 
		multiplication, as
		\[
			A\phi(\vec{a}) = A(p\vec{a}) = p(A\vec{a}) = \phi(A\vec{a})
		\]
		so $\phi$ is a group homomorphism between $\mathds{Z}_{p}^{L}$ and $\mathds{Z}_{p^{2}}^{L}$. This means cycle lengths are preserved under $\phi$, since if
		$\alpha$ is the cycle length of $\vec{a}$ modulo $p$, then
		\[
			A^{\alpha}\phi(\vec{a}) = \phi(A^{\alpha}\vec{a}) = \phi(\vec{a})
		\]
		Therefore, if $B \equiv 0$, $p\vec{v}$ will have maximal cycle length.
		
		Continuing this case, assume
		\[
			A^{\omega} \equiv I \mod{p^3}
		\]
		so $B \equiv 0 \bmod{p^3}$ as well. The exact same reasoning can be applied again, leading to the vector ${p^2}\vec{v}$ having maximal cycle length in this
		case.
		
		In fact, if $A^{\omega} \equiv I \bmod{p^h}$, then the vector $p^{h-1}\vec{v}$ will have maximal cycle length. So, for prime-powered LCAs with $B \equiv 0$,
		a maximal cycle length vector will always exist.
		
		What if $B \not\equiv 0$? Let $p^{k}$ be the first prime-power modulus so that
		\[
			A^{\omega} \equiv I + p^{k-1}B \mod{p^k}, \quad B \in \mathds{Z}_{p}^{L \times L}, \, B \not\equiv 0
		\]
		If this is the case, then the cycle length of $A$ modulo $p^k$ must be $p\omega$. To see this, consider calculating $A^{2\omega}$:
		\begin{align*}
			A^{2\omega} \, &\equiv (I + p^{k-1}B)^2 \\
			            \, &\equiv I + 2p^{k-1}B + p^{2k-2}B^2 \\
						\, &\equiv I + 2p^{k-1}B \mod{p^k}
		\end{align*}
		Calculating $A^{2\omega}$ results in the $B$ term being multiplied by 2. We can use induction to form a general rule:
		\begin{align*}
			A^{(b+1)\omega} \, &\equiv (I + bp^{k-1}B)(I + p^{k-1}B) \\
			                \, &\equiv I + bp^{k-1}B + p^{k-1}B + bp^{2k-2}B^{2} \\
							\, &\equiv I + (b+1)p^{k-1}B \mod{p^k}
		\end{align*}
		By induction, then,
		\[
			A^{b\omega} \equiv I + bp^{k-1}B
		\]
		and
		\[
			A^{p\omega} \equiv I + pp^{k-1}B \equiv I + p^{k}B \equiv I \mod{p^k}
		\]
		Since the cycle length of $A$ modulo $p^k$ must be some multiple of $\omega$ (since if it wasn't, the modular reduction down to modulo $p^{k-1}$ wouldn't 
		work), and since $p\omega$ is the first multiple that iterates $A$ to $I$, the cycle length of $A$ must be $p\omega$ modulo $p^k$.
		
		Now, to find a vector of maximal cycle length in this case, we need to find a vector with cycle length $p\omega$. We don't immediately know of any vectors
		with this cycle length, but we do know that the vector $p^{k-1}\vec{v}$ has cycle length $\omega$ (based on the homomorphism argument above). We can also 
		guarantee that at least one vector must have a cycle length that \emph{doesn't} divide $\omega$. To see why, consider the expression
		\[
			A^{\omega} \equiv I + p^{k-1}B \mod{p^k}, \quad B \in \mathds{Z}_{p}^{L \times L}
		\]
		from above. We know $B \not\equiv 0$, so at least one vector does not get sent to $\vec{0}$ under iteration by $B$. Let's call this vector $\vec{u}$,
		where $\vec{u} \in \mathds{Z}_{p}^{L}$ and $B\vec{u} \not\equiv \vec{0} \bmod{p}$. Then
		\begin{align*}
			A^{\omega}\vec{u} \, &\equiv (I + p^{k-1}B)\vec{u} \\
							  \, &\equiv \vec{u} + p^{k-1}B\vec{u} \mod{p^k}
		\end{align*}
		$p^{k-1}B\vec{u}$ cannot be $\vec{0}$, which shows that $\vec{u}$ cannot have a cycle length that divides $\omega$. If it did, then iterating $\omega$ times
		would bring the vector back to where it started.
		
		Now, it's possible that $\vec{u}$ already has the maximal cycle length of $p\omega$, in which case we don't need to do anything else. For the rest of this 
		argument, we'll assume that it doesn't have maximal cycle length.
		
		We now have $p^{k-1}\vec{v}$, a vector with cycle length $\omega$, and $\vec{u}$, a vector whose cycle length doesn't divide $\omega$. What happens if we
		add the two vectors? Let $\vec{m} = \vec{u} + p^{k-1}\vec{v}$. Can we say anything about the cycle length of $\vec{m}$?
		
		Well, we can guarantee that the cycle length of $\vec{m}$ doesn't divide $\omega$ in the same way we proved it for $\vec{u}$:
		\begin{align*}
			A^{\omega}\vec{m} \, &\equiv (I + p^{k-1}B)(\vec{u} + p^{k-1}\vec{v}) \\
			                  \, &\equiv \vec{u} + p^{k-1}B\vec{u} + p^{k-1}\vec{v} + p^{2k-2}B\vec{v} \\
							  \, &\equiv \vec{m} + p^{k-1}B\vec{u} \mod{p^k}
		\end{align*}
		As before, $p^{k-1}B\vec{u}$ cannot be $\vec{0}$, so $\vec{m}$ cannot have a cycle length that divides $\omega$.
		
		We can also show that $\vec{m}$ cannot have a cycle length that divides $p\omega$ but isn't a factor/multiple of $\omega$. To see why, consider iterating
		$\vec{m}$ by $A$ for any number of iterations:
		\[
			A^{c}\vec{m} \equiv A^{c}\vec{u} + p^{k-1}A^{c}\vec{v} \mod{p^k}
		\]
		What's important to note about this expression is that it'll result in two terms: one multiplied by $p^{k-1}$ and one that isn't. The result of iterating the 
		term which doesn't have the $p^{k-1}$ factor will affect the term that does have the $p^{k-1}$ term, \emph{but not vice versa}. The term which doesn't have a
		$p^{k-1}$ factor will iterate independently of the term which does have the $p^{k-1}$ factor.
		
		How will the term without the $p^{k-1}$ factor iterate? We can't be entirely sure how this term will affect the term with the $p^{k-1}$ factor, but we
		know exactly how this term alone, separated from everything else, will iterate. It'll iterate exactly how $\vec{u}$ iterates modulo $p^{k-1}$; any other bits 
		will be absorbed by the term with the $p^{k-1}$ factor. Now, we don't know a whole lot about how $\vec{u}$ iterates modulo $p^{k-1}$, but we know one thing: 
		it \emph{must} have a cycle length that divides $\omega$, since the cycle length of $A$ modulo $p^{k-1}$ is $\omega$.
		
		What does all this mean? Because we know that the term without the $p^{k-1}$ factor has a cycle length that divides $\omega$, any number of iterations that
		isn't a factor/multiple of $\omega$ can't possibly be the cycle length of $\vec{m}$, since its term without the $p^{k-1}$ term wouldn't have cycled back to
		where it has to be. This directly shows that the cycle length of $\vec{m}$ cannot divide $p\omega$ and not be a factor/multiple of $\omega$ at the same time.
		
		Combining our results, this shows that the cycle length of $\vec{m}$ is restricted to one possible value: $p\omega$. Therefore, $\vec{m}$ is a vector with
		maximal cycle length.
		
		What about for modulo $p^{k+1}$ and beyond? This is fairly straightforward. Simply construct a new vector $\vec{m}_2 = \vec{u}_2 + p\vec{m}$ where the
		cycle length of $\vec{u}_2$ doesn't divide $p\omega$, the cycle length of $A$ under the previous modulus. $\vec{u}_2$ is guaranteed to exist in the same
		way that $\vec{u}$ was guaranteed to exist modulo $p^k$. The logic used modulo $p^k$ directly translates to $p^{k+1}$. With this, induction can be used
		to prove that any prime-power LCA has at least one vector of maximal cycle length.
		
	\bibliographystyle{plainnat}
	\bibliography{refs.bib}
\end{document}