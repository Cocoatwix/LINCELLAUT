
\documentclass[a4paper, reqno, 12pt]{amsart}

\usepackage[T1]{fontenc}
\usepackage[margin=3cm]{geometry}
\usepackage[parfill]{parskip}

\usepackage{setspace}
\setstretch{1.25}

\begin{document}
	From a few examples I've tried, it seems like CCMs may have the ability to give us insight into which cycle lengths exist in a given LCA system. Below is an example of
	a particular matrix that shows most of what I've noticed. I don't know whether any of this actually means anything or not--it could very well be the case that everything
	I've found up to now could be a coincidence.
	
	Let's say $A = 
	\begin{bmatrix}
		\begin{smallmatrix}
			2 & 2 & 3 \\
			0 & 6 & 8 \\
			8 & 6 & 6
		\end{smallmatrix}
	\end{bmatrix}$ mod 9. This matrix has a multiplicative order of 24, and it creates vectors with cycle lengths 1, 3, 8, and 24. It turns out that creating CCMs for the
	matrix $A$ tells us a great deal (in this case, at least) about the possible cycle lengths.
	
	For instance, $C_{8 \rightarrow 1} = A^0 + A^8 + A^{16} \equiv 
	\begin{bmatrix}
		\begin{smallmatrix}
			4 & 2 & 4 \\
			4 & 5 & 1 \\
			8 & 4 & 2
		\end{smallmatrix}
	\end{bmatrix}$. If we row-reduce this matrix:
	\begin{align*}
		& 
		\begin{bmatrix}
			4 & 2 & 4 \\
			4 & 5 & 1 \\
			8 & 4 & 2
		\end{bmatrix} \\
		7 \times \text{R}1 & 
		\begin{bmatrix}
			1 & 5 & 1 \\
			4 & 5 & 1 \\
			8 & 4 & 2
		\end{bmatrix} \\
		\text{R}2 - 4 \times \text{R}1 &
		\begin{bmatrix}
			1 & 5 & 1 \\
			0 & 3 & 6 \\
			8 & 4 & 2
		\end{bmatrix} \\
		\text{R}3 + \text{R}1 &
		\begin{bmatrix}
			1 & 5 & 1 \\
			0 & 3 & 6 \\
			0 & 0 & 3
		\end{bmatrix} \\
		\text{R}2 + \text{R}3 &
		\begin{bmatrix}
			1 & 5 & 1 \\
			0 & 3 & 0 \\
			0 & 0 & 3
		\end{bmatrix}
	\end{align*}
	This is as much as we can reduce this matrix mod 9. The important thing to note is that we can't reduce this matrix to the identity.
	
	Now, if we were to take another CCM, say $C_{4 \rightarrow 1} = A^0 + A^4 + A^8 + A^{12} + A^{16} + A^{20} \equiv
	\begin{bmatrix}
		\begin{smallmatrix}
			2 & 5 & 0 \\
			6 & 3 & 5 \\
			2 & 3 & 3
		\end{smallmatrix}
	\end{bmatrix}$, and perform row reduction, we see:
	\begin{align*}
		& 
		\begin{bmatrix}
			2 & 5 & 0 \\
			6 & 3 & 5 \\
			2 & 3 & 3
		\end{bmatrix} \\
		5 \times \text{R}1 &
		\begin{bmatrix}
			1 & 7 & 0 \\
			6 & 3 & 5 \\
			2 & 3 & 3
		\end{bmatrix} \\
		\text{R}2 + 3 \times \text{R}1 & 
		\begin{bmatrix}
			1 & 7 & 0 \\
			0 & 6 & 5 \\
			2 & 3 & 3
		\end{bmatrix} \\
		\text{R}3 - 2 \times \text{R}1 &
		\begin{bmatrix}
			1 & 7 & 0 \\
			0 & 6 & 5 \\
			0 & 7 & 3 
		\end{bmatrix} \\
		\text{R}2 \leftrightarrow \text{R}3 &
		\begin{bmatrix}
			1 & 7 & 0 \\
			0 & 7 & 3 \\
			0 & 6 & 5
		\end{bmatrix} \\
		\text{R}2 - \text{R}3 &
		\begin{bmatrix}
			1 & 7 & 0 \\
			0 & 1 & 7 \\
			0 & 6 & 5
		\end{bmatrix} \\
		\text{R}3 + 3 \times \text{R}2 &
		\begin{bmatrix}
			1 & 7 & 0 \\
			0 & 1 & 7 \\
			0 & 0 & 8
		\end{bmatrix} \\
		8 \times \text{R}3 &
		\begin{bmatrix}
			1 & 7 & 0 \\
			0 & 1 & 7 \\
			0 & 0 & 1
		\end{bmatrix} \\
		\text{R}2 + 2 \times \text{R}3 &
		\begin{bmatrix}
			1 & 7 & 0 \\
			0 & 1 & 0 \\
			0 & 0 & 1
		\end{bmatrix} \\
		\text{R}1 + 2 \times \text{R}2 &
		\begin{bmatrix}
			1 & 0 & 0 \\
			0 & 1 & 0 \\
			0 & 0 & 1
		\end{bmatrix}
	\end{align*}
	Unlike before, this CCM can be reduced to the identity. Why is this significant? Well, for this specific matrix, there are only three vectors with cycle lengths of
	1: $
	\begin{bmatrix}
		\begin{smallmatrix}
			0 \\
			0 \\
			0
		\end{smallmatrix}
	\end{bmatrix}, \, 
	\begin{bmatrix}
		\begin{smallmatrix}
			3 \\
			3 \\
			6
		\end{smallmatrix}
	\end{bmatrix}, \, \text{and }
	\begin{bmatrix}
		\begin{smallmatrix}
			6 \\
			6 \\
			3
		\end{smallmatrix}
	\end{bmatrix}$. It turns out that, if you set up the equation $C_{4 \rightarrow 1}\vec{v} \equiv \vec{a}$, where $\vec{a}$ is one of the three vectors with cycle length
	1, and row reduce, you'll end up with $I\vec{v} \equiv \vec{a}$; $\vec{a}$ ends up back at itself after all the row operations. This means that $\vec{v}$ must necessarily
	be the same as $\vec{a}$ for the matrix equation to have a solution. But then this implies that the only vectors with "cycle lengths" of 4 are the ones with cycle
	lengths of 1 (since $C_{4 \rightarrow 1}$ is supposed to convert vectors of cycle length 4 to 1). So, this seems to show that no vectors with a \emph{true} cycle length
	of 4 exist, which is indeed the case for our matrix $A$.
	
	Comparing this to the $C_{8 \rightarrow 1}$ case, we see that $C_{8 \rightarrow 1}\vec{v} \equiv \vec{b}$, where $\vec{b}$ is some vector with cycle length 8, is a
	matrix equation with multiple solutions. In this case, it isn't necessarily true that $\vec{v} \equiv \vec{b}$, which seems to suggest that vectors with cycle length
	8 do exist.
	
	This is the gist of what I've noticed. A few other interesting matrices I tried were $
	\begin{bmatrix}
		\begin{smallmatrix}
			1 & 1 \\
			1 & 0
		\end{smallmatrix}
	\end{bmatrix} \bmod{5}, \,
	\begin{bmatrix}
		\begin{smallmatrix}
			0 & 0 & 1 \\
			1 & 0 & 1 \\
			0 & 1 & 8
		\end{smallmatrix}
	\end{bmatrix} \bmod{9}, \text{ and }
	\begin{bmatrix}
		\begin{smallmatrix}
			4 & 1 & 6 \\
			6 & 2 & 2 \\
			5 & 1 & 6
		\end{smallmatrix}
	\end{bmatrix} \bmod{9}
	$. All of these seem to behave in a similar way, which leads me to believe that there's something kind of interesting going on. 
	
	My one major concern with this whole process is how embedded vectors come into play. As an example, take $A = 
	\begin{bmatrix}
		\begin{smallmatrix}
			4 & 8 & 4 \\
			2 & 15 & 16 \\
			14 & 19 & 4
		\end{smallmatrix}
	\end{bmatrix} \bmod{25}$. This matrix has a cycle length of 310. Taking the matrix's cycle length mod 5 (which is 62) and making a CCM, we get $C_{62 \rightarrow 1}
	\equiv 
	\begin{bmatrix}
		\begin{smallmatrix}
			0 & 5 & 5 \\
			20 & 0 & 0 \\
			5 & 0 & 20
		\end{smallmatrix}
	\end{bmatrix}$. Row-reducing this matrix results in $5I$, which corresponds to all the embedded vectors in the LCA mod 25. In this specific case, this is the correct 
	behaviour: only the embedded vectors have a cycle length of 62. However, there's other CCM behaviour involving embedded vectors that isn't always "correct". For instance,
	$C_{p\omega \rightarrow \omega} \equiv pI \bmod{p^2}$ for odd prime $p$ and for some matrices with cycle length $\omega$ mod $p$. It could very
	easily be the case that CCMs can't see certain types of cycles or are focused only on select cycles in specific cases. I don't know enough to say anything certain. 
	
	In any case, this entire process is, at the very least, interesting, and it may lead to a new way to view LCAs under prime-powered moduli, a way separate from minimal
	polynomials and the like.
\end{document}