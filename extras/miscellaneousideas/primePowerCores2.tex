
\documentclass[a4paper, reqno, 12pt]{amsart}

\usepackage[T1]{fontenc}
\usepackage[margin=3cm]{geometry}
\usepackage[parfill]{parskip}

\usepackage{setspace}
\setstretch{1.25}

\usepackage{dsfont}

\usepackage{natbib}

%Generalised dimension notation
\newcommand\dimdi[1]{\text{dimdi}(#1)}

%Core of A
\newcommand\core[2]{\text{core}_{#1}(#2)}

%Column space
\newcommand\colsp[1]{\text{colsp}(#1)}

%Span
\newcommand\vecspan[1]{\text{span}(#1)}

%Kernel
%\newcommand\ker[1]{\text{ker}(#1)}

\begin{document}
	A set of vectors $V = \{\vec{v}_{1}, \vec{v}_{2}, \vec{v}_{3}, \cdots \}$ is \emph{dimensionally independent} iff
	\[
		\forall \vec{v} \in V, \quad \vecspan{\vec{v}} \cap \vecspan{V\setminus\{\vec{v}\}} = \{\vec{0}\}
	\]
	
	Dimensional independence is a generalisation of linear independence since, if the set $V = \{\vec{v}_{1}, \vec{v}_{2}, \vec{v}_{3}, \cdots \}$ is
	linearly independent, then the only solution to
	\[
		a_{1}\vec{v}_{1} + a_{2}\vec{v}_{2} + \cdots \equiv \vec{0}
	\]
	is the trivial solution for the constants $\{a_i\}$. This means the rearranged equation
	\[
		-a_{i}\vec{v}_{i} \equiv a_1\vec{v}_1 + a_2\vec{v}_2 + \cdots + a_{i-1}\vec{v}_{i-1} + a_{i+1}\vec{v}_{i+1} + \cdots
	\]
	has no solutions, which means $V$ is dimensionally independent since no vector in $\vecspan{\vec{v}_i}$ is in $\vecspan{V \setminus \{\vec{v}_i\}}$ except
	the zero vector.
	
	In particular, dimensional independence is a useful generalisation of linear independence when considering vectors from the module $\mathds{Z}_{p^k}^{L}$, the set
	of all $L \times 1$ vectors with components from the integers modulo $p^k$, a power of a prime number. Note that we'll only be considering the case where $k > 1$, 
	since if $k = 1$ then $\mathds{Z}_{p^k}^{L}$ is a vector space and using dimensional independence instead of linear independence wouldn't
	grant any additional utility. 
	
	To understand why dimensional independence is useful in this case, consider the vectors $
	\begin{bmatrix}
		\begin{smallmatrix}
			5 \\
			0
		\end{smallmatrix}
	\end{bmatrix}
	$ and $
	\begin{bmatrix}
		\begin{smallmatrix}
			0 \\
			5
		\end{smallmatrix}
	\end{bmatrix}
	$ modulo 25. In the module $\mathds{Z}_{25}^{2}$, these two vectors are not linearly independent since
	\[
		5
		\begin{bmatrix}
			\begin{smallmatrix}
				5 \\
				0
			\end{smallmatrix}
		\end{bmatrix}
		+
		5
		\begin{bmatrix}
			\begin{smallmatrix}
				0 \\
				5
			\end{smallmatrix}
		\end{bmatrix}
		\equiv \vec{0} \mod{25}
	\]
	However, these two vectors are pointing in orthogonal directions. In some sense, the two vectors are still independent of one another as one cannot be created as 
	a linear combination of the other. Thus, dimensional independence provides a useful metric for when such a situation arises.
	
	Using dimensional independence, we can define the \emph{di-dimension} of a set of vectors $V$ as the minimum number of dimensionally independent vectors needed to 
	span $V$. The di-dimension of $V$ will be denoted as $\dimdi{V}$.
	
	As with the definition of dimensional independence, di-dimension is a generalisation of the more typical definition of dimension used with vector spaces, as
	the dimension of a vector space is the number of linearly independent vectors needed to span the space. Given some module, the number of vectors in the span of 
	$n$ dimensionally independent vectors can, at most, be the same as the number of vectors in the span of $n$ linearly independent vectors, so the dimension of a 
	vector space agrees with the di-dimension of the same space.
	
	The main interest in introducing the concepts of dimensional independence and di-dimension is to understand how the core of a matrix changes when increasing 
	the exponent of a prime-power modulus. The \emph{core} of a matrix $A \in \mathds{Z}_{n}^{L \times L}$ (where $\mathds{Z}_{n}^{L \times L}$ is the set of 
	all $L \times L$ matrices modulo n) is the largest submodule $S$ of $\mathds{Z}_{n}^{L}$ (the set of all $L \times 1$ vectors modulo n) such that
	\[
		AS \equiv S \mod{n}
	\]
	In other words, $S$ is the largest submodule where matrix multiplication by $A$ creates a bijection from $S$ to itself. One can think of $A$ as merely shuffling
	the vectors within the core around as opposed to squeezing multiple vectors together in the mapping. The core of $A$ is denoted as $\core{n}{A}$, where $n$ is the
	relevant modulus.
	
	Another valid definition for the core of a matrix makes use of the matrix's column space:
	\[
		\core{n}{A} = \colsp{A^\tau} \mod{n}
	\]
	where $\tau$ is defined to be the smallest positive integer such that
	\[
		\colsp{A^\tau} = \colsp{A^{\tau + 1}} \mod{n}
	\]
	The equivalency of these definitions is based on the fact that $\colsp{A^\tau}$ gives a submodule where, after multiplication by $A$, each vector in it is mapped 
	to by only one other vector in the submodule, creating a bijection from the submodule to itself. It is also the largest such submodule as the column space of the 
	matrix dictates which vectors can be mapped to by multiplication. No larger submodule with this property exists because the matrix itself doesn't allow for a 
	bigger one to exist. Such a submodule is also guaranteed to exist as, for a general square matrix,
	\[
		\colsp{A^{i}} \supseteq \colsp{A^{i+1}}
	\]
	so the column space either shrinks or stays the same after each multiplication by $A$. If $\colsp{A^{i}} = \colsp{A^{i+1}}$, then 
	$\forall c \in \mathds{Z}^{+}, \, \colsp{A^{i}} = \colsp{A^{i+c}}$ since $A(\colsp{A^i}) = \colsp{A^i}$. Eventually, the column space will reach one of 
	these fixed submodules, or the column space will reduce itself to $\{\vec{0}\}$, at which point the column space will forcibly be fixed since 
	$\forall A, \, A\vec{0} \equiv \vec{0}$.
	
	Our main concern is calculating $\dimdi{\core{p^k}{A}}$ when $\dimdi{\core{p}{A}}$ is known.
	
	The easiest case is when $\core{p}{A} = \{\vec{0}\}$. If this is the case, then for some positive integer $\tau$ we know
	\[
		A^{\tau} \equiv 0 \mod{p}
	\]
	since if the core is only the zero vector, $A$ must have mapped all vectors to it, and this can only happen with the zero matrix. Lifting the above equation
	to a higher-powered modulus, we see
	\[
		A^{\tau} \equiv pB \mod{p^k}, \quad B \in \mathds{Z}^{L \times L}
	\]
	since this is the form the matrix must take if it is to reduce to 0 mod $p$. Raising both sides to the $k$-th power,
	\[
		A^{k\tau} \equiv p^{k}B^{k} \equiv 0 \mod{p^k}
	\]
	If $\core{p}{A} = \{\vec{0}\}$, then $\core{p^k}{A} = \{\vec{0}\}$ since the matrix will always iterate to the zero matrix eventually.
	
	Analogous to how the dimension of $\{\vec{0}\}$ is 0, the di-dimension of $\{\vec{0}\}$ is also 0. So,
	\[
		\dimdi{\core{p}{A}} = 0 \implies \dimdi{\core{p^k}{A}} = 0
	\]
	
	What about when $\core{p}{A} \neq \{\vec{0}\}$? If $\core{p}{A}$ has more than the zero vector in it, then we can write it as the span of a set of vectors since
	the core of $A$ is by definition a submodule. Let $\core{p}{A} = \vecspan{B}$, where $B = \{\vec{v}_{1}, \vec{v}_{2}, \vec{v}_{3}, \cdots\}$. Without loss of 
	generality, we can assume all of $\vec{v}_{i}$ are linearly independent. If they weren't, one or more of $\vec{v}_{i}$ could be rewritten as a linear 
	combination of the other vectors, meaning it could be removed from the set and the span wouldn't change.
	
	Each of $\vec{v}_{i}$ is linearly independent, and therefore dimensionally independent, meaning $\dimdi{\core{p}{A}} = |B|$.
	
	Now, what can we say about $\dimdi{\core{p^k}{A}}$ in this case? By Proposition 5 in \citet{Mendivil2012}, we know that $\core{p}{A}$ is embedded within 
	$\core{p^k}{A}$, so at the very least, $\dimdi{\core{p^k}{A}} \geq \dimdi{\core{p}{A}}$, since it takes at least $\dimdi{\core{p}{A}}$ dimensionally independent
	vectors to span $\core{p}{A}$.
	
	Is it possible for $\dimdi{\core{p^k}{A}} > \dimdi{\core{p}{A}}$? Let's assume it is, meaning
	\[
		\core{p^k}{A} = \vecspan{\{\vec{u}_{1}, \vec{u}_{2}, \vec{u}_{3}, \cdots, \vec{u}_{|B|}, \vec{u}_{|B|+1}\}}
	\]
	so $\core{p^k}{A}$ requires at least one more dimensionally independent vector to span it.
	
	What do we know about each $\vec{u}_{i}$? At the very least, we can define $\vec{u}_{1}$ to $\vec{u}_{|B|}$ as lifts of $\vec{v}_{1}$ to $\vec{v}_{|B|}$. A 
	\emph{lift} of a vector $\vec{v}$ from mod $p^\alpha$ to $p^{\alpha + \beta}$ is defined as a vector of the form
	\[
		\vec{v} + p^{\alpha}\vec{a}, \quad \vec{a} \in \mathds{Z}_{p^\beta}^{L}
	\]
	A lift of $\vec{v}$ from mod $p^\alpha$ to $p^{\alpha+\beta}$ has the property that, when reduced modulo $p^\alpha$, it will reduce back to $\vec{v}$, even if the
	lift is not congruent to $\vec{v}$ modulo $p^{\alpha+\beta}$.
	
	How do we know $\vec{u}_1$ to $\vec{u}_{|B|}$ can be written as lifts of $\vec{v}_1$ to $\vec{v}_{|B|}$? Because our focus is on vectors modulo $p^k$, we can 
	consider \emph{every} vector $\vec{a}$ modulo $p^k$ as the lift of some vector $\vec{b}$ modulo $p$:
	\[
		\vec{a} \equiv \vec{b} + p\vec{c}, \quad \vec{c} \in \mathds{Z}_{p^{k-1}}^L \mod{p^k}
	\]
	A consequence of this is that the dynamics of all vectors modulo $p^k$ contain within them the dynamics of some vector modulo $p$:
	\begin{alignat*}{2}
		A\vec{a} \, &\equiv A\vec{b} + pA\vec{c} && \mod{p^k} \\
		         \, &\equiv A\vec{b} && \mod{p}
	\end{alignat*}
	If we think of all vectors modulo $p^k$ as having a "base" part $\vec{a}_b$ and an "embedded" part $\vec{a}_e$, where
	\[
		\vec{a} \equiv \vec{a}_b + p\vec{a}_e
	\]
	then the behaviour of $\vec{a}_b$ modulo $p^k$ for each iteration of $\vec{a}$ will exactly mirror the dynamics of $\vec{a}_b$ modulo $p$. By \emph{iteration}, 
	we mean a vector obtained by repeatedly multiplying a vector by some power of an update matrix. By \emph{dynamics}, we mean the set of all iterations of some 
	vector. The behaviour of $\vec{a}_e$ for each iteration of $\vec{a}$ is slightly more complex since parts of $\vec{a}_b$ can "bleed" into $\vec{a}_e$, causing 
	its behaviour to be dictated by more than matrix multiplication.
	
	In any case, the base part of a vector modulo $p^k$ will behave in a predictable way. In fact, since each vector modulo $p^k$ can be written as a lift of some 
	vector modulo $p$, this means the base part will behave exactly the same as the dynamics of some vector modulo $p$. This means that, for all vectors in
	$\core{p}{A}$, their behaviour will be mimicked by the base part of some lift vector modulo $p^k$.
	
	We can say a little bit more. Proposition 5 in \citet{Mendivil2012} shows that if $\vec{v} \in \core{p}{A}$, then $p^{k-1}\vec{v} \in \core{p^k}{A}$. This is
	how it ensures the dynamics of $\core{p}{A}$ are embedded within $\core{p^k}{A}$. Thankfully, a useful property of lifts is that, if 
	$\vec{a} \in \mathds{Z}_{p^k}^L$ is a lift of $\vec{v} \in \mathds{Z}_{p}^L$, then
	\[
		p^{k-1}\vec{a} \equiv p^{k-1}\vec{v} \mod{p^k}
	\]
	
	What this means is that the vectors $\vec{u}_1$ to $\vec{u}_{|B|}$ must necessarily be able to be written as lifts of $\vec{v}_1$ to $\vec{v}_{|B|}$ since both
	$p^{k-1}\vec{v}_i$ and a lift of $\vec{v}_i$ must be in $\core{p^k}{A}$, and lifts of $\vec{v}_i$ have $p^{k-1}\vec{v}_i$ in their spans.
	
	So $\vec{u}_1$ to $\vec{u}_{|B|}$ are lifts of $\vec{v}_1$ to $\vec{v}_{|B|}$, and we know they're dimensionally independent by construction since lifts of
	linearly independent vectors ($\vec{v}_1$ to $\vec{v}_{|B|}$) remain linearly independent. Where does that leave $\vec{u}_{|B|+1}$, the extra vector we
	assumed we needed to span $\core{p^k}{A}$?
	
	$\vec{u}_{|B|+1}$ is needed to span $\core{p^k}{A}$, so it must be an element of $\core{p^k}{A}$. This means $\vec{u}_{|B|+1}$ must be able
	to be written as a linear combination of the vectors in $B$, plus some embedded part. This is because, upon reducing modulo $p$, $\vec{u}_{|B|+1}$ must be in
	$\core{p}{A}$. Otherwise, $\vec{u}_{|B|+1}$ would be a lift of a vector not in $\core{p}{A}$, meaning the update matrix $A$ couldn't act as a bijection on
	$\vec{u}_{|B|+1}$ modulo $p$ and by extension modulo $p^k$.
	
	Thus, $\vec{u}_{|B|+1}$ will look something like
	\[
		\vec{u}_{|B|+1} \equiv x_1\vec{v}_1 + x_2\vec{v}_2 + \cdots + x_{|B|}\vec{v}_{|B|} + p\vec{V}, 
		\quad x_i \in \mathds{Z}_{p^k}, \, \vec{V} \in \mathds{Z}_{p^{k-1}}^{L} \mod{p^k}
	\]
	Since $\vec{V}$ can be practically any vector in $\mathds{Z}_{p^{k-1}}^L$, there's no loss in generality in assuming that, instead of using $\vec{v}_1$
	to $\vec{v}_{|B|}$, we used \emph{lifts} of $\vec{v}_1$ to $\vec{v}_{|B|}$. The vector $\vec{V}$ could simply be changed to account for the change in embedded
	part. With this, we can write $\vec{u}_{|B|+1}$ as
	\[
		\vec{u}_{|B|+1} \equiv x_1\vec{u}_1 + x_2\vec{u}_2 + \cdots + x_{|B|}\vec{u}_{|B|} + p\vec{V}, 
		\quad x_i \in \mathds{Z}_{p^k}, \, \vec{V} \in \mathds{Z}_{p^{k-1}}^{L} \mod{p^k}
	\]
	The vectors $\vec{u}_1$ to $\vec{u}_{|B|}$ are guaranteed to be in $\core{p^k}{A}$, so we can practically ignore them for the purpose of determining whether
	we need $\vec{u}_{|B|+1}$ to span $\core{p^k}{A}$. This means we can say
	\[
		\core{p^k}{A} = \vecspan{\{\vec{u}_1, \vec{u}_2, \cdots, \vec{u}_{|B|}, p\vec{V}\}}
	\]
	Is $p\vec{V} \in \vecspan{\{\vec{u}_1, \vec{u}_2, \cdots, \vec{u}_{|B|}\}}$? We know that $p\vec{V} \in \core{p^k}{A}$, which implies that 
	$\vec{V} \bmod{p} \in \core{p}{A}$. But then $\vec{V}$ is simply a linear combination of $\vec{v}_1$ to $\vec{v}_{|B|}$...
	
	\bibliographystyle{plainnat}
	\bibliography{refs.bib}
\end{document}
