
\documentclass[a4paper, 12pt, reqno]{amsart}

\usepackage[T1]{fontenc}
\usepackage[margin=3cm]{geometry}
\usepackage[parfill]{parskip} %Spaces for paragraphs

%Line spacing
\usepackage{setspace}
\setstretch{1.25}

\usepackage{dsfont} %Double stroke font

\usepackage{amsmath}

\usepackage{natbib}

%For adding spacing before section titles
\newcommand{\ssection}[1]{\vspace{1cm}\section{#1}}
\newcommand{\ssubsection}[1]{\vspace{0.25cm}\subsection{#1}}

%Defining an environment for examples
\newcounter{mathexample}[section]
\newenvironment{mathexample}
{
	\refstepcounter{mathexample} %Increment environment counter
	\textbf{Example \themathexample} 
	\\
}
{
	\vspace{1cm}
}

\title{Finding vectors with maximal cycle length within modules}
\author{Zach Strong}
\date{\today}

\begin{document}
	\maketitle
	
	\section{What?}
	For any matrix $A$ with elements in $\mathds{Z}_{N}$, we have that, for some $\tau, \omega \in \mathds{Z}^{+}$,
	\[
		A^{\tau} \equiv A^{\tau + \omega} \mod{N}
	\]
	
	This is true because there are only a finite number of elements in $\mathds{Z}_{N}$, and so there must be a
	finite number of matrices that $A^{k}$ can iterate to. At some point, it must iterate back to a matrix it
	has already iterated to, at which point it'll fall into a cycle where it iterates to the same few matrices
	over and over again.
	
	Let's call $\tau$ the \emph{transient length} of the matrix, and $\omega$ the \emph{cycle length} of the matrix.
	To be more specific, $\omega$ is called the cycle length when it's the smallest nonzero value with the
	property above for some transient length $\tau$. The transient length can be thought of as the smallest
	iteration of the matrix $A$ which will appear as a later iteration; it's the smallest iteration of $A$ such that
	$A^{\tau}$ is in a cycle.
	
	Vectors can also have cycle lengths and transient lengths. If
	\[
		A^{\tau}\vec{v} \equiv A^{\tau + \omega}\vec{v} \mod{N}
	\]
	for some $\tau, \omega \in \mathds{Z}^{+}$ (not necessarily the same values that the matrix had above), then
	$\omega$ is the \emph{cycle length} of the vector if it's the smallest nonzero value with the property above
	for some $\tau$. The \emph{transient length} of the vector is $\tau$, the first iteration of the vector such
	that the vector will eventually iterate back to it after enough matrix multiplications with $A$.
	
	We know the cycle length of a vector $\vec{v}$ iterated by a specific matrix must divide that matrix's cycle 
	length. To see this, let $A$ be the update matrix, $\tau$ be the matrix's transient length, $\omega$ be the 
	matrix's cycle length, and $k$ be the vector's cycle length. By definition of cycle lengths, we know the 
	following:
	\[
		A^{\tau}\vec{v}          \equiv 
		A^{\tau + k}\vec{v}      \equiv
		A^{\tau + 2k}\vec{v}     \equiv
		\cdots                   \equiv
		A^{\tau + nk}\vec{v}     \equiv
		A^{\tau + \omega}\vec{v} \mod{N}
	\]
	
	Either $nk = \omega$, meaning $k$ divides $\omega$, or $\omega$ is slightly bigger than a multiple of $k$.
	If the second case is true, then let $k_{2} = \omega - nk$. Note that $k_{2} < k$ since $\omega$ must be
	less than $(n+1)k$ in this case. We have that
	\begin{align*}
		       & \, A^{\tau + k_{2}}\vec{v}            \\
		\equiv & \, A^{\tau + nk + k_{2}}\vec{v}       \\
		\equiv & \, A^{\tau + nk + \omega - nk}\vec{v} \\
		\equiv & \, A^{\tau + \omega}\vec{v}           \\
		\equiv & \, A^{\tau}\vec{v}                    \mod{N}
	\end{align*}
	so $k$ can't be the cycle length of our vector since $k_{2}$ is smaller and has the same property.
	Repeating this same process, using $k_{2}$ in place of $k$, leads to the same conclusion: either $k_{2}$
	is a multiple of $\omega$, or there exists some smaller value for the cycle length. This line of
	reasoning continues until either we reach a cycle length which is a multiple of $\omega$, or our
	vector's cycle length is $1$, meaning it divides $\omega$ by default. Therefore, a vector's cycle
	length must divide the corresponding matrix's cycle length.
	
	The question we're interested in is whether, given any update matrix $A$, we can find a vector
	with the same cycle length as the matrix. The following section outlines a process for finding
	such a vector.
	
	\ssection{How?}
	For our purposes, it's helpful to think of matrices as collections of column vectors. For the
	following examples, let
	\[
		A = 
		\begin{bmatrix}
			\vec{v}_{1} & \vec{v}_2 & \cdots & \vec{v}_L
		\end{bmatrix}
	\]
	be an $L \, \text{by} \, L$ update matrix mod $N$. 
	
	Since we're only concerned with cycle lengths, we can restrict our attention to the "core" 
	of each matrix/module, the largest set of all vectors $S$ in our module such that $AS = S$. This
	allows us to ignore any complications regarding transient lengths.

	\ssubsection{Matrices with maximal cycle length column vectors}
	In this case, there exists at least one column vector in an iteration of $A$ whose cycle length is 
	equal to the matrix's cycle length. This case occurs when at least one column vector of $A$ iterates 
	to a unique vector for every unique iteration of $A$ (in the cycle of $A$).
	
	\begin{mathexample}
		\label{ex:max_cycle_col_vects}
		The matrix $A =
			\begin{bmatrix}
				\begin{smallmatrix}
					0 & 10 \\
					1 & 0
				\end{smallmatrix}
			\end{bmatrix}
		$
		mod $11$ has a cycle length of $4$:
		
		\begin{align*}
			A^{1} & \equiv
			\begin{bmatrix}
				0 & 10 \\
				1 & 0 
			\end{bmatrix} &
			A^{2} & \equiv
			\begin{bmatrix}
				10 & 0 \\
				0 & 10
			\end{bmatrix} \\
			A^{3} & \equiv
			\begin{bmatrix}
				0 & 1 \\
				10 & 0
			\end{bmatrix} &
			A^{4} & \equiv
			\begin{bmatrix}
				1 & 0 \\
				0 & 1
			\end{bmatrix}
		\end{align*}
		\[
			A^{5} \equiv A \mod{11}
		\]
		
		Notice that both column vectors iterate to different vectors for each iteration in the cycle of $A$. 
		Therefore, the column vectors in the cycle have maximal cycle lengths, so any of them can be used. 
		Similarly, for any matrix where a column vector iterates to a different vector for each iteration of 
		the matrix (within the matrix's cycle, so not including any matrices whose iterations are less than the
		transient length), that column vector has maximal cycle length.
	\end{mathexample}
	
	\ssubsection{Matrices with non-maximal coprime cycle length column vectors}
	In this case, things get a little more complex. Let's look at an example that better illustrates this.
	
	\begin{mathexample}
		\label{ex:two_two_three_col_vects}
		Take the matrix $ A =
			\begin{bmatrix}
				\begin{smallmatrix}
					4 & 0 & 4 \\
					5 & 3 & 2 \\
					3 & 3 & 1
				\end{smallmatrix}
			\end{bmatrix}
		$
		mod $6$. This matrix has a cycle length of $6$:
		
		\begin{align*}
			A^{1} & \equiv
			\begin{bmatrix}
				4 & 0 & 4 \\
				5 & 3 & 2 \\
				3 & 3 & 1
			\end{bmatrix} &
			A^{2} & \equiv
			\begin{bmatrix}
				4 & 0 & 2 \\
				5 & 3 & 4 \\
				0 & 0 & 1
			\end{bmatrix} &
			A^{3} & \equiv
			\begin{bmatrix}
				4 & 0 & 0 \\
				5 & 3 & 0 \\
				3 & 3 & 1
			\end{bmatrix} \\
			A^{4} & \equiv
			\begin{bmatrix}
				4 & 0 & 4 \\
				5 & 3 & 2 \\
				0 & 0 & 1
			\end{bmatrix} &
			A^{5} & \equiv
			\begin{bmatrix}
				4 & 0 & 2 \\
				5 & 3 & 4 \\
				3 & 3 & 1
			\end{bmatrix} &
			A^{6} & \equiv
			\begin{bmatrix}
				4 & 0 & 0 \\
				5 & 3 & 0 \\
				0 & 0 & 1
			\end{bmatrix}
		\end{align*}
		\[
			A^{7} \equiv A \mod{6}
		\]
		
		Unlike example \ref{ex:max_cycle_col_vects}, the column vectors in this matrix cycle back to previous 
		vectors during the matrix's iteration. The two leftmost vectors have a cycle length of $2$, while the 
		rightmost vector has a cycle length of $3$. How can we find a vector with maximal cycle length $6$ in 
		this instance?
		
		The trick is to use linear combinations of the column vectors to construct vectors with higher 
		and higher cycle lengths until we reach the maximal cycle length. For this example, we only need
		to construct one such vector.
		
		Consider the vectors 
		$\vec{v}_{2} =
			\begin{bmatrix}
				\begin{smallmatrix}
					0 \\
					3 \\
					3
				\end{smallmatrix}
			\end{bmatrix}
		$ and 
		$\vec{v}_{3} = 
			\begin{bmatrix}
				\begin{smallmatrix}
					4 \\
					2 \\
					1
				\end{smallmatrix}
			\end{bmatrix}$.
		From the matrix's iteration, we know that $\vec{v}_{2}$ has a cycle length of $2$ and
		$\vec{v}_{3}$ has a cycle length of $3$. So, the sum of these vectors, $\vec{w} = \vec{v}_{2} + 
		\vec{v}_3$, could have a cycle length no greater than $\text{lcm}(2, 3) = 6$, since after $6$ 
		iterations, both $\vec{v}_{2}$ and $\vec{v}_{3}$ would iterate back to themselves, meaning 
		their sum would also iterate back to itself. In fact, we can show the cycle length of $\vec{w}$ 
		is \emph{exactly} $6$. 
		
		Because we know $\vec{w}$ must cycle after $6$ iterations, its cycle length must divide $6$. 
		Otherwise, it would have to be the case that some vector in the cycle of $\vec{w}$ would iterate 
		to different vectors at different times. For instance, if we assume the cycle length of $\vec{w}$
		is $4$, that would mean that, after four iterations and no sooner, $\vec{w}$ would iterate back 
		to itself, but then two iterations later it would also iterate back to $\vec{w}$ since we know 
		from above it must iterate back after $6$ iterations. This isn't possible with our single matrix 
		iteration rule. No matter how many iterations one took to get to a vector, that vector will always 
		iterate to the same next vector (since we use the same matrix and modulus each time to iterate). 
		With this in mind, the only cycle lengths $\vec{w}$ could have are $1, 2, 3,$ and $6$, since these
		are the cycle lengths which would keep the cycle length consistent given how we know $\vec{w}$ must
		iterate back after $6$ iterations.
		
		Assume the cycle length is $1$. This would imply $\vec{w}$ iterates to itself for every iteration.
		However:
		\begin{align*}
								 & \, A^{2}\vec{w}                        \\
			\equiv     & \, A^{2}(\vec{v}_{2} + \vec{v}_{3})    \\
			\equiv     & \, A^{2}\vec{v}_{2} + A^{2}\vec{v}_{3} \\
			\equiv     & \, \vec{v}_{2} + A^{2}\vec{v}_{3}      \\
			\not\equiv & \, \vec{v}_{2} + \vec{v}_{3}
		\end{align*}
		
		After $2$ iterations, $\vec{v}_{2}$ iterates back to itself, but $\vec{v}_{3}$ doesn't. The only
		vector we could add to $\vec{v}_{2}$ to get $\vec{w}$ is $\vec{v}_{3}$, and we know that
		$A^{2}\vec{v}_{3} \not\equiv \vec{v}_{3}$ since its cycle length is 3, not 2. Therefore, we can
		conclude that $\vec{w}$ cannot have a cycle length of $1$ since it iterates to a vector that
		isn't itself.
		
		Now, assume the cycle length of $\vec{w}$ is $2$. This implies that, after $2$ iterations,
		$\vec{w}$ iterates back to itself. From above, however, we showed that it doesn't. Therefore,
		$\vec{w}$ cannot have a cycle length of $2$.
		
		Now, assume the cycle length of $\vec{w}$ is $3$. This implies that, after $3$ iterations,
		$\vec{w}$ iterates back to itself. Using the same reasoning as above, we see that this
		isn't possible:
		\begin{align*}
								 & \, A^{3}\vec{w}                        \\
			\equiv     & \, A^{3}(\vec{v}_{2} + \vec{v}_{3})    \\
			\equiv     & \, A^{3}\vec{v}_{2} + A^{3}\vec{v}_{3} \\
			\equiv     & \, A\vec{v}_{2} + \vec{v}_{3}      \\
			\not\equiv & \, \vec{v}_{2} + \vec{v}_{3}
		\end{align*}
		
		Therefore, $\vec{w}$ cannot have a cycle length of $3$.
		
		This leaves us with only one possible cycle length: $6$. So the cycle length of $\vec{w}$ is $6$,
		which in this case is the maximal cycle length for the given matrix.
	\end{mathexample}
	
	For a given matrix with column vectors that have cycle lengths less than the matrix's cycle length, 
	and whose cycle lengths are coprime, we can add these vectors together to create a vector with a higher 
	cycle length than the vectors we added. The reason for why these cycle lengths must be coprime is
	discussed in example \ref{ex:general_coprime_cycle_lengths}. For example \ref{ex:two_two_three_col_vects}, 
	we only had to construct one vector before we reached the maximal cycle length. In general, the vector 
	we created could then be added to another column vector to create a vector with an even greater cycle 
	length. We can repeat this step as many times as needed in order to reach the matrix's cycle length.
	
	\vspace{1cm}
	Let's consider a more general example, one that doesn't rely on a particular matrix, but rather
	on particular column vector cycle lengths.
	
	\begin{mathexample}
		\label{ex:general_col_vects}
		Let $A = 
			\begin{bmatrix}
				\vec{v}_{1} & \vec{v}_{2} & \vec{v}_{3}
			\end{bmatrix}
		$
		where $\vec{v}_{1}$ has a cycle length of $4$, $\vec{v}_{2}$ has a cycle length of $5$, 
		and $\vec{v}_{3}$ has a cycle length of $7$. The cycle length of $A$ is 
		$\text{lcm}(4, 5, 7) = 140$, since $140$ is the first possible value where the cycles of 
		all column vectors align to return to their original vectors. Is it possible for us to 
		construct a vector with a cycle length of $140$? 
		
		Let $\vec{w}_{1} = \vec{v}_{1} + \vec{v}_{3}$. We know the cycle length of $\vec{w}_{1}$
		is at most $28$ since both $\vec{v}_{1}$ and $\vec{v}_{3}$ will have cycled back to where
		they started after $28$ iterations (lcm$(4, 7) = 28$). We also know that, like before, 
		the cycle length of $\vec{w}_{1}$ must divide $28$, giving us $1$, $2$, $4$, $7$, $14$, and
		$28$ as possible cycle lengths.
		
		Referring back to example \ref{ex:two_two_three_col_vects}, one may think we'd have to 
		check every possible cycle length in order to rule out all but $28$. However, since the 
		cycle lengths of $\vec{v}_{1}$ and $\vec{v}_{3}$ are coprime, we can rule out every 
		possible cycle length but $28$ just by checking multiples of their cycle lengths.
		
		Assume the cycle length of $\vec{w}_{1}$ is $4$. This would imply that, after four
		iterations, $\vec{w}_{1}$ iterates back to itself. However:
		\begin{align*}
								 & \, A^{4}\vec{w}_{1}                    \\
			\equiv     & \, A^{4}(\vec{v}_{1} + \vec{v}_{3})    \\
			\equiv     & \, A^{4}\vec{v}_{1} + A^{4}\vec{v}_{3} \\
			\equiv     & \, \vec{v}_{1} + A^4\vec{v}_{3}        \\
			\not\equiv & \, \vec{v}_{1} + \vec{v}_{3}
		\end{align*}
		
		We see that $\vec{w}_{1}$ can't possibly iterate back to itself after $4$ iterations,
		so we can rule out $4$ as the cycle length of $\vec{w}_{1}$. However, we can also use this to
		rule out both $1$ and $2$ as cycle lengths. A cycle length of $2$ would mean $\vec{w}_{1}$ would 
		have to iterate back after $2$ cycles of $2$ iterations each, or $4$ iterations total. We've
		shown this isn't possible. The same logic goes for a cycle length of $1$: if the cycle length 
		was $1$, then $\vec{w}_{1}$ would have to iterate back to itself after $4$ cycles of $1$ iteration, 
		which we've shown isn't possible.
		
		The only other cycle length possibility we need to check for $\vec{w}_{1}$ is $14$. Using
		the same logic as before, we can show that a cycle length of $14$ isn't possible, which
		implies that a cycle length of $7$ isn't possible, either. That leaves us with $28$, the
		only possible cycle length for $\vec{w}_{1}$.
		
		So we now have a vector $\vec{w}_{1}$ which has a cycle length of $28$. Using this
		vector, we can now construct another vector with an even greater cycle length by
		adding to it a vector with a cycle length which is coprime to $28$. Let's use $\vec{v}_{2}$,
		which has a cycle length of $5$.
		
		Let $\vec{w}_{2} = \vec{w}_{1} + \vec{v}_{2}$. Following the same steps as before, we deduce
		that $\vec{w}_{2}$ must have a cycle length no greater than lcm$(5, 28) = 140$, and in fact
		its cycle length is exactly $140$, giving us a vector with cycle length $140$, which is
		the maximal cycle length for the given matrix.
		
		Even without knowing the specific vectors which comprise the matrix $A$, we can construct a 
		vector with maximal cycle length just by using information about the cycle lengths of the
		matrix's column vectors.
	\end{mathexample}
	
	What if we don't know the exact cycle lengths? Let $A^\tau = 
		\begin{bmatrix}
			\vec{v}_{1} & \vec{v}_{2} & \vec{v}_{3} & \cdots & \vec{v}_{L}
		\end{bmatrix}
	$,
	where $\tau$ is the transient length of $A$, the cycle length of $\vec{v}_{1}$ is $\omega_{1}$, 
	the cycle length of $\vec{v}_{2}$ is $\omega_{2}$, etc. We define $A^\tau$ rather than just $A$ 
	so that we don't need to worry about transient regions; the column vectors in $A^\tau$ are guaranteed 
	to be in a cycle. Each unique $\omega_{i}$ is coprime to each other, meaning a set of cycle lengths 
	such as $2, 5, 21, 5$ is allowed, despite the duplicate cycle length, since each unique cycle length 
	in the set is coprime to every other unique element. As before, the matrix would have a cycle length 
	of lcm$(\omega_{1}, \omega_{2}, \cdots, \omega_{L})$. Even without knowing the specific cycle lengths 
	of the matrix's column vectors, can we still construct a vector with maximal cycle length?
	
	Firstly, let's make a subset of our column vectors whose cycle lengths have a least 
	common multiple which is the same as the entire set of column vectors. For instance, 
	if our matrix had four column vectors with cycle lengths $3$, $4$, $5$, and $4$, then 
	one such set could be the three leftmost vectors with cycle lengths $3$, $4$, and $5$, since 
	all of \{3, 4, 5\} are coprime, and lcm$(3, 4, 5) = \text{lcm}(3, 4, 5, 4) = 60$. Another 
	possible set could be the two rightmost vectors and the leftmost vector with cycle length $3$, 
	$4$, and $5$ since, again, they're coprime, and lcm$(3, 4, 5) = \text{lcm}(3, 4, 5, 4) = 60$.
	
	So, we should end up with some set $V = \{\vec{v}_{x_{1}}, \vec{v}_{x_{2}}, \cdots, \vec{v}_{x_{n}}\}$
	where lcm$(\omega_{x_{1}}, \omega_{x_{2}}, \cdots, \omega_{x_{n}}) = 
	$ lcm$(\omega_{1}, \omega_{2}, \cdots, \omega_{L})$.
	Creating this set may seem like an unnecessary detour, but it'll allow for the following steps 
	to be written in a more straightforward way.
	
	Now, let $\vec{w}_{1} = \vec{v}_{x_{1}} + \vec{v}_{x_{2}}$. As with the previous examples,
	we know the cycle length of $\vec{w}_{1}$ is at most lcm$(\omega_{x_{1}}, \omega_{x_{2}})$,
	and that the true cycle length must divide this value.
	
	However, checking each possible cycle length in this context is much more difficult;
	we don't have explicit values for $\omega_{x_{1}}$ and $\omega_{x_{2}}$! However, by 
	construction of our set $V$, we know that $\omega_{x_{1}}$ and $\omega_{x_{2}}$ are
	coprime. What this means is that, using the multiples of $\omega_{x_{1}}$ 
	and $\omega_{x_{2}}$, we can rule out every possible divisor of lcm$(\omega_{x_{1}}, 
	\omega_{x_{2}})$, except for lcm$(\omega_{x_{1}}, \omega_{x_{2}})$ itself.
	
	An example better illustrates why this works. 
	
	\begin{mathexample}
		\label{ex:specific_coprime_cycle_lengths}
		Consider two vectors with cycle lengths $12$ and $35$. Their sum would correspond to a vector 
		with a cycle length that divides lcm$(12, 35) = 420$, so our possible cycle lengths form a
		pretty long list:	$1$, $2$, $3$, $4$, $5$, $6$, $7$, $10$, $12$, $14$ $15$, $20$, $21$, $28$,
		$30$, $35$, $42$, $60$, $70$, $84$, $105$, $140$, $210$, and $420$.
		
		By checking possible cycle length in the same manner as in example \ref{ex:general_col_vects}, 
		we could show that cycle lengths of $210$ $(35 \times 6)$, $140$ $(35 \times 4)$, $84$ $(12 \times 7)$, 
		and $60$ $(12 \times 5)$ are impossible. By showing these cycle lengths are impossible, we're also 
		showing cycle lengths of all their factors are impossible (as discussed in example 
		\ref{ex:general_col_vects}). The only possible cycle length left is $420$.
		
		What's important here is that, within the factors of the multiples of $12$ and $35$ (up to 
		but not including $420$), every divisor of $420$ appears at least once.
	\end{mathexample}
	
	To show that the above logic holds for any two coprime numbers, we can consider the general case.
	
	\begin{mathexample}
		\label{ex:general_coprime_cycle_lengths}
		Consider two coprime numbers $x_1$ and $x_2$ with prime factorisations
		\begin{align*}
			x_1 &= {p_1}{p_2} \cdots {p_m} \\
			x_2 &= {q_1}{q_2} \cdots {q_n}
		\end{align*}
		Their lcm is then
		\[
			\text{lcm}(x_1, x_2) = {p_1}{p_2} \cdots {p_m}{q_1}{q_2} \cdots {q_n}
		\]
		Any factor $F$ of the lcm (that isn't the lcm itself) will have at least one factor 
		$p_i$ or $q_i$ removed from its product. So, if $F$ is missing at least one $q_i$ term
		in its factorisation, we can take $x_1$ and multiply it by the $q_i$ terms in $F$ to get 
		a multiple of $x_1$ which has $F$ as a factor. Similarly, if $F$ is missing at least one 
		$p_i$ term in its factorisation, we can take $x_2$ and multiply it by the $p_i$ terms in
		$F$ to get a multiple of $x_2$ which has $F$ as a factor.
		
		Importantly, each of the multiples of $x_1$ and $x_2$ we end up with in the process above
		will give numbers less than lcm$(x_1, x_2)$ since there will always be at least one
		prime factor missing from the resulting number.
		
		This means that, given two coprime numbers $x_1$ and $x_2$, every factor of lcm$(x_1, x_2)$,
		except the lcm itself, will appear as a factor in the multiples of $x_1$ or $x_2$ at least
		once for multiples less than the lcm.
	\end{mathexample}
	
	Example \ref{ex:general_coprime_cycle_lengths} says we can check all possible cycle
	lengths for $\vec{w}_{1}$ just by checking the multiples of $\omega_{x_{1}}$ and 
	$\omega_{x_{2}}$, since the multiples of $\omega_{x_1}$ and $\omega_{x_2}$ will contain
	all possible factors of lcm$(\omega_{x_1}, \omega_{x_2})$ (except the lcm itself). 
	This is useful, as now we can more easily reason that the cycle length of $\vec{w}_{1}$ 
	must be lcm$(\omega_{x_{1}}, \omega_{x_{2}})$.
	
	For any number $k$ to be a cycle length for $\vec{w}_{1}$, we must have that
	\[
		A^{k}\vec{w}_{1}                          \equiv 
		A^{k}(\vec{v}_{x_{1}} + \vec{v}_{x_{2}})  \equiv 
		A^{k}\vec{v}_{x_{1}} + A^k\vec{v}_{x_{2}} \equiv
		\vec{v}_{x_{1}} + \vec{v}_{x_{2}}
	\]
	
	We know that, since we're only checking multiples of $\omega_{x_{1}}$ and $\omega_{x_{2}}$,
	either $A^{k}\vec{v}_{x_{1}}$ reduces to $\vec{v}_{x_{1}}$ or $A^k\vec{v}_{x_{2}}$ reduces to
	$\vec{v}_{x_{2}}$. This means that, to get the sum $\vec{v}_{x_{1}} + \vec{v}_{x_{2}}$, it
	must be the case that $k$ is a multiple of both cycle lengths $\omega_{x_{1}}$ and $\omega_{x_{2}}$
	so that both terms reduce. This happens exactly when $k = \text{lcm}(\omega_{x_{1}},
	\omega_{x_{2}})$. Therefore, we can conclude that the cycle length of $\vec{w}_{1}$ will \emph{always}
	be lcm$(\omega_{x_{1}}, \omega_{x_{2}})$.
	
	Now, it's simply a matter of continuing to construct new vectors by taking our current $\vec{w}$ and
	adding the next vector from $V$. All of the cycle lengths of the vectors in $V$ are coprime, so
	the cycle length of $\vec{w}_{i}$ will be the least common multiple of the cycle length of $\vec{w}_{i-1}$
	and the cycle length of whatever vector we pulled from $V$. Thus, once we've exhausted all the vectors
	from $V$, our resulting $\vec{w}$ will have a cycle length equal to the least common multiple of all
	the cycle lengths of the vectors in $V$, which we've defined to be the cycle length of $A$, or the maximal
	cycle length. 
	
	\ssubsection{General Procedure}
	Using the above processes, we can formulate a rather straightforward procedure for creating a vector 
	with maximal cycle length for matrices with column vectors in their cycles with coprime cycle lengths,
	or when a column vector already has the maximal cycle length.
	
	\begin{enumerate}
		\item Calculate the cycle lengths of all the column vectors which compose the chosen matrix's cycle.
		\item Create a subset of these column vectors $V$ where the cycle lengths are coprime and where
		the least common multiple of the vectors' cycle lengths is equal to the least common multiple of all 
		the matrix's column vectors' cycle lengths.
		\item Let $\vec{w}$ equal the sum of the vectors in $V$. $\vec{w}$ will have maximal cycle length.
	\end{enumerate}
	
	\ssection{Who?}
	This document was created to (hopefully!) extend proposition 3 made in \citet{Mendivil2012}.
	Because matrices with column vectors with coprime cycle lengths have a vector whose cycle length is 
	equivalent to the matrix's cycle length (the maximal cycle length), then we automatically know all 
	other vectors must have a cycle length that divides this maximal cycle length. This extends the previous 
	proposition, which showed that all vectors had cycle lengths that divided the maximal cycle length for prime 
	moduli.
	
	\bibliographystyle{plainnat}
	\bibliography{refs.bib}
\end{document}
