
\documentclass[a4paper, reqno, 12pt]{amsart}

\usepackage[T1]{fontenc}
\usepackage[margin=3cm]{geometry}
\usepackage[parfill]{parskip}

\usepackage{setspace}
\setstretch{1.25}

\usepackage{dsfont}
\usepackage{amsmath}
\usepackage{amssymb}
%\usepackage{nicematrix} %Used for adding dividers in complicated matrices

%\usepackage{array} %For nicer tables

\usepackage{natbib}

%spans
\newcommand\vecspan[1]{\textup{span}(#1)}

%Defining an environment for propositions
\newcounter{propcounter}[section]
\newenvironment{proposition}[1]
{
	\refstepcounter{propcounter} %Increment environment counter
	\textbf{Proposition \thesection.\thepropcounter.} \emph{#1}
	
	\emph{Justification.}
}
{
}

\begin{document}
	\section{Some General Vector Stuff}
		A set of vectors $V$ is called \emph{dimensionally independent} if
		\[
			\forall \vec{v} \in V, \quad \vecspan{\vec{v}} \cap \vecspan{V \setminus \{\vec{v}\}} = \{\vec{0}\}
		\]
		
		\begin{proposition}{Given a set of dimensionally-independent vectors $V \subseteq \mathds{Z}_{p^k}^L$ and a vector 
		$\vec{w} = \sum_{i=0}^{k-1} p^i\vec{u}_i,\, \vec{u}_i \in \mathds{Z}_p^L$, then
		$\vec{w} \not\in \vecspan{V} \implies \vecspan{\sum_{i \in J} p^i\vec{u}_i} \cap \vecspan{V} = \{\vec{0}\} \textup{ \& } 
		\vecspan{V \cup \{\sum_{i \in J} p^i\vec{u}_i\}} = \vecspan{V \cup \{\vec{w}\}}$ 
		modulo $p^k$, with $p$ a prime, $k \in \mathds{Z}^+, J \neq \emptyset$.}
			Let $J$ be the set of all integer indices $0 \leq j < k$ such that $p^j\vec{u}_j \not\in \vecspan{V}$. Let $\ell$ be the smallest element in $J$. 
			Let $\vec{m} = \sum_{i \in J} p^i\vec{u}_i$. Let $\vec{n} = \vec{w} - \vec{m}$. By construction, $\vec{n} \in \vecspan{V}$. We know 
			$\vec{w} = \vec{m} + \vec{n} \not\in \vecspan{V}$, so $\vec{m} \not\in \vecspan{V}$. 
			
			To see why $\vecspan{\vec{m}} \cap \vecspan{V} = \{\vec{0}\}$, assume it isn't true. Then there would exist an integer $1 \leq c \leq k-\ell$ such that 
			$p^c\vec{m} \in \vecspan{V}$.
		\end{proposition}
\end{document}
