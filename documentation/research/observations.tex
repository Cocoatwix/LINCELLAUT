
\documentclass[a4paper, reqno, 12pt]{amsart}

\usepackage[T1]{fontenc}
\usepackage[margin=3cm]{geometry}
\usepackage[parfill]{parskip}

\usepackage{setspace}
\setstretch{1.25}

\usepackage{dsfont}
\usepackage{amsmath}
\usepackage{amssymb}
%\usepackage{nicematrix} %Used for adding dividers in complicated matrices

\usepackage{array} %For nicer tables

\usepackage{natbib}

%Defining choose and permutation notation
\newcommand\combine[2]{{}^{#1}C_{#2}}
%\newcommand\permute[2]{{}^{#1} \hspace{-0.05cm} P_{#2}}

%Defining an environment for observations
\newcounter{obscounter}[section]
\newenvironment{observation}[2]
{
	\refstepcounter{obscounter} %Increment environment counter
	\addcontentsline{toc}{subsection}{\thesection .\theobscounter \;\;\;\; #2}
	%\makeatother
	\textbf{Observation \thesection .\theobscounter} 
	- #1
	
	\textbf{Justification} \\
}
{
	\vspace{1cm}
}

\begin{document}
	\tableofcontents
	\section{Explanation}
		This LaTeX file serves as a growing catalogue for any interesting and/or relevant observations regarding linear cellular automata over $\mathds{Z}_N$. The results
		listed here should not be taken as proofs, but rather as things that \emph{may} be true and are worth looking into. To the best of my ability, every observation
		here will be listed with as rigorous a justification as possible to explain why I believe each observation to be true. Again, \textbf{these justifications are not 
		necessarily proofs}. They very well might be, but they're simply my attempt to show why I believe something to be true. 
		
		The following section will list all the relevant definitions and terms needed to understand the observations herein. The remaining sections will list the 
		observations, grouped by topic and sorted in no particular order.
		
	\section{Terms, Definitions, Notation}
		\subsection{Notation}
			\begin{description}
				\item[Divisibility] If $b$ divides $a$ (meaning $\frac{a}{b}$ results in a remainder of 0), this is represented as $b \, | \, a$. If $b$ doesn't divide $a$, 
				this is represented as $b \nmid a$.
				
				\item[Set of All Matrices] The set of all $L_1$ by $L_2$ matrices with elements in $\mathds{Z}$ will be represented by $\mathds{Z}^{L_1 \times L_2}$. 
				Similarly, the set of all $L_1$ by $L_2$ matrices with elements in $\mathds{Z}_{N}$ will be represented by $\mathds{Z}_{N}^{L_1 \times L_2}$.
				
				\item[Set of All Polynomials] The set of all polynomials with coefficients in $\mathds{Z}_{N}$ will be represented as $\mathds{Z}_{N}[x]$.
			\end{description}
			
		\subsection{Miscellaneous}
			\begin{description}
				\item[Factorial] $x! = \prod_{i \, = \, 1}^{x} i$, the product of all integers from 1 to $x$.
			
				\item[Combinations] The number of ways to choose $y$ items from a collection of $x$, disregarding order, will be denoted as $\combine{x}{y}$. This can also
				be computed explicitly as $\frac{x!}{y!(x-y)!}$.
				
				\item[Monic Polynomial] A polynomial whose leading term has a coefficient of 1.
				
				\item[Cycle Converting Matrix (CCM)] A matrix of the form $C_{\beta \rightarrow \alpha} \equiv \sum_{i\,=\,0}^{\frac{\beta-\alpha}{\alpha}} A^{\alpha i}$ for
				some other update matrix $A \in \mathds{Z}_{N}^{L \times L}$, where $\alpha \, | \, \beta$ and $\beta \, | \, \omega$, where $\omega$ is the cycle length
				of $A$.
			\end{description}
			
		\subsection{Linear Cellular Automata}
			\begin{description}
				\item[Multiplicative Order] The \emph{multiplicative order} of some object $\alpha$ is the smallest positive integer $c$ such that $\alpha^c \equiv 1$.
			
				\item[Module] For the purposes of this document, a \emph{module} will refer to sets of ordered tuples of the integers modulo some integer, i.e. 
				$\mathds{Z}_{N}^{L}$. They can be thought of as vector spaces with less strict definitions.
			
				\item[Cycle Length] The \emph{cycle length} of a vector $\vec{v}$ under iteration by $A$ is defined to be the smallest positive integer $\omega$ such that
				$A^{\tau + \omega}\vec{v} \equiv A^{\tau}\vec{v}$ for some nonnegative integer $\tau$, called the \emph{transient length} of the vector $\vec{v}$. The cycle 
				length can be thought of as the number of iterations needed for $\vec{v}$ to return back to itself. The \emph{cycle length} of the matrix $A$ is defined to 
				be the smallest positive integer $\omega$ such that $A^{\tau + \omega} \equiv A^{\tau}$ for some nonnegative integer $\tau$, called the \emph{transient 	
				length} of the matrix $A$.
			
				\item[Transient Length] The \emph{transient length} of a vector $\vec{v}$ under iteration by $A$ is the largest nonnegative integer $\tau$ such that 
				$A^{\tau}\vec{v} \not\equiv A^{\tau + c}\vec{v} \;\; \forall c \in \mathds{Z}^+$. It can be thought of as the number of iterations needed for the vector $
				\vec{v}$ to fall into a cycle.
			
				\item["iterating"] The process of multiplying a vector or matrix by some update matrix. For instance, iterating the vector $\vec{v}$ twice means to calculate
				$A^{2}\vec{v}$.
			
				\item[Maximal cycle length] Under iteration my a matrix $A$, the \emph{maximal cycle length} for a vector is the cycle length of the matrix $A$. It's unknown
				whether a vector with maximal cycle length always exists in $\mathds{Z}_{p^k}^{L}$.
			
				\item["iterating back to itself"] Used to describe the process of a vector or matrix iterating enough times to return back to its "original value". The 
				meaning of "original value" may change depending on context.
				
				\item[Matrix Lift] A \emph{lift} of a matrix $A \in \mathds{Z}_{p^k}^{L \times L}$ is any matrix $B \in \mathds{Z}_{p^{k+n}}^{L \times L}$ with the property
				that $B \equiv A \bmod{p^k}$. This means $B = A + \sum_{i\,=\,0}^{n-1} p^{k+i}G_i, \, G_i \in \mathds{Z}_{p}^{L \times L}$.
				
				\item[Minimal Polynomial] The \emph{minimal polynomial} of a matrix $A \in \mathds{Z}_{p}^{L \times L}$ is the monic polynomial $\mu(x)$ of least degree such 
				that $\mu(A) \equiv 0$. 
			
				\item[Minimal Annihilating Polynomial] The \emph{minimal annihilating polynomial} of a vector $\vec{v} \in \mathds{Z}_{p}^{L}$ is the monic polynomial $\rho(
				x)$ of least degree such that, for a given matrix $A \in \mathds{Z}_{p}^{L \times L}$, $\rho(A)\vec{v} \equiv 0$.
			\end{description}
			
	\section{Fields}
		\label{sec:fields}
		\begin{observation}{For all elements $n \in \mathds{Z}_{p}, n \leq \frac{p-1}{2}$, no two distinct elements can square to the same number.}
		{$\forall m, n \in \mathds{Z}_{p}, m, n \leq \frac{p-1}{2}, m \not\equiv n \implies m^2 \not\equiv n^2$}
			\label{obs:zmodpsquares}
			In order for no two distinct elements in $\mathds{Z}_{p}$ less than or equal to $\frac{p-1}{2}$ to square to the same thing, it must be the case that the first
			$\frac{p-1}{2}+1$ square numbers (using $0^2$ as the first) are all distinct. As an example, for $\mathds{Z}_7$, the first $\frac{7-1}{2}+1=4$ square 
			numbers are 0, 1, 4, and 2. Since these are all distinct, we know for all $n \leq \frac{p-1}{2}, p=7$, no two numbers square to the same thing.
			
			If we were to list out these square numbers in order (including one extra to show where they start repeating), taking into account the amount added to one 
			square number to get the next, we'd get:
			
			%A modern masterpiece
			\begin{center}
				\begin{tabular}{c}
					\begin{tabular}{*{4}{b{0.3cm}}}
						\tiny{+1} & \tiny{+3} & \tiny{+5} & \tiny{+7}
					\end{tabular} \\
					\begin{tabular}{*{4}{p{0.3cm}}}
						$\curvearrowright$ & $\curvearrowright$ & $\curvearrowright$ & $\curvearrowright$
					\end{tabular} \\
					\begin{tabular}{*{5}{p{0.3cm}}}
						0 & 1 & 4 & 2 & 2
					\end{tabular}
				\end{tabular}
			\end{center}
			
			where the arithmetic is carried out mod 7. What's interesting to note is that, to get from one square number to the next, we simply add a sequence of odd
			numbers. For example, to get from 1 to 2, we add +3, then +5. In fact, for any prime modulus we use, going from one square number to the next requires
			simply adding a sequence of some consecutive odd numbers.
			
			This gives us a way to potentially show that none of the square numbers we're interested in will ever repeat. If we could find a way to create a consecutive
			sequence of odd numbers (within our bounds) that sums to a multiple of our prime $p$, then that means there would be a way to hop from one square number
			to the same square number (since adding any multiple of $p$ is the same as adding zero mod $p$). Inversely, if there's no such way to find a consecutive
			sequence of odd numbers like this, then we've shown that all square numbers within our bounds are unique.
			
			Using the list above as reference, we can see that, if we're in $\mathds{Z}_p$, our odd numbers range from 1 to $p-2$. Let's use $a$ to index the start of our
			sequence and $b$ to index the end of our sequence. If our starting odd number is written as $2a+1$, then $0 \leq a \leq \lfloor\frac{p}{2}\rfloor-1$. If our
			ending odd number is written as $2b+1$, then $a \leq b \leq \lfloor\frac{p}{2}\rfloor-1$.
			
			%Don't mess with me; I can do whatever I want
			\begin{figure*}[b]
				\[
					\overbrace{\hspace{3cm}}^{b+1}
					\vspace{1.4cm}
				\]
				\[
					a \;
					\Biggl\{ 
					\vspace{-3cm}
					\hspace{3.7cm}
				\]
				\[
				\begin{matrix}
					\bullet & \bullet & \bullet & \bullet & \bullet & \bullet \\
					\bullet & \bullet & \bullet & \bullet & \bullet & \bullet \\
					\bullet & \bullet & \bullet & \bullet & \bullet & \bullet \\
					\circ   & \circ   & \circ   & \bullet & \bullet & \bullet \\
					\circ   & \circ   & \circ   & \bullet & \bullet & \bullet \\
					\circ   & \circ   & \circ   & \bullet & \bullet & \bullet
				\end{matrix}
				\]
				\caption{A visual representation of what the sum of odd numbers could look like mod 13, with $a=3$ and $b=5$.
				The sum is equal to the number of black dots.}
			\end{figure*}
			
			Using $a$ and $b$, we can write the sum of our resulting sequence of odd numbers as
			\[
				(b+1)^2 - a^2 = (b+1-a)(b+1+a)
			\]
			What's the minimum and maximum value the terms on the RHS of the equation can be?
			Using the bounds we created for $a$ and $b$:
			\begin{alignat*}{3}
				\text{min}(b+1-a) &= 0+1-0                                      &&= 1                                    \\
				\text{max}(b+1-a) &= \left\lfloor\frac{p}{2}\right\rfloor-1+1-0 &&= \left\lfloor\frac{p}{2}\right\rfloor \\
				\text{min}(b+1+a) &= 0+1+0                                      &&= 1                                    \\
				\text{max}(b+1+a) &= \left\lfloor\frac{p}{2}\right\rfloor-1+1+\left\lfloor\frac{p}{2}\right\rfloor-1 &&= 2\left\lfloor\frac{p}{2}\right\rfloor-1
			\end{alignat*}
			
			$p$ is prime, so for $(b+1)^2-a^2$ to be a multiple of $p$, one of $(b+1-a)$ or $(b+1+a)$ must be a multiple of $p$. However, from the minimum and
			maximum calculations above, both these factors are bounded between 1 and $2\left\lfloor\frac{p}{2}\right\rfloor-1$, which is strictly less than $p$. Therefore,
			there's no possible way either of these factors can be a multiple of $p$. Therefore, within our given bounds, a consecutive sequence of odd numbers can never
			sum to a multiple of $p$, and therefore no numbers $\leq \frac{p-1}{2}$ in $\mathds{Z}_p$ can square to the same number. 
		\end{observation}
		
		\begin{observation}{For all elements $n \in \mathds{Z}_{p^2}, n \leq \frac{p^2-1}{2}, \, p \nmid n$, no two distinct elements can square to the same number.}
		{$\forall m, n \in \mathds{Z}_{p^2}, m, n \leq \frac{p^2-1}{2}, p \nmid m, p \nmid n, m \not\equiv n \implies m^2 \not\equiv n^2$}
			The idea is very similar to Observation \ref{sec:fields}.\ref{obs:zmodpsquares}. If we look at the differences between differing square numbers, we see the
			difference is given by sums of consecutive odd numbers. If we let the starting odd number be represented as $2a+1$ and the ending odd number as $2b+1$, the
			sum of all the odd numbers from $2a+1$ to $2b+1$ is given by
			\[
				(b+1)^2 - a^2 = (b+1-a)(b+1+a)
			\]
			
			In this case, we want to ensure that this sum is never a multiple of $p^2$ within our bounds. Otherwise, there would exist a pair of square numbers in our bound
			with the same value mod $p^2$. Of course, in $\mathds{Z}_{p^2}$, any multiple of $p$ will square to 0, so we ignore these by saying that $p$ can't divide our 
			original numbers before squaring. 
			
			Setting our bounds as $0 \leq a \leq \lfloor\frac{p^2}{2}\rfloor-1$ and $a \leq b \leq \lfloor\frac{p^2}{2}\rfloor-1$ ensures we only consider squaring numbers in
			$\mathds{Z}_{p^2}$ which are $\leq \frac{p^2-1}{2}$. Computing the minimum and maximum value for our factors in the sum of consecutive odd numbers as before, we 
			see:
			\begin{alignat*}{3}
				\text{min}(b+1-a) &= 0+1-0                                        &&= 1                                      \\
				\text{max}(b+1-a) &= \left\lfloor\frac{p^2}{2}\right\rfloor-1+1-0 &&= \left\lfloor\frac{p^2}{2}\right\rfloor \\
				\text{min}(b+1+a) &= 0+1+0                                        &&= 1                                      \\
				\text{max}(b+1+a) &= \left\lfloor\frac{p^2}{2}\right\rfloor-1+1+\left\lfloor\frac{p^2}{2}\right\rfloor-1 &&=
				                     2\left\lfloor\frac{p^2}{2}\right\rfloor-1
			\end{alignat*}
			Using these bounds, neither factors in the sum of consecutive odd number can never have a factor of $p^2$ in them. Also, because we explicitly assume that our
			sequence won't start or end on a multiple of $p$-th odd number (e.g. if our prime was 7, then we'd never start or end the sequence on the 7th odd number, the
			14th, 21st, etc., where 1 is treated as the zeroth odd number), a factor of $p^2$ can never be created by multiplying two factors with $p$ in them. Therefore, 
			with our given restrictions and bounds, there's no way the sum of consecutive odd numbers can be a multiple of $p^2$, which means for any two numbers in 
			$\mathds{Z}_{p^2}$ which are $\leq \frac{p^2-1}{2}$ are aren't multiples of $p$, their squares won't be equal.
		\end{observation}
		
	\section{Polynomials, Roots, Field Extensions}
		\label{sec:polynomials}
		\begin{observation}{Given two positive integers $i$ and $j$ with $i \geq j$, 
		\[
			\text{gcd}(x^i-1, \, x^j-1) = x^{\text{gcd}(i, \, j)} - 1 
		\]}{$\text{gcd}(x^i-1, \, x^j-1) = x^{\text{gcd}(i, \, j)} - 1$}
			\label{obs:primegcdexpression}
			\textbf{First observation -} If $j \nmid i$, then $(x^i - 1) \bmod{(x^j - 1)} \equiv x^{i \bmod{j}} - 1$.
			
			To find $(x^i - 1) \bmod{(x^j - 1)}$, we need to divide $x^i - 1$ by $x^j - 1$ and take the remainder. Performing this division, we get
			\[
				\frac{x^i - 1}{x^j - 1} = x^{i-j} + x^{i-2j} + x^{i-3j} + \cdots + x^{i-nj} \text{ R } x^{i-nj} - 1
			\]
			where R signifies the remainder of the division and where $nj$ is the largest positive multiple of $j$ such that $i - nj \geq 0$. $i - nj$ is the same as $i 
			\bmod{j}$, so we can rewrite the above division as
			\[
				\frac{x^i - 1}{x^j - 1} = \left(\sum_{c \, = \, 0}^{\bigl\lfloor\frac{i}{j}\bigl\rfloor} x^{i-cj}\right) \text{ R } \, x^{(i \bmod{j})} - 1
			\]
			
			\textbf{Second observation - } If $j \, | \, i$, then $x^i - 1 \bmod{x^j - 1} \equiv 0$.
			
			Repeating the same division as before, we see
			\[
				\frac{x^i - 1}{x^j - 1} = x^{i-j} + x^{i-2j} + x^{i-3j} + \cdots + x^{i-(n-1)j} + 1
			\]
			The difference in this case is that, unlike when $j \nmid i$, $i-nj = 0$, since $j$ is a factor of $i$. This means the division works out more nicely in that
			it's possible to multiply $x^j - 1$ by another polynomial and get $x^i - 1$ \emph{exactly} as opposed to only getting close.
			
			This leaves us with a remainder of 0 in this case.
			
			These observations allow us to find the gcd of our two polynomials using the Euclidean algorithm. To find the gcd of $x^i - 1$ and $x^j - 1$ using the Euclidean
			algorithm, we divide the two polynomials and find the remainder. If the remainder is zero, the divisor of the division is taken to be the gcd. If the remainder
			isn't zero, we repeat this process using the divisor as the new dividend and the remainder as the new divisor, repeating as many times as necessary until the
			remainder is zero, at which point we can find the gcd of the two original polynomials by looking at the most recent divisor.
			
			This allows us to check the claim above. If $j \, | \, i$, then gcd$(x^i - 1, \, x^j - 1) = x^j - 1$ directly from the Euclidean algorithm and our second
			observation.
			
			If $j \nmid i$, then we'll end up computing a sequence of divisions until the remainder is zero:
			\[
				\frac{x^i - 1}{x^j - 1}, \, \frac{x^j - 1}{x^{(i \bmod j)} - 1}, \, \frac{x^{(i \bmod j)} - 1}{x^{(j \bmod{(i \bmod j)})} - 1}, \, \cdots
			\]
			
			What's interesting about this sequence is that the exponents on each polynomial mimic how $i$ and $j$ would behave under the Euclidean algorithm on their own;
			the sequence of divisions we get trying to find gcd$(i, \, j)$ gives us the exponents for our polynomials above:
			\[
				\frac{i}{j}, \, \frac{j}{i \bmod{j}}, \, \frac{i \bmod{j}}{j \bmod{(i \bmod{j})}}, \, \cdots
			\]
			This is a direct consequence of our first observation. Since the Euclidean algorithm finishing for gcd$(i, \, j)$ means the divisor divides the 
			dividend for the resulting division, the exponents for the corresponding polynomial division in gcd$(x^j-1, \, x^i-1)$ will also divide, putting us in the first 
			case where the division will result in a remainder of zero. 
			
			This means that the number of divisions we need to compute for gcd$(x^j - 1, \, x^i - 1)$ is exactly the same as the number of divisions we need to compute for
			gcd$(i, \, j)$. We know the Euclidean algorithm will end when the divisor of the resulting division is the gcd of the two starting values, so we know that the
			exponent on the divisor of the last division for gcd$(x^j-1, \, x^i-1)$ must be gcd$(i, \, j)$, since that will be the value of the divisor of the last division
			while calculating gcd$(i, \, j)$.
			
			Putting these two cases together suggests that gcd$(x^i-1, \, x^j-1) = x^{\text{gcd}(i, \, j)}-1$.
		\end{observation}
		
		\begin{observation}{Given a root $\alpha$ of an irreducible monic polynomial $\mu(x) \in \mathds{Z}_{p}[x]$, where the degree of $\mu(x)$ is $d$, $\alpha$ must have 
		a multiplicative order that divides $p^d - 1$ in $\mathds{Z}_p$.}
		{Order of roots of irreducible polynomials}
			\label{obs:irreducibleroots}
			The root $\alpha$ is a purely algebraic object, one that's defined entirely by the monic polynomial given. Therefore, any other object that satisfies the same
			algebraic definition given should share the same properties as $\alpha$, at least up to the properties that the monic polynomial guarantees. In this case, it's
			helpful to imagine a matrix $A$ whose minimal polynomial is $\mu(x)$. By properties of the polynomial, this tells us that $A \in \mathds{Z}_{p}^{d \times d}$.
			
			Because the minimal polynomial of $A$ is $\mu(x)$, $A$ is also a sort of root of the polynomial $\mu(x)$. For us, this means that the multiplicative order of $A$
			should match with the multiplicative order of $\alpha$ (since powers of both $\alpha$ and $A$ will take the same algebraic form and abide by the same algebraic
			restrictions).
			
			Now, we can make use of Observation \ref{sec:matrices}.\ref{obs:onebigcyclelength} to deduce properties of $A$. Because the minimal polynomial of $A$ is 
			irreducible, that means iterating vectors by $A$ can create at most 2 different cycle lengths, one of which being 1 while the other is some value $\geq 1$.
			We only care about this second cycle length as, when these vectors iterate back to themselves, the matrix $A$ must cycle back to itself, too (since all vectors
			would be iterated back to their original values, $A$ would necessarily have to iterate to $I$). This means the matrix's multiplicative order is tied to the 
			possible values this second cycle length can have. From Observation \ref{sec:matrices}.\ref{obs:onebigcyclelength}, we know that all the possible cycle lengths
			must divide $p^d - 1$, meaning the multiplicative order of $A$ must also divide $p^d - 1$.
			
			Because the multiplicative order of $A$ divides $p^d - 1$, the multiplicative order of $\alpha$ must also divide $p^d - 1$.
		\end{observation}
		
		\begin{observation}{For $i \geq 1$, the polynomial $x^{p^i} - x \in \mathds{Z}_{p}[x]$ is the product of all monic irreducible polynomials in $\mathds{Z}_{p}[x]$
		whose degree divides $i$.}{Properties of $x^{p^i} - x \in \mathds{Z}_{p}[x]$}		
			Firstly, let's factor this polynomial as
			\[
				x(x^{p^i - 1} - 1) \mod{p}
			\]
			Clearly, the irreducible factor $x$ will always be in this polynomial. We'll now focus our attention only on the second factor:
			\[
				x^{p^i - 1} - 1 \mod{p}
			\]
			This polynomial behaves similarly to a "roots of unity" polynomial, in that every factor defines some root that has an order that divides $p^i-1$. In fact, 
			because we're in a field ($\mathds{Z}_{p}$), this polynomial has \emph{every} factor that defines a root with order that divides $p^i-1$. Also, because we're in a
			finite field, we can always factor out some constant multiple from each factor to make them monic, so we'll assume all factors we're dealing with are monic.
			
			Because of Observation \ref{sec:polynomials}.\ref{obs:irreducibleroots}, we know that all irreducible polynomials of degree $i$ have roots with orders that 
			divide $p^i-1$. Therefore, the polynomial $x^{p^i-1}-1$ must contain all irreducible monic polynomials of degree $i$.
			
			As well, we can show that all irreducible monic polynomials of degree $j, \, j \, | \, i$ are also factors in $x^{p^i-1}-1$. If $j \, | \, i$, then 
			$p^j-1 \, | \, p^i-1$ by Observation \ref{sec:polynomials}.\ref{obs:primegcdexpression}. This means any irreducible monic polynomial of degree $j$ will have 
			roots with orders that divide $p^i-1$, meaning they're factors of $x^{p^i-1}-1$.
			
			Now let's say we're given an irreducible monic polynomial with degree $k, \, k \nmid i$. By Observation \ref{sec:polynomials}.\ref{obs:primegcdexpression}, we
			know that if the order of the root from the $k$-th degree polynomial divides $p^i-1$, it must also divide $p-1$. However, every root with order that divides
			$p-1$ would simply be a linear factor of the form $x-c, \, c \in \mathds{Z}_{p}$. Therefore, no irreducible monic polynomial with degree $k$ is a factor of 
			$x^{p^i-1}-1$.
			
			The last thing we need to show is that no irreducible factor repeats in the factorisation of $x^{p^i-1}-1$. Taking the derivative of this polynomial, we see
			\[
				\frac{\mathrm{d}}{\mathrm{d}x} x^{p^i-1}-1 \equiv \, (p^i-1)x^{p^i-2} \mod{p}
			\]
			For any root to be a repeated root of $x^{p^i-1}-1$, it must also be a root of its derivative. The only root of the derivative above is 0, which we know isn't
			a root of $x^{p^i-1}-1$. Therefore, none of the roots of $x^{p^i-1}-1$ are repeated.
			
			So, this should be enough to show that the polynomial $x(x^{p^i-1}-1) = x^{p^i}-x$ is the product of all irreducible monic polynomials over $\mathds{Z}_{p}[x]$
			whose degree divides $i$.
		\end{observation}
		
		\begin{observation}{For a polynomial $q(x) \in \mathds{Z}_{p}[x]$, 
		\[
			q'(x) \equiv 0 \implies (q(x^{\frac{1}{p}}))^p \equiv q(x) \mod{p}
		\]}
		{$q'(x) \equiv 0 \implies (q(x^{\frac{1}{p}}))^p \equiv q(x)$}
			If $q'(x) \equiv 0$, this means $q(x)$ is a polynomial in $x^p$, since these are the polynomial terms which get sent to 0 under differentiation. This means
			$q(x)$ looks something like
			\[
				q(x) = c_0 + {c_1}x^p + {c_2}x^{2p} + \cdots + {c_n}x^{np} \mod{p}
			\]
			Plugging in $x = x^{\frac{1}{p}}$, we get
			\begin{align*}
				q(x^{\frac{1}{p}}) &= c_0 + {c_1}(x^{\frac{1}{p}})^p + {c_2}(x^{\frac{1}{p}})^{2p} + \cdots + {c_n}(x^{\frac{1}{p}})^{np} \\
				                   &= c_0 + {c_1}x + {c_2}x^2 + \cdots + {c_n}x^n \mod{p}
			\end{align*}
			Then, raising this entire expression to the $p$-th power gives
			\[
				(q(x^{\frac{1}{p}}))^p = (c_0 + {c_1}x + {c_2}x^2 + \cdots + {c_n}x^n)^p \mod{p}
			\]
			This expression seems difficult to simplify, but we can make use of a few observations to help us.
			
			Firstly, the resulting expression will be a polynomial with terms from $x^0$ to $x^{np}$:
			\[
				\equiv d_0 + d_1{x} + d_2{x^2} + \cdots + d_{np}{x^{np}} \mod{p}
			\]
			How will these terms be created? Each one will the the sum of all possible ways to create that specific power of $x$. For example, our $x^4$ term will be the
			sum of all possible $(x)(x)(x)(x)$ terms (four $x$ terms multiplied together, then multiplied by a bunch of constants), as well as all $(x^2)(x)(x)$ terms (an 
			$x^2$ term multiplied by two $x$ terms, then multiplied by a bunch of constants), $(x^3)(x)$ terms, $(x^2)(x^2)$ terms, and $x^4$ terms directly. How many 
			different ways are there to create these terms?
			
			Imagine a simpler example, where $(q(x^{\frac{1}{p}}))^p$ looked like
			\begin{align*}
				(q(x^{\frac{1}{p}}))^p = & \, (c_0 + {c_1}x + {c_2}x^2 + {c_3}x^3 + {c_4}x^4) \, \times \\
				                         & \, (c_0 + {c_1}x + {c_2}x^2 + {c_3}x^3 + {c_4}x^4) \, \times \\
				                         & \, (c_0 + {c_1}x + {c_2}x^2 + {c_3}x^3 + {c_4}x^4) \, \times \\
				                         & \, (c_0 + {c_1}x + {c_2}x^2 + {c_3}x^3 + {c_4}x^4) \mod{p}
			\end{align*}
			To get a $(x^2)(x)(x)$ term, for instance, we need to pick one line to get a $x^2$ from, and two lines to get a $x$ from. The number of ways to do this is
			$(\combine{4}{1})(\combine{3}{2})$. Similarly, every possible term we can create with various powers of $x$ can be counted using some product of combination
			functions. It's important to note that each term we count this way will be exactly the same. Each $x^2$ from each line has the same coefficient, as does any
			other power we choose. Therefore, the resulting term we get at the end of this multiplication will look like 
			${c_2}{c_1}{c_1}{c_0}(\combine{4}{1})(\combine{3}{2})x^{2}xx$.
			
			Another important note is that these terms can be chosen in any order. Instead of choosing the $x^2$ term first, we just as easily could have chosen the two
			$x$ terms first, in which case our resulting term would look like ${c_1}{c_1}{c_2}{c_0}(\combine{4}{2})(\combine{2}{1})xxx^2$. The two results are equivalent.
			
			Going back to our original problem, this means that, for some term $(x^\alpha)(x^\beta)(x^\gamma)\cdots$, the resulting term will be given by 
			$D(\combine{p}{A})(\combine{p - A}{B})(\combine{p - A+B}{C})\cdots(x^\alpha)(x^\beta)(x^\gamma)\cdots$ where $D$ is the product of all the relevant 
			coefficients, and $A, \, B, \, C, \, \cdots$ is the number of times each power of $x$ appears. Because these can be in any order, we'll assume 
			$A, \, B, \, C, \, \cdots$ are in descending order. Now, if none of $A, \, B, \, C, \, \cdots$ are equal to $p$, then the first combination function, 
			$\combine{p}{A}$, will result in a factor of $p$. Modulo $p$, our entire term will subsequently disappear because of it. In fact, the majority of valid 
			combinations of $A, \, B, \, C, \, \cdots$ won't have any numbers equal to $p$. The only possible terms that'll result in one of $A, \, B, \, C, \, \cdots$ being
			equal to $p$ are $(x^0), \, (x)(x)(x)\cdots, \, (x^2)(x^2)(x^2)\cdots, \, (x^3)(x^3)(x^3)\cdots, \, \cdots$, the terms with $p$ copies of the same power of $x$. 
			These are the terms that won't reduce to zero modulo $p$.
			
			Clearly, the terms $(x^0), \, (x)(x)(x)\cdots, \, (x^2)(x^2)(x^2)\cdots, \, (x^3)(x^3)(x^3)\cdots, \, \cdots$ will result in an expression similar to
			\[
				(q(x^{\frac{1}{p}}))^p \equiv d_0 + {d_p}x^p + {d_{2p}}x^{2p} + \cdots + {d_{np}}x^{np} \mod{p}
			\]
			Can we show that the coefficients $d_i$ match with the coefficients of $q(x)$? Expanding each term out gives us
			\begin{align*}
				\equiv & \, {c_0}{c_0}\cdots + ({c_1}{c_1}\cdots){x}xx\cdots + ({c_2}{c_2}\cdots){x^2}{x^2}{x^2}\cdots + \cdots + ({c_n}{c_n}\cdots)
				{x^n}{x^n}{x^n}\cdots \\
				\equiv & \, {{c_0}^p} + {{c_1}^p}x^p + {{c_2}^p}(x^2)^p + \cdots + {{c_n}^p}(x^n)^p \mod{p}
			\end{align*}
			In $\mathds{Z}_{p}$, every number has a multiplicative order that divides $p-1$, so $c^p \equiv c \;\; \forall c \in \mathds{Z}_{p}$. We can simplify our above
			expression to
			\[
				\equiv c_0 + {c_1}x^p + {c_2}x^{2p} + \cdots + {c_n}x^{np} \mod{p}
			\]
			This expression is exactly $q(x)$. So $q'(x) \equiv 0$ should imply that $(q(x^{\frac{1}{p}}))^p \equiv q(x) \bmod{p}$.
		\end{observation}
	
	\section{Matrices, Matrix Lifts}
		\label{sec:matrices}
		\begin{observation}{$A\vec{v} = \lambda\vec{v} \implies p(A)\vec{v} = p(\lambda)\vec{v}$.}
		{$A\vec{v} = \lambda\vec{v} \implies p(A)\vec{v} = p(\lambda)\vec{v}$}
			Let $p(x) = c_0 + c_1x + c_2x^2 + \cdots + c_nx^n$. Computing $p(A)\vec{v}$, we see
			\begin{align*}
				  & \, p(A)\vec{v} \\
				= & \, (c_0I + c_1A + c_2A^2 + \cdots + c_nA^n)\vec{v} \\
				= & \, c_0\vec{v} + c_1A\vec{v} + c_2A^2\vec{v} + \cdots + c_nA^n\vec{v} \\
				= & \, c_0\vec{v} + c_1\lambda\vec{v} + c_2\lambda^2\vec{v} + \cdots + c_n\lambda^n\vec{v} \\
				= & \, (c_0 + c_1\lambda + c_2\lambda^2 + \cdots + c_n\lambda^n)\vec{v} \\
				= & \, p(\lambda)\vec{v}
			\end{align*}
		\end{observation}
		
		\begin{observation}{Given an invertible matrix $A \in \mathds{Z}^{L \times L}$ and some odd prime $p$, 
		\[
			A^\omega \equiv I + p^{k-1}B \bmod{p^k} \implies A^{p\omega} \equiv I + p^{k}B \bmod{p^{k+1}}
		\]
		for some integer $k \geq 2$ and some $B \in \mathds{Z}_{p}^{L \times L}$.}
		{$A^\omega \equiv I + p^{k-1}B \bmod{p^k} \implies A^{p\omega} \equiv I + p^{k}B \bmod{p^{k+1}}$}
			\label{obs:abcmatrixrelation}
			If $A^\omega \equiv I + p^{k-1}B \bmod{p^k}$, then we know
			\[
				A^\omega \equiv I + p^{k-1}B + p^{k}C \mod{p^{k+1}}
			\]
			for some $C \in \mathds{Z}_{p}^{L \times L}$. With a bit of algebraic manipulation, we see
			\[
				A^{2\omega} \equiv I + 2p^{k-1}B + 2p^{k}C + p^{2k-2}B^{2} \mod{p^{k+1}}
			\]
			for $k \geq 2$. We can now use induction to show what higher powers of $A$ will look like. Let $\Delta_{n} = \frac{n(n+1)}{2}$ be the $n$th triangular number. 
			We can then rewrite $A^{2\omega}$ as
			\[
				A^{2\omega} \equiv I + 2p^{k-1}B + 2p^{k}C + \Delta_{2-1}p^{2k-2}B^{2} \mod{p^{k+1}}
			\]
			Let's assume that, for some integer $r < p$, 
			\[
				A^{r\omega} \equiv I + rp^{k-1}B + rp^{k}C + \Delta_{r-1}p^{2k-2}B^{2} \mod{p^{k+1}}
			\]
			If we multiply both sides by $A^\omega$, we get
			\begin{align*}
				       & \, A^{\omega}A^{r\omega} \\
				\equiv & \, (I + p^{k-1}B + p^{k}C)(I + rp^{k-1}B + rp^{k}C + \Delta_{r-1}p^{2k-2}B^{2}) \\
				\equiv & \, I + (r+1)p^{k-1}B + (r+1)p^{k}C + (r + \Delta_{r-1})p^{2k-2}B^2 \\
				\equiv & \, I + (r+1)p^{k-1}B + (r+1)p^{k}C + (r + \frac{(r-1)r}{2})p^{2k-2}B^2 \\
				\equiv & \, I + (r+1)p^{k-1}B + (r+1)p^{k}C + (\frac{(r+1)r}{2})p^{2k-2}B^2 \\
				\equiv & \, I + (r+1)p^{k-1}B + (r+1)p^{k}C + \Delta_{r}p^{2k-2}B^2 \\
				\equiv & \, A^{(r+1)\omega} \mod{p^{k+1}}
			\end{align*}
			From this, we see
			\begin{align*}
				             A^{r\omega} & \, \equiv I + rp^{k-1}B + rp^{k}C + \Delta_{r-1}p^{2k-2}B^{2} \\
				\implies A^{(r+1)\omega} & \, \equiv I + (r+1)p^{k-1}B + (r+1)p^{k}C + \Delta_{r}p^{2k-2}B^{2} \mod{p^{k+1}}
			\end{align*}
			Via the principle of mathematical induction, we can say
			\[
				A^{n\omega} \equiv I + np^{k-1}B + np^{k}C + \Delta_{n-1}p^{2k-2}B^{2} \mod{p^{k+1}}
			\]
			for $k \geq 2$ and $p \neq 2$x. Then, simply plugging $n = p$ into this expression gives
			\begin{align*}
				A^{p\omega} & \, \equiv I + pp^{k-1}B + pp^{k}C + \Delta_{p-1}p^{2k-2}B^{2}   \\
				            & \, \equiv I + p^{k}B + p^{k+1}C + \frac{(p-1)p}{2}p^{2k-2}B^{2} \\
							& \, \equiv I + p^{k}B + \frac{(p-1)p}{2}p^{2k-2}B^{2} \mod{p^{k+1}}
			\end{align*}
			Since we know $p \neq 2$, we can simplify the third term in this sum:
			\begin{align*}
				\equiv & \, I + p^{k}B + \frac{p-1}{2}p^{2k-1}B^{2} \\
				\equiv & \, I + p^{k}B \mod{p^{k+1}}
			\end{align*}
			for $k \geq 2$. 
		\end{observation}
		
		\begin{observation}{Assume $A \in \mathds{Z}^{L \times L}$ is an invertible matrix where $A^\omega \equiv I \bmod{p}$, $p \neq 2$ (so $\omega$ is the cycle length 
		of $A$). If, for the smallest $k > 1$, $B \neq 0$, $A^\omega \equiv I + p^{k-1}B \bmod{p^k}$, and the vector $\vec{v}$ has maximal cycle length mod $p^k$, then for 
		all $h > 0$, $\vec{v}$ will also have maximal cycle length mod $p^h$.}
		{Maximal cycle vector mod $p^k$, $\implies$ maximal cycle vector mod $p^{k+h}$}
			From Proposition 3 in \citet{Mendivil2012}, we know a vector with maximal cycle length always exists under a prime modulus. Let's say the vector $\vec{v}$
			has maximal cycle length $\omega$ under iteration by $A$ mod $p$.
			
			Modulo $p^2$, there are only two possibilities for the cycle length of $A$: either it stays as $\omega$--which means $\vec{v}$ will necessarily still have maximal
			cycle length--or it increases to $p\omega$. If the cycle length stays the same, then the cycle length of $A$ mod $p^3$ will have the same two choices. If the
			cycle length stays the same mod $p^3$, then the cycle length mod $p^4$ will have the same choice, and so on. We'll assume that at some modulus, say $p^k$, the 
			cycle length of $A$ will increase to $p\omega$. Otherwise, $\vec{v}$ will always have maximal cycle length for all mod $p^k$ anyway.
			
			If the cycle length of $A$ increases to $p\omega$ mod $p^k$, then we can write that
			\[
				A^\omega \equiv I + p^{k-1}B \mod{p^k}
			\]
			since $A$ must reduce to $I$ for all moduli $p^h$ where $h < k$. From our assumptions, $B \neq 0$, so there exists at least one vector that doesn't iterate to 
			zero under $B$. If one of those vectors happens to be $\vec{v}$, then $\vec{v}$ also has maximal cycle length mod $p^k$ since its cycle length must both be a
			multiple of $\omega$ and must divide the matrix's cycle length $p\omega$. Since $p$ is prime, $n\omega \, | \, p\omega \implies n \in \{1, \, p\}$, and if
			$B\vec{v} \neq \vec{0}$, then $A^{\omega}\vec{v} \neq \vec{v}$, so the cycle length of $\vec{v}$ can't possibly be $\omega$. This leaves the only remaining
			possible cycle length for $\vec{v}$ as $p\omega$, which is the maximal cycle length mod $p^k$.
			
			Now, we can make use of Observation \ref{sec:matrices}.\ref{obs:abcmatrixrelation}, which says that
			\[
				A^{\omega} \equiv I + p^{k-1}B \bmod{p^k} \implies A^{p\omega} \equiv I + p^{k}B \bmod{p^{k+1}}
			\]
			We know $\vec{v}$ has a cycle length of $p\omega \bmod{p^k}$, so its cycle length must be a multiple of $p\omega \bmod{p^{k+1}}$. However, it must also divide
			the matrix's cycle length mod $p^{k+1}$, which is $p^{2}\omega$. By the same logic as before, this means $\vec{v}$ must also have maximal cycle length
			mod $p^{k+1}$.
			
			This same line of reasoning should extend to any mod $p^{k+n}$, meaning $\vec{v}$ will have maximal cycle length for all moduli which are powers of $p$.
		\end{observation}
		
		\begin{observation}{If $A \in \mathds{Z}_{p}^{L \times L}$ is an invertible matrix with irreducible monic minimal polynomial $\mu(x)$, then under iteration by $A$,
		there are only two possible cycle lengths for a vector in $\mathds{Z}_{p}^{L}$ (possibly the same), one of which being 1. As well, these cycle lengths must divide
		$p^L-1$.}{Cycles under matrices w/ irreducible minimal polynomials}
			\label{obs:onebigcyclelength}
			The Minimal Polynomial Theorem described in \citet{Patterson2008}, says that, given a vector's minimal annihilating polynomial, we can compute that vector's
			cycle length by computing the minimal annihilating polynomial's order. We also know that the minimal annihilating polynomial for any vector must be a factor of 
			the matrix's minimal polynomial due to the Primary Decomposition Theorem. 
			
			However, since our matrix's minimal polynomial is irreducible, it can't be factored into smaller factors. This means each vector in $\mathds{Z}_{p}^{L}$ must
			have the same minimal annihilating polynomial: the matrix's minimal polynomial. So, every vector in the space must have the same cycle length, dictated by the
			order of the minimal polynomial, except for the zero vector, which is already zero and thus will always have a cycle length of 1, separate from the rest (unless
			the order of the minimal polynomial happens to be 1, in which case all vectors will have a cycle length of 1).
			
			As well, since there are $p^L-1$ nonzero vectors in $\mathds{Z}_{p}^{L}$, the order of the minimal polynomial (and thus the cycle length for all nonzero vectors)
			must divide $p^L-1$. Otherwise, there's no possible way the matrix would be invertible (multiple vectors would have to be mapped to the same vector).
		\end{observation}
		
		\begin{observation}{Given an invertible matrix $A \in \mathds{Z}_{p}^{L \times L}$ with minimal polynomial in the form $(\lambda-c)^{2}q(\lambda)$, $A$ will have a 
		cycle length in the form $np$ for some $n \in \mathds{Z}$.}
		{Repeated root in minimal polynomial $\implies \omega = np$}
			If the cycle length of $A$ is $\omega$, then
			\begin{align*}
				         & A^\omega \equiv I \\
				\implies & A^\omega - I \equiv 0 \mod{p}
			\end{align*}
			This means the minimal polynomial of $A$ must divide $\lambda^\omega - 1$. So
			\begin{align*}
				         & (\lambda-c)^{2}q(\lambda) \;\, | \;\, \lambda^\omega - 1 \\
				\implies & (\lambda-c)^{2} \;\, | \;\, \lambda^\omega - 1 \\
				\implies & (\lambda-c) \;\, | \;\, \lambda^\omega - 1
			\end{align*}
			If we were to divide these polynomials, we see
			\[
				\frac{\lambda^\omega-1}{\lambda-c} = \lambda^{\omega-1} + c\lambda^{\omega-2} + c^{2}\lambda^{\omega-3} + \cdots + c^{\omega-2}\lambda + c^{\omega-1}
				\implies c^{\omega} \equiv 1 \mod{p}
			\]
			This isn't too surprising. However, we also know that $(\lambda-c)^{2} \;\, | \;\, \lambda^\omega - 1$, so $\lambda-c$ must divide the quotient above.
			Dividing them, we see
			\begin{align*}
				& \frac{\sum_{i\;=\;0}^{\omega-1} c^{i}\lambda^{\omega-1-i}}{\lambda-c} = 
				\lambda^{\omega-2} + 2c\lambda^{\omega-3} + 3c^{2}\lambda^{\omega-4} + \cdots + (\omega-2)c^{\omega-3}\lambda + (\omega-1)c^{\omega-2} \\
				\implies & (1-\omega)c^{\omega-1} \equiv c^{\omega-1} \\
				\implies & \omega \equiv 0 \mod{p}
			\end{align*}
			This means $\omega$ must be a multiple of $p$ when the minimal polynomial has a repeated root in the form $\lambda-c$. Theoretically, this should also extend
			to when $c$ is a field extension of $\mathds{Z}_p$, so non-linear repeated roots in the minimal polynomial should also cause the cycle length of $A$ to be a
			multiple of $p$.
		\end{observation}
		
		\begin{observation}{After a certain number of iterations, lifts of invertible matrices "merge" into the same matrix.}
		{Merging Matrix Lifts}
			Sadly, I don't have anything more than examples to back this one up, but based on the few examples I've tried, this seems to be a common behaviour.
			
			As an easy example, consider the matrix $A = 
			\begin{bmatrix}
				\begin{smallmatrix}
					1 & 1 \\
					1 & 0
				\end{smallmatrix}
			\end{bmatrix}$ 
			mod 5. If we "lift" this matrix to mod 25 (meaning we take some lift of it mod 25) and label our lift as $B$, then it seems as though
			\[
				B^5 \equiv A^5 \equiv
				\begin{bmatrix}
					8 & 5 \\
					5 & 3
				\end{bmatrix} \mod{25}
			\]
			no matter what lift we choose. In this way, after 5 iterations, the lifts of $A$ all seem to "merge" into the same matrix.
			
			This isn't specific to this matrix or modulus, either. As another random example, take $A = 
			\begin{bmatrix}
				\begin{smallmatrix}
					0 & 8 \\
					13 & 17
				\end{smallmatrix}
			\end{bmatrix}$
			mod 19. If we lift this matrix to mod 361 ($19^2$), then all lifts seem to marge after 380 iterations. For instance:
			\[
				\begin{bmatrix}
					0 & 8 \\
					13 & 17
				\end{bmatrix}^{380}
				\equiv
				\begin{bmatrix}
					57 & 141 \\
					108 & 359
				\end{bmatrix}^{380}
				\equiv
				\begin{bmatrix}
					333 & 0 \\
					0 & 333
				\end{bmatrix} \mod{361}
			\]
		\end{observation}
		
		\begin{observation}{All 2 by 2 matrices mod a prime power have a vector with maximal cycle length}
		{$\forall A \in \mathds{Z}_{p^k}^{2 \times 2}$, a vector with maximal cycle length exists}
			My reasoning is explained in \citet{Strong2022maximal2x2}.
		\end{observation}
		
	\section{Cycle Converting Matrices (CCMs)}
		\begin{observation}{$(C_{\alpha \rightarrow \beta})(C_{\beta \rightarrow \gamma}) \equiv C_{\alpha \rightarrow \gamma}$.}
		{$(C_{\alpha \rightarrow \beta})(C_{\beta \rightarrow \gamma}) \equiv C_{\alpha \rightarrow \gamma}$.}
			By definition of the CCM:
			\begin{align*}
				  & (C_{\alpha \rightarrow \beta})(C_{\beta \rightarrow \gamma}) \\
				= & \left(\sum_{i\,=\,0}^{\frac{\alpha-\beta}{\beta}} A^{\beta i}\right)\left(\sum_{j\,=\,0}^{\frac{\beta-\gamma}{\gamma}} A^{\gamma j}\right) \\
				= & \, I(I + A^{\gamma} + A^{2\gamma} + \cdots + A^{\beta - 2\gamma} + A^{\beta - \gamma}) \\
				+ & \, A^{\beta}(I + A^{\gamma} + A^{2\gamma} + \cdots + A^{\beta - 2\gamma} + A^{\beta - \gamma}) \\
				+ & \, A^{2\beta}(I + A^{\gamma} + A^{2\gamma} + \cdots + A^{\beta - 2\gamma} + A^{\beta - \gamma}) \\
				+ & \, \cdots \\
				+ & \, A^{\alpha - 2\beta}(I + A^{\gamma} + A^{2\gamma} + \cdots + A^{\beta - 2\gamma} + A^{\beta - \gamma}) \\
				+ & \, A^{\alpha - \beta}(I + A^{\gamma} + A^{2\gamma} + \cdots + A^{\beta - 2\gamma} + A^{\beta - \gamma})
			\end{align*}
			
			This creates a sequence of increasing powers of $A$, from $I$ to $A^{\alpha-\gamma}$, each different by a factor of $A^{\gamma}$. This can be rewritten as
			\begin{align*}
				= & \, \sum_{i\,=\,0}^{\frac{\alpha-\gamma}{\gamma}} A^{\gamma i} \\
				= & \, C_{\alpha \rightarrow \gamma}
			\end{align*}
		\end{observation}
		
		\begin{observation}{For an invertible matrix $A \in \mathds{Z}_{p^k}^{L \times L}$ with cycle length $\omega \bmod{p^k}$, if the cycle length of $A$ increases to
		$p\omega \bmod{p^{k+1}}$, then $C_{p\omega \rightarrow \omega} \equiv pI \bmod{p^{k+1}}$ for $p \neq 2$.}
		{$C_{p\omega \rightarrow \omega} \equiv pI \bmod{p^{k+1}}, p \neq 2$}
			Computing $C_{p\omega \rightarrow \omega} \bmod{p^{k+1}}$, we see
			\begin{align*}
				       & \, C_{p\omega \rightarrow \omega} \\
				\equiv & \, \sum_{i\,=\,0}^{p\omega-\omega} A^{\omega i} \\
				\equiv & \, I + A^\omega + A^{2\omega} + \cdots + A^{p\omega-2\omega} + A^{p\omega-\omega} \\
				\equiv & \, I + (I + p^kB) + (I + 2p^kB) + \cdots + (I + (p-2)p^kB) + (I + (p-1)p^kB), \, B \in \mathds{Z}_{p}^{L \times L} \\
				\equiv & \, pI + \frac{p(p-1)}{2}p^kB
			\end{align*}
			If $p \neq 2$,
			\[
				\equiv pI \mod{p^{k+1}}
			\]
		\end{observation}
		
		\begin{observation}{Given an invertible matrix with cycle length $\omega$, the CCM $C_{\alpha \rightarrow \beta}$ with $\alpha \, | \, \omega$, $\beta \, | \, \omega
		$, $\alpha, \beta < \omega$ will never be the zero matrix.}
		{$\alpha < \omega \implies C_{\alpha \rightarrow \beta} \not\equiv 0$}
			If an invertible matrix $A$ has cycle length $\omega$, this means 
			\[
				A^{\omega} - I \equiv 0
			\]
			and
			\[
				A^{c} - I \not\equiv 0 \quad \forall c < \omega
			\]
			
			If $c$ happens to divide $\omega$, then the above expression can be factored as
			\begin{align*}
				       & \, A^{c} - I \\
				\equiv & \, (I + A^{d} + A^{2d} + \cdots + A^{c-d})(A^{d} - I), \quad d \, | \, c \\
				\equiv & \, C_{c \rightarrow d}(A^{d} - I)
			\end{align*}
			We know this expression can't equal 0, so none of its factors can equal 0, either. This means $C_{c \rightarrow d} \not\equiv 0 \; \forall c < \omega$.
			What's more, $c$ and $d$ can be any factors of $\omega$ we want, provided they're less than $\omega$, so we can guarantee that all CCMs without an $\omega$
			as one of their subscript numbers can't equal the zero matrix.
		\end{observation}
		
		\bibliographystyle{plainnat}
		\bibliography{refs.bib}
\end{document}
