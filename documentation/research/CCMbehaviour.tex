
%Reference articles for styling:
% tex.stackexchange.com/questions/628300
% tex.stackexchange.com/questions/79134
% tex.stackexchange.com/questions/103508

% http://tug.ctan.org/tex-archive/macros/latex/contrib/nicematrix/nicematrix.pdf

\documentclass[a4paper, 12pt, reqno]{amsart}

\usepackage[T1]{fontenc}
\usepackage[margin=3cm]{geometry}
\usepackage[parfill]{parskip}

\usepackage{setspace}
\setstretch{1.25}

\usepackage{dsfont}
\usepackage{amsmath}
\usepackage{nicematrix} %Used for adding dividers in complicated matrices

\usepackage{natbib}

\begin{document}
	\section{Intro}
	For a given linear cellular automata (LCA) with a module of the form $\mathds{Z}_{n}^L$ and a matrix $A$ from $\mathds{Z}_{n}^{L \times L}$ (the set of all $L$ by $L$ 
	matrices with entries in $\mathds{Z}_n$), it's useful to know the cycle lengths of all the vectors under iteration by $A$. The \emph{cycle length} of a vector $\vec{v}$ 
	is defined to be the smallest positive integer $\omega$ such that
	\[
		A^{\tau + \omega}\vec{v} \, \equiv A^{\tau + 2\omega}\vec{v}
	\]
	for the smallest possible nonnegative integer $\tau$, which is called the \emph{transient length} of the vector. We'll also refer to the cycle length of the matrix 
	itself, which is defined as the smallest positive integer $\omega$ such that
	\[
		A^{\tau + \omega} \equiv A^{\tau + 2\omega}
	\]
	for the smallest possible nonnegative integer $\tau$, which is the transient length of the matrix.
	
	Note that transient lengths are nonzero only if $A$ is not invertible. For the majority of this file, we'll only be considering invertible matrices, so we won't need to
	worry about transient lengths (i.e. all transient lengths will be zero). When $\tau$ is zero, the above relations become much simpler:
	\begin{align*}
		         A^{\tau + \omega}\vec{v} \, & \equiv A^{\tau + 2\omega}\vec{v} \\
		\implies A^{\omega}\vec{v}        \, & \equiv A^{2\omega}\vec{v}        \\
		\implies \vec{v}                  \, & \equiv A^{\omega}\vec{v}
	\end{align*}
	and
	\begin{align*}
		         A^{\tau + \omega} \, & \equiv A^{\tau + 2\omega} \\
		\implies A^{\omega}        \, & \equiv A^{2\omega}        \\
		\implies I                 \, & \equiv A^{\omega}
	\end{align*}
	where $I$ is the $L \times L$ identity matrix.
	
	This file will also refer to multiplying vectors by the matrix $A$ as \emph{iterating} the vector. For instance, iterating a vector twice refers to multiplying a vector
	by $A^2$. A vector \emph{iterating to itself} refers to when multiplying a vector by some power of the matrix $A$ produces the same vector, i.e.
	\[
		A^c\vec{v} \, \equiv \vec{v}
	\]
	where $c$ is some integer.
	
	It's often useful to find vectors of particular cycle lengths for analysis or comparison purposes. One way to do this would simply be to iterate a bunch of vectors 
	until one with the cycle length of interest is found. However, depending on the size of the system and on the computational resources available, this could take a very 
	long time. Is there a more efficient way to do this?
	
	A few different methods exist that make working with the cycle lengths of vectors easier (particularly in \citet{Patterson2008}), but this document aims to detail a
	different approach. Consider the module $\mathds{Z}_{p}^L$ and some matrix $A$ in $\mathds{Z}_{p}^{L \times L}$. If we assume $A$ is invertible, that the cycle length 
	of $A$ is 30, and that $\vec{v}$ is some vector in $\mathds{Z}_{p}^L$, then the vector
	\[
		(I + A^{6} + A^{12} + A^{18} + A^{24})\vec{v}
	\]
	has a cycle length that divides 6. Why? If we were to multiply this vector by $A^6$, we'd get
	\begin{align*}
		       & \ A^{6}(I + A^{6} + A^{12} + A^{18} + A^{24})\vec{v} \\
		\equiv & \ (A^{6} + A^{12} + A^{18} + A^{24} + A^{30})\vec{v} \\
		\equiv & \ (A^{6} + A^{12} + A^{18} + A^{24} + I)\vec{v}      \\
		\equiv & \ (I + A^{6} + A^{12} + A^{18} + A^{24})\vec{v}
	\end{align*}
	
	Multiplying this vector by $A^6$ gets us back to the original vector, meaning its cycle length must divide 6, since a vector with a cycle length that doesn't divide 6 
	wouldn't iterate to itself after 6 iterations.  We can't immediately conclude its cycle length is 6 since this same relation would hold if the vector also had a 
	cycle length of 1, 2, or 3. However, the fact that we can take any vector from the module, multiply it by this relatively simple matrix expression, and get a vector 
	whose cycle length is narrowed down to a few specific values, is certainly noteworthy. The rest of this file will try and make sense of these matrix expressions, and 
	how we might be able to use them to deduce properties of the vectors in the module.
	
	\section{Cycle Converting Matrices}
	Given a module $\mathds{Z}_{n}^{L}$ and an invertible matrix $A \in \mathds{Z}_{n}^{L \times L}$ with cycle length $\omega$, a \emph{cycle converting matrix} (CCM) 
	\emph{from $\alpha$ to $\beta$} is a matrix $C$ of the form
	\[
		C_{\alpha \rightarrow \beta} = \sum_{i \, = \, 0}^{\alpha - \beta} A^{\beta i}
	\]
	where $\alpha$ and $\beta$ are factors of $\omega$ (or $\omega$ itself), and $\alpha \geq \beta$. This file will sometimes refer to $\beta$ as the \emph{target cycle 
	length} of the CCM. The reason for the restrictions on $\alpha$ and $\beta$ is straightforward. If $\alpha$ or $\beta$ weren't factors of $\omega$, then vectors with 
	those cycle lengths couldn't exist under iteration by $A$. In other words, all vectors in $\mathds{Z}_{n}^L$ must have cycle lengths that divide $\omega$. 
	
	Since $A$ is invertible, we know $A^{\omega} \equiv I$. If the cycle length $p$ of some vector $\vec{v}$ wasn't a factor of $\omega$, there would be some $\epsilon < p$ 
	where $\omega = np + \epsilon$, meaning
	\begin{alignat*}{4}
		         & A^{\omega}\vec{v}        \  & \equiv & \ A^{np + \epsilon}\vec{v}  \\
		\implies & A^{\omega}\vec{v}        \  & \equiv & \ A^{np}A^{\epsilon}\vec{v} \\
		\implies & A^{-np}A^{\omega}\vec{v} \  & \equiv & \ A^{\epsilon}\vec{v}       \\
		\implies & A^{-np}I\vec{v}          \  & \equiv & \ A^{\epsilon}\vec{v}       \\
		\implies & A^{-np}\vec{v}           \  & \equiv & \ A^{\epsilon}\vec{v}       \\
		\implies & A^{-np}A^{2np}\vec{v}    \  & \equiv & \ A^{\epsilon}\vec{v}       \\
		\implies & A^{np}\vec{v}            \  & \equiv & \ A^{\epsilon}\vec{v}       \\
		\implies & \vec{v}                  \  & \equiv & \ A^{\epsilon}\vec{v}
	\end{alignat*}
	so $p$ can't be the cycle length of $\vec{v}$ since there's a smaller positive integer which satisfies the same property. This process can be repeated until the cycle
	length of $\vec{v}$ is a factor of the matrix's cycle length. So $\alpha$ and $\beta$ must be factors of $\omega$ simply because those are the only cycle lengths which
	are possible.
	
	The second restriction, $\alpha \geq \beta$, is a result of the fact that adding iterations of the same vector can never result in a vector with a higher cycle length,
	so CCMs can't ever increase the cycle length of a given vector. To see why, imagine a general vector $\vec{w}$ which is a linear combination of iterations of some
	other vector $\vec{v}$:
	\[
		\vec{w} \equiv x_{0}\vec{v} + x_{1}A\vec{v} + x_{2}A^{2}\vec{v} + \cdots
	\]
	If the cycle length of $\vec{v}$ is $p$, then
	\begin{align*}
		A^{p}\vec{w} & \equiv x_{0}A^{p}\vec{v} + x_{1}AA^{p}\vec{v} + x_{2}A^{2}A^{p}\vec{v} + \cdots \\
		             & \equiv x_{0}\vec{v} + x_{1}A\vec{v} + x_{2}A^{2}\vec{v} + \cdots                \\
					 & \equiv \vec{w}
	\end{align*}
	so the cycle length of $\vec{w}$ must divide $p$, therefore it's unable to go above $p$.
	
	Now understanding the restrictions, how do CCMs behave? Consider the example from earlier, where our matrix $A$ is invertible and has a cycle length of 30. Let's say we
	have a vector $\vec{v}$ that has a cycle length of 10. Using this vector, we can find a vector whose cycle length divides 5 by using the CCM:
	\[
		C_{10 \rightarrow 5} = I + A^{5}
	\]
	The vector $C_{10 \rightarrow 5}\vec{v}$ must now have a cycle length that divides 5 since
	\begin{align*}
	           & \, A^{5}C_{10 \rightarrow 5}\vec{v} \\
		\equiv & \, A^{5}(I + A^{5})\vec{v}          \\
		\equiv & \, A^{5}\vec{v} + A^{10}\vec{v}     \\
		\equiv & \, A^{5}\vec{v} + \vec{v}           \\
		\equiv & \, (A^{5} + I)\vec{v}               \\
		\equiv & \, C_{10 \rightarrow 5}\vec{v}
	\end{align*}
	
	In general, whenever we're given a vector with a known cycle length, we can use a CCM to "convert" the vector to one whose cycle length divides a smaller number. If 
	we're given a vector whose cycle length we don't know, we at least know its cycle length divides the cycle length of the matrix, so a CCM of the form 
	$C_{\omega \rightarrow \alpha}$ can be used, where $\omega$ is the cycle length of the matrix $A$.
	
	Unfortunately, in practice, using CCMs isn't as straightforward as we'd like. For starters, the zero vector exists. The cycle length of $\vec{0}$ is always 1, meaning 
	its cycle length divides any other cycle length. What this means is that, when multiplying a vector by a CCM, it's possible that we'll end up converting it to the zero
	vector, which isn't very useful. Even worse, what's stopping the CCM itself from being the zero matrix? 
	
	As a simple example, consider the matrix $A = 
	\begin{bmatrix}
		\begin{smallmatrix}
			1 & 1 \\
			1 & 0
		\end{smallmatrix}
	\end{bmatrix}$ modulo 5. This matrix has a cycle length of 20. Under iteration by this matrix, 4 vectors have a cycle length of 4 (the eigenvectors of this matrix).
	Forming $C_{20 \rightarrow 4}$, we see
	\[
		C_{20 \rightarrow 4} \equiv 
		\begin{bmatrix}
			1 & 0 \\
			0 & 1
		\end{bmatrix} + 
		\begin{bmatrix}
			0 & 3 \\
			3 & 2
		\end{bmatrix} + 
		\begin{bmatrix}
			4 & 1 \\
			1 & 3
		\end{bmatrix} + 
		\begin{bmatrix}
			3 & 4 \\
			4 & 4
		\end{bmatrix} + 
		\begin{bmatrix}
			2 & 2 \\
			2 & 0
		\end{bmatrix} \equiv 0 \mod{5}
	\]
	meaning we can't use the CCM to find any of these vectors; every vector we try will be sent to $\vec{0}$. While this example is extremely specific, it shows that CCMs 
	do have the potential to become the zero matrix, even when vectors with the given cycle length exist. This raises an interesting question: when do CCMs become zero?
	What do these zeros mean? 
	
	As another example, consider the matrix $A = 
	\begin{bmatrix}
		\begin{smallmatrix}
			4 & 5 \\
			1 & 1
		\end{smallmatrix}
	\end{bmatrix}$ modulo 7. This matrix has a cycle length of 6. 6 vectors have a cycle length of 3. Computing $C_{6 \rightarrow 3}$, we get
	\[
		C_{6 \rightarrow 3} \equiv 
		\begin{bmatrix}
			1 & 0 \\
			0 & 1
		\end{bmatrix} +
		\begin{bmatrix}
			4 & 4 \\
			5 & 3
		\end{bmatrix} = 
		\begin{bmatrix}
			5 & 4 \\
			5 & 4
		\end{bmatrix} \mod{7}
	\]
	From the CCM, we see that all vectors with cycle lengths that divide 3 must lie along the span of the vector
	$\begin{bmatrix}
		\begin{smallmatrix}
			1 \\
			1
		\end{smallmatrix}
	\end{bmatrix}$. The span of this vector has 7 vectors in it. Not including the zero vector, that gives us all 6 vectors with a cycle length of 3. Unlike in the
	previous example, this CCM gives us a lot of info. What's different between these two cases? What causes the CCM to give "unjustified" results in the first example,
	but not in the second?
	
	\section{CCMs, Eigenvectors, and Prime Moduli}
	The general case of CCMs is difficult to deal with, so let's restrict our attention to prime moduli for the time being. Prime moduli have a number of useful properties
	which allow us to make sense of why CCMs behave the way they do.
	
	Let's consider the case where $A$ is invertible, the modulus is prime, and where the module in question has an eigenbasis (a basis of eigenvectors under iteration by
	$A$). If we let $\vec{u}_1$, $\vec{u}_2$, $\cdots$ represent a set of linearly-independent eigenvectors, then we can write our matrix $A$ as
	\[
	\arraycolsep=0.33cm
		A = \begin{bNiceArray}{*{2}{c:}c}
			a_{1,1}\vec{u}_1 + a_{1,2}\vec{u}_2 + \cdots & a_{2,1}\vec{u}_1 + a_{2,2}\vec{u}_2 + \cdots & \cdots
		\end{bNiceArray}
	\]
	where each column vector of $A$ is represented as a linear combination of eigenvectors. For the next step, it's useful to write each eigenvector as an eigenvector 
	multiplied by an eigenvalue. Because $\vec{u}_1$, $\vec{u}_2$, $\cdots$ are an arbitrary collection of eigenvectors, we can easily rewrite the matrix $A$ as
	\[
	\arraycolsep=0.33cm
		A = \begin{bNiceArray}{*{2}{c:}c}
			\lambda_{1}a_{1,1}\vec{u}_1 + \lambda_{2}a_{1,2}\vec{u}_2 + \cdots & \lambda_{1}a_{2,1}\vec{u}_1 + \lambda_{2}a_{2,2}\vec{u}_2 + \cdots & \cdots
		\end{bNiceArray}
	\]
	
	Now, if we let $\alpha$ be some factor of the matrix's cycle length $\omega$, then we can write $C_{\omega \rightarrow \alpha}$ as
	\begin{align*}
		C_{\omega \rightarrow \alpha} & \equiv I + A^{\alpha} + A^{2\alpha} + \cdots + A^{\omega - \alpha}                      \\
		                              & \equiv A^{-1}A + A^{\alpha - 1}A + A^{2\alpha - 1}A + \cdots + A^{\omega - \alpha - 1}A
	\end{align*}
	Because the column vectors of $A$ are composed of eigenvectors, these multiplications can be carried out fairly easily:
	\begin{align*}
		A^{-1}A & \equiv A^{-1}\begin{bNiceArray}{*{2}{c:}c}
			\lambda_{1}a_{1,1}\vec{u}_1 + \lambda_{2}a_{1,2}\vec{u}_2 + \cdots & \lambda_{1}a_{2,1}\vec{u}_1 + \lambda_{2}a_{2,2}\vec{u}_2 + \cdots & \cdots
		\end{bNiceArray} \\
		& \equiv \begin{bNiceArray}{*{2}{c:}c}
			a_{1,1}\vec{u}_1 + a_{1,2}\vec{u}_2 + \cdots & a_{2,1}\vec{u}_1 + a_{2,2}\vec{u}_2 + \cdots & \cdots
		\end{bNiceArray}
	\end{align*}
	\begin{align*}
		A^{\alpha - 1}A & \equiv A^{\alpha - 1}\begin{bNiceArray}{*{2}{c:}c}
			\lambda_{1}a_{1,1}\vec{u}_1 + \lambda_{2}a_{1,2}\vec{u}_2 + \cdots & \lambda_{1}a_{2,1}\vec{u}_1 + \lambda_{2}a_{2,2}\vec{u}_2 + \cdots & \cdots
		\end{bNiceArray} \\
		& \equiv \begin{bNiceArray}{*{2}{c:}c}
			\lambda_{1}^{\alpha}a_{1,1}\vec{u}_1 + \lambda_{2}^{\alpha}a_{1,2}\vec{u}_2 + \cdots & 
			\lambda_{1}^{\alpha}a_{2,1}\vec{u}_1 + \lambda_{2}^{\alpha}a_{2,2}\vec{u}_2 + \cdots & 
			\cdots
		\end{bNiceArray}
	\end{align*}
	\begin{center}
		$\vdots$
	\end{center}
	
	Each term in this sequence will be the same collection of terms for each column vector, just with the powers of the eigenvalues changed. Adding all these terms
	together will give us $C_{\omega \rightarrow \alpha}$. To make this easier, let's only focus on the first column vector. The other column vectors of 
	$C_{\omega \rightarrow \alpha}$ will be created similarly. The first column vector will look something like:
	\begin{align*}
		       & \, a_{1,1}\vec{u}_1 + a_{1,2}\vec{u}_2 + \cdots + \lambda_{1}^{\alpha}a_{1,1}\vec{u}_1 + \lambda_{2}^{\alpha}a_{1,2}\vec{u}_2 + \cdots \\
		\equiv & \, a_{1,1}(1 + \lambda_{1}^{\alpha} + \lambda_{1}^{2\alpha} + \cdots)\vec{u}_{1} + 
				    a_{1,2}(1 + \lambda_{2}^{\alpha} + \lambda_{2}^{2\alpha} + \cdots)\vec{u}_{2} + \cdots 
	\end{align*}
	Our column vector for $C_{\omega \rightarrow \alpha}$ becomes a comparatively simple sum of eigenvectors. Again, let's further hone our attention to the eigenvalue sums
	attached to each eigenvector, say the one for $\vec{u}_{1}$:
	\[
		1 + \lambda_{1}^{\alpha} + \lambda_{1}^{2\alpha} + \cdots + \lambda_{1}^{\omega - \alpha}
	\]
	Algebraically, this expression tells us a lot about how each eigenvector will behave when creating a CCM. Firstly, if our eigenvector's cycle length doesn't divide the 
	target cycle length of the CCM ($\alpha$ in this case), this expression will evaluate to zero. For instance, let's say our eigenvector $\vec{u}_{1}$ has a cycle 
	length of 6. If our matrix $A$ had a cycle length of 24, then calculating $C_{24 \rightarrow 4}$ would give expressions similar to
	\[
		(1 + \lambda_{1}^{4} + \lambda_{1}^{8} + \cdots + \lambda_{1}^{20})\vec{u}_{1}
	\]
	If we were to iterate this vector 4 times, it should cycle back to itself. However, we know the cycle length of $\vec{u}_{1}$ is 6, and 6 doesn't divide 4, so
	this vector couldn't possibly have a cycle length of 4. Furthermore, this vector is a multiple of $\vec{u}_{1}$, so whatever vector we end up with, it'll still be on the
	span of eigenvectors with eigenvalue $\lambda_{1}$. However, under a prime moduli, all vectors along the span of an eigenvector (except for $\vec{0}$) must have the
	same cycle length. If they didn't, it would contradict the fact that multiples of a given vector can never have a greater cycle length than the original vector (due to
	linearity). Therefore, because our eigenvector has a cycle length that doesn't divide our target cycle length, the only vector it can convert to that's consistent with
	the algebraic expression we get is the zero vector.
	
	This holds generally: if we try to convert an eigenvector to a cycle length other than one that its cycle length divides, the expression will evaluate to zero. However,
	this isn't the only time these expressions can become zero. Even if the target cycle length is a multiple of an eigenvector's cycle length, the expression can still 
	evaluate to zero if the cycle length of the matrix $A$ has a multiple of the prime modulus in it.
	
	As an example, we'll again consider the matrix $A = 
	\begin{bmatrix}
		\begin{smallmatrix}
			1 & 1 \\
			1 & 0
		\end{smallmatrix}
	\end{bmatrix}$ modulo 5. This matrix has a cycle length of 20, which is $5 \times 4$. Conveniently, this matrix has a set of eigenvectors along the span of
	$\begin{bmatrix}
		\begin{smallmatrix}
			1 \\
			2
		\end{smallmatrix}
	\end{bmatrix}$ with cycle lengths of 4 and an eigenvalue of 3. If we were to compute $C_{20 \rightarrow 4}$, we'd get expressions similar to
	\begin{align*}
		       & \, (1 + 3^{4} + 3^{8} + 3^{12} + 3^{16})\vec{u}_{1} \\
		\equiv & \, (1 + 1 + 1 + 1 + 1)\vec{u}_{1}                   \\
		\equiv & \, (5)\vec{u}_{1}                                   \\
		\equiv & \, (0)\vec{u}_{1}                                   \\
		\equiv & \, \vec{0} \mod{5}
	\end{align*}
	
	When the target cycle length matches the eigenvector's cycle length, the $\lambda$ term turns into a sum of ones. If the cycle length of the matrix $A$ is a multiple
	of the prime modulus used, and the target cycle length isn't, then the sum will always contain a multiple of the prime number of ones, meaning it'll evaluate to
	zero. For LCAs with an eigenbasis, this phenomenon explains entirely why the CCMs sometimes evaluate to zero, even when vectors with the desired cycle lengths exist.
	For this particular example, there's a little more going on, since there isn't an eigenbasis under the matrix 
	$\begin{bmatrix}
		\begin{smallmatrix}
			1 & 1 \\
			1 & 0
		\end{smallmatrix}
	\end{bmatrix}$ modulo 5. However, this particular matrix serves as a useful example for many of the quirks of CCMs, so it was an easy example to use.
	
	Putting this all together, we see that, for prime moduli and LCAs with an eigenbasis, the behaviour of CCMs boils down to how the cycle lengths of the eigenvectors
	relate to the target cycle length, and whether the matrix has a multiple of the prime modulus in its cycle length. If an eigenvector's cycle length doesn't divide the
	target cycle length, it'll disappear from the CCM's column vectors. If it does divide the target cycle length, it'll appear only if either the matrix doesn't have
	a factor of the prime in its cycle length, or both the matrix's cycle length has a factor of the prime, and the target cycle length has the same number of factors of
	the prime (as in this case, the $\lambda$ term won't have a multiple of the prime number of ones).
	
	This gives a rough idea as to why CCMs are able to "convert" vectors to particular cycle lengths. By creating a matrix with only particular eigenvectors in its column
	vectors--specifically those eigenvectors with cycle lengths that divide the target cycle length--, one can multiply a vector by that matrix and guarantee that the 
	resulting vector is a linear combination of eigenvectors, all with cycle lengths that divide the target cycle length. Therefore, that linear combination must also have
	a cycle length that divides the target cycle length due to linearity.
	
	This doesn't leave us with a full description of CCM behaviour, but it's a starting point. The next section will dive a little deeper into when CCMs evaluate to zero.
	
	\section{CCMs and Minimal Polynomials}
	If a matrix $A$ has a cycle length which contains a factor of the prime modulus, it's possible for the matrix's CCMs to evaluate to zero (or, at least, some of the CCM's
	column vectors) even when vectors of the target cycle length exist. It turns out that there's a semi-surprising link between a matrix's minimal polynomial and its cycle
	length, particularly when a factor of the prime is present in the cycle length.
	
	For reference, the minimal polynomial of a matrix $A$ is the monic polynomial $\mu(x)$ of least degree such that
	\[
		\mu(A) \equiv 0
	\]
	By determining the "order" of the factors of the minimal polynomial (see \citet{Patterson2008}), the cycle lengths of all the vectors in the module can be determined.
	It can also be used to determine the cycle length of the matrix itself by determining the maximal cycle length (see \citet{Mendivil2012}). 
	
	In essence, the process of determining the order of the factors of the minimal polynomial involves determining the smallest positive integer $c$ such that, for your
	factor $f(\lambda)$:
	\[
		f(\lambda) \ | \ \lambda^{c} - 1
	\]
	
	Using this idea, we can make a relatively simple conclusion about the cycle length of our matrix $A$ when there's a repeated factor in the form $\lambda - x$ in the
	minimal polynomial. Let's assume $\lambda - x$ divides $\lambda^{c} - 1$. Performing polynomial division, we see
	\[
		\frac{\lambda^{c} - 1}{\lambda - x} = \lambda^{c-1} + x\lambda^{c-2} + x^{2}\lambda^{c-3} + \cdots + x^{c-2}\lambda + x^{c-1} \iff x^{c} \equiv 1
	\]
	In order for $(\lambda - x)^{2}$ to divide $\lambda^{c} - 1$, $\lambda - x$ must divide the quotient above. Assuming that it does divide, we see
	\[
		\frac{\sum_{i \, = 0}^{c-1} x^{i}\lambda^{c-1-i}}{\lambda - x} = \lambda^{c-2} + 2x\lambda^{c-3} + 3x^{2}\lambda^{c-4} + \cdots + 
		                                                                 (c-2)x^{c-3}\lambda + (c-1)x^{c-2}
	\]
	which is true only when
	\[
		(1-c)x^{c-1} \equiv x^{c-1}
	\]
	Simplifying the congruence we get, we see
	\[
		c \equiv np \quad \text{for } n \in \mathds{W}
	\]
	where $p$ is our prime modulus. This means that, when we have a repeated linear factor in our minimal polynomial, the order of our minimal polynomial,
	which ends up being the cycle length of our matrix $A$, must be a multiple of the prime modulus used. 
	
	So, when our minimal polynomial has repeated linear terms, the CCMs of the matrix $A$ are likely to be "unjustified" in the sense that they evaluate to zero even when
	vectors of the target cycle length exist. Unfortunately, proving this relation in the other direction is a lot harder (i.e. if the CCMs of the matrix are unjustified,
	then the minimal polynomial must have a repeated linear root). If this other direction could be proved, it would serve as the basis for understanding when CCMs can and
	can't be used to gain insights on the cycle lengths of vectors under some update matrix $A$.
	
	\section{CCMs and Prime-powered Moduli}
	Up to now, we've focused our attention on when the modulus used is prime. How do CCMs behave when the modulus is a prime-power? Unlike in the prime case, we can't
	make use of minimal polynomials (as they don't really work in composite moduli), nor can we rely on the nice properties of the integers mod $p$ (which forms a field
	rather than a ring). However, there are a few observations which can be made.
	
	First, it's helpful to define lift matrices. Given a matrix $A \in \mathds{Z}_{p^{k-h}}^{L \times L}$, a \emph{lift} of the matrix $A$ in mod $p^k$ is any matrix $M$ in 
	the form
	\[
		M = A + p^{k-1}B_{1} + p^{k-2}B_{2} + \cdots + p^{k-h}B_{k-h} \text{ for matrices } B_{i} \in \mathds{Z}_{p}^{L \times L}
	\]
	On their own, these matrices have a lot of nice properties which make analysing prime-powered LCAs easier. One particularly useful property is that every matrix in
	$\mathds{Z}_{p^k}^{L \times L}$ is a lift of some matrix in $\mathds{Z}_{p}^{L \times L}$. Another useful property (which arises entirely out of the arithmetic 
	relationship between $\mathds{Z}_{p^{k-1}}$ and $\mathds{Z}_{p^{k}}$) is that, for invertible matrices $A \in \mathds{Z}_{p^{k-1}}^{L \times L}$ and their corresponding 
	lifts $M_{i} \in \mathds{Z}_{p^k}^{L \times L}$,
	\begin{equation}
		\label{imp:liftToTheOmega}
		A^{\omega} \equiv I \text{ mod } p^{k-1} \implies M_{i}^{\omega} \equiv I + p^{k-1}B \text{ mod } p^{k} \text{ for some matrix } B \in \mathds{Z}_{p}^{L \times L}
	\end{equation}
	This property can be used to deduce the form of particular CCMs under prime-powered moduli.
	
	Say we're given a matrix $A \in \mathds{Z}_{p^{k-1}}^{L \times L}$ that has cycle length $\omega$, and a lift of this matrix $M \in \mathds{Z}_{p^{k}}^{L \times L}$
	with cycle length $p\omega$. It may seem overly restricting to assign the cycle lengths in this manner, but this is actually the most common relationship between a
	matrix in $\mathds{Z}_{p^{k-1}}^{L \times L}$ and its lift in $\mathds{Z}_{p^{k}}^{L \times L}$. In fact, it's one of only two possibilities. The other possible
	cycle length for $M$ is $\omega$ itself. These being the only two possibilities are a direct result from implication \ref{imp:liftToTheOmega}. In any case, given these
	two matrices, calculating $C_{p\omega \rightarrow \omega}$ for $M$ results in the following:
	\begin{align*}
		C_{p\omega \rightarrow \omega} & \equiv \sum_{i \, = \, 0}^{p\omega - \omega} M^i \\
		                               & \equiv \sum_{i \, = \, 0}^{p\omega - \omega} (I + ip^{k-1}B), \ B \in \mathds{Z}_{p}^{L \times L} \\
									   & \equiv pI + p^{k-1}(\frac{p(p-1)}{2})B \\
	\end{align*}
	Assuming $p$ is an odd prime:
	\begin{align*}
		& \equiv pI + p^{k}(\frac{p-1}{2})B \\
		& \equiv pI \mod{p^{k}}
	\end{align*}
	
	So, if $p \neq 2$, then for any invertible matrix $M \in \mathds{Z}_{p^{k}}^{L \times L}$ which has cycle length $p\omega$, and which is the lift of a matrix in 
	$\mathds{Z}_{p^{k-1}}^{L \times L}$ with cycle length $\omega$, the CCM $C_{p\omega \rightarrow \omega} = pI$. This result is very similar to Lemma 1 in 
	\citet{Mendivil2012}, though Mendivil's result arises through the lens of analysing vectors, while the result here arises through analysing matrices (specifically CCMs).
	
	This result exposes a sort of "blind spot" for CCMs under prime-powered moduli. The column space of the matrix $pI$ represents the set of all vectors which are 
	"embedded" into higher prime-power moduli from lower prime-power moduli; the behaviour of these vectors mimics the behaviour of LCAs under lower-powered moduli. The
	fact that the CCM will always give this matrix in these scenarios suggests that CCMs are unable to see certain types of vectors under prime-powered moduli.
	
	\textbf{Note: I'll most likely add more to this document once I've had more time to look into CCM behaviour when an eigenbasis doesn't exist.}
	
	\bibliographystyle{plainnat}
	\bibliography{refs.bib}
	
\end{document}
