
\documentclass[a4paper, 12pt, reqno]{amsart}

\usepackage[T1]{fontenc}
\usepackage[margin=3cm]{geometry}
\usepackage[parfill]{parskip} %Spaces for paragraphs

%Line spacing
\usepackage{setspace}
\setstretch{1.25}

\usepackage{dsfont} %Double stroke font

\usepackage{amsmath}

\usepackage{natbib}

%For adding spacing before section titles
\newcommand{\ssection}[1]{\vspace{1cm}\section{#1}}
\newcommand{\ssubsection}[1]{\vspace{0.25cm}\subsection{#1}}

\author{Zach Strong}
\title{Exploring matrix cores within prime-powered modules}
\date{\today}

\begin{document}
	\maketitle
	
	\ssection{Introduction}
	Given a module and a matrix $A$, there are two possibilities for each vector $\vec{v}$ in the module: 
	either $A$ maps at least one vector to $\vec{v}$, or it doesn't. Vectors that don't get mapped to are 
	said to be part of the \emph{transient region} of the given system. Any vector which gets mapped to 
	only from vectors within the transient region are also said to be part of the transient region. Any 
	vectors which aren't part of the transient region are said to be in the \emph{core} of the given system.
	
	More concretely, given some module $\mathds{Z}_{N}^{L}$ and an $L$ by $L$ matrix $A$ with elements 
	in $\mathds{Z}_{N}$, the \emph{core} of $A$ (denoted by $S$) is the largest subset of the module such that
	\[
		AS = S
	\]
	One can think of the core as the largest set of vectors that gets shuffled around when multiplied by $A$.
	For the following discussion, we'll let core$(A)$ represent this set $S$ for the matrix $A$. The modulus 
	used will be apparent from context.
	
	The number of vectors in a matrix's core--the core's cardinality--changes depending on the modulus used. As a
	simple example, the matrix $A = 
		\begin{bmatrix}
			\begin{smallmatrix}
				2 & 0 \\
				0 & 2
			\end{smallmatrix}
		\end{bmatrix}
	$
	has 49 vectors in its core mod 7, but only 9 vectors mod 6. We're interested in whether there's any sort of
	relationship between the cardinalities of a core under a prime modulus verses a core under a prime-powered modulus
	(e.g. 7 vs $7^2$). Many other relations exist between prime and prime-powered moduli (see \citet{Patterson2008}
	and \citet{Mendivil2012}), so it's natural to assume that the core would show some relation as well.
	
	\ssection{Properties of the Core}
	Firstly, given some $L$ by $L$ matrix $A$, how should we treat core$(A)$? Computationally, the core of $A$ is
	obtained by repeatedly multiplying $\mathds{Z}_{N}^{L}$ by $A$ until the set stops shrinking in size (cardinality). 
	At this point, all vectors in the transient region have been removed from the set, and we're left with a set of 
	vectors which shuffle amongst themselves under multiplication by $A$; this set is core$(A)$. Notably, this set 
	is defined entirely by matrix multiplication. If $\tau$ is the transient length of $A$ (see \citet{Strong2022maximal}), 
	then
	\[
		\text{core}(A) = A^{\tau}\mathds{Z}_{N}^{L}
	\]
	
	The core of $A$, then, is the span of the column vectors of $A^{\tau}$:
	\[
		\text{core}(A) = \text{colsp}(A^{\tau})
	\]
	
	So, the core of $A$ can be manipulated as a submodule of $\mathds{Z}_{N}^{L}$; the core isn't some arbitrary
	collection of vectors with no structure. This greatly reduces the possible sets core$(A)$ could be for
	any matrix $A$.
	
	\ssection{Cores Under a Prime Modulus}
	Before considering the prime-powered case, it's beneficial to get an understanding of the prime case.
	Let $A$ be an $L$ by $L$ matrix with elements in $\mathds{Z}_{p}$ for some prime $p$.	The prime case has
	one slight difference which makes it easier to work with than any other case: $\mathds{Z}_{p}$ is a field,
	meaning $\mathds{Z}_{p}^{L}$ can be treated as a vector space rather than just a module.

	For us, this gives an even greater restriction on what core$(A)$ can look like. To see why, consider what
	the span of a vector can look like in a vector space verses a module. In a vector space $\mathds{Z}_{p}^{L}$,
	each vector (except for $\vec{0}$) must have $p$ vectors in its span, while in a module $\mathds{Z}_{N}^{L}$,
	each vector only needs to have a number of vectors which is a \emph{divisor} of $N$ in its span.
	
	For example, in the vector space $\mathds{Z}_{7}^{2}$, the vector $\vec{v} = 
		\begin{bmatrix}
			\begin{smallmatrix}
				2 \\
				4
			\end{smallmatrix}
		\end{bmatrix}
	$
	has 7 vectors in its span:
	\begin{align*}
		0\vec{v} \equiv
		\begin{bmatrix}
			0 \\
			0
		\end{bmatrix} &
		\quad &
		1\vec{v} \equiv
		\begin{bmatrix}
			2 \\
			4
		\end{bmatrix} &
		\quad &
		2\vec{v} \equiv
		\begin{bmatrix}
			4 \\
			1
		\end{bmatrix} &
		\quad &
		3\vec{v} \equiv
		\begin{bmatrix}
			6 \\
			5
		\end{bmatrix} \\
		4\vec{v} \equiv
		\begin{bmatrix}
			1 \\
			2
		\end{bmatrix} &
		\quad &
		5\vec{v} \equiv
		\begin{bmatrix}
			3 \\
			6
		\end{bmatrix} &
		\quad &
		6\vec{v} \equiv
		\begin{bmatrix}
			5 \\
			3
		\end{bmatrix} &
		\quad
	\end{align*}
	
	In the module $\mathds{Z}_{6}^{2}$, the same vector $\vec{v}$ only has 3 vectors in its span:
	\[
		0\vec{v} \equiv
		\begin{bmatrix}
			0 \\
			0
		\end{bmatrix}
		\quad
		1\vec{v} \equiv
		\begin{bmatrix}
			2 \\
			4
		\end{bmatrix}
		\quad
		2\vec{v} \equiv
		\begin{bmatrix}
			4 \\
			2
		\end{bmatrix}
		\quad
		3\vec{v} \equiv 0\vec{v}
	\]
	
	All of this is to say that the span of a vector in a vector space can be more "full" compared to when it's in
	a module. In a vector space, each multiple of a vector is unique (except for $\vec{0}$), whereas in a module
	this isn't necessarily the case.
	
	Because core$(A)$ is created from the span of a set of vectors, this means that, under a prime modulus, its 
	cardinality will always be a power of the modulus. More specifically,
	\[
		|\text{core}(A)| = p^{\text{dim}(\text{core}(A))} \quad \text{for prime moduli} \,\, p 
	\]
	
	Here, $|\text{core}(A)|$ refers to the cardinality of core$(A)$ and dim(core$(A)$) refers to the number
	of directionally-independent vectors (vectors which don't share any vectors in their spans, except for $\vec{0}$) 
	needed to span core$(A)$, the \emph{dimension} of core$(A)$. This equation may look complicated, but all it 
	says is that the vectors that span core$(A)$ for a prime modulus must have "full" spans; the vectors that 
	span core$(A)$ have unique multiples.
	
	Another way to think of this relation is that core$(A)$ cannot have any "gaps" in it for prime moduli. If a
	vector in the span of core$(A)$ points in a specific direction, all the vectors along that direction must
	also be in core$(A)$.
	
	Note that, if core$(A) = \{\vec{0}\}$, then we say dim(core$(A)) = 0$, and the above equation still holds.
	
	\ssection{Relationship Between Cores Under a Prime Modulus and a Prime-Powered Modulus}
	Moving up to the prime-powered case, we note that $\mathds{Z}_{p^{k}}^{L}$ is no longer a vector space, so
	we can't use the same arguments as above to restrict the structure of core$(A)$. However, we can instead 
	make use of the relationships between prime moduli and prime-powered moduli to make conclusions about how 
	cores behave in this setting.
	
	A natural question one may ask, for an $L$ by $L$ matrix $A$, is whether there's any connection between
	core$(A)$ in $\mathds{Z}_{p}^{L}$ vs in $\mathds{Z}_{p^k}^{L}$.
	
	\ssubsection{Lifts}
	It's helpful to understand the concept of a "lift vector" when dealing with prime-powered moduli.
	Let's let
	\[
		\vec{a} = 
		\begin{bmatrix}
			c_{1}  \\
			c_{2}  \\
			\vdots \\
			c_{L}
		\end{bmatrix} \mod{p^k}
	\]
	and let $\vec{a}$ be in some cycle. If we then shift our attention to mod $p^{k+1}$, we find that
	vectors in the form
	\[
		\vec{A}_{i} = 
		\begin{bmatrix}
			p^{k}b_{1} + c_{1} \\
			p^{k}b_{2} + c_{2} \\
			\vdots             \\
			p^{k}b_{L} + c_{L}
		\end{bmatrix} \mod{p^{k+1}}
	\]
	will behave very similarly to $\vec{a}$ mod $p^{k}$. In fact, if we only concern ourselves with the
	$c_i$ terms for both vectors, their iterations will be identical, despite using different moduli.
	This is because any value bigger than or equal to $p^k$ will be fed into the $p^{k}b_{i}$ terms,
	leaving the $c_i$ terms to mimic their behaviour in the mod $p^k$ case. We'll call these $\vec{A}_{i}$ 
	vectors \emph{lifts} of the vector $\vec{a}$. Every vector mod $p^k$ will have corresponding lifts
	mod $p^{k+1}$.
	
	Not every lift of $\vec{a}$ will be in a cycle since the $p^{k}b_{i}$ terms can behave differently.
	However, every lift will eventually lead into a cycle composed of lifts of the cycle of $\vec{a}$. 
	This is merely because there are a finite number of possibilities for the $p^{k}b_{i}$ terms, and 
	these terms don't affect the $c_i$ terms in any way; the $c_i$ terms already cycle, and the $p^{k}b_i$ 
	terms have to cycle eventually.
	
	Another useful trait of any lift $\vec{A}_i$ mod $p^{k+1}$ is that they always reduce to their corresponding
	$\vec{a}$ mod $p^k$, since reducing $\vec{A}_i$ mod $p^k$ would remove the $p^{k}b_i$ terms and retain
	the $c_i$ terms.
	
	The idea of lifts helps relate prime-powered moduli of different exponents, since lifts give us a
	relationship between the behaviour of vectors under both moduli. Note that lifts should be considered
	different from "embed" vectors, which are the type of vectors used in proposition 5 within \citet{Mendivil2012}.
	A vector mod $p^k$ can have several lifts mod $p^{k+1}$ which behave similarly, while that same vector
	mod $p^k$ can only have one embed vector mod $p^{k+1}$ which behaves identically up to homomorphism.
	
	In fact, embed vectors are actually multiples of lift vectors since
	\[
		p\vec{a} \equiv p\vec{A}_i \mod{p^{k+1}}
	\]
	
	\ssubsection{Dimensions of the Core}
	Let's look at the dimension of the core. By proposition 5 in \citet{Mendivil2012}, we know
	the behaviour of core$(A)$ under multiplication by $A$ in $\mathds{Z}_{p^{k}}^{L}$ is embedded in the 
	behaviour of core$(A)$ in $\mathds{Z}_{p^{k+1}}^{L}$. This tells us that the dimension of core$(A)$ in 
	$\mathds{Z}_{p^{k+1}}^{L}$ is at least the same as in $\mathds{Z}_{p^{k}}^{L}$. Otherwise, there's no 
	way the behaviour of the core could be embedded in this way.
	
	We can also show that dim(core$(A)$) in $\mathds{Z}_{p^{k+1}}^{L}$ is at most the same as it was in
	$\mathds{Z}_{p^k}^{L}$. Let
	\[
		\text{core}(A) = \text{span}(\{\vec{v}_{1}, \vec{v}_{2}, \cdots, \vec{v}_{b}\}) \quad \text{for} 
		\,\, \mathds{Z}_{p^k}^{L}
	\]
	We'll assume that, on top of the $b$ vectors we know we need, an extra vector $\vec{w}$ is needed to
	span core$(A)$ in $\mathds{Z}_{p^{k+1}}^{L}$.
	
	Now, since $\vec{w}$ is in the core, that means it's part of a cycle. If $\vec{w}$ cycles mod $p^{k+1}$, 
	then it must also cycle mod $p^{k}$ since $\vec{w}$ must be the lift of some vector mod $p^{k}$. This 
	implies that $\vec{w}$ reduced mod $p^{k}$ must be in core$(A)$ for $\mathds{Z}_{p^k}^{L}$. This means 
	we can write $\vec{w}$ as
	\[
		\vec{w} \equiv x_{1}\vec{j}_{1} + x_{2}\vec{j}_{2} + \cdots + x_{b}\vec{j}_{b} + p^{k}\vec{V} \mod{p^{k+1}}
	\]
	for some lifts $\vec{j}_1 \cdots \vec{j}_b$ of the vectors $\vec{v}_1 \cdots \vec{v}_b$, some constants 
	$x_{1} \cdots x_{b} \in \mathds{Z}_{p^{k+1}}$ and some vector $\vec{V} \in \mathds{Z}_{p}^{L}$. Without loss
	of generality, we can assume all of $x_{i}\vec{j}_i$ are in cycles (and therefore in core$(A)$) since, if 
	one of them wasn't, we could just iterate the vector until we reach an alternate lift for $\vec{v}_i$ which 
	does cycle. Also, since we know the vectors $\vec{v}_1 \cdots \vec{v}_b$ span core$(A)$ for $\mathds{Z}_{p^k}^L$, 
	they must be dimensionally-independent by construction. This means the lifts $\vec{j}_1 \cdots \vec{j}_b$ are 
	also dimensionally-independent, which means each one adds an extra dimension to core$(A)$ in 
	$\mathds{Z}_{p^{k+1}}^L$.
	
	Does the vector $p^{k}\vec{V}$ add another dimension? We can rewrite this vector as
	\[
		p^{k}\vec{V} = p(p^{k-1}\vec{V}) \mod{p^{k+1}}
	\]
	which shows that, by proposition 5 in \citet{Mendivil2012}, $p^{k}\vec{V}$ must be an embed vector for
	some vector in $\mathds{Z}_{p^k}^L$. By definition, embed vectors are simply $p\vec{a}$ mod $p^{k+1}$ 
	for some vector $\vec{a}$ mod $p^k$. We know the vector $\vec{a}$ must cycle since $p^{k}\vec{V}$ must
	cycle (due to $\vec{w}$ being in a cycle), which means $p^{k}\vec{V}$ must be able to be written as
	\begin{align*}
		p^{k}\vec{V} & \equiv p\vec{a}                                                     & \mod{p^{k+1}} \\
								 & \equiv p(y_{1}\vec{v}_1 + y_{2}\vec{v}_2 + \cdots + y_{b}\vec{v}_b) & \mod{p^{k+1}} \\
								 & \equiv p(y_{1}\vec{j}_1 + y_{2}\vec{j}_2 + \cdots + y_{b}\vec{j}_b) & \mod{p^{k+1}}
	\end{align*}
	for some constants $y_1 \cdots y_b \in \mathds{Z}_{p^{k+1}}$. So $p^{k}\vec{V}$ can be written
	in terms of vectors which we've already accounted for in the dimension of core$(A)$ for 
	$\mathds{Z}_{p^{k+1}}^L$, so it cannot add an extra dimension. This means $\vec{w}$ isn't needed
	to fully span core$(A)$ for $\mathds{Z}_{p^{k+1}}^L$.
	
	Therefore, the core mod $p^{k}$ must have the same dimension as the core mod $p^{k+1}$.
	This means that cores under prime-powered moduli must have the same dimension as the core
	under the corresponding prime modulus.
	
	\ssubsection{Cardinalities of the Core}
	Thinking back to our discussion of cores under prime moduli, we found that
	\[
		|\text{core}(A)| = p^{\text{dim}(\text{core}(A))} \quad \text{for prime moduli} \,\, p 
	\]
	Is there a similar relation for prime-powered moduli? 
	
	Given a set of vectors $\vec{v}_1 \cdots \vec{v}_b$ which spans core$(A)$ for $\mathds{Z}_p^L$,
	we know that we can find lifts for these vectors for $\mathds{Z}_{p^k}^L$ which are part of
	the core. We also know that, since the vectors which span core$(A)$ in the prime case
	necessarily have full spans, the lifts we find will have full spans, too.
	
	To see why this is true, let's look at an example. The vector $\vec{v} = 
		\begin{bmatrix}
			\begin{smallmatrix}
				8 \\
				12
			\end{smallmatrix}
		\end{bmatrix}
	$
	mod 49 ($7^{2}$) has a full span since each of its multiples are unique. Neither 8 nor 12 is 
	a multiple of 7, and thus their additive orders (the number of times they must be added to get 
	zero) are each 49. This means the entire vector's additive order is also 49.
	
	If we move up to mod 343 ($7^{3}$), there are 49 vectors which reduce to $\vec{v}$ mod 49, 
	all of them taking the form 
	\[
	\vec{w} =
		\begin{bmatrix}
			49a + 8 \\
			49b + 12
		\end{bmatrix}
		=
		\begin{bmatrix}
			49a + 7 + 1 \\
			49b + 7 + 5
		\end{bmatrix}
	\]
	
	As is evident from this form, none of the 49 possible lifts of $\vec{v}$ can have components 
	that are multiples of 7, and therefore all their additive orders are 343. 
	
	This is a general property: if a number isn't a multiple of the prime under a prime-powered
	modulus, then its lifts will also not be multiples of the prime under any prime-powered
	modulus (of the same prime, of course). This means that, if a vector has maximum additive
	order under some prime-powered modulus, then its lifts will also have maximum additive
	order under any prime-powered modulus (again, of the same prime).
	
	Using this fact, we instantly know that the relationship surrounding the core's cardinality
	under a prime modulus also extends to a prime-powered modulus. We've shown the dimension of 
	the core has to remain fixed for all prime-powered moduli deriving from the same prime, and
	of these dimensions, we can take lifts of the vectors which span the core for the prime case
	and use them to span the prime-powered case. These vectors will all have full spans, so the
	cardinality of the core of a matrix $A$ in these cases can be written as
	\[
		|\text{core}(A)| = (p^{k})^{\text{dim(core}(A))} \text{ for prime-powered moduli } p^{k}
	\]
	
	This tells us that, for all prime-powered moduli, the size of the core will always be
	maximal. Given the dimension of a core and a prime-powered modulus, we can instantly
	find out what the cardinality of the core must be.
	
	\ssection{Significance}
	There are a few reasons why it's beneficial to understand how the cores of matrices
	behave under prime-powered moduli.
	
	Knowing that cores under prime-powered moduli must always be "full" allows a proper
	basis to be formed for the core without the use of an irregularly-shaped module
	(a module where different moduli are used for each vector component). This, in turn,
	allows for matrices to more easily be restricted to the core, which allows specific
	ideas and results to be applied to them (such as the result in \citet{Strong2022maximal}).
	
	As well, having a better general idea of what the core looks and acts like allows better
	conclusions to be drawn about its converse: the transient region. By better understanding
	and enumerating the core of a matrix, questions such as the number of transient vectors
	can be reframed in terms of the core, an arguably easier construct to understand.
	
	\bibliographystyle{plainnat}
	\bibliography{refs.bib}

\end{document}