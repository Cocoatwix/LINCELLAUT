
\documentclass[a4paper, reqno, 12pt]{amsart}

\usepackage[T1]{fontenc}
\usepackage[margin=3cm]{geometry}
\usepackage[parfill]{parskip}

\usepackage{setspace}
\setstretch{1.25}

\usepackage{dsfont}
\usepackage{amssymb}
%\usepackage{upgreek} %Uppercase Greek letters

\usepackage{array}

\usepackage{natbib}

%Cyclespace notation
\newcommand\cycsp[1]{\mathcal{S}_{#1}}

%Cycle length notation
%\newcommand\cyclen[1]{\upgreek_{#1}}

%span
\newcommand\vecspan[1]{\text{span}(#1)}

\begin{document}
	\section{Definitions \& Notation}
		\begin{description}
			\item[Set of all $L \times 1$ vectors modulo $n$] Represented as $\mathds{Z}_{n}^{L}$.
			
			\item[Set of all $M \times N$ matrices] Represented as $\mathds{Z}^{M \times N}$
			
			\item[Set of all $M \times N$ matrices modulo $n$] Represented as $\mathds{Z}_{n}^{M \times N}$.
			
			\item[Transient length of $A$] The largest positive integer $\tau$ such that, for all positive integers $c$,
			\[
				A^{\tau-1} \not\equiv A^{\tau - 1 + c}
			\]
			The transient length is said to be 0 if no such positive integer $\tau$ exists.
			
			\item[Transient length of $\vec{v}$ under $A$] The largest positive integer $\tau$ such that, for all positive integers $c$,
			\[
				A^{\tau-1}\vec{v} \not\equiv A^{\tau-1+c}\vec{v}
			\]
			The transient length is said to be 0 if no such positive integer $\tau$ exists.
			
			\item[Cycle length of $A$] The smallest positive integer $\omega$ such that
			\[
				A^{\tau} \equiv A^{\tau + \omega}
			\]
			for the matrix's transient length $\tau$.
			
			\item[Cycle length of $\vec{v}$ under $A$] The smallest positive integer $\omega$ such that
			\[
				A^{\tau}\vec{v} \equiv A^{\tau + \omega}\vec{v}
			\]
			for the vector's transient length $\tau$.
			
			\item[Iteration by $A$] The product of a vector and a matrix $A$. Repeated iteration takes a previous iteration and uses it as the vector of
			another iteration.
			
			\item[Minimal polynomial of $A$] The monic polynomial of least degree $\lambda(x)$ such that
			\[
				\lambda(A) \equiv 0 \mod{p}
			\]
			for prime $p$.
			
			\item[Minimal annihilating polynomial of $\vec{v}$ under $A$] The monic polynomial of least degree $\lambda(x)$ such that 
			\[
				\lambda(A)\vec{v} \equiv \vec{0} \mod{p}
			\]
			for prime $p$.
			
			\item[Cyclespace of $\vec{v}$ under $A$] The submodule defined by 
			\[
				\cycsp{\vec{v}} = \text{span}(\{\vec{v}, A\vec{v}, A^{2}\vec{v}, \cdots\})
			\]
			where $A$ is the update matrix being used. The update matrix $A$ will be apparent from context.
			
			\item[Maximal vector] A vector with the same cycle length as the update matrix being used.
			
			\item[LCA] Linear Cellular Automata. Refers to a system involving a module, an update matrix acting on that module, and a modulus.
		\end{description}
		
	\section{Primary Decomposition Into Cyclespaces}
		Let $A \in \mathds{Z}^{L \times L}$, $L \in \mathds{Z}^{+}$, with $\ker(A) = \{\vec{0}\}$. We'll denote the characteristic
		polynomial of $A$ modulo $p$ as $\kappa(x)$ and the minimal polynomial of $A$ modulo $p$ as $m(x)$, where $p$ is prime.
		
		Let the prime factorisation of $m(x)$ be $f_{1}^{n_{1}}(x)f_{2}^{n_{2}}(x)\cdots f_{g}^{n_{g}}(x)$ with each $f_i^{n_i}$ monic and relatively prime. 
		The Primary Decomposition Theorem says that each factor $f_i^{n_i}$ corresponds to a subspace $U_i$ such that, for all vectors $\vec{u} \in U_i$,
		$f_i^{n_i}(A)\vec{u} \equiv \vec{0}$. It also guarantees that each $U_i \neq \{\vec{0}\}$, and that each one is disjoint from one another, except for the 
		zero vector.
		
		It's guaranteed that, for each $U_i$, there exists at least one vector $\vec{v}_i \in U_i$ whose minimal annihilating polynomial is exactly $f_i^{n_i}$.
		Otherwise, if such a vector didn't exist, then $m(x)$ wouldn't be the minimal polynomial of $A$ as a polynomial of lesser degree would annihilate all
		vectors in the vector space. It's also guaranteed that each $U_i$ is invariant under iteration by $A$. In order for a vector to "jump" from one $U_i$
		to another via iteration by $A$, $A\vec{v}$ would have to be annihilated by a polynomial that isn't an annihilating polynomial of $\vec{v}$. But
		\begin{align*}
			         & q(A)(A\vec{v}) \equiv \vec{0}                             \\
			\implies & A(q(A)\vec{v}) \equiv \vec{0}                             \\
			\implies & A\vec{w} \equiv \vec{0}, \quad \vec{w} \not\equiv \vec{0} \\
			\implies & \ker(A) \neq \{\vec{0}\}
		\end{align*}
		and by our assumptions, $\ker(A) = \{\vec{0}\}$. Therefore, when $\ker(A) = \{\vec{0}\}$, vectors can never "jump" from one $U_i$ to another from 
		iteration by $A$.
		
		In the case where $\kappa(x) \equiv m(x)$, $U_i = \cycsp{\vec{v}_i}$ where the minimal annihilating polynomial of $\vec{v}$ is $f_i^{n_i}$. Why?
		$m(x)$ will have the same degree as $\kappa(x)$, so the sum of the degrees of each $f_i^{n_i}$ will equal the degree of $\kappa(x)$. We also know the 
		dimension of each $U_i$ will at least be the degree of $f_i^{n_i}$. This is because the degree of a vector's minimal annihilating polynomial dictates how many 
		iterations by $A$ it takes before a vector can be written as a linear combination of previous iterations, and we know $\vec{v}_i \in U_i$ has $f_i^{n_i}$ as 
		its minimal annihilating polynomial.
		
		Since $\kappa(x) \equiv m(x)$, the dimension of any $U_i$ can be no more than the degree of $f_i^{n_i}$. If it was, at least one pair of $U_i$ subspaces
		would share more than the zero vector since the sum of the dimensions of each $U_i$ would be greater than the dimension of the entire vector space (given 
		by the degree of $\kappa(x)$). Therefore, in this case, the dimension of each $U_i$ is exactly the degree of each $f_i^{n_i}$. It follows that 
		$U_i = \cycsp{\vec{v}_i}$ since $U_i$ is invariant under iteration by $A$, and because any vector in $U_i$ implies that all multiples of that vector are 
		in $U_i$ (i.e. $U_i$ is a subspace).
		
	\section{Forming a Maximal Basis}
		If $A$ is the same as defined above, and if $\kappa(x) \equiv m(x) \bmod{p}$, with $p$ an odd prime, then we can form a basis for $\mathds{Z}_{p}^{L}$ using 
		maximal vectors under $A$ as follows:
		
		Let $\vec{v}_1, \vec{v}_2, \cdots, \vec{v}_g$ be the set of vectors whose cyclespaces define the subspaces $U_1$ to $U_g$ guaranteed by the Primary
		Decomposition Theorem, and let $d_1, d_2, \cdots, d_g$ be the dimensions of $U_1$ to $U_g$. Because all of $U_i$ are disjoint (except for the zero vector),
		Proposition 2 in \citet{Mendivil2012} says that adding together one vector from each $U_i$ will result in a vector whose cycle length is the least common
		multiple of each vector's cycle length in the sum. If we choose our vectors from each of $U_i$ to have maximal cycle length in their respective subspace,
		the resulting vector created will necessarily be a maximal vector under $A$.
		
		Thankfully, we know each of $\vec{v}_i$ to have maximal cycle length in their respective $U_i$ since they are generating vectors for each subspace.
		By constructing two sets of vectors $\{\vec{\alpha}\}$ and $\{\vec{\beta}\}$ where each vector in both sets is maximal and where 
		$\{\vec{\alpha}\} \cup \{\vec{\beta}\}$ forms a basis for the vector space, we can create a maximal basis.
		
		Define the two sets of vectors as:
		\[
			\vec{\alpha}_{i,j} \equiv \vec{v}_1 + \vec{v}_2 + \cdots + \vec{v}_{i-1} + A^{j}\vec{v}_i + \vec{v}_{i+1} + \cdots + \vec{v}_g, \quad
			i,j \in \mathds{Z}^+, 1 \leq i \leq g, 1 \leq j < d_i
		\]
		and
		\begin{align*}
			\vec{\beta}_1 \, &\equiv \vec{v}_1 + \vec{v}_2 + \cdots \vec{v}_g \\
			\vec{\beta}_i \, &\equiv \vec{v}_1 + \vec{v}_2 + \cdots + \vec{v}_{i-1} + 2\vec{v}_i + \vec{v}_{i+1} + \cdots + \vec{v}_g, \quad
			 i \in \mathds{Z}^+, \, 2 \leq i \leq g
		\end{align*}
		
		Each vector in $\{\vec{\alpha}\}$ is linearly independent from one another since only one $\vec{\alpha}_{i,j}$ will have the vector $A^{j}\vec{v}_i$ in it,
		and this vector provides a unique span to any other vector.
		
		Showing the vectors in $\{\vec{\beta}\}$ to be linearly independent is a little more involved. Lining each $\vec{\beta}$ up, we see
		\begin{center}
			\begin{tabular}{*{11}{c}}
				$\vec{\beta}_1$ & $\equiv$ & $\vec{v}_1$ & + & $\vec{v}_2$  & + & $\vec{v}_3$  & + & $\vec{v}_4$  & + & $\cdots$ \\
				$\vec{\beta}_2$ & $\equiv$ & $\vec{v}_1$ & + & $2\vec{v}_2$ & + & $\vec{v}_3$  & + & $\vec{v}_4$  & + & $\cdots$ \\
				$\vec{\beta}_3$ & $\equiv$ & $\vec{v}_1$ & + & $\vec{v}_2$  & + & $2\vec{v}_3$ & + & $\vec{v}_4$  & + & $\cdots$ \\
				$\vec{\beta}_4$ & $\equiv$ & $\vec{v}_1$ & + & $\vec{v}_2$  & + & $\vec{v}_3$  & + & $2\vec{v}_4$ & + & $\cdots$ \\
				$\vdots$        & $\equiv$ & $\vdots$    & + & $\vdots$     & + & $\vdots$     & + & $\vdots$     & + & $\ddots$
			\end{tabular}
		\end{center}
		which can be rewritten using a matrix equation:
		\[
			\begin{bmatrix}
				\vec{\beta}_1 \\
				\vec{\beta}_2 \\
				\vec{\beta}_3 \\
				\vec{\beta}_4 \\
				\vdots        \\
				\vec{\beta}_g
			\end{bmatrix}
			\equiv
			\begin{bmatrix}
				1      & 1      & 1      & 1      & \cdots & 1      \\
				1      & 2      & 1      & 1      & \cdots & 1      \\
				1      & 1      & 2      & 1      & \cdots & 1      \\
				1      & 1      & 1      & 2      & \cdots & 1      \\
				\vdots & \vdots & \vdots & \vdots & \ddots & \vdots \\
				1      & 1      & 1      & 1      & \cdots & 2
			\end{bmatrix}
			\begin{bmatrix}
				\vec{v}_1 \\
				\vec{v}_2 \\
				\vec{v}_3 \\
				\vec{v}_4 \\
				\vdots    \\
				\vec{v}_g
			\end{bmatrix}
		\]
		If we can show that this coefficient matrix is invertible, we can show that the vectors in $\{\vec{\beta}\}$ are linearly independent. One way to show this 
		matrix is invertible is to show that the equation
		\[
			\begin{bmatrix}
				1      & 1      & 1      & 1      & \cdots & 1      \\
				1      & 2      & 1      & 1      & \cdots & 1      \\
				1      & 1      & 2      & 1      & \cdots & 1      \\
				1      & 1      & 1      & 2      & \cdots & 1      \\
				\vdots & \vdots & \vdots & \vdots & \ddots & \vdots \\
				1      & 1      & 1      & 1      & \cdots & 2
			\end{bmatrix}
			\vec{a} \equiv \vec{0}
		\]
		has only the trivial solution $\vec{a} \equiv \vec{0}$. Let
		\[
			I = 
			\begin{bmatrix}
				1      & 0      & 0      & 0      & \cdots & 0      \\
				0      & 1      & 0      & 0      & \cdots & 0      \\
				0      & 0      & 1      & 0      & \cdots & 0      \\
				0      & 0      & 0      & 1      & \cdots & 0      \\
				\vdots & \vdots & \vdots & \vdots & \ddots & \vdots \\
				0      & 0      & 0      & 0      & \cdots & 1
			\end{bmatrix}
		\]
		and
		\[
			N =
			\begin{bmatrix}
				1      & 1      & 1      & 1      & \cdots & 1      \\
				1      & 1      & 1      & 1      & \cdots & 1      \\
				1      & 1      & 1      & 1      & \cdots & 1      \\
				1      & 1      & 1      & 1      & \cdots & 1      \\
				\vdots & \vdots & \vdots & \vdots & \ddots & \vdots \\
				1      & 1      & 1      & 1      & \cdots & 1
			\end{bmatrix}
		\]
		and
		\[
			L = 
			\begin{bmatrix}
				1      & 0      & 0      & 0      & \cdots & 0      \\
				0      & 0      & 0      & 0      & \cdots & 0      \\
				0      & 0      & 0      & 0      & \cdots & 0      \\
				0      & 0      & 0      & 0      & \cdots & 0      \\
				\vdots & \vdots & \vdots & \vdots & \ddots & \vdots \\
				0      & 0      & 0      & 0      & \cdots & 0
			\end{bmatrix}
		\]
		Then
		\[
			\begin{bmatrix}
				1      & 1      & 1      & 1      & \cdots & 1      \\
				1      & 2      & 1      & 1      & \cdots & 1      \\
				1      & 1      & 2      & 1      & \cdots & 1      \\
				1      & 1      & 1      & 2      & \cdots & 1      \\
				\vdots & \vdots & \vdots & \vdots & \ddots & \vdots \\
				1      & 1      & 1      & 1      & \cdots & 2
			\end{bmatrix}
			= I + N - L
		\]
		
		Now, we'd like to show that the equation
		\[
			(I + N - L)\vec{a} \equiv \vec{0}
		\]
		has only the trivial solution for $\vec{a}$. Rearranging and expanding the vectors, we see
		\[
			\begin{bmatrix}
				a_1    \\
				a_2    \\
				a_3    \\
				\vdots \\
				a_g
			\end{bmatrix}
			+
			\begin{bmatrix}
				c      \\
				c      \\
				c      \\
				\vdots \\
				c
			\end{bmatrix}
			\equiv
			\begin{bmatrix}
				a_1    \\
				0      \\
				0      \\
				\vdots \\
				0
			\end{bmatrix}
		\]
		where $c \equiv a_1 + a_2 + \cdots + a_g$. This implies $c \equiv 0$, so
		\[
			\begin{bmatrix}
				a_1    \\
				a_2    \\
				a_3    \\
				\vdots \\
				a_g
			\end{bmatrix}
			\equiv
			\begin{bmatrix}
				a_1    \\
				0      \\
				0      \\
				\vdots \\
				0
			\end{bmatrix}
		\]
		This implies that $a_2, a_3, \cdots, a_g \equiv 0$, so
		\[
			\vec{a} \equiv
			\begin{bmatrix}
				a_1    \\
				0      \\
				0      \\
				\vdots \\
				0
			\end{bmatrix}
		\]
		But we know $N\vec{a} \equiv 0$, meaning $a_1 \equiv c \equiv 0$. Therefore $\vec{a} \equiv \vec{0}$. This means the coefficient
		matrix is invertible, meaning the vectors in the set $\{\vec{\beta}\}$ are linearly independent.
		
		The vectors in the set $\{\vec{\alpha}\} \cup \{\vec{\beta}\}$ are also linearly independent, as none of $\vec{\beta}$ can be in $\{\vec{\alpha}\}$ since
		each $\vec{\beta}$ has an $A^{j}\vec{v}_i$ term in its sum which cannot be removed by any other $\vec{\beta}$, and none of $\vec{\alpha}$ can be in 
		$\{\vec{\beta}\}$ since none of $\vec{\alpha}$ have an $A^{j}\vec{v}_i$ term in their sums.
		
		So $\{\vec{\alpha}\} \cup \{\vec{\beta}\}$ forms a linearly independent set of $\sum_{i\,=\,0}^{g}(d_i - 1) + g = \sum_{i\,=\,0}^{g}(d_i)$ vectors, which
		is enough to span the entire vector space and therefore forms a basis. Also, each $\vec{\alpha}$ and $\vec{\beta}$ is a maximal vector by construction, so 
		this basis is a maximal basis.
		
	\section{Maximal Basis for LCAs with Prime-Power Moduli}
		If a maximal basis exists for an LCA with an odd prime modulus, then that same basis can be used for LCAs with a modulus which is a power of that prime and
		which use the same update matrix. Granted, the vectors will no longer necessarily be maximal, but they can be used to prove that at least one maximal vector 
		exists in the prime-power case.
		
		Let $A \in \mathds{Z}^{L \times L}$ be an invertible matrix mod $p$, an odd prime, where $\kappa(A) \equiv m(A)$ and $L \in \mathds{Z}^+$. In this case,
		we know a maximal basis exists. Denote this basis as $V$. Also, let the cycle length of $A$ modulo $p$ be denoted as $\omega$.
		
		Now, consider the LCA created using the same update matrix $A$ but with $p^2$ as the modulus. There are two possibilities for the cycle length of $A$ 
		modulo $p^2$: either it stays the same or it increases by a factor of $p$.
		
		To see why, consider the equation
		\[
			A^\omega \equiv I \mod{p}
		\]
		We know this is true since $A$ is invertible and since the cycle length of $A$ mod $p$ is $\omega$. Raising this equation to mod $p^2$, we see
		\[
			A^\omega \equiv I + pB \mod{p^2}, \quad B \in \mathds{Z}_p^{L \times L}
		\]
		since $A^\omega$ must reduce down to $I$ mod $p$. If $B \equiv 0$, then the cycle length of $A$ remains unchanged. If $B \not\equiv 0$, then the cycle 
		length must have increased. How much does it increase? Using induction, one can show that
		\[
			A^{a\omega} \equiv I + apB \mod{p^2}
		\]
		Plugging in $a = p$, we see
		\[
			A^{p\omega} \equiv I + ppB \equiv I + p^2B \equiv I \mod{p^2}
		\]
		We know the cycle length of $A$ modulo $p^2$ must be some multiple of $\omega$ since $A^c$ modulo $p^2$ must always reduce to $A^c$ modulo $p$ for
		some integer $c$. Therefore, we can conclude that the cycle length of $A$ modulo $p^2$ is $p\omega$ if $B \not\equiv 0$.
		
		If $B \equiv 0 \bmod{p^2}$, then a maximal vector clearly exists since any vector from $V$ will still be maximal---the cycle length
		wouldn't have changed. Going from modulo $p^2$ to $p^3$, the same argument as above can be used: either the cycle length stays the same or it increases by
		a factor of $p$. 
		
		Let $p^k$ be the first modulus where the cycle length increases. Then we have
		\[
			A^\omega \equiv I + p^{k-1}B \mod{p^k}, \quad B \not\equiv 0
		\]
		We know that $V$ spans the entire module by construction, so at least one $\vec{h} \in V$ will not be in the kernel of $B$ since $B \not\equiv 0$. 
		But then this means
		\[
			A^{\omega}\vec{h} \not\equiv \vec{h} \mod{p^k}
		\]
		so the cycle length of $\vec{h}$ is greater than $\omega$. Much like matrices, cycle lengths for vectors must be multiples of their cycle lengths for
		lower-power moduli since the vector's iterations must always be able to reduce to lower-power moduli. We know the first power of $A^\omega$ that doesn't have 
		a $B$ in its sum is $A^{p\omega}$, so the cycle length of $\vec{h}$ must be $p\omega$. 
		
		The only possible way this couldn't be true is if, for some integer $1 \leq a \leq p$, $a\vec{h} \equiv 0 \bmod{p^k}$, but this can't happen since all the 
		vectors in $V$ are plucked directly from $\mathds{Z}_p^L$ so they can't have a factor of $p$ in them which would cause torsion.
		
		Another result we can show using induction is that, for odd prime $p$,
		\[
			A^{\omega} \equiv I + p^{k-1}B \bmod{p^k} \implies A^{p\omega} \equiv I + p^{k}B \bmod{p^{k+1}}
		\]
		where the matrix $B$ stays the same on each side of the implication. This means, for odd prime, the vector $\vec{h}$ found above will \emph{always} have
		maximal cycle length since, due to $\vec{h}$ not being in the kernel of $B$, we can make the same argument about its cycle length increasing for each
		prime-power modulus, showing each time that its cycle length must be the same as the matrix's, and therefore being maximal.
		
	\bibliographystyle{plainnat}
	\bibliography{refs.bib}

\end{document}